\documentclass[12pt]{article}
\usepackage[utf8]{inputenc}
\usepackage[T1]{fontenc}
\usepackage{amsmath}
\usepackage{amsfonts}
\usepackage{amssymb}
\usepackage[version=4]{mhchem}
\usepackage{stmaryrd}
\usepackage{hyperref}
\hypersetup{colorlinks=true, linkcolor=blue, filecolor=magenta, urlcolor=cyan,}
\urlstyle{same}
\usepackage{physics}
\usepackage{graphicx}

\usepackage{listings} % Required for insertion of code
\usepackage{xcolor} % Required for custom colors

% Define custom colors
\definecolor{codegreen}{rgb}{0,0.6,0}
\definecolor{codegray}{rgb}{0.5,0.5,0.5}
\definecolor{codepurple}{rgb}{0.58,0,0.82}
\definecolor{backcolour}{rgb}{0.95,0.95,0.92}

% Setup the style for code listings
\lstdefinestyle{mystyle}{
    backgroundcolor=\color{backcolour},   
    commentstyle=\color{codegreen},
    keywordstyle=\color{magenta},
    numberstyle=\tiny\color{codegray},
    stringstyle=\color{codepurple},
    basicstyle=\ttfamily\footnotesize,
    breakatwhitespace=false,         
    breaklines=true,                 
    captionpos=b,                    
    keepspaces=true,                 
    numbers=left,                    
    numbersep=5pt,                  
    showspaces=false,                
    showstringspaces=false,
    showtabs=false,                  
    tabsize=2
}

% Activate the style
\lstset{style=mystyle}

\title{PROBLEMS: }

\author{}
\date{}


\begin{document}
\maketitle
Physics 125b

Problem set number 3

Due midnight Wednesday, January 24, 2024

READING: Section 17.3 in Shankar on degenerate perturbation theory. Also sections 18.1 and 18.2 on time-dependent perturbation theory.\\
\emph{I have an extension for 2 days, so this is due on the 26th.}
\section{}
\begin{enumerate}
  \setcounter{enumi}{8}
  \item We have solved the Schrödinger equation for the Hydrogen atom with Hamiltonian:
\end{enumerate}

$$
H_{0}=\frac{p^{2}}{2 m}-\frac{e^{2}}{r}
$$

The kinetic energy term is non-relativistic - the actual kinetic energy will have relativistic corrections.
(a) Obtain an expression for the next order relativistic (kinetic energy) correction to the energy spectrum of hydrogen. It is convenient to avoid taking multiple derivatives by using the unperturbed Schrödinger equation to eliminate them. Thus, write your expression in terms of the unperturbed energies and expectation values of $\frac{e^{2}}{r}$ and $\left(\frac{e^{2}}{r}\right)^{2}$. Do not actually do the integration over $r$ here, but reduce the problem to such integrals (which you may just write as expectation values). Make sure you understand all of your steps.
\subsection{}
If we want to take into account relativistic effects, the kinetic energy is:
\begin{equation}
  T = \sqrt{p^2c^2 + m^2c^4} - mc^2 = \frac{p^2}{2m} - \frac{p^4}{8m^3c^2} + \cdots
\end{equation}
So we want to find the effect of the extra term, or perturbation:
\begin{equation}
  H' = -\frac{P^4}{8m^3c^2}
\end{equation}
Since the perturbation doesn't depend on any angular coordinates, we just care about the diagonal elements in the $\ket{nlm}$ basis (we show in problem 12 that the off-diagonal elements vanish). Then we have:
\begin{equation}
  E^{1}_T = \bra{nlm}H'\ket{nlm} = -\frac{\bra{nlm}P^4\ket{nlm}}{8m^3c^2}
\end{equation}
It hubs to notice that:
\begin{equation}
  P^4 = 4m^2\left(\frac{P^2}{2m} \right)^2 = 4m^2\left(H^0 + \frac{e^2}{r} \right)^2
\end{equation}
So we can get for the matrix element:
\begin{equation}
  E^{1}_T = -\frac{1}{2mc^{2}}  \left[ (E_0)^{2}+e^4\left\langle\frac{1}{r^2}\right\rangle+2 E_0e^2\left\langle\frac{1}{r}\right\rangle\right]
\end{equation}

(b) Now apply your formula to obtain the first-order relativistic kinetic energy correction to the ground state energy of hydrogen. Express your answer as a multiple of the unperturbed ground state energy, and also calculate the size of the correction in $\mathrm{eV}$.
\subsection{}
We know both:
\begin{equation}
  \left\langle\frac{1}{r}\right\rangle = -\frac{2E_0}{e^{2}}
\end{equation}
and:
\begin{equation}
  \left\langle\frac{1}{r^2}\right\rangle = \frac{4E_{0}^{2}n}{l + 1 / 2}
\end{equation}
So we get:
\begin{equation}
  E^{1}_T = -\frac{1}{2mc^{2}}  \left[ (E_0)^{2}+e^4\frac{4E_{0}^{2}n}{l + 1 / 2}-4E_0e^2\frac{2E_0}{e^{2}}\right]
\end{equation}
Factoring out the $E_0$ term, we get:
\begin{equation}
  E^{1}_T = - \frac{E_0^2}{2mc^2} \left[-3 + \frac{4n}{l + 1 / 2} \right]
\end{equation}
The value of $E_0$ for the ground state of hydrogen is:
\begin{equation}
  E_0 = -\frac{mc^2\alpha^2}{2n^2}
\end{equation}
where $\alpha$ is the fine structure constant:
Substituting this in, we get for the ground state:
\begin{equation}
  E^{1}_T = \frac{mc^2\alpha^2}{8n^4} \left[3 - \frac{4n}{l + 1 / 2} \right]
\end{equation}
For the hydrogen ground state, we have $n=1$ and $l=0$, so:
\begin{equation}
  E_T^1 = -\frac{5mc^2\alpha^2}{8} 
\end{equation}
This give a first order correction to the energy of the ground state of hydrogen of:
\begin{equation}
  E^{1}_T = -0.0009067 eV
\end{equation}
% Inline Python code in the document
\begin{lstlisting}[language=Python]
# Constants
m_e = 9.109e-31  # Electron mass in kg
c = 2.998e8      # Speed of light in m/s
alpha = 1/137    # Fine structure constant
eV = 1.602e-19   # Joules per electron volt

# First-order relativistic correction (in Joules)
E1_T_joules = - (5 * m_e * c**2 * alpha**4) / 8

# Convert to electron volts
E1_T_eV = E1_T_joules / eV
E1_T_joules, E1_T_eV

\end{lstlisting}
\section{}
\begin{enumerate}
  \setcounter{enumi}{9}
  \item The nucleus is not a point, though it is to a good approximation in computing the atomic energy levels of light atoms. We may use perturbation theory to estimate the effect of the nuclear size on the energy levels. We'll take a simple model for our nucleus as a sphere of radius $\rho$ containing a uniform charge distribution, with total charge $Z e$, where $e$ is the proton charge.
\end{enumerate}

(a) What is the potential energy of the electron, assuming a one-electron atom? Use this to write down an expression for a potential as a perturbation to our point-like nucleus Hamiltonian.
\subsection{}
The volume of the sphere is given by:
\begin{equation}
  V = \frac{4}{3} \pi \rho^3
\end{equation}
Since the charge density is constant true out the sphere and the total charge is $Ze$, we have:
\begin{equation}
  \sigma  = \frac{3Ze}{4\pi\rho^3}
\end{equation}
Inside the sphere when $r < \rho$, the enclosed charge is:
\begin{equation}
  Q = \frac{4}{3} \pi r^3 \sigma = \frac{3Ze}{4\pi\rho^3} \frac{4}{3} \pi r^3 = Ze \frac{r^3}{\rho^3}
\end{equation}
Then we might use Gauss's law to determine the electric field inside the sphere:
\begin{equation}
  \mathbf{E} = \frac{Q}{4\pi\epsilon_0 r^2} = \frac{Ze}{4\pi\epsilon_0\rho^3} r
\end{equation}
To find the potential energy and side of the nucleus, we ingrate the electric field::
\begin{equation}
  V = \int \mathbf{E} \cdot d\mathbf{r} = \int_{r}^{\rho } \frac{Ze}{4\pi\epsilon_0\rho^3} r dr = \frac{Ze}{4\pi\epsilon_0\rho^3} \frac{r^2}{2} \Big|_{r}^{\rho} = \frac{Ze}{4\pi\epsilon_0\rho^3} \frac{\rho^2}{2} - \frac{Ze}{4\pi\epsilon_0\rho^3} \frac{r^2}{2} = \frac{Ze(-r^2 + \rho^2)}{8\pi\epsilon_0\rho^3}
\end{equation}
We know the solutions for the point-like nucleus Hamiltonian with $V = -\frac{Ze^2}{4\pi\epsilon_0 r}$, so we can write the perturbation as:
\begin{equation}
  H' = -\frac{Ze(-r^2 + \rho^2)}{8\pi\epsilon_0\rho^3} - -\frac{Ze^2}{4\pi\epsilon_0 r}
\end{equation}



(b) Assume the nucleus is small compared with the atom. The one-electron wave function is $\psi_{n l m}(r, \theta, \phi)=R_{n l}(r) Y_{l m}(\theta, \phi)$. What is the expectation value of your perturbation with respect to this state? Note that the assumption of a small nucleus suggests that you can just evaluate the radial wave function at $r=0$ (which you may leave in the form $R_{n \ell}(0)$ ), since the wave function varies little over the extent of the nucleus.
\subsection{}
we want to consider:
\begin{equation}
  \bra{\psi_{n l m}(r, \theta, \phi)}H'\ket{\psi_{n l m}(r, \theta, \phi)} = \bra{\psi_{n l m}(r, \theta, \phi)}-\frac{Ze^2}{4\pi\epsilon_0 r} - -\frac{Ze(-r^2 + \rho^2)}{8\pi\epsilon_0\rho^3}\ket{\psi_{n l m}(r, \theta, \phi)}
\end{equation}
This integral is:
\begin{equation}
  \int_{0}^{\infty} \int_{0}^{\pi} \int_{0}^{2 \pi} \psi_{n l m}^{*}(-\frac{Ze^2}{4\pi\epsilon_0 r} - -\frac{Ze(-r^2 + \rho^2)}{8\pi\epsilon_0\rho^3}) \psi_{n l m} r^{2} \sin \theta d r d \theta d \phi
\end{equation}
Now, the perturbation doesn't have any angular dependence, so we can integrate those out:
\begin{equation}
  = 4 \pi \int_{0}^{\infty} R_{nl}(r)^{*}(-\frac{Ze^2}{4\pi\epsilon_0 r} - -\frac{Ze(-r^2 + \rho^2)}{8\pi\epsilon_0\rho^3}) R_{nl}(r) r^{2} d r
\end{equation}
We also notice that the perturbation is only non zero inside of the nucleus and also the radial wave functions don't change much in this interval:
\begin{equation}
  = 4 \pi R_{nl}(0)^2 \int_{0}^{\rho}(-\frac{Ze^2}{4\pi\epsilon_0 r} - -\frac{Ze(-r^2 + \rho^2)}{8\pi\epsilon_0\rho^3}) r^{2} d r
\end{equation}
For the integral, SymPy gives:
\begin{equation}
  = 4 \pi R_{nl}(0)^2 \left(-\frac{Ze^2\rho^2}{8\pi\epsilon_0} + \frac{Ze\rho^2}{60\pi\epsilon_0} \right)
\end{equation}
% Inline Python code in the document
\begin{lstlisting}[language=Python]
from sympy import symbols, pi, integrate

# Defining the symbols
r, rho, Z, e, epsilon_0, R_nl_0 = symbols('r rho Z e epsilon_0 R_nl_0')

# Defining the integrand
integrand = (-Z*e**2 / (4*pi*epsilon_0*r) - -Z*e*(-r**2 + rho**2) / (8*pi*epsilon_0*rho**3)) * r**2

# Performing the integration
integral_result = 4 * pi * R_nl_0**2 * integrate(integrand, (r, 0, rho))
integral_result
\end{lstlisting}
Now, we knows that the radial wave function at $r=0$ is:
\begin{equation}
  R_{nl}(0) = \left(\frac{2 Z}{n a_{0}}\right)^{3 / 2} \sqrt{\frac{(n-1) !}{2 nn !}} L_{n-1}^{1}\left(0\right) = \left(\frac{2 Z}{n a_{0}}\right)^{3 / 2} \sqrt{\frac{1}{2n^2}}\binom{n}{n-1} = 2 \left(\frac{Z}{n a_{0}}\right)^{3 / 2}
\end{equation}
So we get for the matrix element:
\begin{equation}
  = 8 \pi  \left(\frac{Z}{n a_{0}}\right)^{3} \left(-\frac{Ze^2\rho^2}{8\pi\epsilon_0} + \frac{Ze\rho^2}{60\pi\epsilon_0} \right)
\end{equation}
We can simplify this to:
\begin{equation}
  = \frac{Z^{4} e \rho^{2} \cdot \left(2 - 15 e\right)}{15 a_{0}^{3} \epsilon_{0} n^{3}}
\end{equation}
% Inline Python code in the document
\begin{lstlisting}[language=Python]
from sympy import pi, symbols, simplify, latex

# Define symbols
Z, n, a_0, e, rho, epsilon_0 = symbols('Z n a_0 e rho epsilon_0', real=True, positive=True)

# The given expression to be simplified
expression = 8 * pi * (Z / (n * a_0))**3 * (-Z * e**2 * rho**2 / (8 * pi * epsilon_0) + Z * e * rho**2 / (60 * pi * epsilon_0))

# Simplify the expression
simplified_expression = simplify(expression)
print(latex(simplified_expression))
\end{lstlisting}
(c) Thus, you have estimated the first order correction to the energy levels due to the finite nuclear size. What do you obtain for states with $l>0$ ? What do you obtain for $s$-wave states? Do you need to worry about degeneracy of energy levels? Why or why not?
\subsection{}
For $l > 0$, we have:
\begin{equation}
  R_{nl}(0) = 0
\end{equation}
because of the $\frac{2 Z r}{n a_{0}}$ term. This can also be justified from a chemical intuition knowing that the $l > 0$ states have a node at the origin. For $s$-wave states, we have:
\begin{equation}
  R_{nl}(0) = \left(\frac{2 Z}{n a_{0}}\right)^{3 / 2} \sqrt{\frac{(n-1) !}{2 nn !}} L_{n-1}^{1}\left(0\right)
\end{equation}
I don't have to worry about the gender icy of the energy levels, because the radial wave functions are different for different $n$ values, so the energy levels are not degenerate.

% \end{lstlisting}

We may use here that the one-electron wave function is (letting $a_{0} \rightarrow a_{0} / Z$ from our result for hydrogen):

$$
\psi_{n \ell m}(\mathbf{x})=Y_{\ell m}(\Omega) R_{n \ell}(r),
$$

with

$$
R_{n \ell}(r)=\left(\frac{2 Z}{n a_{0}}\right)^{3 / 2} \sqrt{\frac{(n-\ell-1) !}{2 n(n+\ell) !}}\left(\frac{2 Z r}{n a_{0}}\right)^{\ell} e^{-Z r / n a_{0}} L_{n-\ell-1}^{2 \ell+1}\left(\frac{2 Z r}{a_{0} n}\right),
$$

and the associated Laguerre polynomials given by

$$
L_{n-\ell-1}^{2 \ell+1}(z)=\sum_{k=0}^{n-\ell-1} \frac{(-1)^{k}}{k !}\left(\begin{array}{c}
n+\ell \\
n-\ell-1-k
\end{array}\right) z^{k}
$$

Aside: The various conventions for Laguerre polynomials can be confusing. You may find the square root term appearing in the expression for $R_{n \ell}(r)$ given in some references as

$$
\sqrt{\frac{(n-\ell-1) !}{2 n[(n+\ell) !]^{3}}}
$$

This corresponds to a different normalization for the associated Laguerre polynomials [referred to as the "physicist convention" in Wikipedia (\href{https://en}{https://en}. \href{http://wikipedia.org/wiki/Laguerre_polynomials}{wikipedia.org/wiki/Laguerre\_polynomials})]. Sometimes these polynomials are defined with an overall minus sign compared with our definition. Also, the lower index $(n-\ell-1)$ we use here (as does Shankar) is the degree of the polynomial. Often the associated Laguerre polynomials are labeled instead with a lower index $n+\ell$ (equal to the sum of the upper and lower indices in our convention). This is just another convention for naming the same polynomial. Everybody seems to agree on the convention for the upper index, $2 \ell+1$.
\section{}
\begin{enumerate}
  \setcounter{enumi}{10}
  \item We discussed the spin-orbit correction to the hydrogen energy levels. In this example as well as others, we needed to evaluate certain expectation values with respect to the hydrogen wave functions. There are some handy "tricks" towards these evaluations. I refer you in particular to the discussion on pages $470-1$ of the text. Thus, considering the $\left|\psi_{n l m}\right\rangle$ hydrogen wave functions, evaluate:
\end{enumerate}

(a)

$$
\left\langle\frac{1}{r^{2}}\right\rangle \equiv\left\langle\psi_{n l m}\left|\frac{1}{r^{2}}\right| \psi_{n l m}\right\rangle
$$
\subsection{}
The radial equation for the hydrogen atom is:
\begin{equation}
  \frac{d^2u(r)}{dr^2} + \left[ \frac{2\mu}{\hbar^2}\left( E + \frac{Ze^2}{4\pi\epsilon_0r} \right) - \frac{l(l+1)}{r^2} \right] u(r) = 0
\end{equation}
Following the discussion on pg. 470 Shankar, we can remove the perturbation $\frac{\lambda}{r^2}$ from the Coulumbic term $\frac{Ze^2}{4\pi\epsilon_0r}$, which we will call $V(r)$, by making the substitution:
\begin{equation}
  V(r) = V_0(r) + \frac{\lambda}{r^2}
\end{equation}
so we get:
\begin{equation}
  \frac{d^2u(r)}{dr^2} + \left[ \frac{2\mu}{\hbar^2}\left( E + V_0(r) + \frac{\lambda}{r^2} \right) - \frac{l(l+1)}{r^2} \right] u(r) = 0
\end{equation}
Then we adjust the centrigul term from $l\rightarrow l'$:
\begin{equation}
  \frac{d^2u(r)}{dr^2} + \left[ \frac{2\mu}{\hbar^2}\left( E + V_0(r) \right) - \frac{l'(l'+1)}{r^2} \right] u(r) = 0
\end{equation}
Since we only modify the differential equation, the energy relation still holds:
\begin{equation}
  E(l')= - \frac{me^4}{2\hbar^2(k+l'+1)^2}
\end{equation}
We now equatee the different parts of eqn, 32 and 33:
\begin{equation}
  \frac{\hbar^2l(l+1)}{2\mu} + \lambda = \frac{\hbar^2l'(l'+1)}{2\mu}
\end{equation}
Multiplying through by $\frac{2\mu}{\hbar^2}$:
\begin{equation}
  l'(l'+1) + \frac{2\mu\lambda}{\hbar^2} = l(l+1)
\end{equation}
Evaluating this with $\lambda = 0$ implies either:
\begin{equation}
  l' = l \quad \text{or} \quad l' = -l - 1
\end{equation}
The latter would imply that $l' < 0$, which is not possible. Therefore, we have:
\begin{equation}
  l' = l
\end{equation}
Now, we recall in class, we derived that:
\begin{equation}
  \left\langle\frac{1}{r^{2}}\right\rangle = \left[\pdv{E}{\lambda}\right]_{\lambda=0} = \left[\pdv{E}{l'}\pdv{l'}{\lambda}\right]_{\lambda=0}
\end{equation}
Differentating the expression for $E(l')$ with respect to $l'$, we get:
\begin{equation}
  \pdv{E}{l'} = + \frac{me^4}{\hbar^2} \frac{1}{(k+l'+1)^3}
\end{equation}
Differentating equation 36 w.r.t. $\lambda$, we get:
\begin{equation}
  2\left(l' + \frac{1}{2}\right) dl' = \frac{2m}{\hbar^2} d\lambda
\end{equation}
So:
\begin{equation}
  \frac{dl'}{d\lambda} = \frac{m}{\hbar^2} \frac{1}{(l' + \frac{1}{2})^2}
\end{equation}
Now we combine both derivatives:
\begin{equation}
  \left[\pdv{E}{l'}\pdv{l'}{\lambda}\right]_{\lambda=0} = \frac{me^4}{\hbar^2} \frac{1}{(k+l'+1)^3} \frac{m}{\hbar^2} \frac{1}{(l' + \frac{1}{2})}
\end{equation}
Since the Bohr radius is given by:
\begin{equation}
  a_0 = \frac{4\pi\epsilon_0\hbar^2}{me^2}
\end{equation}
We can write:
\begin{equation}
  \left\langle\frac{1}{r^{2}}\right\rangle = \frac{16\pi^2\epsilon_0^2}{a_0^2(k+l'+1)^3} \frac{1}{(l' + \frac{1}{2})}
\end{equation}
And the we recognize that:
\begin{equation}
  \left\langle\frac{1}{r^{2}}\right\rangle = \frac{16\pi^2\epsilon_0^2}{a_0^2(n)^3} \frac{1}{(l + \frac{1}{2})}
\end{equation}
\subsection{}
$$
\left\langle\frac{1}{r^{3}}\right\rangle .
$$
The radial part of the Hamiltonian is:
\begin{equation}
  -\frac{\hbar^2}{2m}\left(\frac{1}{r^2}\frac{d}{dr}\left(r^2\frac{d}{dr}\right)\right)
\end{equation}
If we define $\hat{P}_r = -i\hbar\left(\pdv{r}+\frac{1}{r}\right)$, then we can write the radial part of the Hamiltonian as:
\begin{equation}
 \frac{\hat{P}_r^2}{2m} = -\frac{\hbar^2}{2m}\left(\frac{1}{r^2}\frac{d}{dr}\left(r^2\frac{d}{dr}\right)\right)
\end{equation}
Due to Ehrenfests theorem, this operator cannot change with time. Therefore, we can write:
\begin{equation}
  \dv{t}\langle\hat{P}_r\rangle = 0 = \langle \left[\hat{H}, \hat{P}_r\right]\rangle
\end{equation}
We can write the commutator as:
\begin{equation}
  \left[\hat{H}, \hat{P}_r\right] = \left[\frac{\hat{P}_r^2}{2m}, \hat{P}_r\right] = \frac{1}{2m}\left[\hat{P}_r^2, \hat{P}_r\right] = \frac{1}{2m}\left[\hat{P}_r, \hat{P}_r\right]\hat{P}_r = 0
\end{equation}
I wasn't able to get further than this.
\section{}
\begin{enumerate}
  \setcounter{enumi}{11}
  \item Let us consider an example of the use of degenerate stationary state perturbation theory. Thus, let us take the hydrogen atom, with unperturbed Hamiltonian $H_{0}=$ $\frac{P^{2}}{2 m}-\frac{\alpha}{r}$, and consider the effect of putting this atom in a uniform external electric field: $\mathbf{E}=E \hat{\mathbf{e}}_{z}$. We are interested in calculating, to first order in perturbation theory, the shifts in the $n=2$ energy levels ( $n$ is the principal quantum number here). Note that the $n=2$ level is four-fold degenerate, corresponding to the eigenstates (in notation $n \ell_{m_{\ell}}$ ): $\left|2 S_{0}\right\rangle,\left|2 P_{1}\right\rangle,\left|2 P_{0}\right\rangle,\left|2 P_{-1}\right\rangle$, neglecting spins.
\subsection{}
(a) Write down the perturbing potential, $V$. [Note that we need only consider the electron's coordinates, relative to the nucleus - why?] Calculate the commutator $\left[V, L_{z}\right]$, and hence determine the matrix elements of $V$ between states with different eigenvalues of $L_{z}$.
\end{enumerate}
Given that the charge of the electron is $\sqrt{\alpha }$ and the electric filed only acts in the z direction, we have:
\begin{equation}
  V = -\sqrt{\alpha} E z
\end{equation}
Similar to the Born-Oppenheimer approximation, we may consider the nucleus as fixed , because it is much more massive, and the electron as moving around it. Then we may consider the electron's coordinates relative to the nucleus. $L_z$ is defined as:
\begin{equation}
  L_z = -i\hbar \left( x \frac{\partial}{\partial y} - y \frac{\partial}{\partial x} \right)
\end{equation}
Then we have:
\begin{equation}
  \left[V, L_{z}\right] = V(z)L_z(x,y) - L_z(x,y)V(z) = 0
\end{equation}
The commutator vanishes because $L_z$ commutes with $z$, as can be observed due to the fact that $L_z$ has no dependence on $z$, and $V$ is a function of $z$ only. This means that $V$ and $L_z$ share a common set of eigenfunctions, and we may use the eigenfunctions of $L_z$ to determine the matrix elements of $V$ between states with different eigenvalues of $L_z$. \\
$V$ in spherical coordinates is:
\begin{equation}
  V = -\sqrt{\alpha} E r \cos \theta
\end{equation}
The matrix elements of $V$ between states with different eigenvalues of $L_z$ are:
\begin{equation}
  \bra{2n_{m_l}}V\ket{2n_{m_l'}} = -\sqrt{\alpha} E \bra{2n_{m_l}}r \cos \theta\ket{2n_{m_l'}}
\end{equation}
We will create a matrix in this basis of states for the perturbation $V$:
\begin{equation}
  \begin{bmatrix}
    \bra{2s_{0}}V\ket{2s_{0}} & \bra{2s_{0}}V\ket{2p_{1}} & \bra{2s_{0}}V\ket{2p_{0}} & \bra{2s_{0}}V\ket{2p_{-1}} \\
    \bra{2p_{1}}V\ket{2s_{0}} & \bra{2p_{1}}V\ket{2p_{1}} & \bra{2p_{1}}V\ket{2p_{0}} & \bra{2p_{1}}V\ket{2p_{-1}} \\
    \bra{2p_{0}}V\ket{2s_{0}} & \bra{2p_{0}}V\ket{2p_{1}} & \bra{2p_{0}}V\ket{2p_{0}} & \bra{2p_{0}}V\ket{2p_{-1}} \\
    \bra{2p_{-1}}V\ket{2s_{0}} & \bra{2p_{-1}}V\ket{2p_{1}} & \bra{2p_{-1}}V\ket{2p_{0}} & \bra{2p_{-1}}V\ket{2p_{-1}}
  \end{bmatrix}
\end{equation}
The perturbation has no dependence on the azimuthal angle $\phi$, so it cannot connect states with different $m_l$ values. In addition, the diagonal matrix elements vanish because the perturbation is an operator with odd parity, and the angular functions for the diagonal elements combine to form a function with even parity, and so the integral will be zero. Therefore, we can simplify to:
\begin{equation}
  \begin{bmatrix}
    0 & 0 & \bra{2s_{0}}V\ket{2p_{0}} & 0 \\
    0 & 0 & 0 & 0 \\
    \bra{2p_{0}}V\ket{2s_{0}} & 0 & 0 & 0 \\
    0 & 0 & 0 & 0
  \end{bmatrix}
\end{equation}
\subsection{}
(b) You should have found a "selection rule" which simplifies the problem. What is the degeneracy that needs to be addressed in the problem now that you have made this calculation?
\subsubsection{}
The selection rule that I found is 2 pronged: the $m_l$ values must be the same and the $l$ values must differ by one. With these selection rules in mind, the degeneracy that needs to be addressed is only the four states that have the same $m_l$ value, but different $l$ values. This is the $n = 2$ level, which is four-fold degenerate, but only two-fold degenerate when we consider the selection rules.

(c) Using the invariance of the hydrogen atom Hamiltonian under parity, write down the remaining matrix elements of $V$ which need to be determined, and compute their values.
\subsubsection{}
The remaining matrix elements are:
\begin{equation}
  \begin{bmatrix}
    0 & 0 & \bra{2s_{0}}V\ket{2p_{0}} & 0 \\
    0 & 0 & 0 & 0 \\
    \bra{2p_{0}}V\ket{2s_{0}} & 0 & 0 & 0 \\
    0 & 0 & 0 & 0
  \end{bmatrix}
\end{equation}
Using SymPy, I found that:
\begin{equation}
  \bra{2s_{0}}V\ket{2p_{0}} = \bra{2p_{0}}V\ket{2s_{0}} = 3Ea_0 \sqrt{\alpha }
\end{equation}
% Inline Python code in the document
\begin{lstlisting}[language=Python]
from sympy import symbols, cos, integrate, pi, sqrt, exp, oo, conjugate, sin
from sympy.abc import r, theta, phi

# Define constants
a0, E, alpha = symbols('a0 E alpha', real=True, positive=True)

# Define the wave functions
psi_200 = 1/sqrt(4*pi) * 1/(2*sqrt(2)) * 1/a0**(3/2) * (2 - r/a0) * exp(-r/(2*a0))
psi_210 = sqrt(3/(4*pi)) * cos(theta) * 1/(4*sqrt(6)) * 1/a0**(3/2) * 2 * (r/a0) * exp(-r/(2*a0))

# Define the perturbing potential V in spherical coordinates
V = -sqrt(alpha) * E * r * cos(theta)

# Function to calculate the matrix element
def matrix_element(psi1, psi2, V):
    integrand = conjugate(psi1) * V * psi2 * r**2 * sin(theta)
    return integrate(integrand, (r, 0, oo), (theta, 0, pi), (phi, 0, 2*pi))

# Compute the matrix elements
element_200_210 = matrix_element(psi_200, psi_210, V)
element_210_200 = matrix_element(psi_210, psi_200, V)

element_200_210, element_210_200


\end{lstlisting}
Further simplifications probably would have been insightful, but I did not perform them.
% Inline Python code in the document


Hydrogenic wave functions you might need are:

$$
\begin{aligned}
& \psi_{200}=\frac{1}{\sqrt{4 \pi}} \frac{1}{2 \sqrt{2}} \frac{1}{a_{0}^{3 / 2}}\left(2-r / a_{0}\right) e^{-r / 2 a_{0}} \\
& \psi_{210}=\sqrt{\frac{3}{4 \pi}} \cos \theta \frac{1}{4 \sqrt{6}} \frac{1}{a_{0}^{3 / 2}} 2 \frac{r}{a_{0}} e^{-r / 2 a_{0}} .
\end{aligned}
$$

You may also want to evaluate the handy integral:

$$
I_{k} \equiv \int_{0}^{\infty} x^{k} e^{-x} d x, \quad k \geq 0
$$

(d) Now complete your degenerate perturbation theory calculation to determine the splitting of the states in the applied electric field. Calculate numerical splittings (in eV) for an applied field of $100 \mathrm{kV} / \mathrm{cm}$. Also, estimate the "typical" electric field felt by the electron, due to the nucleus, in a hydrogen atom. Was the use of perturbation theory reasonable for this problem?
\subsubsection{}
considering that $\alpha = \frac{e^2}{4\pi\epsilon_0}$ and $a_0 = \frac{4\pi\epsilon_0\hbar^2}{m_e e^2}$, my script gives:
\begin{equation}
  \bra{2s_{0}}V\ket{2p_{0}} = \bra{2p_{0}}V\ket{2s_{0}} = 0.0151 eV
\end{equation}
I was not able to get my calculation for the typical electric field felt by the electron due to the nucleus in a hydrogen atom to yield physical results, but I read on the web that this is on the order of $10^1$ eV over the length of the Bohr radius. So, the use of perturbation theory was reasonable for a $100$ kV/cm electric field.
% Inline Python code in the document
\begin{lstlisting}[language=Python]
from sympy import symbols, pi, sqrt, exp, integrate, sin, conjugate, oo, cos, N

# Define physical constants
e = 1.602e-19  # Elementary charge in Coulombs
epsilon_0 = 8.854e-12  # Permittivity of free space in F/m
a0 = 5.29177210903e-11  # Bohr radius in meters

# Define symbols for the problem
r, theta, phi, E, alpha = symbols('r theta phi E alpha', positive=True, real=True)

# Define the wave functions for 2p0 and 2s0 states
# 2s0 state
psi_200 = 1/sqrt(4*pi) * 1/(2*sqrt(2)) * 1/a0**(3/2) * (2 - r/a0) * exp(-r/(2*a0))
# 2p0 state
psi_210 = sqrt(3/(4*pi)) * cos(theta) * 1/(4*sqrt(6)) * 1/a0**(3/2) * 2 * (r/a0) * exp(-r/(2*a0))

# Define the perturbing potential V in spherical coordinates
alpha_value = e**2 / (4 * pi * epsilon_0)  # Value of alpha for hydrogen atom
V = -sqrt(alpha_value) * E * r * cos(theta)

# Function to calculate the matrix element
def matrix_element(psi1, psi2, V):
    integrand = conjugate(psi1) * V * psi2 * r**2 * sin(theta)
    return integrate(integrand, (r, 0, oo), (theta, 0, pi), (phi, 0, 2*pi))

# Define the electric field (100 kV/cm in SI units)
E_field_SI = 100e3 / 100  # 100 kV/cm in V/m

# Calculate the matrix elements for the given electric field
element_200_210 = N(matrix_element(psi_200, psi_210, V.subs(E, E_field_SI)))
element_210_200 = N(matrix_element(psi_210, psi_200, V.subs(E, E_field_SI)))

# Convert energy shifts from Joules to eV
J_to_eV = 1 / (1.602e-19)  # Conversion factor from Joules to eV
energy_shift_200_210_eV = element_200_210 * J_to_eV
energy_shift_210_200_eV = element_210_200 * J_to_eV

element_200_210, element_210_200, energy_shift_200_210_eV, energy_shift_210_200_eV


\end{lstlisting}
\end{document}