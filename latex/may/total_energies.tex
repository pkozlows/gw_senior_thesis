\documentclass[12pt]{article}
\usepackage{amsmath}
\usepackage{amssymb}
\usepackage{graphicx}
\usepackage{physics}
\usepackage{hyperref}

\usepackage{listings} % Required for insertion of code
\usepackage{xcolor} % Required for custom colors

% Define custom colors
\definecolor{codegreen}{rgb}{0,0.6,0}
\definecolor{codegray}{rgb}{0.5,0.5,0.5}
\definecolor{codepurple}{rgb}{0.58,0,0.82}
\definecolor{backcolour}{rgb}{0.95,0.95,0.92}

% Setup the style for code listings
\lstdefinestyle{mystyle}{
    backgroundcolor=\color{backcolour},   
    commentstyle=\color{codegreen},
    keywordstyle=\color{magenta},
    numberstyle=\tiny\color{codegray},
    stringstyle=\color{codepurple},
    basicstyle=\ttfamily\footnotesize,
    breakatwhitespace=false,         
    breaklines=true,                 
    captionpos=b,                    
    keepspaces=true,                 
    numbers=left,                    
    numbersep=5pt,                  
    showspaces=false,                
    showstringspaces=false,
    showtabs=false,                  
    tabsize=2
}

% Activate the style
\lstset{style=mystyle}



\title{Total energy functionals}
\author{Patryk Kozlowski}
\date{\today}
\begin{document}
\maketitle
We have the MO coefficients from a prior mean field calculation $C_{\mu p}$. I computed a form for a linearized GW density matrix in MO basis $\gamma _{pq}$. Now I want to alter the $C_{\mu p}$ to take into account this new density matrix. We can diagonalize the $\gamma _{pq}$ to get the wave function rotations and orbital energies:
\begin{equation}
    \boldsymbol{\gamma }\boldsymbol{\psi } = \epsilon \boldsymbol{\psi }
\end{equation}
Then we can get the new $C_{\mu p'}$ by:
\begin{equation}
    C_{\mu p'} = \sum_{p} C_{\mu p} \psi _{pp'}
\end{equation}
Then, I need to orthonormalize the new $C_{\mu p'}$ using the overlap matrix $S_{pp'}$ with
\begin{equation}
    C_{\mu p} = C_{\mu p'} (S^{-1/2})_{pp'} C _{p' \nu}
\end{equation}

Now that we have the density matrix at our disposition we want to use it to evaluate total energies. Any expression for the total energy will have the following form:
\begin{equation}
    E_{\text{tot}}=T_s [? ] + V_H [\gamma ] + V_x [\gamma ] + V_{\text{nuc, elec}} [\gamma ] + V_{\text{nuc, nuc}}+ E_{\text{corr}} [? ]
\end{equation}
where $V_H[\gamma ]$ is the Hartree energy, $V_x [\gamma ]$ is the exchange energy, $V_{\text{nuc, elec}} [\gamma ]$ is the nuclear-electron attraction energy, and $V_{\text{nuc, nuc}}$ is the nuclear-nuclear repulsion energy. These terms remain constant to different total energy functions. $T_s$, which is defined as the kinetic energy, and $E_{\text{corr}}$, which is the correlation energy, are the terms that will change across different definitions of total energy functionals since either can contain a portion of correlation. $E_{\text{corr}}$ can either be evaluated using the Klein or Galitskii-Migdal formula. The Klein formula is given by:
\begin{equation}
    E_{\text{corr}}^{\text{Klein}} = \frac{1}{2} \text{Tr} (\Omega ^{\text{RPA}} - A^{\text{TDA}})
    = \frac{1}{2} \text{Tr} (\Omega ^{\text{RPA}} ) - \frac{1}{2} \text{Tr} (A^{\text{TDA}})
\end{equation}
where $\Omega ^{\text{RPA}}$ are the RPA excitation energies and $A^{\text{TDA}}$ is from the Tamm-Dancoff approximation. This is also known as the plasmon-pole form.

However, we know that the trace is invariant to unitary transformations and also cyclic permutations, so:
\begin{equation}
    \text{Tr} (A^{\text{TDA}}) = \text{Tr} (A^{\text{TDA}} UU^{\dagger}) = \text{Tr} (U^{\dagger} A^{\text{TDA}} U) = \text{Tr} (\Omega ^{\text{TDA}})
\end{equation}
So with $E_{\text{corr}}^{\text{Klein}}$ we are quantifying the correlation energy by considering the difference between the RPA and TDA excitation energies for a given system. Since we found earlier that the TDA captures more correlation than the RPA, we can expect that $E_{\text{corr}}^{\text{Klein}}$ will be negative, or stabilizing, as we would expect.
\begin{equation}
E_{\mathrm{c}}^{\mathrm{GM}}=-\sum_{i a m} \frac{M_{i a, m}^2}{\epsilon_a-\epsilon_i+\Omega_m}
\end{equation}
but this expression is given in terms of spin orbits, so after the integration we gain a factor of 2
\begin{equation}
E_{\mathrm{c}}^{\mathrm{GM}}=-2 \sum_{i a m} \frac{M_{i a, m}^2}{\epsilon_a-\epsilon_i+\Omega_m}
\end{equation}
where we have our transition densities:
\begin{equation}
    M_{pq,u} = \sum_{ia} [pq|ia] X_{ia}^u
\end{equation}
where $X_{ia}^u$ are the excitation vectors. After the spin integration, this simplified to:
\begin{equation}
    M_{pq,u} = \sqrt{2} \sum_{ia} (pq|ia) X_{ia}^u
\end{equation}
We want to test 3 different total energy functionals:
\begin{equation}
    E_{\text{tot}} = T_s [\gamma _{0}] + V_H [\gamma ] + V_x [\gamma ] + V_{\text{nuc, elec}} [\gamma ] + V_{\text{nuc, nuc}} + E_{\text{corr}} ^{\text{Klein}}[\gamma_0 ]
\label{eq:klein1}
\end{equation}
\begin{equation}
    E_{\text{tot}} = T_s [\gamma ] + V_H [\gamma ] + V_x [\gamma ] + V_{\text{nuc, elec}} [\gamma ] + V_{\text{nuc, nuc}} + E_{\text{corr}} ^{\text{Klein}}[\gamma_0]
\label{eq:klein2}
\end{equation}
\begin{equation}
    E_{\text{tot}}^{GM} = T_s [\gamma _{0} ] + V_H [\gamma _{0} ] + V_x [\gamma _{0} ] + V_{\text{nuc, elec}} [\gamma _{0} ] + V_{\text{nuc, nuc}} + E_{\text{corr}} ^{\text{GM}}[\gamma _{0}]
\end{equation}

I have been struggling a little bit with how to get The
\begin{equation}
    T_s[\gamma _0] + V_{\text{nuc, elec}} [\gamma]
\end{equation}
In my first line of code, I am generating the kinetic integrals in the AO basis: $\mathrm{T}_{\mu \nu}$. Then, I rotate them into the MO basis using the noninteracting coefficients $T_{pq}=\sum_{\mu \nu} C_{\mu p} T_{\mu \nu} C_{\nu q}$. Finally, I get $T_{\mu \nu}$ back from a rotation via $\gamma ^{GW}\rightarrow S_{\mu p}$ like $T_{\mu \nu} = \sum_{pq} S_{\mu p} T_{pq} S_{\nu q}$. But I am getting answers that are far off, so is this the right way to do it?\\\\
Specifically, we will want to see the difference between \ref{eq:klein1} and \ref{eq:klein2}. 
We are interested in formulating a table, which will corroborate vitals putin MOLGW, FHI-AIMS, and my implementation for the Galitskii-Migdal correlation energies. Then, I want to show how my energy for the Galitskii-Migdal deviates from that reported in the paper using HF, since mine uses the $G_0W_0$ density matrix that we investigated and the paper just uses the HF density matrix.
Then I want to make a separate table, which compares the results of my Klein implemtation, for both of the interacting and noninteracting cases, with the responding result from PySCF. I will comment about how the discrepancy originates from the fact that the PySCF implementation uses and axillary bases, wail I didn't. Lastly, I want to make a table which documents the deviations between my Klein folly interacting and noninteracting cases cover for the total energies and the correlation energies. For the first table, I will just consider the two atoms Ne and He. For the second two tables, I will consider the water molecule, mating, and licl, to show more versatility.
\end{document}