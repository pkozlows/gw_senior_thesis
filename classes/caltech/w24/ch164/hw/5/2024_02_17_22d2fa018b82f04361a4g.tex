\documentclass[12pt]{article}
\usepackage[utf8]{inputenc}
\usepackage[T1]{fontenc}
\usepackage{amsmath}
\usepackage{amsfonts}
\usepackage{amssymb}
\usepackage[version=4]{mhchem}
\usepackage{stmaryrd}
\usepackage{graphicx}
\usepackage{physics}

\usepackage{listings} % Required for insertion of code
\usepackage{xcolor} % Required for custom colors

% Define custom colors
\definecolor{codegreen}{rgb}{0,0.6,0}
\definecolor{codegray}{rgb}{0.5,0.5,0.5}
\definecolor{codepurple}{rgb}{0.58,0,0.82}
\definecolor{backcolour}{rgb}{0.95,0.95,0.92}

% Setup the style for code listings
\lstdefinestyle{mystyle}{
    backgroundcolor=\color{backcolour},   
    commentstyle=\color{codegreen},
    keywordstyle=\color{magenta},
    numberstyle=\tiny\color{codegray},
    stringstyle=\color{codepurple},
    basicstyle=\ttfamily\footnotesize,
    breakatwhitespace=false,         
    breaklines=true,                 
    captionpos=b,                    
    keepspaces=true,                 
    numbers=left,                    
    numbersep=5pt,                  
    showspaces=false,                
    showstringspaces=false,
    showtabs=false,                  
    tabsize=2
}

% Activate the style
\lstset{style=mystyle}


\title{Ch/ChE 164 Winter 2024 
 Homework Problem Set \#5 
 Due Date: Thursday, February 22, 2024 @ 11:59pm PT 
 Out of 100 Points }

\author{}
\date{}


\begin{document}
\maketitle
\section{}
\begin{enumerate}
  \item (15 points) (4.15 from Chandler) Consider an isomerization process $A \rightleftharpoons B$, where $A$ and $B$ refer to the different isomer states of a molecule. Imagine that the process takes place in a dilute gas, and that $\Delta \epsilon$ is the energy difference between state $A$ and state $B$. According to the Boltzmann distribution law, the equilibrium ratio of $A$ and $B$ populations is given by
\end{enumerate}


\begin{equation*}
\frac{\left\langle N_{A}\right\rangle}{\left\langle N_{B}\right\rangle}=\frac{g_{A}}{g_{B}} e^{-\beta \Delta \epsilon}, \tag{1}
\end{equation*}


where $g_{A}$ and $g_{B}$ are the degeneracies of states $A$ and $B$, respectively. Show how this same result follows from the condition of chemical equilibria, $\mu_{A}=\mu_{B}$.
\subsection{}
The total partition function for the system will be given by:
\begin{equation}
Q=Q^A Q^B=\frac{1}{N_{A} !} q_A^{N_A} \frac{1}{N_{B} !} q_B^{N_B}
\end{equation}
where $N_A$ and $N_B$ are the number of molecules in the states $A$ and $B$ respectively. 
The equation for the home holds free energy is:
\begin{equation}
F=-k_B T \log Q
\end{equation}
So we plug in to get:
\begin{equation}
F=-k_B T \log \left(\frac{1}{N_{A} !} q_A^{N_A} \frac{1}{N_{B} !} q_B^{N_B}\right) = -k_B T \log \left(\frac{1}{N_{A} !} q_A^{N_A}\right) -k_B T \log \left(\frac{1}{N_{B} !} q_B^{N_B}\right) = 
\end{equation}
Further separation of the logarithms gives:
\begin{equation}
  = kT\left( \ln \left(N_A!\right) - N_A \ln q_A + \ln \left(N_B!\right) - N_B \ln q_B\right)
\end{equation}
We can use the Stirling approximation to get:
\begin{equation}
  = kT\left( N_A \ln N_A - N_A - N_A \ln q_A + N_B \ln N_B - N_B - N_B \ln q_B\right)
\end{equation}
Now, we know that the derivative of this free energy with respect to the number of particles in a state gives the chemical potential for the state:
\begin{equation}
  \mu_{A,B} = \frac{\partial F}{\partial N_{A,B}}
\end{equation}
For the state $A$ we get:
\begin{equation}
  \mu_A = kT \left(\log{\left(N_{A} \right)} - \log{\left(q_{A} \right)}\right)
\end{equation}
And for the state $B$ we get:
\begin{equation}
  \mu_B = kT \left(\log{\left(N_{B} \right)} - \log{\left(q_{B} \right)}\right)
\end{equation}
% Inline Python code in the document
\begin{lstlisting}[language=Python]
from sympy import symbols, diff, log, latex

# Define symbols
kT, N_A, N_B, q_A, q_B = symbols('kT N_A N_B q_A q_B')

# Define the free energy expression using Stirling's approximation
F = kT * (N_A * log(N_A) - N_A - N_A * log(q_A) + N_B * log(N_B) - N_B - N_B * log(q_B))

# Compute the derivatives
mu_A = diff(F, N_A)
mu_B = diff(F, N_B)

# Display the results in LaTeX
mu_A_latex = mu_A.simplify().doit()
mu_B_latex = mu_B.simplify().doit()

print(latex(mu_A_latex))
# latex(mu_B_latex)
\end{lstlisting}
The equilibrium contention tells us that we can set both equations equal to each other:
\begin{equation}
  \mu_A = \mu_B \rightarrow kT \left(\log{\left(N_{A} \right)} - \log{\left(q_{A} \right)}\right) = kT \left(\log{\left(N_{B} \right)} - \log{\left(q_{B} \right)}\right)
\end{equation}
We also consider that we have the form for the single partial partition functions:
\begin{equation}
  q_A = g_A e^{-\beta \epsilon_A}
\end{equation}
\begin{equation}
  q_B = g_B e^{-\beta \epsilon_B}
\end{equation}
We can use these to get:
\begin{equation}
  \log{\left(q_{A} \right)} = \log{\left(g_A e^{-\beta \epsilon_A} \right)} = \log{\left(g_A \right)} - \beta \epsilon_A
\end{equation}
\begin{equation}
  \log{\left(q_{B} \right)} = \log{\left(g_B e^{-\beta \epsilon_B} \right)} = \log{\left(g_B \right)} - \beta \epsilon_B
\end{equation}
We can use these to get:
\begin{equation}
  kT \left(\log{\left(N_{A} \right)} - \log{\left(q_{A} \right)}\right) = kT \left(\log{\left(N_{A} \right)} - \log{\left(g_A \right)} + \beta \epsilon_A\right)
\end{equation}
\begin{equation}
  kT \left(\log{\left(N_{B} \right)} - \log{\left(q_{B} \right)}\right) = kT \left(\log{\left(N_{B} \right)} - \log{\left(g_B \right)} + \beta \epsilon_B\right)
\end{equation}
Setting but size equate again:
\begin{equation}
  kT \left(\log{\left(N_{A} \right)} - \log{\left(g_A \right)} + \beta \epsilon_A\right) = kT \left(\log{\left(N_{B} \right)} - \log{\left(g_B \right)} + \beta \epsilon_B\right)
\end{equation}
We can cancel the common factor out font from both sites and exponential both sites:
\begin{equation}
  \exp{\left(\log{\left(N_{A} \right)} - \log{\left(g_A \right)} + \beta \epsilon_A\right)} = \exp{\left(\log{\left(N_{B} \right)} - \log{\left(g_B \right)} + \beta \epsilon_B\right)}
\end{equation}
We can use the properties of the logarithm to get:
\begin{equation}
  \frac{N_A}{g_A} e^{\beta \epsilon_A} = \frac{N_B}{g_B} e^{\beta \epsilon_B}
\end{equation}
Rearranging the terms now:
\begin{equation}
  \frac{N_A}{N_B} = \frac{g_A}{g_B} e^{\beta \epsilon_B - \beta \epsilon_A}
\end{equation}
Defining the energy difference as $\Delta \epsilon = \epsilon_A - \epsilon_B$ we get:
\begin{equation}
  \frac{N_A}{N_B} = \frac{g_A}{g_B} e^{-\beta \Delta \epsilon}
\end{equation}

\section{}
\begin{enumerate}
  \setcounter{enumi}{1}
  \item (20 points) (4.16 from Chandler) Consider the system described in problem 1. The canonical partition function is
\end{enumerate}


\begin{equation*}
Q=\frac{q^{N}}{N !} \tag{2}
\end{equation*}


where $N$ is the total number of molecules, and $q$ is the Boltzmann weighted sum over all single molecule states, both those associated with isomers of type $A$ and those associated with isomers of type $B$.

(a) Show that one may partition the sum and write


\begin{equation*}
Q=\sum_{P} \exp \left\{-\beta F\left(N_{A}, N_{B}\right)\right\} \tag{3}
\end{equation*}


with


\begin{equation*}
-\beta F\left(N_{A}, N_{B}\right)=\log \left[\left(N_{A} ! N_{B} !\right)^{-1} q_{A}^{N_{A}} q_{B}^{N_{B}}\right], \tag{4}
\end{equation*}


where $\sum_{P}$ is over all the partitions of $N$ molecules into $N_{A}$ molecules of type $A$ and $N_{B}$ molecules of type $B, q_{A}$ is the Boltzmann weighted sum over states of isomer $A$, and $q_{B}$ is similarly defined.
\subsection{}
A single practical partition function is defined as:
\begin{equation}
q = \sum_{i} e^{-\beta \epsilon_i}
\end{equation}
In this case, we have only two options; state $A$ and state $B$. So the single particle partition function is just a sum of the partition functions for each isomer
\begin{equation}
q = q_A + q_B
\end{equation}
where:
\begin{equation}
q_A = g_A e^{-\beta \epsilon_A}
\end{equation}
\begin{equation}
q_B = g_B e^{-\beta \epsilon_B}
\end{equation}
The total partition function is then:
\begin{equation}
Q=\frac{q^{N}}{N !} = \frac{\left(q_A + q_B\right)^{N}}{N !}
\end{equation}
This resembles the binomial expansion:
\begin{equation}
\left(a+b\right)^N = \sum_{N} \binom{N}{N_A} a^{N_A} b^{N_B}
\end{equation}
where the sum is over all the partitions of $N$ molecules into $N_{A}$ molecules of type $A$ and $N_{B}$ molecules of type $B$. So, we get:
\begin{equation}
Q=\frac{1}{N!}\sum_{N_A = 0}^{
N} \frac{N!}{N_A! N_B!} q_A^{N_A} q_B^{N_B} = \sum_{P} \frac{q_A^{N_A} q_B^{N_B}}{N_A! N_B!}
\end{equation}
Taking the logarithm and then exponentiating the inside of the sum we get:
\begin{equation}
Q = \sum_{P} \exp{\left(\log{\left(\frac{q_A^{N_A} q_B^{N_B}}{N_A! N_B!}\right)}\right)}
\end{equation}
And we know that this should be the same as:
\begin{equation}
Q = \sum_{P} \exp{\left(-\beta F\left(N_{A}, N_{B}\right)\right)}
\end{equation}
So, by inspection, we have:
\begin{equation}
-\beta F\left(N_{A}, N_{B}\right)=\log \left[\left(N_{A} ! N_{B} !\right)^{-1} q_{A}^{N_{A}} q_{B}^{N_{B}}\right]
\end{equation}


(b) Show that the condition of chemical equilibria is identical to finding the partitioning that minimizes the Helmholtz free energy


\begin{equation*}
\frac{\partial F}{\partial\left\langle N_{A}\right\rangle}=\frac{\partial F}{\partial\left\langle N_{B}\right\rangle}=0 \tag{5}
\end{equation*}


subject to the constraint that $\left\langle N_{A}\right\rangle+\left\langle N_{B}\right\rangle=N$ is fixed.
\subsection{}
Solving for $F(N_A, N_B)$ we get:
\begin{equation}
F(N_A, N_B) = -kT \log \left[\left(N_{A} ! N_{B} !\right)^{-1} q_{A}^{N_{A}} q_{B}^{N_{B}}\right]
\end{equation}
Separating the logarithm we get:
\begin{equation}
F(N_A, N_B) = -kT \left(\log{q_{A}^{N_{A}} } + \log{q_{B}^{N_{B}} } - \log{N_{A} ! } - \log{N_{B} ! }\right)
\end{equation}
We can use the Stirling approximation to get:
\begin{equation}
F(N_A, N_B) = -kT \left(N_{A} \log{q_{A}} + N_{B} \log{q_{B}} - N_{A} \log{N_{A}} + N_{A} - N_{B} \log{N_{B}} + N_{B}\right)
\end{equation}
Taking the appropriate derivatives gives:
\begin{equation}
\frac{\partial F}{\partial N_A} = kT \log\left(\frac{N_A}{q_A}\right)
\end{equation}
We set this equal to 0:
\begin{equation}
  \log\left(\frac{N_A}{q_A}\right) = 0 \rightarrow \frac{N_A}{q_A} = 1 \rightarrow N_A = q_A
\end{equation}
\begin{equation}
  \frac{\partial F}{\partial N_B} = kT \log\left(\frac{N_B}{q_B}\right)
\end{equation}
We set this equal to 0:
\begin{equation}
  \log\left(\frac{N_B}{q_B}\right) = 0 \rightarrow \frac{N_B}{q_B} = 1 \rightarrow N_B = q_B
\end{equation}
% Inline Python code in the document
\begin{lstlisting}[language=Python]
from sympy import symbols, diff, log

# Define symbols
k, T, N_A, N_B, q_A, q_B = symbols('k T N_A N_B q_A q_B', real=True, positive=True)

# Helmholtz free energy function
F = -k*T*(N_A*log(q_A) + N_B*log(q_B) - N_A*log(N_A) + N_A - N_B*log(N_B) + N_B)

# Derivatives with respect to N_A and N_B
dF_dN_A = diff(F, N_A)
dF_dN_B = diff(F, N_B)

dF_dN_A.simplify(), dF_dN_B.simplify()

\end{lstlisting}
\section{}
\begin{enumerate}
  \setcounter{enumi}{2}
  \item (15 points) (4.25 from Chandler) Use the information compiled in Chapter 8 of Hill's Introduction to Statistical Thermodynamics to calculate the equilibrium constant, $K$, for the reaction $I_{2} \rightleftharpoons 2 I$ when the reaction occurs in the dilute gas phase at $T=1000^{\circ} \mathrm{K}$. (Note: For the electronic partition function, you need to consider the difference in degeneracies of the ground states of the iodine atom and molecule, $g_{e, I}=4, g_{e, I_{2}}=1$. This is based on the molecular orbital theory. You are not required to show this.)
\subsection{}
We want to start by considering the individual partition functions for the molecule. We use the following formula to compute the one for the molecule:
\begin{equation}
\begin{aligned}
q_{\mathrm{int}}=g_{0, e}\left(2 I_A+1\right)\left(2 I_B+1\right) e^{-\beta \epsilon_{00}} q_{\mathrm{vi}} q_{\mathrm{ro}} / \sigma_{A B}
\end{aligned}
\end{equation}
where we have:
$q_{\mathrm{ro}}=T / \Theta_{\mathrm{ro}}$. Also the degeneracy of the iodine molecule is $g_{e, I_{2}}=1$. 
and then the total partition function for the species will be:
$$
q_i=\left(V / \lambda_i^3\right) q_i^{(\text {int })},
$$
where $q_i^{(\text {int })}$ is for species $i$ the $q_{\text {int }}(T)$ of the previous section, and
$$
\lambda_i=h / \sqrt{2 \pi m_i k_B T}
$$
The nuclear spin quantum numbers are going to be the same for the diatomic $I_A=I_B=\frac{5}{2}$. Additionally, since we are dealing with a diatomic $\sigma _{A B}=2$. We get the rotational temperature from Hill as $\theta_{\text {rot }}=0.054 \mathrm{K}$. The vibrational temperature is $\theta_{\text {vib }}=310 \mathrm{K}$. We choose for the $\epsilon_{0,0}=-D_0=-1.54 \text{eV}$ and then use $q_v=\frac{1}{1-\exp \left(-\theta_v / T\right)}$. By doing this, we note that we have chosen the ground state energy as $-D_0=-D_e+\frac{1}{2}\hbar\omega $. Following the same procedure for the iodine atom, we just get that:
\begin{equation}
  q_{\text{int}}= 4
\end{equation}
\section{}
  \item (50 pts.) Consider a one-dimensional monatomic crystal of $N$ atoms with equilibrium nearest neighbor spacing a. In order to minimize boundary effects, assume a periodic boundary condition, i.e., $x_{N+1}=x_{1}$ where $x_{i}$ is the position of the $i$ th atom. If the atoms interact only with nearest neighbors via a potential $u\left(x_{i+1}-x_{i}\right)$,

\end{enumerate}

(i) show that the energy of the crystal can be written in the form


\begin{equation*}
H=\frac{m}{2} \sum_{i=1}^{N} \dot{\xi}_{i}^{2}+\frac{K}{2} \sum_{i=1}^{N}\left(\xi_{i+1}-\xi_{i}\right)^{2}+N u(a) \tag{6}
\end{equation*}


to quadratic order in the displacement $\xi_{i} \equiv x_{i}-x_{i}^{(0)}$, where $x_{i}^{(0)}$ is the equilibrium position of the $i$ th atom. What is $K$ ?
\subsection{}
\subsubsection{}
These definitions mean that $\dot{\xi}_{i}$ is the velocity and is equal to the time derivative of $\xi_{i}=x_{i}-x_{i}^{(0)}\rightarrow \dot{\xi}_{i}=\dot{x}_{i}$. So the first term is just the kinetic anergy:
\begin{equation}
  \frac{m}{2}\sum_{i=1}^{N} v_{i}^{2} = \frac{m}{2}\sum_{i=1}^{N} \dot{\xi}_{i}^{2}
\end{equation}
The potential energy is:
\begin{equation}
  u(x_{i+1}-x_{i}) = u(x_{i+1}-x_{i+1}^{(0)}+x_{i+1}^{(0)}-x_{i}^{(0)}+x_{i}^{(0)}-x_{i}) = u(\xi_{i+1}-\xi_{i}+a)
\end{equation}

We want to tailor expand this latter function in powers of $\xi_{i+1}-\xi_{i}$:
\begin{equation}
  u(\xi_{i+1}-\xi_{i}+a) = u(a) + \left(\xi_{i+1}-\xi_{i}\right)u'(a) + \frac{1}{2}\left(\xi_{i+1}-\xi_{i}\right)^2u''(a) + \ldots
\end{equation}

(ii) Now decompose configurations of atoms into normal modes.
\subsubsection{}
(a) Define normal modes $\eta_{k}$ such that $\xi_{j}$ is a linear superposition of $\eta_{k}$

\begin{equation*}
\xi_{j}=\frac{1}{\sqrt{2 N}} \sum_{k} \eta_{k} e^{i(j a k)} \tag{7}
\end{equation*}


Show that the periodic boundary condition leads to $k=\frac{2 \pi n}{N a}$ where $n$ is any integer.

Show further that adding $\frac{2 \pi}{a}$ to $k$ does not change $\xi_{j}$. Therefore there are only $N$ independent modes. We choose $n \in\left[-\frac{N}{2}, \frac{N}{2}-1\right]$ (assuming $N$ even).
\subsection{}
I think this is similar to how quantization comes from the boundary conditions in the particle in a box.

(b) Show that the fact that the $\xi_{j}$ 's are real leads to $\eta_{k}^{*}=\eta_{-k}$, where $\eta_{k}^{*}$ is the complex conjugate of $\eta_{k}$.
\subsection{}
This has to be related to taking the complex contract of a plain wave exponential.

(iii) It can be shown that the normal mode coordinates $\eta_{k}$ diagonalize the Hamiltonian:


\begin{equation*}
\sum_{j}\left(\xi_{j+1}-\xi_{j}\right)^{2}=\sum_{k>0}\left[\left(\eta_{k}^{R}\right)^{2}+\left(\eta_{k}^{I}\right)^{2}\right] 4 \sin ^{2}\left(\frac{1}{2} k a\right) \tag{8}
\end{equation*}


and


\begin{equation*}
\sum_{j} \dot{\xi}_{j}^{2}=\sum_{k>0}\left[\left(\dot{\eta}_{k}^{R}\right)^{2}+\left(\dot{\eta}_{k}^{I}\right)^{2}\right] \tag{9}
\end{equation*}


where $\eta_{k}^{R}$ and $\eta_{k}^{I}$ are the real and imaginary parts of $\eta_{k}$. What is the frequency for each normal mode $\omega_{k}$ ? What is the speed of sound for this model? (The speed of sound is defined as $\left.\frac{d \omega_{k}}{d k}\right|_{k=0}$ ).
\subsection{}
The frequencies of the normal moods are going to be given by the agent values of this diagonal matrix.

(iv) Show that in a large 1-D solid, the density (or degenercy) of normal modes with the frequencies between $\omega$ and $\omega+d \omega$ is


\begin{equation*}
g(\omega) d \omega=\frac{2 N}{\pi \omega_{m} \sqrt{1-\left(\omega / \omega_{m}\right)^{2}}} d \omega \tag{10}
\end{equation*}


where $\omega_{m}$ is the maximum frequency of normal modes.

(v) Compare the Debye model and the exact results.

(a) If one is to make the Debye model for this 1-D solid, what is its Debye temperature $\Theta_{D}$ ? Why is the Debye frequency $\omega_{D}$ larger than the maximum frequency allowed in the system $\omega_{m}$ ?

(b) Show that one gets the same dependency of the heat capacity on temperature at very low temperature with the Debye approximation and with the exact degeneracy. Why does the Debye model give an accurate result even though unphysical normal modes are considered?


\end{document}