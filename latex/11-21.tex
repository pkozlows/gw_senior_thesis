\documentclass[12pt]{article}
\usepackage{amsmath}
\usepackage{physics}
\usepackage{graphicx}
\usepackage[linesnumbered,ruled,vlined]{algorithm2e}
\author{Patryk Kozlowski}
\title{G0W0}
\date{\today}
\begin{document}
\maketitle
\section{Implementation}
I want to discuss two equations today:
\begin{equation}
    \Sigma_{pp}^{\text{correlation}}(\omega) = \sum_{\mu }^{\text{RPA}}\left(\sum_{j}^{\text{occupied}} \frac{V_{pj}^{\mu }V_{pj}^{\mu }}{\omega -(\varepsilon _{j}-\Omega  _{\mu })}+ \sum_{b}^{\text{virtual}} \frac{V_{pb}^{\mu }V_{bp}^{\mu }}{\omega -(\varepsilon _{b}+\Omega  _{\mu })}\right)
\end{equation}
and the iterative procedure:
\begin{equation}
    \delta_{pq}F_{pq}^{\text{HF}}[\gamma^{\text{HF}}] + \Sigma_{p}^{\text{corr}}(\varepsilon_{p}^{\text{QP}}) = \varepsilon_{p}^{\text{QP}}
\end{equation}
My current understanding is that for now we have initially
\begin{equation}
    \varepsilon_{p}^{\text{QP}} = \varepsilon_{p}^{\text{HF}}
\end{equation}
which is just the molecular orbital energies from my initial HF calculation. Also, something like
\begin{equation}
    \delta_{pq}F_{pq}^{\text{HF}}[\gamma^{\text{HF}}]
\end{equation}
is just a diagonal matrix with the molecular orbital energies from my HF calculation. I know that we are making a whole lot of approximations, so what do you recommend that I play with next?
\section{Spin Integration}
We previously discussed that for the electron repulsion integrals in terms of spin orbitals
\begin{equation}
    (ia|jb) \rightarrow (i_{\alpha }a_{\alpha }|j_{\beta }b_{\beta }) , (i_{\beta }a_{\beta }|j_{\alpha }b_{\alpha }); (i_{\alpha }a_{\alpha }|j_{\alpha }b_{\alpha }) , (i_{\beta }a_{\beta }|j_{\beta }b_{\beta })
\end{equation}
We label them as 1, 2, 3, and 4 respectively. 
So we have a singlet and triplet CSF respectively
\begin{equation}
    \ket{\Psi _{S}} = \frac{1}{\sqrt{2}} \left(\ket{1}+\ket{2}\right)
\end{equation}
and
\begin{equation}
    \ket{\Psi _{T}} = \frac{1}{\sqrt{2}} \left(\ket{3}-\ket{4}\right)
\end{equation}
In these CSFs, the total spin is conserved.
How can I start to think about getting different factors from here e.g. 2 for
\begin{equation}
    A_{iajb}=\delta _{ij} \delta _{ab} \left(\varepsilon _{a}-\varepsilon _{i}\right) + (\underline{i} \underline{a} | \underline{j} \underline{b} )
\end{equation}
to
\begin{equation}
    A_{iajb}=\delta _{ij} \delta _{ab} \left(\varepsilon _{a}-\varepsilon _{i}\right) + 2(ia|jb)
\end{equation}
and $\sqrt{2}$ for 
\begin{equation}
    W_{p,q,i,a} = \sum_{\underline{p,q,i,a}} (\underline{p} \underline{q} | \underline{i} \underline{a} )
\end{equation}
to
\begin{equation}
    W_{p,q,i,a} = \sqrt{2} \sum_{p,q,i,a} (pq|ia)
\end{equation}
I assume an understanding of this concept will also be helpful for making $F_{pq}[\gamma^{DFT}]$. Right now I am just using $F_{pq}[\gamma^{HF}]$, which is just a simple diagonal approximation.
\section{Fock Matrix}
My current understanding of the Fock operator in the diagonal approximation is
\begin{equation}
    f_{pp}(\textbf{r})\ket{\phi _{p}} = H_{\text{core}}(\textbf{r})\ket{\phi _{p}} + 2*J - K
\end{equation}
I assume that all of the pieces here I can just get from $\gamma _{DFT}$.
I recall you said earlier that I should also take spin into account i.e. just calling $mf.get_fock()$ wont work. Could you try to cover this again?
\section{Filipp Furche}
I have come to the idea that doing research with him this summer is going to be a good idea. I will get a exposure to a different quantum chemistry code and will meet new people. I took a brief look at what he does and I saw that he does a thought of applications, but I also saw that he does something with RPA theory. So far I have not been able to get far enough in the project to get exposure to the RPa, but it seems like something that I would want to dive deeper into diving the summer. Do you have any thoughts about what I would want to ask him to do?
\end{document}