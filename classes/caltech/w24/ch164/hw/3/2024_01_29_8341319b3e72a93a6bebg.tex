\documentclass[10pt]{article}
\usepackage[utf8]{inputenc}
\usepackage[T1]{fontenc}
\usepackage{amsmath}
\usepackage{amsfonts}
\usepackage{amssymb}
\usepackage[version=4]{mhchem}
\usepackage{stmaryrd}

\title{Ch/ChE 164 Winter 2024 
 Homework Problem Set \#3 }

\author{}
\date{}


\def\AA{\mathring{\mathrm{A}}}

\begin{document}
\maketitle
Due Date: Thursday Feb 1, 2024 @ 11:59pm
\section{}
\begin{enumerate}
  \item (from Chandler 3.22)
\end{enumerate}

(a) (14 pts.) Consider a region within a fluid described by the van der Waals equation

$$
\beta p=\frac{\rho}{1-b \rho}-\beta a \rho^{2}
$$

where $\rho=\langle N\rangle / V$. The volume of the region is $L^{3}$. Due to the spontaneous fluctuations in the system, the instantaneous value of the density in that region can differ from its average by an amount $\delta \rho$. Determine, as a function of $\beta, \rho, a, b$, and $L^{3}$, the typical relative size of these fluctuations; that is, evaluate $\left\langle(\delta \rho)^{2}\right\rangle^{1 / 2} / \rho$. Demonstrate that when one considers observations of a macroscopic system (i.e., the size of the region becomes macroscopic, $L^{3} \rightarrow \infty$ ) the relative fluctuations become negligible.
\subsection{}
We have a fixed temperature $\beta = 1/kT$ and a fixed volume $V=L^3$. The relevant thermo dynamic potential for this problem is going to be the Helmholtz free energy. The differential of the Helmholtz free energy is given by:
\begin{equation}
  dF=-SdT-pdV+\mu dN
\end{equation}
Taking into account that the volume and the temperature are fixed, it becomes:

(b) (9 pts.) A fluid is at its "critical point" when

$$
\left(\frac{\partial \beta p}{\partial \rho}\right)_{\beta}=\left(\frac{\partial^{2} \beta p}{\partial \rho^{2}}\right)_{\beta}=0
$$

Determine the critical point density and temperature for a fluid obeying the van der Waals equation. That is compute $\beta_{c}$ and $\rho_{c}$ as a function of $a$ and $b$.

(c) (9 pts.) Focus attention on the subvolume of size $L^{3}$ in the fluid. Suppose $L^{3}$ is 100 times the space filling volume of a molecule - that is, $L^{3} \approx 100 b$. For this region in the fluid, compute the relative size of the density fluctuations when $\rho=\rho_{c}$, and the temperature is $10 \%$ above the critical temperature. Repeat this calculation for temperatures $0.1 \%$ and $0.001 \%$ from the critical temperature.

(d) (8 pts.) Light that we can observe with our eyes has wavelengths of order of $1000 \AA$. Fluctuations in density cause changes in the index of refraction, and those changes produce scattering of light. Therefore, if a region of fluid $1000 \AA$ across contains significant density fluctuations, we will visually observe these fluctuations. On the basis of the type of calculation performed in part (b), determine how close to the critical point a system must be before critical fluctuations become optically observable (that is, when the quantity $\left\langle(\delta \rho)^{2}\right\rangle^{1 / 2} / \rho$ is of order one). The phenomenon of long wavelength density fluctuations in a fluid approaching the critical point is known as opalescence. (Note: You will need to estimate the size of $b$, and to do this you should note that the typical diameter of a small molecule is around $5 \AA$ ).
\section{}
\begin{enumerate}
  \setcounter{enumi}{1}
  \item (i) (15 points) Consider a perfect crystal originally comprised of $N$ molecules on an equal number of lattice sites. The creation of vacancies in the lattice (or equivalently adding "surface" sites) is a thermally activated process with Boltzmann factor, $e^{-w / k T}$ where $w$ is the energy required to bring each of the molecules to the surface.
\end{enumerate}

If $n$ of these $N$ molecules move to the surface and leave $n$ vacant lattice sites, show that the partition function is well approximated by

$$
Q(N, T)=\sum_{n=0}^{\infty} \frac{(N+n) !}{n ! N !} e^{-n w / k T}
$$

Considering the thermodynamic limit and that $w$ is several $k T$, find the maximal term in the above sum to show that

$$
\frac{n}{N+n}=e^{-w / k T}
$$
\subsection{}
Each vacancy might be accompanied by an energy $\omega$ (the energy of a surface site) and the total energy of the microstate would, therefore, be $n\times \omega$.
As usual the Boltzmann factor is $e^{-\beta E}$, where $\beta=1/kT$ and $E=n\omega$.
Considering how canonical ensemble partition function is defined, we have:
\begin{equation}
  Q(N,T)=\sum_{n=0}^{n_{max}}\Omega (N,n)e^{-\beta E}
\end{equation}
where $\Omega (N,n)$ is the number of microstates with $N$ molecules and $n$ vacancies and $n_{max}$ is the maximum number of vacancies that can be created.
Considering that we have $N$ molecules and $n$ vacancies, there are $n$ ways to arrange the $N+n$ spaces, so:
\begin{equation}
  \Omega (N,n)= \binom{N+n}{n} = \frac{(N+n)!}{n!N!}
\end{equation}
Combining our results we have, for the partition function:
\begin{equation}
  Q(N,T)=\sum_{n=0}^{n_{max}}\frac{(N+n)!}{n!N!}e^{-\beta n\omega}
\end{equation}
Now, we wish to show that the sum can well be extended to infinity, by considering where the argument of the sum is maximized. We can do this by taking the derivative of the argument of the sum with respect to $n$, or alternatively, by taking the logarithm of the argument and then differentiating. We want to set this equal to zero, so we have:
\begin{equation}
  \frac{d}{dn}\ln\left(\frac{(N+n)!}{n!N!}e^{-\beta n\omega}\right)=0
\end{equation}
Applying Stirling's approximation to the logarithm of the factorials, we have:
\begin{equation}
  \ln(t_{n}) \approx (N+n)\ln (N+n)-n\ln n-N\ln N-\beta n\omega
\end{equation}
Now, we want to Taylor expand the logarithm of the argument of the sum around the maximum term, which we will call $n^{*}$. We have:
\begin{equation}
  \ln(t_{n}) \approx \ln(t_{n^{*}})+\frac{d\ln(t_{n})}{dn}\bigg|_{n=n^{*}}(n-n^{*})+\frac{1}{2}\frac{d^2\ln(t_{n})}{dn^2}\bigg|_{n=n^{*}}(n-n^{*})^2
\end{equation}
We know that the first derivative of the logarithm of the argument of the sum is zero at the maximum term, so the first term of the Taylor expansion is zero:
\begin{equation}
  \ln(t_{n}) \approx \ln (t_{n^{*}})+\frac{1}{2}\frac{d^2\ln(t_{n})}{dn^2}\bigg|_{n=n^{*}}(n-n^{*})^2
\end{equation}
We can now take the exponential of both sides of the equation, and we have:
\begin{equation}
  t_{n} \approx t_{n^{*}}\times \exp\left(\frac{1}{2}\frac{d^2\ln(t_{n})}{dn^2}\bigg|_{n=n^{*}}(n-n^{*})^2\right)
\end{equation}
\subsection{}
(ii) (15 pts.) Evaluate the above full partition function and then obtain $\langle n\rangle$ by suitable differentiation of $\log Q$. (Hint: Think of $Q(N, V, T)$ as a power series $Q(N, V, T)=\sum_{n=0}^{\infty} a_{n} z^{n}$.) Are the results for $\langle n\rangle$ from the maximum term and the full partition function the same?\\
We have the partition function:
\begin{equation}
  Q(N,T)=\sum_{n=0}^{\infty}\frac{(N+n)!}{n!N!}e^{-n\omega/kT}
\end{equation}
We can consider $z=e^{-\omega/kT}$ and $a_{n}=\frac{(N+n)!}{n!N!}$, so that we have:
\begin{equation}
  Q(N,T)=\sum_{n=0}^{\infty} a_{n} z^{n}
\end{equation}
We can recreate $a_{n}$ by considering:
\begin{equation}
  a_{n}=\frac{(N+n)!}{n!N!}=\frac{(N+n)(N+n-1)...(N+1)}{n!}
\end{equation}
We recognize the numerator as:
\begin{equation}
  \frac{\partial^n f}{\partial x^n}
\end{equation}
where $f=\frac{1}{(1+x^N)}$.
\section{}
\begin{enumerate}
  \setcounter{enumi}{2}
  \item (30 pts.) Show that the entropy for the Bose-Einstein and Fermi-Dirac gas can be written in the form
\end{enumerate}

$$
S=-k \sum_{\alpha=1}^{\infty}\left[\left\langle n_{\alpha}\right\rangle \ln \left\langle n_{\alpha}\right\rangle \mp\left(1 \pm\left\langle n_{\alpha}\right\rangle\right) \ln \left(1 \pm\left\langle n_{\alpha}\right\rangle\right)\right]
$$

where $\left\langle n_{\alpha}\right\rangle$ is the average occupation number for state $\alpha$.
\subsection{}
The Bose-Einstein and Fermi-Dirac distributions are given by:
\begin{equation}
  \left\langle n_{\alpha}\right\rangle=\frac{1}{e^{\beta\left(\epsilon_{\alpha}-\mu\right)}\mp 1}
\end{equation}
where the upper sign is for Bose-Einstein and the lower sign is for Fermi-Dirac.


\end{document}