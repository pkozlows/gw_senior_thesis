\documentclass{article}
\usepackage{amsmath}
\usepackage{physics}
\begin{document}
\section{}
We consider the quantum mechanics of a particle in the earth's gravitational field:
\begin{align}
    V(r) &= -\frac{GMm}{r} \tag{1} \\
    &= -\frac{GMm}{R + z} \tag{2} \\
    &\approx -\frac{GMm}{R} + mgz \tag{3}
\end{align}
where
\begin{itemize}
    \item $M =$ mass of earth
    \item $m =$ mass of particle
    \item $r =$ distance from center of earth
    \item $G =$ Newton's gravitational constant
    \item $R =$ radius of earth
    \item $z =$ height of particle above surface of earth
    \item $g = \frac{GM}{R^2}$
\end{itemize}
We may drop the constant term in our discussion, and consider only the $mgz$ piece, with $z \ll R$. We further assume that no angular momentum is involved, and treat this as a one-dimensional problem. Finally, assume that the particle is unable to penetrate the earth's surface.

\subsection{Question}
(a) Make a WKB calculation for the energy spectrum of the particle.
\subsection{Answer}
First, we want to figure out our classical turning points. Since the particle can't penetrate the earth's surface, we have:
\begin{equation}
    V(z) =
\begin{cases}
    mgz, & z \geq 0 \\
    \infty, & z < 0
\end{cases}
\end{equation}
Then, we also have
\begin{equation}
    p = \sqrt{2m(E - mgz)}
\end{equation}
So, the $f(E)$ is given by:
\begin{equation}
    f(E) = \int_{z_1}^{z_2} \sqrt{2m(E - mgz)} dz\\
= \int_{0}^{\infty} \sqrt{2m(E - mgz)} dz
\end{equation}

\end{document}
