\documentclass[12pt]{article}
\usepackage[utf8]{inputenc}
\usepackage[T1]{fontenc}
\usepackage{amsmath}
\usepackage{amsfonts}
\usepackage{amssymb}
\usepackage[version=4]{mhchem}
\usepackage{stmaryrd}
\usepackage{graphicx}
\usepackage{physics}

\usepackage{listings} % Required for insertion of code
\usepackage{xcolor} % Required for custom colors

% Define custom colors
\definecolor{codegreen}{rgb}{0,0.6,0}
\definecolor{codegray}{rgb}{0.5,0.5,0.5}
\definecolor{codepurple}{rgb}{0.58,0,0.82}
\definecolor{backcolour}{rgb}{0.95,0.95,0.92}

% Setup the style for code listings
\lstdefinestyle{mystyle}{
    backgroundcolor=\color{backcolour},   
    commentstyle=\color{codegreen},
    keywordstyle=\color{magenta},
    numberstyle=\tiny\color{codegray},
    stringstyle=\color{codepurple},
    basicstyle=\ttfamily\footnotesize,
    breakatwhitespace=false,         
    breaklines=true,                 
    captionpos=b,                    
    keepspaces=true,                 
    numbers=left,                    
    numbersep=5pt,                  
    showspaces=false,                
    showstringspaces=false,
    showtabs=false,                  
    tabsize=2
}

% Activate the style
\lstset{style=mystyle}


\title{Ch/ChE 164 Winter 2024 
 Homework Problem Set \#5 
 Due Date: Thursday, February 22, 2024 @ 11:59pm PT 
 Out of 100 Points }

\author{}
\date{}


\begin{document}
\maketitle
\section{}
\begin{enumerate}
  \item (15 points) (4.15 from Chandler) Consider an isomerization process $A \rightleftharpoons B$, where $A$ and $B$ refer to the different isomer states of a molecule. Imagine that the process takes place in a dilute gas, and that $\Delta \epsilon$ is the energy difference between state $A$ and state $B$. According to the Boltzmann distribution law, the equilibrium ratio of $A$ and $B$ populations is given by
\end{enumerate}


\begin{equation*}
\frac{\left\langle N_{A}\right\rangle}{\left\langle N_{B}\right\rangle}=\frac{g_{A}}{g_{B}} e^{-\beta \Delta \epsilon}, \tag{1}
\end{equation*}


where $g_{A}$ and $g_{B}$ are the degeneracies of states $A$ and $B$, respectively. Show how this same result follows from the condition of chemical equilibria, $\mu_{A}=\mu_{B}$.
\subsection{}
The total partition function for the system will be given by:
\begin{equation}
Q=Q^A Q^B=\frac{1}{N_{A} !} q_A^{N_A} \frac{1}{N_{B} !} q_B^{N_B}
\end{equation}
where $N_A$ and $N_B$ are the number of molecules in the states $A$ and $B$ respectively. 
The equation for the home holds free energy is:
\begin{equation}
F=-k_B T \log Q
\end{equation}
So we plug in to get:
\begin{equation}
F=-k_B T \log \left(\frac{1}{N_{A} !} q_A^{N_A} \frac{1}{N_{B} !} q_B^{N_B}\right) = -k_B T \log \left(\frac{1}{N_{A} !} q_A^{N_A}\right) -k_B T \log \left(\frac{1}{N_{B} !} q_B^{N_B}\right) = 
\end{equation}
Further separation of the logarithms gives:
\begin{equation}
  = kT\left( \ln \left(N_A!\right) - N_A \ln q_A + \ln \left(N_B!\right) - N_B \ln q_B\right)
\end{equation}
We can use the Stirling approximation to get:
\begin{equation}
  = kT\left( N_A \ln N_A - N_A - N_A \ln q_A + N_B \ln N_B - N_B - N_B \ln q_B\right)
\end{equation}
Now, we know that the derivative of this free energy with respect to the number of particles in a state gives the chemical potential for the state:
\begin{equation}
  \mu_{A,B} = \frac{\partial F}{\partial N_{A,B}}
\end{equation}
For the state $A$ we get:
\begin{equation}
  \mu_A = kT \left(\log{\left(N_{A} \right)} - \log{\left(q_{A} \right)}\right)
\end{equation}
And for the state $B$ we get:
\begin{equation}
  \mu_B = kT \left(\log{\left(N_{B} \right)} - \log{\left(q_{B} \right)}\right)
\end{equation}
% Inline Python code in the document
\begin{lstlisting}[language=Python]
from sympy import symbols, diff, log, latex

# Define symbols
kT, N_A, N_B, q_A, q_B = symbols('kT N_A N_B q_A q_B')

# Define the free energy expression using Stirling's approximation
F = kT * (N_A * log(N_A) - N_A - N_A * log(q_A) + N_B * log(N_B) - N_B - N_B * log(q_B))

# Compute the derivatives
mu_A = diff(F, N_A)
mu_B = diff(F, N_B)

# Display the results in LaTeX
mu_A_latex = mu_A.simplify().doit()
mu_B_latex = mu_B.simplify().doit()

print(latex(mu_A_latex))
# latex(mu_B_latex)
\end{lstlisting}
The equilibrium contention tells us that we can set both equations equal to each other:
\begin{equation}
  \mu_A = \mu_B \rightarrow kT \left(\log{\left(N_{A} \right)} - \log{\left(q_{A} \right)}\right) = kT \left(\log{\left(N_{B} \right)} - \log{\left(q_{B} \right)}\right)
\end{equation}
We also consider that we have the form for the single partial partition functions:
\begin{equation}
  q_A = g_A e^{-\beta \epsilon_A}
\end{equation}
\begin{equation}
  q_B = g_B e^{-\beta \epsilon_B}
\end{equation}
We can use these to get:
\begin{equation}
  \log{\left(q_{A} \right)} = \log{\left(g_A e^{-\beta \epsilon_A} \right)} = \log{\left(g_A \right)} - \beta \epsilon_A
\end{equation}
\begin{equation}
  \log{\left(q_{B} \right)} = \log{\left(g_B e^{-\beta \epsilon_B} \right)} = \log{\left(g_B \right)} - \beta \epsilon_B
\end{equation}
We can use these to get:
\begin{equation}
  kT \left(\log{\left(N_{A} \right)} - \log{\left(q_{A} \right)}\right) = kT \left(\log{\left(N_{A} \right)} - \log{\left(g_A \right)} + \beta \epsilon_A\right)
\end{equation}
\begin{equation}
  kT \left(\log{\left(N_{B} \right)} - \log{\left(q_{B} \right)}\right) = kT \left(\log{\left(N_{B} \right)} - \log{\left(g_B \right)} + \beta \epsilon_B\right)
\end{equation}
Setting but size equate again:
\begin{equation}
  kT \left(\log{\left(N_{A} \right)} - \log{\left(g_A \right)} + \beta \epsilon_A\right) = kT \left(\log{\left(N_{B} \right)} - \log{\left(g_B \right)} + \beta \epsilon_B\right)
\end{equation}
We can cancel the common factor out font from both sites and exponential both sites:
\begin{equation}
  \exp{\left(\log{\left(N_{A} \right)} - \log{\left(g_A \right)} + \beta \epsilon_A\right)} = \exp{\left(\log{\left(N_{B} \right)} - \log{\left(g_B \right)} + \beta \epsilon_B\right)}
\end{equation}
We can use the properties of the logarithm to get:
\begin{equation}
  \frac{N_A}{g_A} e^{\beta \epsilon_A} = \frac{N_B}{g_B} e^{\beta \epsilon_B}
\end{equation}
Rearranging the terms now:
\begin{equation}
  \frac{N_A}{N_B} = \frac{g_A}{g_B} e^{\beta \epsilon_B - \beta \epsilon_A}
\end{equation}
Defining the energy difference as $\Delta \epsilon = \epsilon_A - \epsilon_B$ we get:
\begin{equation}
  \frac{N_A}{N_B} = \frac{g_A}{g_B} e^{-\beta \Delta \epsilon}
\end{equation}

\section{}
\begin{enumerate}
  \setcounter{enumi}{1}
  \item (20 points) (4.16 from Chandler) Consider the system described in problem 1. The canonical partition function is
\end{enumerate}


\begin{equation*}
Q=\frac{q^{N}}{N !} \tag{2}
\end{equation*}


where $N$ is the total number of molecules, and $q$ is the Boltzmann weighted sum over all single molecule states, both those associated with isomers of type $A$ and those associated with isomers of type $B$.

(a) Show that one may partition the sum and write


\begin{equation*}
Q=\sum_{P} \exp \left\{-\beta F\left(N_{A}, N_{B}\right)\right\} \tag{3}
\end{equation*}


with


\begin{equation*}
-\beta F\left(N_{A}, N_{B}\right)=\log \left[\left(N_{A} ! N_{B} !\right)^{-1} q_{A}^{N_{A}} q_{B}^{N_{B}}\right], \tag{4}
\end{equation*}


where $\sum_{P}$ is over all the partitions of $N$ molecules into $N_{A}$ molecules of type $A$ and $N_{B}$ molecules of type $B, q_{A}$ is the Boltzmann weighted sum over states of isomer $A$, and $q_{B}$ is similarly defined.
\subsection{}
A single practical partition function is defined as:
\begin{equation}
q = \sum_{i} e^{-\beta \epsilon_i}
\end{equation}
In this case, we have only two options; state $A$ and state $B$. So the single particle partition function is just a sum of the partition functions for each isomer
\begin{equation}
q = q_A + q_B
\end{equation}
where:
\begin{equation}
q_A = g_A e^{-\beta \epsilon_A}
\end{equation}
\begin{equation}
q_B = g_B e^{-\beta \epsilon_B}
\end{equation}
The total partition function is then:
\begin{equation}
Q=\frac{q^{N}}{N !} = \frac{\left(q_A + q_B\right)^{N}}{N !}
\end{equation}
This resembles the binomial expansion:
\begin{equation}
\left(a+b\right)^N = \sum_{N} \binom{N}{N_A} a^{N_A} b^{N_B}
\end{equation}
where the sum is over all the partitions of $N$ molecules into $N_{A}$ molecules of type $A$ and $N_{B}$ molecules of type $B$. So, we get:
\begin{equation}
Q=\frac{1}{N!}\sum_{N_A = 0}^{
N} \frac{N!}{N_A! N_B!} q_A^{N_A} q_B^{N_B} = \sum_{P} \frac{q_A^{N_A} q_B^{N_B}}{N_A! N_B!}
\end{equation}
Taking the logarithm and then exponentiating the inside of the sum we get:
\begin{equation}
Q = \sum_{P} \exp{\left(\log{\left(\frac{q_A^{N_A} q_B^{N_B}}{N_A! N_B!}\right)}\right)}
\end{equation}
And we know that this should be the same as:
\begin{equation}
Q = \sum_{P} \exp{\left(-\beta F\left(N_{A}, N_{B}\right)\right)}
\end{equation}
So, by inspection, we have:
\begin{equation}
-\beta F\left(N_{A}, N_{B}\right)=\log \left[\left(N_{A} ! N_{B} !\right)^{-1} q_{A}^{N_{A}} q_{B}^{N_{B}}\right]
\end{equation}


(b) Show that the condition of chemical equilibria is identical to finding the partitioning that minimizes the Helmholtz free energy


\begin{equation*}
\frac{\partial F}{\partial\left\langle N_{A}\right\rangle}=\frac{\partial F}{\partial\left\langle N_{B}\right\rangle}=0 \tag{5}
\end{equation*}


subject to the constraint that $\left\langle N_{A}\right\rangle+\left\langle N_{B}\right\rangle=N$ is fixed.
\subsection{}
We will start by assuming the equilibrium condition of the chemical potentials:
\begin{equation}
\mu_A = \mu_B
\end{equation}
we will use this to prove equation 5.
Since the contract variable of the chemical potential is $\beta N$, we have the relation:
\begin{equation}
\mu = -\frac{\partial \ln Q}{\partial \beta N}
\end{equation}
and therefore for the state $A$ we get:
\begin{equation}
\mu_A = -\frac{\partial \ln Q}{\partial \beta N_A}
\end{equation}
and for the state $B$ we get:
\begin{equation}
\mu_B = -\frac{\partial \ln Q}{\partial \beta N_B}
\end{equation}
The condition for the chemical equilibrium gives:
\begin{equation}
\frac{\partial F }{\partial N_A} = \frac{\partial F }{\partial N_B}
\end{equation}
But because we have the constraint that $N_A + N_B = N$, this means that:
\begin{equation}
\frac{\partial F }{\partial N_A} = -\frac{\partial F }{\partial N_B}
\end{equation}
For both identities to hold, we need to have:
\begin{equation}
\frac{\partial F }{\partial N_A} = \frac{\partial F }{\partial N_B} = 0
\end{equation}
% \subsection{}
% Solving for $F(N_A, N_B)$ we get:
% \begin{equation}
% F(N_A, N_B) = -kT \log \left[\left(N_{A} ! N_{B} !\right)^{-1} q_{A}^{N_{A}} q_{B}^{N_{B}}\right]
% \end{equation}
% Separating the logarithm we get:
% \begin{equation}
% F(N_A, N_B) = -kT \left(\log{q_{A}^{N_{A}} } + \log{q_{B}^{N_{B}} } - \log{N_{A} ! } - \log{N_{B} ! }\right)
% \end{equation}
% We can use the Stirling approximation to get:
% \begin{equation}
% F(N_A, N_B) = -kT \left(N_{A} \log{q_{A}} + N_{B} \log{q_{B}} - N_{A} \log{N_{A}} + N_{A} - N_{B} \log{N_{B}} + N_{B}\right)
% \end{equation}
% Taking the appropriate derivatives gives:
% \begin{equation}
% \frac{\partial F}{\partial N_A} = kT \log\left(\frac{N_A}{q_A}\right)
% \end{equation}
% We set this equal to 0:
% \begin{equation}
%   \log\left(\frac{N_A}{q_A}\right) = 0 \rightarrow \frac{N_A}{q_A} = 1 \rightarrow N_A = q_A
% \end{equation}
% \begin{equation}
%   \frac{\partial F}{\partial N_B} = kT \log\left(\frac{N_B}{q_B}\right)
% \end{equation}
% We set this equal to 0:
% \begin{equation}
%   \log\left(\frac{N_B}{q_B}\right) = 0 \rightarrow \frac{N_B}{q_B} = 1 \rightarrow N_B = q_B
% \end{equation}
% % Inline Python code in the document
% \begin{lstlisting}[language=Python]
% from sympy import symbols, diff, log

% # Define symbols
% k, T, N_A, N_B, q_A, q_B = symbols('k T N_A N_B q_A q_B', real=True, positive=True)

% # Helmholtz free energy function
% F = -k*T*(N_A*log(q_A) + N_B*log(q_B) - N_A*log(N_A) + N_A - N_B*log(N_B) + N_B)

% # Derivatives with respect to N_A and N_B
% dF_dN_A = diff(F, N_A)
% dF_dN_B = diff(F, N_B)

% dF_dN_A.simplify(), dF_dN_B.simplify()

% \end{lstlisting}
\section{}
\begin{enumerate}
  \setcounter{enumi}{2}
  \item (15 points) (4.25 from Chandler) Use the information compiled in Chapter 8 of Hill's Introduction to Statistical Thermodynamics to calculate the equilibrium constant, $K$, for the reaction $I_{2} \rightleftharpoons 2 I$ when the reaction occurs in the dilute gas phase at $T=1000^{\circ} \mathrm{K}$. (Note: For the electronic partition function, you need to consider the difference in degeneracies of the ground states of the iodine atom and molecule, $g_{e, I}=4, g_{e, I_{2}}=1$. This is based on the molecular orbital theory. You are not required to show this.)
\subsection{}
We want to start by considering the individual partition functions for the molecule. We use the following formula to compute the one for the molecule:
\begin{equation}
\begin{aligned}
q_{\mathrm{int}}=g_{0, e}\left(2 I_A+1\right)\left(2 I_B+1\right) e^{-\beta \epsilon_{00}} q_{\mathrm{vi}} q_{\mathrm{ro}} / \sigma_{A B}
\end{aligned}
\end{equation}
where we have:
$q_{\mathrm{ro}}=T / \Theta_{\mathrm{ro}}$. Also the degeneracy of the iodine molecule is $g_{e, I_{2}}=1$. 
and then the total partition function for the species will be:
$$
q_i=\left(V / \lambda_i^3\right) q_i^{(\text {int })},
$$
where $q_i^{(\text {int })}$ is for species $i$ the $q_{\text {int }}(T)$ of the previous section, and
$$
\lambda_i=h / \sqrt{2 \pi m_i k_B T}
$$
is the thermal de Broglie wavelength of species $i$.
The mass that goes into this is simply that of the iodine atom or the combined mass of two iodine atoms for the diatomic.
The nuclear spin quantum numbers are going to be the same for the diatomic $I_A=I_B=\frac{5}{2}$. Additionally, since we are dealing with a diatomic $\sigma _{A B}=2$. We get the rotational temperature from Hill as $\theta_{\text {rot }}=0.054 \mathrm{K}$. The vibrational temperature is $\theta_{\text {vib }}=310 \mathrm{K}$. We choose for the $\epsilon_{0,0}=-D_0=-1.54 \text{eV}$ and then use $q_v=\frac{1}{1-\exp \left(-\theta_v / T\right)}$. By doing this, we note that we have chosen the ground state energy as $-D_0=-D_e+\frac{1}{2}\hbar\omega $. Following the same procedure for the iodine atom, we just get that:
\begin{equation}
  q_{\text{int}}= 4
\end{equation}
The condition for chemical equilibrium will be $\sum_{\nu} \nu_{i} \mu_{i}=0$. So for the reaction $I_{2} \rightleftharpoons 2 I$ we get:
\begin{equation}
  \mu_{I_2} = 2\mu_{I}
\end{equation}
This gives the following algorithm. Since we know the partition function for the system, we can find the free energy using:
\begin{equation}
  F = -kT \log Q
\end{equation}
We can then use the free energy to find the chemical potential using:
\begin{equation}
  \mu = \frac{\partial F}{\partial N}
\end{equation}
We can then use the chemical potential to find the equilibrium constant using:
\begin{equation}
  K = e^{-\beta \Delta G}
\end{equation}
where $\Delta G = \mu_{I_2} - 2\mu_{I}$. 
Performing the calculations we get:
\begin{equation}
  K = 0.00110582457135813 \frac{\text{mol}}{m^3}
\end{equation}
% Inline Python code in the document
\begin{lstlisting}[language=Python]
from sympy import symbols, exp, sqrt, pi, log, solve

# Defining symbols
T, V, h, k_B, Theta_rot, Theta_vib, epsilon_0_0, m_I, g_e_I2, g_e_I, sigma_AB = symbols(
    'T V h k_B Theta_rot Theta_vib epsilon_0_0 m_I g_e_I2 g_e_I sigma_AB')

# Given constants
given_constants = {
    h: 6.62607015e-34,  # Planck's constant, J*s
    k_B: 1.380649e-23,  # Boltzmann's constant, J/K
    Theta_rot: 0.054,  # Rotational temperature, K
    Theta_vib: 310,  # Vibrational temperature, K
    epsilon_0_0: -1.54 * 1.602176634e-19,  # Ground state energy, J (converted from eV)
    m_I: 126.90447 * 1.660539040e-27,  # Mass of iodine atom, kg (converted from u to kg)
    g_e_I2: 1,  # Degeneracy of iodine molecule
    g_e_I: 4,  # Degeneracy of iodine atom
    sigma_AB: 2,  # Symmetry number for I2
    V: 1  # Assuming volume of 1 m^3 for simplicity
}

# Nuclear spin quantum numbers for iodine (diatomic), assuming I_A = I_B = 5/2
I_A = I_B = 5/2

# Calculate partition functions
# Thermal de Broglie wavelength
lambda_I = h / sqrt(2 * pi * m_I * k_B * T)
lambda_I2 = h / sqrt(2 * pi * (2 * m_I) * k_B * T)

# Internal partition function q_int for I2
q_int_I2 = g_e_I2 * (2 * I_A + 1) * (2 * I_B + 1) * exp(-epsilon_0_0 / (k_B * T)) * (T / Theta_rot) * (1 / (1 - exp(-Theta_vib / T))) / sigma_AB

# Internal partition function q_int for I
q_int_I = g_e_I  # For iodine atom, considering only electronic degeneracy

# Total partition function for I2 and I
q_I2 = (V / lambda_I2**3) * q_int_I2
q_I = (V / lambda_I**3) * q_int_I

# Simplify expressions with given constants
q_I2_simplified = q_I2.subs(given_constants)
q_I_simplified = q_I.subs(given_constants)

# Calculate chemical potentials for I2 and I using the simplified partition functions
mu_I2 = -k_B * T * log(q_I2_simplified / V)
mu_I = -k_B * T * log(q_I_simplified / V)

# Calculate the equilibrium constant K for the reaction I2 <-> 2I
Delta_mu = mu_I2 - 2 * mu_I
K_expression = exp(Delta_mu / (k_B * T))
K_simplified = K_expression.simplify()

# Substitute T=1000 K and V=1 m^3 into the expression for K to calculate its value at T=1000 K
K_value_at_1000K = K_simplified.subs({T: 1000, V: 1})

# Evaluate the expression
K_value_at_1000K.evalf()

#  Avogadro's number
N_A = 6.02214076e+23  # mol^-1

# Divide K by Avogadro's number to convert to terms of moles
K_mols_divided = K_value_at_1000K / N_A

K_mols_divided.evalf()

\end{lstlisting}
\section{}
  \item (50 pts.) Consider a one-dimensional monatomic crystal of $N$ atoms with equilibrium nearest neighbor spacing a. In order to minimize boundary effects, assume a periodic boundary condition, i.e., $x_{N+1}=x_{1}$ where $x_{i}$ is the position of the $i$ th atom. If the atoms interact only with nearest neighbors via a potential $u\left(x_{i+1}-x_{i}\right)$,

\end{enumerate}

(i) show that the energy of the crystal can be written in the form


\begin{equation*}
H=\frac{m}{2} \sum_{i=1}^{N} \dot{\xi}_{i}^{2}+\frac{K}{2} \sum_{i=1}^{N}\left(\xi_{i+1}-\xi_{i}\right)^{2}+N u(a) \tag{6}
\end{equation*}


to quadratic order in the displacement $\xi_{i} \equiv x_{i}-x_{i}^{(0)}$, where $x_{i}^{(0)}$ is the equilibrium position of the $i$ th atom. What is $K$ ?
\subsection{}
\subsubsection{}
These definitions mean that $\dot{\xi}_{i}$ is the velocity and is equal to the time derivative of $\xi_{i}=x_{i}-x_{i}^{(0)}\rightarrow \dot{\xi}_{i}=\dot{x}_{i}$. So the first term is just the kinetic anergy:
\begin{equation}
  \frac{m}{2}\sum_{i=1}^{N} v_{i}^{2} = \frac{m}{2}\sum_{i=1}^{N} \dot{\xi}_{i}^{2}
\end{equation}
The potential energy is:
\begin{equation}
  u(x_{i+1}-x_{i}) = u(x_{i+1}-x_{i+1}^{(0)}+x_{i+1}^{(0)}-x_{i}^{(0)}+x_{i}^{(0)}-x_{i}) = u(\xi_{i+1}-\xi_{i}+a)
\end{equation}

We want to tailor expand this latter function in powers of $\xi_{i+1}-\xi_{i}$:
\begin{equation}
  u(\xi_{i+1}-\xi_{i}+a) = u(a) + \left(\xi_{i+1}-\xi_{i}\right)u'(a) + \frac{1}{2}\left(\xi_{i+1}-\xi_{i}\right)^2u''(a) + \ldots
\end{equation}
Our expression for the total potential energy is going to involve a sum of this expression $N$ atoms:
\begin{equation}
  U = \sum_{i=1}^{N} u(\xi_{i+1}-\xi_{i}+a) = Nu(a) + \sum_{i=1}^{N} \left(\xi_{i+1}-\xi_{i}\right)u'(a) + \frac{1}{2}\sum_{i=1}^{N}\left(\xi_{i+1}-\xi_{i}\right)^2u''(a) + \ldots
\end{equation}
On average the displacements over all of the autumns in the latest finish, so the second term vanishes. We are left with:
\begin{equation}
  U = Nu(a) + \frac{1}{2}\sum_{i=1}^{N}\left(\xi_{i+1}-\xi_{i}\right)^2u''(a)
\end{equation}
So this is where the 2 later terms in our Hulton on come from with $K=u''(a)$.
\subsection{}
(ii) Now decompose configurations of atoms into normal modes.
\subsubsection{}
(a) Define normal modes $\eta_{k}$ such that $\xi_{j}$ is a linear superposition of $\eta_{k}$

\begin{equation*}
\xi_{j}=\frac{1}{\sqrt{2 N}} \sum_{k} \eta_{k} e^{i(j a k)} \tag{7}
\end{equation*}


Show that the periodic boundary condition leads to $k=\frac{2 \pi n}{N a}$ where $n$ is any integer.

Show further that adding $\frac{2 \pi}{a}$ to $k$ does not change $\xi_{j}$. Therefore there are only $N$ independent modes. We choose $n \in\left[-\frac{N}{2}, \frac{N}{2}-1\right]$ (assuming $N$ even).

Because of the periodic bander by conditions, we need to have thet:
\begin{equation}
  \xi_{N+1} = \xi_{1}
\end{equation}
We can use the definition of the normal modes to get:
\begin{equation}
  \xi_{N+1} = \frac{1}{\sqrt{2 N}} \sum_{k} \eta_{k} e^{i(N+1) a k}
\end{equation}
and then:
\begin{equation}
  \xi_{1} = \frac{1}{\sqrt{2 N}} \sum_{k} \eta_{k} e^{i a k}
\end{equation}
This implies that:
\begin{equation}
  1 = e^{i a N k}
\end{equation}
For this condition to be certified, we need:
\begin{equation}
  a N k = 2 \pi n
\end{equation}
where $n$ is any integer. We can then solve for $k$ to get:
\begin{equation}
  k = \frac{2 \pi n}{N a}
\end{equation}


(b) Show that the fact that the $\xi_{j}$ 's are real leads to $\eta_{k}^{*}=\eta_{-k}$, where $\eta_{k}^{*}$ is the complex conjugate of $\eta_{k}$.
\subsection{}
The fact that the $\xi_{j}$ 's are real means that:
\begin{equation}
  \xi_{j} = \xi_{j}^{*}
\end{equation}
We can use the definition of the normal modes to get:
\begin{equation}
  \xi_{j} = \frac{1}{\sqrt{2 N}} \sum_{k} \eta_{k} e^{i j a k}
\end{equation}
and then:
\begin{equation}
  \xi_{j}^{*} = \frac{1}{\sqrt{2 N}} \sum_{k} \eta_{k}^{*} e^{-i j a k}
\end{equation}
This equality can only be accomplished (the exponentials only match) when we have the top sum running over $-k$ instead of $k$. This means that:
\begin{equation}
  \eta_{k}^{*} = \eta_{-k}
\end{equation}

(iii) It can be shown that the normal mode coordinates $\eta_{k}$ diagonalize the Hamiltonian:


\begin{equation*}
\sum_{j}\left(\xi_{j+1}-\xi_{j}\right)^{2}=\sum_{k>0}\left[\left(\eta_{k}^{R}\right)^{2}+\left(\eta_{k}^{I}\right)^{2}\right] 4 \sin ^{2}\left(\frac{1}{2} k a\right) \tag{8}
\end{equation*}


and


\begin{equation*}
\sum_{j} \dot{\xi}_{j}^{2}=\sum_{k>0}\left[\left(\dot{\eta}_{k}^{R}\right)^{2}+\left(\dot{\eta}_{k}^{I}\right)^{2}\right] \tag{9}
\end{equation*}


where $\eta_{k}^{R}$ and $\eta_{k}^{I}$ are the real and imaginary parts of $\eta_{k}$. What is the frequency for each normal mode $\omega_{k}$ ? What is the speed of sound for this model? (The speed of sound is defined as $\left.\frac{d \omega_{k}}{d k}\right|_{k=0}$ ).
\subsection{}
We start by considering the equation:
\begin{equation}
  -m  \ddot{x}= \frac{\partial H}{\partial x}
\end{equation}
We want to consider this equation but specifically for item $j$ and then we can also use the relation $x_j = \xi_j$. We get:
\begin{equation}
  -m  \ddot{\xi}_{j}= \frac{\partial H}{\partial \xi_{j}}
\end{equation}
Our Helmut union was defined as:
\begin{equation}
  H=\frac{m}{2} \sum_{i=1}^{N} \dot{\xi}_{i}^{2}+\frac{K}{2} \sum_{i=1}^{N}\left(\xi_{i+1}-\xi_{i}\right)^{2}+N u(a)
\end{equation}
For the first term all of the terms in the summation will automatically be 0, unless $i=j$. Further more even for the term $i=j$ this involves:
\begin{equation}
  \frac{\partial }{\partial \xi_{j}}\frac{\partial \xi_{j}}{\partial t} = \frac{\partial }{\partial t}\frac{\partial \xi_{j}}{\partial \xi_{j}} = \frac{\partial }{\partial t}1 = 0
\end{equation}
So the first term in the Hamiltonian does not contribute to the equation of motion. The third term also vanishes as it has no dependence on $\xi_j$. The second term only doesn't vanish when we have either $i=j$ or $i=j-1$. Taking the derivative and applying the chain rule gives for the derivative:
\begin{equation}
  \frac{\partial H}{\partial \xi_{j}} = -K\left(\xi_{j+1}-\xi_{j}\right) + K\left(\xi_{j}-\xi_{j-1}\right)
\end{equation}
So for the equation of motion we get:
\begin{equation}
  -m  \ddot{\xi}_{j}= K(2 \xi_{j}-\xi_{j+1}-\xi_{j-1})
\end{equation}
Now substituting in for the four year transform expression for the normal modes:
\begin{equation}
  -\frac{m}{\sqrt{2N}} \sum_{k} \ddot{\eta}_{k} e^{i j a k}= K(2 \frac{1}{\sqrt{2N}} \sum_{k} \eta_{k} e^{i j a k}-\frac{1}{\sqrt{2N}} \sum_{k} \eta_{k} e^{i (j+1) a k}-\frac{1}{\sqrt{2N}} \sum_{k} \eta_{k} e^{i (j-1) a k})
\end{equation}
We can then multiply both sides by $\sqrt{2N} e^{-i j a k}$ and equate the terms in the summations to get:
\begin{equation}
  -m \ddot{\eta}_{k} = K(2 - e^{i a k} - e^{-i a k})\eta_{k}
\end{equation}
Recognizing that $e^{i a k} + e^{-i a k} = 2 \cos(a k)$ we get:
\begin{equation}
\ddot{\eta}_{k} = -\frac{2K}{m} \eta_{k} (1 - \cos(a k)) 
\end{equation}
If we set $f(k) = - \frac{2K}{m} (1 - \cos(a k))$ we get a Sturm-Liouville equation:
\begin{equation}
\ddot{\eta}_{k} = f(k) \eta_{k}
\end{equation}
The general solution to this equation is:
\begin{equation}
\eta_{k} = A \cos(\omega_k t) + B \sin(\omega_k t)
\end{equation}
where $\omega_k$ is the frequency for each normal mode. We can solve for $\omega_k$ by taking the second derivative of the general solution and then substituting it into the equation of motion. We get:
\begin{equation}
\ddot{\eta}_{k} = -A \omega_k^2 \cos(\omega_k t) - B \omega_k^2 \sin(\omega_k t)
\end{equation}
Substituting this into the equation of motion we get:
\begin{equation}
-A \omega_k^2 \cos(\omega_k t) - B \omega_k^2 \sin(\omega_k t) = f(k) (A \cos(\omega_k t) + B \sin(\omega_k t))
\end{equation}
On the left hand side we can factor out a $-\omega_k^2$ and then divide by the general solution to get:
\begin{equation}
-\omega_k^2 = f(k)
\end{equation}
So the frequency for each normal mode $\omega_{k}$ is:
\begin{equation}
\omega_k = \sqrt{\frac{2K}{m} (1 - \cos(a k))}
\end{equation}
The speed of sound for this model is defined as $\left.\frac{d \omega_{k}}{d k}\right|_{k=0}$. At first I just took the derivative of $\omega_k$ with respect to $k$ and then evaluated it at $k=0$.
% Inline Python code in the document
\begin{lstlisting}[language=Python]
from sympy import symbols, cos, sin, sqrt, diff

# Define symbols
K, m, a, k = symbols('K m a k')

# Original omega_k equation
omega_k = sqrt(2*K/m * (1 - cos(a*k)))

# Derivative of omega_k with respect to k
d_omega_k_dk = diff(omega_k, k)

# Evaluate the derivative at k=0
d_omega_k_dk_at_0 = d_omega_k_dk.subs(k, 0)

omega_k, d_omega_k_dk, d_omega_k_dk_at_0

\end{lstlisting}
However, this gave me a divided by zero error, so I have to apply L'Hopital's rule to get the result. This gives me:
\begin{equation}
  \frac{d \omega_{k}}{d k} \eval _{k=0} = \frac{a \sqrt{ K}}{\sqrt{m}}
\end{equation}
% Inline Python code in the document
\begin{lstlisting}[language=Python]
from sympy import limit

# Apply L'Hopital's rule by evaluating the limit of the derivative as k approaches 0
speed_of_sound = limit(d_omega_k_dk, k, 0)

speed_of_sound
\end{lstlisting}


(iv) Show that in a large 1-D solid, the density (or degenercy) of normal modes with the frequencies between $\omega$ and $\omega+d \omega$ is


\begin{equation*}
g(\omega) d \omega=\frac{2 N}{\pi \omega_{m} \sqrt{1-\left(\omega / \omega_{m}\right)^{2}}} d \omega \tag{10}
\end{equation*}


where $\omega_{m}$ is the maximum frequency of normal modes.
\subsection{}
We had the expression for the frequency:
\begin{equation}
\omega_k = \sqrt{\frac{2K}{m} (1 - \cos(a k))} = 2 \sqrt{\frac{K}{m}} \sin\left(\frac{1}{2} a k\right)
\end{equation}
We know from the periodic boundary conditions that $k=\frac{2 \pi n}{N a}$ where $n$ is any integer. We plug this in to get:
\begin{equation}
\omega_k = 2 \sqrt{\frac{K}{m}} \sin\left(\frac{\pi n}{N}\right)
\end{equation}
The maximum possible value of the frequency occurs when $\sin \left(\frac{\pi n}{N}\right) = 1$. This gives us:
\begin{equation}
\omega_m = 2 \sqrt{\frac{K}{m}}
\end{equation}
We know the relation:
\begin{equation}
  \frac{\dd{\omega }}{\dd{k}} = \frac{\dd{\omega }}{\dd{n}} \frac{\dd{n}}{\dd{k}} \rightarrow  \dd{n} \left( \frac{\dd{\omega }}{\dd{k} }\right) = \dd{\omega } \left( \frac{\dd{n}}{\dd{k} }\right)
\end{equation}
Solving for $\dd{n}$ we get:
\begin{equation}
  \dd{n} = \dd{\omega } \left( \frac{\dd{n}}{\dd{k} }\right) \left( \frac{\dd{\omega }}{\dd{k} }\right)^{-1}
\end{equation}
\begin{equation}
  \frac{\dd{\omega }}{\dd{n}}= \frac{\dd{\omega }}{\dd{k} }\left( \frac{\dd{n}}{\dd{k} }\right)^{-1}
\end{equation}
Now, from our previous definition of the frequency, we know that:
\begin{equation}
  \frac{\dd{\omega }}{\dd{n}} = \frac{2\pi}{N} \sqrt{\frac{K}{m}} \cos\left(\frac{\pi n}{N}\right) =\omega_{m} \frac{\pi}{N} \cos\left(\frac{\pi n}{N}\right) 
\end{equation}
Now, we also know the expression for the frequency:
\begin{equation}
  \omega =\omega _{m} \sin\left(\frac{\pi n}{N}\right)
\end{equation}
Not fur how to forced from here.

(v) Compare the Debye model and the exact results.
\subsection{}
What we found was that the density of normal mode with the frequencies between $\omega$ and $\omega+d \omega$ is:
\begin{equation}
g(\omega) d \omega=\frac{2 N}{\pi \omega_{m} \sqrt{1-\left(\omega / \omega_{m}\right)^{2}}} d \omega
\end{equation}

(a) If one is to make the Debye model for this 1-D solid, what is its Debye temperature $\Theta_{D}$ ? Why is the Debye frequency $\omega_{D}$ larger than the maximum frequency allowed in the system $\omega_{m}$ ?
\subsection{}
From the notes, we find that:
\begin{equation}
  \Theta_{D} = \frac{\hbar \omega_{D}}{k_B}
\end{equation}
Although I didn't perform this integration, I remember that at one point we made the approximation that the phonon spectrum is continuous and can be approximated by an integral. Not sure if this has something to do with it.

(b) Show that one gets the same dependency of the heat capacity on temperature at very low temperature with the Debye approximation and with the exact degeneracy. Why does the Debye model give an accurate result even  unphysical normal modes are considered?
\subsection{}
When considering the heat capacity at a very low temperature, only low-energy phonons can be occupied. So even though the Debye model might consider frequencies above what is physical, these modes are not populated at low temperatures.


\end{document}