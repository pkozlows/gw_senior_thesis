% Options for packages loaded elsewhere
\PassOptionsToPackage{unicode}{hyperref}
\PassOptionsToPackage{hyphens}{url}
%
\documentclass[
]{article}
\usepackage{amsmath,amssymb}
\usepackage{iftex}
\ifPDFTeX
  \usepackage[T1]{fontenc}
  \usepackage[utf8]{inputenc}
  \usepackage{textcomp} % provide euro and other symbols
\else % if luatex or xetex
  \usepackage{unicode-math} % this also loads fontspec
  \defaultfontfeatures{Scale=MatchLowercase}
  \defaultfontfeatures[\rmfamily]{Ligatures=TeX,Scale=1}
\fi
\usepackage{lmodern}
\ifPDFTeX\else
  % xetex/luatex font selection
\fi
% Use upquote if available, for straight quotes in verbatim environments
\IfFileExists{upquote.sty}{\usepackage{upquote}}{}
\IfFileExists{microtype.sty}{% use microtype if available
  \usepackage[]{microtype}
  \UseMicrotypeSet[protrusion]{basicmath} % disable protrusion for tt fonts
}{}
\makeatletter
\@ifundefined{KOMAClassName}{% if non-KOMA class
  \IfFileExists{parskip.sty}{%
    \usepackage{parskip}
  }{% else
    \setlength{\parindent}{0pt}
    \setlength{\parskip}{6pt plus 2pt minus 1pt}}
}{% if KOMA class
  \KOMAoptions{parskip=half}}
\makeatother
\usepackage{xcolor}
\setlength{\emergencystretch}{3em} % prevent overfull lines
\providecommand{\tightlist}{%
  \setlength{\itemsep}{0pt}\setlength{\parskip}{0pt}}
\setcounter{secnumdepth}{-\maxdimen} % remove section numbering
\ifLuaTeX
  \usepackage{selnolig}  % disable illegal ligatures
\fi
\IfFileExists{bookmark.sty}{\usepackage{bookmark}}{\usepackage{hyperref}}
\IfFileExists{xurl.sty}{\usepackage{xurl}}{} % add URL line breaks if available
\urlstyle{same}
\hypersetup{
  hidelinks,
  pdfcreator={LaTeX via pandoc}}

\author{}
\date{}

\begin{document}

\textbf{Ch/ChE 164 Winter 2024}

\textbf{Homework Problem Set \#3}

Due Date: Thursday Feb 1, 2024 @ 11:59pm

\textbf{1.} (from Chandler 3.22)

(14 pts.) Consider a region within a fluid described by the van der
Waals equation

{[}\beta p=\frac{\rho}{1-b\rho}-\beta a\rho\^{}\{2\}, \tag{1}{]}

where (\rho=\langle N\rangle/V). The volume of the region is
(L\^{}\{3\}). Due to the spontaneous fluctuations in the system, the
instantaneous value of the density in that region can differ from its
average by an amount (\delta\rho). Determine, as a function of (\beta),
(\rho), (a), (b), and (L\^{}\{3\}), the typical relative size of these
fluctuations; that is, evaluate
(\langle(\delta\rho)\textsuperscript{\{2\}\rangle}\{1/2\}/\rho).
Demonstrate that when one considers observations of a macroscopic system
(\emph{i.e.}, the size of the region becomes macroscopic,
(L\^{}\{3\}\rightarrow\infty)) the relative fluctuations become
negligible. 2. \emph{(9 pts.)} A fluid is at its ``critical point'' when
{[}\left(\frac{\partial\beta p}{\partial\rho}\right)\emph{\{\beta\}=\left(\frac{ \partial^{2}\beta p}{\partial\rho^{2}}\right)}\{\beta\}=0.{]}
(2) Determine the critical point density and temperature for a fluid
obeying the van der Waals equation. That is compute (\beta\emph{\{c\})
and (\rho}\{c\}) as a function of (a) and (b). 3. \emph{(9 pts.)} Focus
attention on the subvolume of size (L\^{}\{3\}) in the fluid. Suppose
(L\^{}\{3\}) is 100 times the space filling volume of a molecule--that
is, (L\^{}\{3\}\approx 100b). For this region in the fluid, compute the
relative size of the density fluctuations when (\rho=\rho\emph{\{c\}),
and the temperature is 10\% above the critical temperature. Repeat this
calculation for temperatures 0.1\% and 0.001\% from the critical
temperature. 4. }(8 pts.)\_ Light that we can observe with our eyes has
wavelengths of order of 1000 A. Fluctuations in density cause changes in
the index of refraction, and those changes produce scattering of light.
Therefore, if a region of fluid 1000 A across contains significant
density fluctuations, we will visually observe these fluctuations. On
the basis of the type of calculation performed in part (b), determine
how close to the critical point a system must be before critical
fluctuations become optically observable (that is, when the quantity
(\langle(\delta\rho)\textsuperscript{\{2\}\rangle}\{1/2\}/\rho) is of
order one). The phenomenon of long wavelength density fluctuations in a
fluid approaching the critical point is known as opalescence. (Note: You
will need to estimate the size of (b), and to do this you should note
that the typical diameter of a small molecule is around 5 A).

\textbf{2.}: (i) \emph{(15 points)} Consider a perfect crystal
originally comprised of (N) molecules on an equal number of lattice
sites. The creation of vacancies in the lattice (or equivalently adding
``surface'' sites) is a thermally activated process with Boltzmann
factor, (e\^{}\{-w/kT\}) where (w) is the energy required to bring each
of the molecules to the surface.

If (n) of these (N) molecules move to the surface and leave (n) vacant
lattice sites, show that the partition function is well approximated by

{[}Q(N,T)=\sum\emph{\{n=0\}\textsuperscript{\{\infty\}\frac{(N+n)!}{n!N!},e}\{-nw/kT\},{]}Considering
the thermodynamic limit and that (w) is several (kT), find the maximal
term in the above sum to show that {[}\frac{n}{N+n}=e\^{}\{-w/kT\}.{]}
2. }(15 pts.)\_ Evaluate the above \emph{full} partition function and
then obtain (\langle n\rangle) by suitable differentiation of (\log Q).
(\emph{Hint:} Think of (Q(N,V,T)) as a power series
(Q(N,V,T)=\sum\emph{\{n=0\}\textsuperscript{\{\infty\}a\_\{n\}z}\{n\}).)
Are the results for (\langle n\rangle) from the maximum term and the
full partition function the same? 3. }(30 pts.)\_ Show that the entropy
for the Bose-Einstein and Fermi-Dirac gas can be written in the form
{[}S=-k\sum\emph{\{\alpha=1\}\^{}\{\infty\}{[}\langle n\_\{\alpha\}\rangle\ln\langle n\_\{\alpha\}
\rangle\mp(1\pm\langle n\_\{\alpha\}\rangle)\ln\left(1\pm\langle n\_\{\alpha\}
\rangle\right){]},{]} where (\langle n}\{\alpha\}\rangle) is the average
occupation number for state (\alpha).

\end{document}
