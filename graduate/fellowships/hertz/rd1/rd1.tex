\documentclass[12pt]{article}
\usepackage{amsmath}
\usepackage{physics}
\author{Patryk Kozlowski}
\title{Hertz Round 1}
\date{\today}
\begin{document}
\maketitle
First, they asked me to explain how I got interested in chemistry and my prior research, so kind of like a brief biography. At the end, I mentioned what I was working on in my senior thesis, so I got some follow-up questions about that. I am not able to handwrite, so the interview might have been slightly different for me, but the idea that I got is before the pandemic they were much more focused on asking candidates to perform computations with their scratch paper, but now over Zoom, it is more about reasoning through conceptual questions. First, they asked me about solar cells because I indicated my interest in sustainability and materials science. Prior to the interview, I thought that I should read a little bit more up on how they work, but retrospectively, I am happy that I didn't; they will probe you about a subject until they find something that you don't know about. So, if I knew more about solar cells, I just would have gotten more difficult questions about them. Then, I got asked a simple question about computational scaling, easier than Leetcode. The interviewer guided me along until I answered the question satisfactorily. Then I got a thought experiment about whether the Earth would be warmer with or without the atmosphere. This pivoted to a question about the greenhouse gas effect accompanied by a visual spectroscopic diagram regarding the absorption of sunlight in different regions of the EM spectrum over screen share, but I wasn't able to figure out this one, so the interviewer decided to move on after some time. I got asked a simple practical thermodynamics question; the interviewer wanted to know if I understood the role of surface area in the heat transfer of a material by asking questions about how the liquid in her coffee cup would cool under various circumstances. Lastly, they asked me if I had any questions and I asked them about how the Hertz could help me get involved in climate activism. The interviewer explained that in the past, the Hertz was more focused on national security, but that this has changed. They remarked that the fellowship helps to pay for graduate school, so maybe I might be applying to the wrong place, but then I clarified that I am very much interested in doing the quantum chemistry that I interviewed for and I only want to pursue climate activism/communication on the side, and they were happy with this answer.
\end{document}