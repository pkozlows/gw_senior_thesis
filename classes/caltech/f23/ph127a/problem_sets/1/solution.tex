\documentclass{article}
\usepackage{amsmath}
\usepackage{physics}
\date{\today}
\title{PH127a Problem Set 1}
\author{Patryk Kozlowski}
\begin{document}
\maketitle
\section{Problem 1: Minimal experimental data to recover all thermodynamics}
\subsection{A}
To begin, let's express \(dU\) and \(dS\) in terms of \(dT\) and \(dV\). Given the relationship:
\begin{equation}
T dS = dU + p dV
\end{equation}

We can write \(dU\) as:
\begin{equation}
dU = T dS - p dV
\end{equation}

Now, let's represent \(dU\) in terms of \(dT\) and \(dV\) as follows:
\begin{equation}
dU = \left( \frac{\partial U}{\partial T} \right)_V dT + \left( \frac{\partial U}{\partial V} \right)_T dV
\end{equation}

Equating the two expressions for \(dU\), we get:
\begin{equation}
T dS - p dV = \left( \frac{\partial U}{\partial T} \right)_V dT + \left( \frac{\partial U}{\partial V} \right)_T dV
\end{equation}

Rearranging and expressing \(dS\) in terms of \(dT\) and \(dV\), we find:
\begin{equation}
dS = \left( \frac{\partial U}{\partial T} \right)_V \frac{1}{T} dT + \left( \frac{\partial U}{\partial V} \right)_T \frac{1}{T} dV + p \frac{1}{T} dV
\end{equation}
\begin{equation}
dS = \left( \frac{\partial U}{\partial T} \right)_V \frac{1}{T} dT + \left( \left( \frac{\partial U}{\partial V} \right)_T + p \right) \frac{1}{T} dV
\end{equation}

Now, we can write the partial derivatives of \(S\) with respect to \(T\) and \(V\) when the opposite perimeter is fixed:
\begin{equation}
\left( \frac{\partial S}{\partial T} \right)_V = \left( \frac{\partial U}{\partial T} \right)_V \frac{1}{T}
\end{equation}
\begin{equation}
\left( \frac{\partial S}{\partial V} \right)_T = \left( \left( \frac{\partial U}{\partial V} \right)_T + p \right) \frac{1}{T}
\end{equation}
Now we consider the mixed derivatives. Since the order of the ventilation does not matter, they are equal:
\begin{equation}
\left( \frac{\partial}{\partial V} \left( \frac{\partial S}{\partial T} \right) \right) = \left( \frac{\partial}{\partial T} \left( \frac{\partial S}{\partial V} \right) \right)
\end{equation}
First, we consider the left and side:
\begin{equation}
\left( \frac{\partial}{\partial V} \left( \frac{\partial S}{\partial T} \right) \right) = \frac{1}{T} \left( \frac{\partial}{\partial V} \left( \left( \frac{\partial U}{\partial T} \right) \right) \right)
\end{equation}
Next, we consider the right hand side:
\begin{equation}
\left( \frac{\partial}{\partial T} \left( \frac{\partial S}{\partial V} \right)_T \right)_V = \left( \frac{\partial}{\partial T} \left( \frac{1}{T} \left( \left( \frac{\partial U}{\partial V} \right)_T + p \right) \right) \right)_V
\end{equation}
Differentiating with respect to $T$, we get:
\begin{equation}
    \left( \frac{\partial^2 S}{\partial V \partial T} \right) = \frac{1}{T} \left( \frac{\partial^2 U}{\partial V \partial T} \right) - \frac{1}{T^2} \left( \left( \frac{\partial U}{\partial V} \right)_T  + p \right) + \frac{1}{T} \left( \frac{\partial p}{\partial T} \right)_V
\end{equation}
Using the fact that the mixed derivatives are equal to each other:
\begin{equation}
    \frac{1}{T} \left( \frac{\partial^2 U}{\ V \partial T}\right) = \frac{1}{T} \left( \frac{\partial^2 U}{\partial V \partial T} \right) - \frac{1}{T^2} \left( \left( \frac{\partial U}{\partial V} \right)_T  + p \right) + \frac{1}{T} \left( \frac{\partial p}{\partial T} \right)_V
\end{equation}
We simplify and by multiplying through by $T^2$ to get the identity that we want, which is:
\begin{equation}
    \left( \frac{\partial U}{\partial V} \right)_T = T \left( \frac{\partial p}{\partial T} \right)_V - p
\end{equation}
So:
\begin{equation}
    dU = C_{V} + \left( T \left( \frac{\partial p}{\partial T} \right)_V - p \right) dV
\end{equation}
Integrating, we get that:
\begin{equation}
    U(T_{2}, V_{2}) - U(T_{1}, V_{1}) = \int_{V_{1}}^{V_{2}} \left( C_{V} + \left( T \left( \frac{\partial p}{\partial T} \right)_V - p \right) dV \right)
\end{equation}
Similarly, for the entropy:
\begin{equation}
   T dS = C_{V} + \left( T \left( \frac{\partial p}{\partial T} \right)_V - p \right) dV + p dV 
\end{equation}
Isolating the derivative of entropy and simplifying, we get:
\begin{equation}
    dS = \frac{C_{V}}{T} + \left( \left( \frac{\partial p}{\partial T} \right)_V \right) dV
\end{equation}
Now, integrating:
\begin{equation}
    S(T_{2}, V_{2}) - S(T_{1}, V_{1}) = \int_{V_{1}}^{V_{2}} \left( \frac{C_{V}}{T} + \left( \left( \frac{\partial p}{\partial T} \right)_V \right) dV \right)
\end{equation}
\subsection{B}
We want to take the temperature derivative of
\begin{equation}
    \left( \frac{\partial U}{\partial V} \right)_T = T \left( \frac{\partial p}{\partial T} \right)_V - p
\end{equation}
We can write this as:
\begin{equation}
    \frac{\partial }{\partial T} \left( \frac{\partial U}{\partial V} \right)_T = \frac{\partial }{\partial T} \left( T \left( \frac{\partial p}{\partial T} \right)_V - p \right)= \left( \frac{\partial p}{\partial T} \right)_V + T \left( \frac{\partial^2 p}{\partial T^2} \right)_V - \left( \frac{\partial p}{\partial T} \right)_V 
\end{equation}
So, we get:
\begin{equation}
    \left(\frac{\partial C_{V}}{\partial V} \right)_T = \frac{\partial }{\partial T} \left( \frac{\partial U}{\partial V} \right)_T = T \left( \frac{\partial^2 p}{\partial T^2} \right)_V
\end{equation}
Similarly, we can express this as in integral:
\begin{equation}
    C_{V}(V, T) - C_{V}(V_{0},T) = \int_{V_{0}}^{V} \left( T \left( \frac{\partial^2 p}{\partial T^2} \right)_V \right) dV
\end{equation}
\subsection{C}
We will begin by calculating the specific heat according to:
\begin{equation}
    C_{V}(V, T) = C_{V}(V_{0},T) + \int_{V_{0}}^{V} \left( T \left( \frac{\partial^2 p}{\partial T^2} \right)_V \right) dV
\end{equation}
We are using the equation of state for the van der Waals gas and reference specific heat for the ideal gas:
\begin{equation}
    C_{V}(V, T) = \frac{3}{2} N k_{B} + \int_{V_{0}}^{V} \left( T \left( \frac{\partial^2 p}{\partial T^2} \right)_V \right) dV
\end{equation}
Solving for the pressure in the vander wall equation of state:
\begin{equation}
    p = \frac{N k_{B} T}{V - N b} - \frac{a N^2}{V^2}
\end{equation}
We want to find the second derivative of the pressure with respect to temperature at a fixed volume. First, we start by taking one reverie:
\begin{equation}
    \left( \frac{\partial p}{\partial T} \right)_V = \frac{N k_{B}}{V - N b}    
\end{equation}
Next, we take the derivative again:
\begin{equation}
    \left( \frac{\partial^2 p}{\partial T^2} \right)_V = -\frac{N k_{B}}{(V - N b)^2}
\end{equation}
Now the second derivative of pressure with respect to temperature is:
\begin{equation}
    \left( \frac{\partial^2 p}{\partial T^2} \right)_V = \frac{N k_{B}}{V - N b} \left( \frac{a}{k_{B} T} \right)^2
\end{equation}
\section{Problem 2: Thermodynamics of a model classical paramagnet}
First we want to compute the partition function of our system. Defining $\beta = \frac{1}{k_{B} T}$, we can write the partition function as:
\begin{equation}
    Z = \int e^{-\beta E} d \Omega
\end{equation}
where we have to find the deferential solid angle $d \Omega = sin \theta d \theta d \phi$.
Given that we have the following expression for the energy:
\begin{equation}
    E = E[\{\vec{m}_i\}] = -\sum_{i=1}^{N} \vec{m}_i \cdot \vec{B}
\end{equation}
Gevent that  $\mathbf{m} = \mu(\sin \theta \cos \phi, \sin \theta \sin \phi, \cos \theta)$ and $\mathbf{B} = (0, 0, B)$
We can write the energy as:
\begin{equation}
    E = -\mu B \sum_{i=1}^{N} \cos \theta_{i}
\end{equation}
So in spherical coordinates the partition some integral becomes:
\begin{equation}
    Z = \int_{0}^{2 \pi} \int_{0}^{\pi} e^{\beta \mu B \sum_{i=1}^{N} \cos \theta_{i}} \sin \theta_{i} d \theta_{i} d \phi_{i} = 2 \pi \int_{0}^{\pi} e^{\beta \mu B \sum_{i=1}^{N} \cos \theta_{i}} \sin \theta_{i} d \theta_{i}
\end{equation}
We can bring the summation down from the exponent and convert it into a product:
\begin{equation}
    Z = 2 \pi \int_{0}^{\pi} \prod_{i=1}^{N} e^{\beta \mu B \cos \theta_{i}} \sin \theta_{i} d \theta_{i} = 2 \pi \prod_{i=1}^{N} \int_{0}^{\pi} e^{\beta \mu B \cos \theta_{i}} \sin \theta_{i} d \theta_{i} = 2 \pi \left( \int_{0}^{\pi} e^{\beta \mu B \cos \theta} \sin \theta d \theta \right)^{N}
\end{equation}
Recognizing this integral as a vessel function with $x = \beta \mu B$:
\begin{equation}
    I (x) = \int_{0}^{\pi} e^{x \cos \theta} \sin \theta d \theta 
\end{equation}


% \subsection{A}
% We start with the two relations:
% \begin{equation}
%     d U = \left(\frac{\partial U}{\partial T}\right)_V d T + \left(\frac{\partial U}{\partial V}\right)_T d V
% \end{equation}
% and
% \begin{equation}
%     d S = \left(\frac{\partial S}{\partial T}\right)_V d T + \left(\frac{\partial S}{\partial V}\right)_T d V
% \end{equation}
% Substituting into the main thermo dynamics equation and solving with the derivative with respect to temperature at fixed volume:
% \begin{equation}
%     \left(\frac{\partial S}{\partial T}\right)_V = \frac{1}{T} \left(\frac{\partial U}{\partial T}\right)_V + \frac{p}{T} \left(\frac{\partial V}{\partial T}\right)_V
% \end{equation}

% When the volume is fixed, we get:
% \begin{equation}
%     \left(\frac{\partial S}{\partial T}\right)_V = \frac{1}{T} \left(\frac{\partial U}{\partial T}\right)_V
% \end{equation}
% When the temperature is fixed, we get:
% \begin{equation}
%     \left(\frac{\partial S}{\partial V}\right)_T = \frac{p}{T}  
% \end{equation}
% Now we take the mixed derivatives. We start with the second derivative of entropy with respect to temperature first and then volume:
% \begin{equation}
%     \frac{\partial}{\partial V} \left(\frac{\partial S}{\partial T}\right) = \frac{\partial}{\partial V} \left(\frac{1}{T} \left(\frac{\partial U}{\partial T}\right)_V\right) = \frac{1}{T} \left(\frac{\partial}{\partial V} \left(\frac{\partial U}{\partial T}\right)_V\right) + \frac{1}{T} \left(\frac{\partial U}{\partial T}\right)_V \left(\frac{\partial}{\partial V} \frac{1}{T}\right)
% \end{equation}
% The first term is 0 because the derivative of the internal energy with respect to temperature at fixed volume is a function of temperature only. The second term is:
% \begin{equation}
%     \frac{1}{T} \left(\frac{\partial U}{\partial T}\right)_V \left(\frac{\partial}{\partial V} \frac{1}{T}\right) = -\frac{1}{T^2} \left(\frac{\partial U}{\partial T}\right)_V \left(\frac{\partial T}{\partial V}\right)
% \end{equation}
% Now we take the other mixed derivative:
% \begin{equation}
%     \frac{\partial}{\partial T} \left(\frac{\partial S}{\partial V}\right) = \frac{\partial}{\partial T} \left(\frac{p}{T}\right) = -\frac{p}{T^2} + \frac{1}{T} \left(\frac{\partial p}{\partial T}\right)_V
% \end{equation}
\end{document}