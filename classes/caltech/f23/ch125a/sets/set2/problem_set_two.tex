\documentclass{article}
\usepackage{amsmath, amssymb}
\usepackage{physics}


\begin{document}

\section{Problem 1: Adding special relativity to the Schr\"{o}dinger equation}
\subsection{}

% Adding special relativity to the Schr\"{o}dinger equation can be achieved from quantizing a classical electromagnetic wave equation.

% \begin{enumerate}
%     \item Start with
%     \[
%     \frac{\partial^2 E}{\partial x^2} - \frac{1}{c^2} \frac{\partial^2 E}{\partial t^2} = 0,
%     \]
%     the plane wave solution
%     \[
%     E(x, t) = E_0 \exp(i(kx - \omega t)),
%     \]
%     and the relations
%     \[
%     \mathcal{E} = \hbar\omega \quad \text{and} \quad p = \hbar k.
%     \]
%     Prove that
%     \[
%     \mathcal{E}^2 = p^2c^2
%     \]
%     for a photon.
We start with the original deferential equation:
\begin{equation}
    \frac{\partial^2 E}{\partial x^2} - \frac{1}{c^2} \frac{\partial^2 E}{\partial t^2} = 0
\end{equation}
first, we consider the plain wave solution:
\begin{equation}
    E(x, t) = E_0 \exp(i(kx - \omega t))
\end{equation}
We take the second partial derivative with respect to $x$:
\begin{equation}
    \frac{\partial^2 E}{\partial x^2} = -k^2 E(x, t)
\end{equation}

We then take the second partial derivative with respect to $t$:
\begin{equation}
    \frac{\partial^2 E}{\partial t^2} = -\omega^2 E(x, t)
\end{equation}

We then plug in the second partial derivatives into the original differential equation:
\begin{equation}
    -k^2 E(x, t) - \frac{1}{c^2} (-\omega^2 E(x, t)) = 0
\end{equation}
Dividing through by the electric field:
\begin{equation}
    -k^2 - \frac{1}{c^2} (-\omega^2) = 0
\end{equation}
We then multiply through by $c^2$ and bring the $k^2$ term to the right hand side:
\begin{equation}
    \omega^2 = c^2 k^2
\end{equation}
Now, the relations are defined as:
\begin{equation}
    \mathcal{E} = \hbar\omega \quad \text{and} \quad p = \hbar k
\end{equation}
So, we have:
\begin{equation}
    \mathcal{E}^2 = \hbar^2 \omega^2 \rightarrow \omega^2 = \frac{\mathcal{E}^2}{\hbar^2}
\end{equation}
Next, we also have:
\begin{equation}
    p^2 = \hbar^2 k^2 \rightarrow k^2 = \frac{p^2}{\hbar^2}
\end{equation}
We then plug in the relations into the original equation:
\begin{equation}
    \omega^2 = c^2 k^2 \rightarrow \frac{\mathcal{E}^2}{\hbar^2} = c^2 \frac{p^2}{\hbar^2}
\end{equation}
We then multiply through by $\hbar^2$:
\begin{equation}
    \boxed{\mathcal{E}^2 = c^2 p^2}
\end{equation}
\subsection{}
We start by using again the relations:
\begin{equation}
    \mathcal{E} = \hbar\omega \quad \text{and} \quad p = \hbar k
\end{equation}
We then plug in the relations into the original equation:
\begin{equation}
    \mathcal{E}^2 = p^2c^2 + m^2c^4 \rightarrow \hbar^2 \omega^2 = \hbar^2 k^2 c^2 + m^2c^4
\end{equation}
We then divide through by $\hbar^2$:
\begin{equation}
    \omega^2 = k^2 c^2 + \frac{m^2c^4}{\hbar^2}
\end{equation}
Multiplying through by negative $\Psi$:
\begin{equation}
    -\omega^2 \Psi = -k^2 c^2 \Psi - \frac{m^2c^4}{\hbar^2} \Psi
\end{equation}
Again, the plain wave solution is defined as:
\begin{equation}
    \Psi(x, t) = \Psi_0 \exp(i(kx - \omega t))
\end{equation}
So, the Laplacian for the plan wave solution is:
\begin{equation}
    \nabla^2 \Psi = -k^2 \Psi(x, t)
\end{equation}
Similarly the second derivative with respect to time of the plane wave solution is:
\begin{equation}
    \frac{\partial^2 \Psi}{\partial t^2} = -\omega^2 \Psi(x, t)
\end{equation}
Recognizing the right hand sites in our equation 16, we plug in to get:
\begin{equation}
    \frac{\partial^2 \Psi}{\partial t^2} = c^{2}\nabla^2 \Psi - \frac{m^2c^4}{\hbar^2} \Psi
\end{equation}
dividing through by the sweet of light squared:
\begin{equation}
    \boxed{\frac{1}{c^2} \frac{\partial^2 \Psi}{\partial t^2} = \nabla^2 \Psi - \frac{m^2c^2}{\hbar^2} \Psi}
\end{equation}
\section{Problem 2}

This problem is a good practice on Dirac notation. The math here is nothing but simple addition / multiplication, but when tied into Dirac notation, it adds a level of hidden sub-text that is confusing.

The Hermitian operator \( H \) acts in a two-dimensional space with orthonormal basis vectors \( |1\rangle \) and \( |2\rangle \). The matrix elements are
\[
\begin{pmatrix}
\langle 1|H|1\rangle & \langle 1|H|2\rangle \\
\langle 2|H|1\rangle & \langle 2|H|2\rangle \\
\end{pmatrix}
=
\begin{pmatrix}
3 & 4 \\
4 & -3 \\
\end{pmatrix}
\tag{1}
\]
The eigenvalues are \( 5 \) and \( -5 \). The column vectors representation of the eigenvalues \( |A\rangle \) and \( |B\rangle \) is
\[
\begin{pmatrix}
\langle 1|A\rangle \\
\langle 2|A\rangle \\
\end{pmatrix}
= \frac{1}{\sqrt{5}}
\begin{pmatrix}
2 \\
1 \\
\end{pmatrix}
\]
and
\[
\begin{pmatrix}
\langle 1|B\rangle \\
\langle 2|B\rangle \\
\end{pmatrix}
= \frac{1}{\sqrt{5}}
\begin{pmatrix}
-1 \\
2 \\
\end{pmatrix}
\tag{2}
\]
\( H \) can be diagonalized by a unitary operator \( U \) (with \( U^\dagger U = I \)), i.e. \( U^\dagger HU = D \) where
\[
\begin{pmatrix}
\langle 1|U|1\rangle & \langle 1|U|2\rangle \\
\langle 2|U|1\rangle & \langle 2|U|2\rangle \\
\end{pmatrix}
= \frac{1}{\sqrt{5}}
\begin{pmatrix}
2 & -1 \\
1 & 2 \\
\end{pmatrix}
\tag{3}
\]
and
\[
\begin{pmatrix}
\langle 1|D|1\rangle & \langle 1|D|2\rangle \\
\langle 2|D|1\rangle & \langle 2|D|2\rangle \\
\end{pmatrix}
= 
\begin{pmatrix}
5 & 0 \\
0 & -5 \\
\end{pmatrix}
\tag{4}
\]
\subsection{Show that the column vectors in (2) are the eigenvectors of (1).}
We start by plugging in the column vectors one at a time into the matrix in (1):
\begin{equation}
    \begin{pmatrix}
        3 & 4 \\
        4 & -3 \\
    \end{pmatrix}
\frac{1}{\sqrt{5}}
    \begin{pmatrix}
        2 \\
        1 \\
    \end{pmatrix}
    =
\frac{1}{\sqrt{5}}
    \begin{pmatrix}
        10 \\
        5 \\
    \end{pmatrix}
=
5 \left(\frac{1}{\sqrt{5}}
    \begin{pmatrix}
        2 \\
        1 \\
    \end{pmatrix}
\right)
\end{equation}
Next, we plug in the second column vector:
\begin{equation}
    \begin{pmatrix}
        3 & 4 \\
        4 & -3 \\
    \end{pmatrix}
\frac{1}{\sqrt{5}}
    \begin{pmatrix}
        -1 \\
        2 \\
    \end{pmatrix}
    =
\frac{1}{\sqrt{5}}
    \begin{pmatrix}
        5 \\
        -10 \\
    \end{pmatrix}
=
-5 \left(\frac{1}{\sqrt{5}}
    \begin{pmatrix}
        -1 \\
        2 \\
    \end{pmatrix}
\right)
\end{equation}
So, they are eigen vectors with eigen values of 5 and negative 5, respectively.
\subsection{Show that \( U^\dagger HU = D \). If we think of our kets as unit vectors, what would this operation physically represent? As in, what if \( H \) was initially \( x \)-hat, and \( U \) made it \( y \)-hat.}
We start by carrying out the matrix multiplication:
\begin{equation}
\frac{1}{\sqrt{5}}
    \begin{pmatrix}
        2 & 1 \\
        -1 & 2 \\
    \end{pmatrix}
    \begin{pmatrix}
        3 & 4 \\
        4 & -3 \\
    \end{pmatrix}
    \frac{1}{\sqrt{5}}
    \begin{pmatrix}
        2 & -1 \\
        1 & 2 \\
    \end{pmatrix}
\end{equation}
Consolidating the constants and carrying out the right-hand side matrix multiplication first, we simplify to:
\begin{equation}
    \frac{1}{5}
    \begin{pmatrix}
        2 & 1 \\
        -1 & 2 \\
    \end{pmatrix}
    \begin{pmatrix}
        10 & 5 \\
        5 & -10 \\
    \end{pmatrix}
=
    \frac{1}{5}
    \begin{pmatrix}
        25 & 0 \\
        0 & -25 \\
    \end{pmatrix}
=
    \begin{pmatrix}
        5 & 0 \\
        0 & -5 \\
    \end{pmatrix}
\end{equation}
So, we have shown that \( U^\dagger HU = D \). If we think of our kets as unit vectors, this operation would represent a rotation of basis. The choice of \(x\)-hat and \(y\)-hat is typically used for column vectors, but the same idea applies. If \( H \) was initially \( x \)-hat, and \( U \) made it \( y \)-hat, then the matrix \( U \) would be a rotation matrix.

3. Since \( H = UDU^{\dagger} \), it also follows that \( H^2 = UD^2U^{\dagger} \) and in general that \( H^n = UD^nU^{\dagger} \). The exponential of \( H \) is therefore given by
\begin{align*}
e^H &= \sum_{n=0}^{\infty} \frac{1}{n!} H^n \\
&= U \left[ \sum_{n=0}^{\infty} \frac{1}{n!} D^n \right] U^{\dagger} \\
&= U e^D U^{\dagger} \\
&= U 
\begin{pmatrix}
e^5 & 0 \\
0 & e^{-5}
\end{pmatrix}
U^{\dagger}
\end{align*}
\subsection{Perform the matrix multiplication on the above right to obtain the values of the four matrix elements of \( e^H \) in the \( \ket{1} \), \( \ket{2} \) basis}

First, we will perform the matrix multiplication on the left hand side:
\begin{equation}
    U
    \begin{pmatrix}
        e^5 & 0 \\
        0 & e^{-5} \\
    \end{pmatrix}
    U^{\dagger}
=
    \frac{1}{\sqrt{5}}
    \begin{pmatrix}
        2 & -1 \\
        1 & 2 \\
    \end{pmatrix}
    \begin{pmatrix}
        e^5 & 0 \\
        0 & e^{-5} \\
    \end{pmatrix}
    U^{\dagger}
=
    \frac{1}{\sqrt{5}}
    \begin{pmatrix}
        2e^5 & -e^{-5} \\
        e^5 & 2e^{-5} \\
    \end{pmatrix}
    U^{\dagger}
\end{equation}
next we plug in four $U^{\dagger}$:
\begin{equation}
    \frac{1}{5}
    \begin{pmatrix}
        2e^5 & -e^{-5} \\
        e^5 & 2e^{-5} \\
    \end{pmatrix}
    \begin{pmatrix}
        2 & 1 \\
        -1 & 2 \\
    \end{pmatrix}
= \boxed{\frac{1}{5}
    \begin{pmatrix}
        4e^5 + e^{-5} & 2e^5 - 2e^{-5} \\
        2e^5 - 2e^{-5} & e^5 + 4e^{-5} \\
    \end{pmatrix}}
\end{equation}
\subsection{compute the four matrix elements of \( e^H \) in the \( \ket{1} \), \( \ket{2} \) basis to show it is the same as above.}

We start with the first aliment, which is:
\begin{equation}
    \bra{1}e^H\ket{1} = e^5 \bra{1}\ket{A}\bra{A}\ket{1} + e^{-5} \bra{1}\ket{B}\bra{B}\ket{1}
\end{equation}
\begin{equation}
    = \frac{4}{5}e^5 + \frac{1}{5}e^{-5}
\end{equation}
Next, we have:
\begin{equation}
    \bra{1}e^H\ket{2} = e^5 \bra{1}\ket{A}\bra{A}\ket{2} + e^{-5} \bra{1}\ket{B}\bra{B}\ket{2}
\end{equation}
\begin{equation}
    = \frac{2}{5}e^5 - \frac{2}{5}e^{-5}
\end{equation}
Next, we have:
\begin{equation}
    \bra{2}e^H\ket{1} = e^5 \bra{2}\ket{A}\bra{A}\ket{1} + e^{-5} \bra{2}\ket{B}\bra{B}\ket{1}
\end{equation}
\begin{equation}
    = \frac{2}{5}e^5 - \frac{2}{5}e^{-5}
\end{equation}
Finally, we have:
\begin{equation}
    \bra{2}e^H\ket{2} = e^5 \bra{2}\ket{A}\bra{A}\ket{2} + e^{-5} \bra{2}\ket{B}\bra{B}\ket{2}
\end{equation}
\begin{equation}
    = \frac{1}{5}e^5 + \frac{4}{5}e^{-5}
\end{equation}
Taking out a factor of $\frac{1}{5}$ and consolidating everything into a matrix format:
\begin{equation}
\boxed{\frac{1}{5}
    \begin{pmatrix}
        4e^5 + e^{-5} & 2e^5 - 2e^{-5} \\
        2e^5 - 2e^{-5} & e^5 + 4e^{-5} \\
    \end{pmatrix}}\end{equation}
This is the same as the matrix we got in part 3.




% The relations are defined as:
% \begin{equation}
%     \mathcal{E} = \hbar\omega \quad \text{and} \quad p = \hbar k
% \end{equation}

\end{document}

