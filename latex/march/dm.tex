\documentclass[12pt]{article}
\usepackage{amsmath}
\usepackage{amssymb}
\usepackage{graphicx}
\usepackage{physics}
\usepackage{hyperref}

\usepackage{listings} % Required for insertion of code
\usepackage{xcolor} % Required for custom colors

% Define custom colors
\definecolor{codegreen}{rgb}{0,0.6,0}
\definecolor{codegray}{rgb}{0.5,0.5,0.5}
\definecolor{codepurple}{rgb}{0.58,0,0.82}
\definecolor{backcolour}{rgb}{0.95,0.95,0.92}

% Setup the style for code listings
\lstdefinestyle{mystyle}{
    backgroundcolor=\color{backcolour},   
    commentstyle=\color{codegreen},
    keywordstyle=\color{magenta},
    numberstyle=\tiny\color{codegray},
    stringstyle=\color{codepurple},
    basicstyle=\ttfamily\footnotesize,
    breakatwhitespace=false,         
    breaklines=true,                 
    captionpos=b,                    
    keepspaces=true,                 
    numbers=left,                    
    numbersep=5pt,                  
    showspaces=false,                
    showstringspaces=false,
    showtabs=false,                  
    tabsize=2
}

% Activate the style
\lstset{style=mystyle}



\title{Linearized $G_0W_0$ Density Matrix}
\author{Patryk Kozlowski}
\date{\today}
\begin{document}
\maketitle
We have the equation for the density matrix:
\begin{equation}
\begin{aligned}
\gamma^\sigma\left(\mathbf{r}_1, \mathbf{r}_2\right)= & \gamma_0^\sigma\left(\mathbf{r}_1, \mathbf{r}_2\right) -\frac{\mathrm{i}}{2 \pi} \int \mathrm{d} \mathbf{r}_3 \mathrm{~d} \mathbf{r}_4 \mathrm{~d} \omega \mathrm{e}^{\mathrm{i \omega \eta}} G_0^\sigma\left(\mathbf{r}_1, \mathbf{r}_3, \omega\right) \Sigma_c^\sigma\left(\mathbf{r}_3, \mathbf{r}_4, \omega\right) G_0^\sigma\left(\mathbf{r}_4, \mathbf{r}_2, \omega\right)
\label{eqn:init_dm}
\end{aligned}
\end{equation}
In order to simplify the integral, Let us consider
\begin{equation}
I = \int \mathrm{d} \mathbf{r}_3 \mathrm{~d} \mathbf{r}_4  G_0^\sigma\left(\mathbf{r}_1, \mathbf{r}_3\right) \Sigma_c^\sigma\left(\mathbf{r}_3, \mathbf{r}_4\right) G_0^\sigma\left(\mathbf{r}_4, \mathbf{r}_2\right)
\end{equation}
The noninteracting Green's function is defined as:
\begin{equation}
G_0\left(\mathbf{r}_1, \mathbf{r}_2, \right) = \sum_{pq} \phi_p^*(\mathbf{r}_1) G_{p q} \phi_q(\mathbf{r}_2)
\end{equation}
and likewise for the self-energy:
\begin{equation}
\Sigma_c\left(\mathbf{r}_1, \mathbf{r}_2\right) = \sum_{pq} \phi_p^*(\mathbf{r}_1) \Sigma_{ p q} \phi_q(\mathbf{r}_2)
\end{equation}
where $G_{p q}$ and $\Sigma_{p q}$ are the matrix elements of the noninteracting Green's function and the self-energy, respectively. We can rewrite the integral as:
\begin{equation}
I = \sum_{pq} \sum_{rs} \sum_{tu} \int \mathrm{d} \mathbf{r}_3 \mathrm{~d} \mathbf{r}_4 \phi_p^*(\mathbf{r}_1) G_{p q} \phi_q(\mathbf{r}_3) \phi_r^*(\mathbf{r}_3) \Sigma_{r s} \phi_s(\mathbf{r}_4) \phi_t^*(\mathbf{r}_4) G_{t u} \phi_u(\mathbf{r}_2)
\end{equation}
We can simplify this expression by using the orthonormality of the basis functions:
\begin{equation}
\int \mathrm{d} \mathbf{r} \phi_q^*(\mathbf{r}) \phi_r(\mathbf{r}) = \delta_{qr}
\end{equation}
So we can simplify the expression to:
\begin{equation}
I = \sum_{pq} \sum_{r} \sum_{t} \phi_p^*(\mathbf{r}_1) G_{p r} \phi_r(\mathbf{r}) \phi_r^*(\mathbf{r}) \Sigma_{r t} \phi_t(\mathbf{r}^\prime) \phi_t^*(\mathbf{r}^\prime) G_{t q} \phi_q(\mathbf{r}_2)
\end{equation}
We use this and then also rewrite equation \ref{eqn:init_dm} in terms of the matrix elements of the density matrix
with the following definition:
\begin{equation}
D_{p q \sigma}=\left\langle p \sigma\left|\gamma^\sigma\right| q \sigma\right\rangle
\end{equation}
By the derivation above, we can rewrite equation \ref{eqn:init_dm} as:
\begin{equation}
    D_{p q \sigma}=\bra{p\sigma } \gamma _0^\sigma \ket{q\sigma } -\frac{\mathrm{i }}{2 \pi} \sum_{r} \sum_{t} \int_{-\infty }^{\infty } \mathrm{d} \omega \mathrm{e}^{\mathrm{i \omega \eta}} \bra{p\sigma } G_0^\sigma (\omega) \ket{r\sigma } \bra{r\sigma } \Sigma_c^\sigma (\omega) \ket{t\sigma } \bra{t\sigma } G_0^\sigma (\omega) \ket{q\sigma }
\label{eqn:mo_dm}
\end{equation}
% We want to convince ourselves that one can move from equation \ref{eqn:init_dm} to \ref{eqn:mo_dm}. Start by considering the definition of the noninteracting Green's function:
% \begin{equation}
% G_{0}(\mathbf{r}_1, \mathbf{r}_2, \omega) = \sum_{pq} \phi_p^*(\mathbf{r}_1) G_{p q}(\omega) \phi_q(\mathbf{r}_2)
% \end{equation}
% Let us first consider the simple case of:
% \begin{equation}
% \int \mathrm{d} \mathbf{r}_3 G_0(\mathbf{r}_1, \mathbf{r}_3)G_0(\mathbf{r}_3, \mathbf{r}_2) = \sum_{pq} \sum_{rs} \int \mathrm{d} \mathbf{r}_3 \phi_p^*(\mathbf{r}_1) G_{p q}(\omega) \phi_q(\mathbf{r}_3) \phi_r^*(\mathbf{r}_3) G_{r s}(\omega) \phi_s(\mathbf{r}_2)
% \end{equation}
% The more complicated case of the transition from equation \ref{eqn:init_dm} to \ref{eqn:mo_dm} can be generalized from this.
% By orthonormality of the basis, we know that $\int \mathrm{d} \mathbf{r}_3 \phi_q(\mathbf{r}_3) \phi_r^*(\mathbf{r}_3) = \delta_{qr}$, so we can simplify the expression to:
% \begin{equation}
% \int \mathrm{d} \mathbf{r}_3 G_0(\mathbf{r}_1, \mathbf{r}_3)G_0(\mathbf{r}_3, \mathbf{r}_2) = \sum_{ps} \sum_{q} \phi_p^*(\mathbf{r}_1) G_{p q}(\omega) \phi_q(\mathbf{r}_2) \phi_q^*(\mathbf{r}_2) G_{q s}(\omega) \phi_s(\mathbf{r}_2)
% \end{equation}
% which in the bra-ket notation is:
% \begin{equation}
% \sum_{ps}\sum_{q}\bra{p} G_0 (\omega) \ket{q} \bra{q} G_0 (\omega) \ket{s}
% \end{equation}
% So we can see that the transition from equation \ref{eqn:init_dm} to \ref{eqn:mo_dm} is valid with the only integral remaining as the one over frequency.
Next, we plug in the following definitions into equation \ref{eqn:mo_dm}:
\begin{equation}
G_{0 p q}^\sigma=\sum_i \frac{\delta_{p q} \delta_{p i}}{\omega-\epsilon_{i \sigma}-\mathrm{i} \eta}+\sum_a \frac{\delta_{p q} \delta_{p a}}{\omega-\epsilon_{a \sigma}+\mathrm{i} \eta}
\end{equation}
and
\begin{equation}
\begin{aligned}
\Sigma_{c p q}^\sigma(\omega)= & \sum_{i s} \frac{w_{p i \sigma}^s w_{q i \sigma}^s}{\omega-\epsilon_{i \sigma}+\Omega_s-\mathrm{i} \eta} +\sum_{a s} \frac{w_{p a \sigma}^s w_{q a \sigma}^s}{\omega-\epsilon_{a \sigma}-\Omega_s+\mathrm{i} \eta}
\end{aligned}
\end{equation}
Plugging in these definitions, we get:
\begin{equation}
\begin{aligned}
D_{p q \sigma}= & \bra{p\sigma } \gamma _0^\sigma \ket{q\sigma } -\frac{\mathrm{i }}{2 \pi} \sum_{r} \sum_{t} \int_{-\infty }^{\infty } \mathrm{d} \omega \mathrm{e}^{\mathrm{i \omega \eta}} \left( \sum_i \frac{\delta_{p r} \delta_{p i}}{\omega-\epsilon_{i \sigma}-\mathrm{i} \eta}+\sum_a \frac{\delta_{p r} \delta_{p a}}{\omega-\epsilon_{a \sigma}+\mathrm{i} \eta} \right)\\
& \left( \sum_{k s} \frac{w_{r k \sigma}^s w_{t k \sigma}^s}{\omega-\epsilon_{k \sigma}+\Omega_s-\mathrm{i} \eta} +\sum_{c s} \frac{w_{r c \sigma}^s w_{t c \sigma}^s}{\omega-\epsilon_{c \sigma}-\Omega_s+\mathrm{i} \eta} \right) \left( \sum_j \frac{\delta_{t q} \delta_{t j}}{\omega-\epsilon_{j \sigma}-\mathrm{i} \eta}+\sum_b \frac{\delta_{t q} \delta_{t b}}{\omega-\epsilon_{b \sigma}+\mathrm{i} \eta} \right)
\end{aligned}
\end{equation}
Let us just distribute the integral, which technically spawns 8 terms. Also note that the delta functions will get rid of the sums over r and t:
\begin{equation}
\begin{aligned}
I = \int_{-\infty }^{\infty }\mathrm{d} \omega \mathrm{e}^{\mathrm{i \omega \eta}} 
& \left( \sum_{ijks} \left( \frac{w_{i k \sigma}^s w_{j k \sigma}^s}{(\omega-\epsilon_{k \sigma}+\Omega_s-\mathrm{i} \eta)(\omega-\epsilon_{i \sigma}-\mathrm{i} \eta)(\omega-\epsilon_{j \sigma}-\mathrm{i} \eta)} \right) \right.\\
& \left. + \sum_{ibks} \left( \frac{w_{i k \sigma}^s w_{b k \sigma}^s}{(\omega-\epsilon_{k \sigma}+\Omega_s-\mathrm{i} \eta)(\omega-\epsilon_{i \sigma}-\mathrm{i} \eta)(\omega-\epsilon_{b \sigma}+\mathrm{i} \eta)} \right) \right.\\
& \left. + \sum_{ijcs} \left( \frac{w_{i c \sigma}^s w_{j c \sigma}^s}{(\omega-\epsilon_{c \sigma}-\Omega_s+\mathrm{i} \eta)(\omega-\epsilon_{i \sigma}-\mathrm{i} \eta)(\omega-\epsilon_{j \sigma}-\mathrm{i} \eta)} \right) \right.\\
& \left. + \sum_{ibcs} \left( \frac{w_{i c \sigma}^s w_{b c \sigma}^s}{(\omega-\epsilon_{c \sigma}-\Omega_s+\mathrm{i} \eta)(\omega-\epsilon_{i \sigma}-\mathrm{i} \eta)(\omega-\epsilon_{b \sigma}+\mathrm{i} \eta)} \right) \right. \\
& \left. + \sum_{ajks} \left( \frac{w_{a k \sigma}^s w_{j k \sigma}^s}{(\omega-\epsilon_{k \sigma}+\Omega_s-\mathrm{i} \eta)(\omega-\epsilon_{a \sigma}+\mathrm{i} \eta)(\omega-\epsilon_{j \sigma}-\mathrm{i} \eta)} \right) \right.\\
& \left. + \sum_{abks} \left( \frac{w_{a k \sigma}^s w_{b k \sigma}^s}{(\omega-\epsilon_{k \sigma}+\Omega_s-\mathrm{i} \eta)(\omega-\epsilon_{a \sigma}+\mathrm{i} \eta)(\omega-\epsilon_{b \sigma}+\mathrm{i} \eta)} \right) \right.\\
& \left. + \sum_{ajcs} \left( \frac{w_{a c \sigma}^s w_{j c \sigma}^s}{(\omega-\epsilon_{c \sigma}-\Omega_s+\mathrm{i} \eta)(\omega-\epsilon_{a \sigma}+\mathrm{i} \eta)(\omega-\epsilon_{j \sigma}-\mathrm{i} \eta)} \right) \right.\\
& \left. + \sum_{abcs} \left( \frac{w_{a c \sigma}^s w_{b c \sigma}^s}{(\omega-\epsilon_{c \sigma}-\Omega_s+\mathrm{i} \eta)(\omega-\epsilon_{a \sigma}+\mathrm{i} \eta)(\omega-\epsilon_{b \sigma}+\mathrm{i} \eta)} \right) \right)
\end{aligned}
\label{eqn:long_integral}
\end{equation}
At this point, we note the following relation between integrals $\oint_{D_{\pm}} f(z) = \int_{-R}^R f(z) + \int_{{C_R}_{\pm}} f(z)$. $D_{\pm}$ is a semicircular domain in either half of the complex plane, ${C_R}_{\pm}$ is the semicircle in the upper or lower part of the complex plane, and $R$ is the radius of the semicircle. We are able to take $R\rightarrow \infty$ and since $f(z)=e^{i\omega \eta}g(z)$, where $g(z)$ is analytic on $D$ except for a finite number of poles, the integral over the semicircle will vanish by Jordan's lemma, leaving us with $\int_{-R=-\infty}^{R=\infty} f(z)= \oint_{D_{\pm}} f(z)$. The contribution over the fully occupied block will be given by the following two terms:
\begin{equation}
\begin{aligned}
I_{ij} =& \sum_{ks} w_{i k \sigma}^s w_{j k \sigma}^s \oint_{D+} \mathrm{d} \omega \mathrm{e}^{\mathrm{i \omega \eta}} \frac{1}{(\omega-\epsilon_{k \sigma}+\Omega_s-\mathrm{i} \eta)(\omega-\epsilon_{i \sigma}-\mathrm{i} \eta)(\omega-\epsilon_{j \sigma}-\mathrm{i} \eta)}\\
& + \sum_{cs} w_{i c \sigma}^s w_{j c \sigma}^s \oint_{D+} \mathrm{d} \omega \mathrm{e}^{\mathrm{i \omega \eta}} \frac{1}{(\omega-\epsilon_{c \sigma}-\Omega_s+\mathrm{i} \eta)(\omega-\epsilon_{i \sigma}-\mathrm{i} \eta)(\omega-\epsilon_{j \sigma}-\mathrm{i} \eta)}
\end{aligned}
\end{equation}
Due to the contour that is chosen for this case, we have poles for the first term at $\omega_{11} = \epsilon _{k\sigma } - \Omega _s + \mathrm{i} \eta$, $\omega_{12} = \epsilon _{i\sigma } + \mathrm{i} \eta$, and $\omega_{13} = \epsilon _{j\sigma } + \mathrm{i} \eta$. For such simple poles, the Cauchy residue theorem simplifies to:
\begin{equation}
\operatorname{Res}_{\omega =\omega _0} f(\omega )= \phi^{}\left(\omega _0\right)
\label{eqn:cauchy_residue}
\end{equation}
where $\phi_{\omega _0}(\omega ) = (\omega - \omega_0) f(\omega )$. For the first of these integrals in the occupied block, we have:
\begin{equation}
f_1(\omega) = \frac{e^{i\omega \eta }}{(\omega-\epsilon_{k \sigma}+\Omega_s-\mathrm{i} \eta)(\omega-\epsilon_{i \sigma}-\mathrm{i} \eta)(\omega-\epsilon_{j \sigma}-\mathrm{i} \eta)}
\end{equation}
Plugging in $\omega_{11} = \epsilon _{k\sigma } - \Omega _s + \mathrm{i} \eta$, we get:
\begin{equation}
\phi_{\omega_{11}}(\omega) = (\omega - \epsilon_{k \sigma} + \Omega_s - \mathrm{i} \eta) f_1(\omega) = \frac{e^{i\omega \eta }}{(\omega-\epsilon_{i \sigma}-\mathrm{i} \eta)(\omega-\epsilon_{j \sigma}-\mathrm{i} \eta)}
\end{equation}
Evaluating this at the pole, we get:
\begin{equation}
\phi_{\omega_{11}}(\epsilon_{k \sigma} - \Omega_s + \mathrm{i} \eta) = \frac{e^{i(\epsilon_{k \sigma} - \Omega_s + \mathrm{i} \eta) \eta }}{(\epsilon_{k \sigma} - \Omega_s + \mathrm{i} \eta-\epsilon_{i \sigma}-\mathrm{i} \eta)(\epsilon_{k \sigma} - \Omega_s + \mathrm{i} \eta-\epsilon_{j \sigma}-\mathrm{i} \eta)}
\end{equation}
In the limit $\eta \to 0$, we get:
\begin{equation}
\boxed{\phi_{\omega_{11}}(\epsilon_{k \sigma} - \Omega_s + \mathrm{i} \eta) = \frac{1}{(\epsilon_{k \sigma} - \Omega_s -\epsilon_{i \sigma})(\epsilon_{k \sigma} - \Omega_s -\epsilon_{j \sigma})}}
\end{equation}
For the other poles, the procedure is similar, so I will just summarise the results:

% Now we consider the second pole at $\omega_{12} = \epsilon _{i\sigma } + \mathrm{i} \eta$. We consider:
% \begin{equation}
% \phi_{\omega_{12}}(\omega) = (\omega - \epsilon_{i \sigma} - \mathrm{i} \eta) f_1(\omega) = \frac{e^{i\omega \eta }}{(\omega-\epsilon_{k \sigma}+\Omega_s-\mathrm{i} \eta)(\omega-\epsilon_{j \sigma}-\mathrm{i} \eta)}
% \end{equation}
% Plugging in $\omega_{12} = \epsilon _{i\sigma } + \mathrm{i} \eta$, we get:
% \begin{equation}
% \phi_{\omega_{12}}(\epsilon_{i \sigma} + \mathrm{i} \eta) = \frac{e^{i(\epsilon_{i \sigma} + \mathrm{i} \eta) \eta }}{(\epsilon_{i \sigma} + \mathrm{i} \eta-\epsilon_{k \sigma}+\Omega_s-\mathrm{i} \eta)(\epsilon_{i \sigma} + \mathrm{i} \eta-\epsilon_{j \sigma}-\mathrm{i} \eta)}
% \end{equation}
% In the limit $\eta \to 0$, we get:
\begin{equation}
\boxed{\phi_{\omega_{12}}(\epsilon_{i \sigma} + \mathrm{i} \eta) = \frac{1}{(\epsilon_{i \sigma} -\epsilon_{k \sigma}+\Omega_s)(\epsilon_{i \sigma} -\epsilon_{j \sigma})}}
\end{equation}
% Now we consider the third pole at $\omega_{13} = \epsilon _{j\sigma } + \mathrm{i} \eta$. We consider:
% \begin{equation}
% \phi_{\omega_{13}}(\omega) = (\omega - \epsilon_{j \sigma} - \mathrm{i} \eta) f_1(\omega) = \frac{e^{i\omega \eta }}{(\omega-\epsilon_{k \sigma}+\Omega_s-\mathrm{i} \eta)(\omega-\epsilon_{i \sigma}-\mathrm{i} \eta)}
% \end{equation}
% Plugging in $\omega_{13} = \epsilon _{j\sigma } + \mathrm{i} \eta$, we get:
% \begin{equation}
% \phi_{\omega_{13}}(\epsilon_{j \sigma} + \mathrm{i} \eta) = \frac{e^{i(\epsilon_{j \sigma} + \mathrm{i} \eta) \eta }}{(\epsilon_{j \sigma} + \mathrm{i} \eta-\epsilon_{k \sigma}+\Omega_s-\mathrm{i} \eta)(\epsilon_{j \sigma} + \mathrm{i} \eta-\epsilon_{i \sigma}-\mathrm{i} \eta)}
% \end{equation}
% In the limit $\eta \to 0$, we get:
\begin{equation}
\boxed{\phi_{\omega_{13}}(\epsilon_{j \sigma} + \mathrm{i} \eta) = \frac{1}{(\epsilon_{j \sigma} -\epsilon_{k \sigma}+\Omega_s)(\epsilon_{j \sigma} -\epsilon_{i \sigma})}}
\end{equation}
By Cauchy's residue theorem,  the first term of the integral will be given by:
\begin{equation}
\begin{aligned}
2\pi i \sum_{ks} w_{i k \sigma}^s w_{j k \sigma}^s & \left( \frac{1}{(\epsilon_{k \sigma} - \Omega_s -\epsilon_{i \sigma})(\epsilon_{k \sigma} - \Omega_s -\epsilon_{j \sigma})} \right. \\
& \left. + \frac{1}{(\epsilon_{i \sigma} -\epsilon_{k \sigma}+\Omega_s)(\epsilon_{i \sigma} -\epsilon_{j \sigma})} \right. \\
& \left. + \frac{1}{(\epsilon_{j \sigma} -\epsilon_{k \sigma}+\Omega_s)(\epsilon_{j \sigma} -\epsilon_{i \sigma})} \right)
\end{aligned}
\end{equation}
We move on to the second integral in the occupied block. It only has two poles in the fully occupied contour: $\omega_{21} = \epsilon _{i\sigma } + \mathrm{i} \eta$ and $\omega_{22} = \epsilon _{j\sigma } + \mathrm{i} \eta$. We have $f_2(\omega)$ as:
\begin{equation}
f_2(\omega) = \frac{e^{i\omega \eta }}{(\omega-\epsilon_{c \sigma}-\Omega_s+\mathrm{i} \eta)(\omega-\epsilon_{i \sigma}-\mathrm{i} \eta)(\omega-\epsilon_{j \sigma}-\mathrm{i} \eta)}
\end{equation}
So $\phi_{\omega_{21}}(\omega _{21})$ is:
\begin{equation}
\boxed{\phi_{\omega_{21}}(\epsilon_{i \sigma} + \mathrm{i} \eta) = \frac{1}{(\epsilon_{i \sigma} -\epsilon_{c \sigma}-\Omega_s)(\epsilon_{i \sigma} -\epsilon_{j \sigma})}}
\end{equation}
Now we consider the second pole at $\omega_{22} = \epsilon _{j\sigma } + \mathrm{i} \eta$
\begin{equation}
\boxed{\phi_{\omega_{22}}(\epsilon_{j \sigma} + \mathrm{i} \eta) = \frac{1}{(\epsilon_{j \sigma} -\epsilon_{c \sigma}-\Omega_s)(\epsilon_{j \sigma} -\epsilon_{i \sigma})}}
\end{equation}
So the second term of the integral will be given by:
\begin{equation}
\begin{aligned}
2\pi i \sum_{cs} w_{i c \sigma}^s w_{j c \sigma}^s & \left( \frac{1}{(\epsilon_{i \sigma} -\epsilon_{c \sigma}-\Omega_s)(\epsilon_{i \sigma} -\epsilon_{j \sigma})} \right. \\
& \left. + \frac{1}{(\epsilon_{j \sigma} -\epsilon_{c \sigma}-\Omega_s)(\epsilon_{j \sigma} -\epsilon_{i \sigma})} \right)
\end{aligned}
\end{equation}


\begin{table}[h]
\centering
\caption{Summary of Poles and their Residues}
\begin{tabular}{|c|c|c|}
\hline
Pole Notation & Position $\omega_0$ & Residue $\phi_{\omega_0}(\omega_0)$ \\
\hline
\multicolumn{3}{|c|}{Series $\omega_1$} \\
\hline
$\omega_{11}$ & $\epsilon_{k \sigma} - \Omega_s + \mathrm{i} \eta$ & $\frac{1}{(\epsilon_{k \sigma} - \Omega_s -\epsilon_{i \sigma})(\epsilon_{k \sigma} - \Omega_s -\epsilon_{j \sigma})}$ \\
$\omega_{12}$ & $\epsilon_{i \sigma} + \mathrm{i} \eta$ & $\frac{1}{(\epsilon_{i \sigma} -\epsilon_{k \sigma}+\Omega_s)(\epsilon_{i \sigma} -\epsilon_{j \sigma})}$ \\
$\omega_{13}$ & $\epsilon_{j \sigma} + \mathrm{i} \eta$ & $\frac{1}{(\epsilon_{j \sigma} -\epsilon_{k \sigma}+\Omega_s)(\epsilon_{j \sigma} -\epsilon_{i \sigma})}$ \\
\hline
\multicolumn{3}{|c|}{Series $\omega_2$} \\
\hline
$\omega_{21}$ & $\epsilon_{i \sigma} + \mathrm{i} \eta$ & $\frac{1}{(\epsilon_{i \sigma} -\epsilon_{c \sigma}-\Omega_s)(\epsilon_{i \sigma} -\epsilon_{j \sigma})}$ \\
$\omega_{22}$ & $\epsilon_{j \sigma} + \mathrm{i} \eta$ & $\frac{1}{(\epsilon_{j \sigma} -\epsilon_{c \sigma}-\Omega_s)(\epsilon_{j \sigma} -\epsilon_{i \sigma})}$ \\
\hline
\end{tabular}
\end{table}
Adding the two terms together, we get:
\begin{equation}
\begin{aligned}
I_{ij} = 2\pi i \Bigg( & \sum_{ks} w_{i k \sigma}^s w_{j k \sigma}^s \left( \frac{1}{(\epsilon_{k \sigma} - \Omega_s - \epsilon_{i \sigma})(\epsilon_{k \sigma} - \Omega_s - \epsilon_{j \sigma})} + \frac{1}{(\epsilon_{i \sigma} - \epsilon_{k \sigma} + \Omega_s)(\epsilon_{i \sigma} - \epsilon_{j \sigma})} \right. \\
& \left. + \frac{1}{(\epsilon_{j \sigma} - \epsilon_{k \sigma} + \Omega_s)(\epsilon_{j \sigma} - \epsilon_{i \sigma})} \right) \\
& + \sum_{cs} w_{i c \sigma}^s w_{j c \sigma}^s \left( \frac{1}{(\epsilon_{i \sigma} - \epsilon_{c \sigma} - \Omega_s)(\epsilon_{i \sigma} - \epsilon_{j \sigma})} + \frac{1}{(\epsilon_{j \sigma} - \epsilon_{c \sigma} - \Omega_s)(\epsilon_{j \sigma} - \epsilon_{i \sigma})} \right) \Bigg)
\end{aligned}
\end{equation}
Let as first just consider the first term:
\begin{equation}
\sum_{ks} w_{i k \sigma}^s w_{j k \sigma}^s \left( \frac{1}{(\epsilon_{k \sigma} - \Omega_s - \epsilon_{i \sigma})(\epsilon_{k \sigma} - \Omega_s - \epsilon_{j \sigma})} + \frac{1}{(\epsilon_{i \sigma} - \epsilon_{k \sigma} + \Omega_s)(\epsilon_{i \sigma} - \epsilon_{j \sigma})} + \frac{1}{(\epsilon_{j \sigma} - \epsilon_{k \sigma} + \Omega_s)(\epsilon_{j \sigma} - \epsilon_{i \sigma})} \right)
\end{equation}
Getting a common denominator for all of the terms:
\begin{equation}
\begin{aligned}
& \frac{1}{(\epsilon_{k \sigma} - \Omega_s - \epsilon_{i \sigma})(\epsilon_{k \sigma} - \Omega_s - \epsilon_{j \sigma})} + \frac{1}{(\epsilon_{i \sigma} - \epsilon_{k \sigma} + \Omega_s)(\epsilon_{i \sigma} - \epsilon_{j \sigma})} + \frac{1}{(\epsilon_{j \sigma} - \epsilon_{k \sigma} + \Omega_s)(\epsilon_{j \sigma} - \epsilon_{i \sigma})} \\
& = \frac{\left( \epsilon_{i\sigma } - \epsilon_{j\sigma } \right)}{\left(\epsilon_{k\sigma } - \Omega _s - \epsilon_{i\sigma }\right)\left(\epsilon_{k\sigma } - \Omega _s - \epsilon_{j\sigma }\right)\left(\epsilon_{i\sigma } - \epsilon_{j\sigma }\right)} - \frac{\left( \epsilon_{k\sigma } - \Omega _s - \epsilon_{j\sigma } \right)}{\left(\epsilon_{k\sigma } - \Omega _s - \epsilon_{i\sigma }\right)\left(\epsilon_{k\sigma } - \Omega _s - \epsilon_{j\sigma }\right)\left(\epsilon_{i\sigma } - \epsilon_{j\sigma }\right)}\\
& + \frac{\left( \epsilon_{k\sigma } - \Omega _s - \epsilon_{i\sigma } \right)}{\left(\epsilon_{k\sigma } - \Omega _s - \epsilon_{i\sigma }\right)\left(\epsilon_{k\sigma } - \Omega _s - \epsilon_{j\sigma }\right)\left(\epsilon_{i\sigma } - \epsilon_{j\sigma }\right)} \\
&=0
\end{aligned}
\end{equation}
So the first term simplifies to zero. The second term is:
\begin{equation}
\sum_{cs} w_{i c \sigma}^s w_{j c \sigma}^s \left( \frac{1}{(\epsilon_{i \sigma} - \epsilon_{c \sigma} - \Omega_s)(\epsilon_{i \sigma} - \epsilon_{j \sigma})} + \frac{1}{(\epsilon_{j \sigma} - \epsilon_{c \sigma} - \Omega_s)(\epsilon_{j \sigma} - \epsilon_{i \sigma})} \right)
\end{equation}
Getting a common denominator for all of the terms:
\begin{equation}
\begin{aligned}
& \frac{1}{(\epsilon_{i \sigma} - \epsilon_{c \sigma} - \Omega_s)(\epsilon_{i \sigma} - \epsilon_{j \sigma})} + \frac{1}{(\epsilon_{j \sigma} - \epsilon_{c \sigma} - \Omega_s)(\epsilon_{j \sigma} - \epsilon_{i \sigma})} \\
& = \frac{\left( \epsilon_{j\sigma } - \epsilon_{c\sigma } - \Omega _s \right)}{\left(\epsilon_{i\sigma } - \epsilon_{c\sigma } - \Omega _s\right)\left(\epsilon_{i\sigma } - \epsilon_{j\sigma }\right)\left(\epsilon_{j\sigma } - \epsilon_{c\sigma } - \Omega _s\right)} - \frac{\left( \epsilon_{i\sigma } - \epsilon_{c\sigma } - \Omega _s \right)}{\left(\epsilon_{i\sigma } - \epsilon_{c\sigma } - \Omega _s\right)\left(\epsilon_{i\sigma } - \epsilon_{j\sigma }\right)\left(\epsilon_{j\sigma } - \epsilon_{c\sigma } - \Omega _s\right)}\\
&=\frac{\left( \epsilon _{j\sigma } - \epsilon _{i\sigma } \right)}{\left(\epsilon _{i\sigma } - \epsilon _{c\sigma } - \Omega _s\right)\left(\epsilon _{j\sigma } - \epsilon _{c\sigma } - \Omega _s\right)\left(\epsilon _{i\sigma } - \epsilon _{j\sigma }\right)}\\
&= -\frac{1}{\left(\epsilon _{i\sigma } - \epsilon _{c\sigma } - \Omega _s\right)\left(\epsilon _{j\sigma } - \epsilon _{c\sigma } - \Omega _s\right)}=-\frac{1}{\left(\Omega _s + \epsilon _{c\sigma } - \epsilon _{i\sigma }\right)\left(\Omega _s + \epsilon _{c\sigma } - \epsilon _{j\sigma }\right)}
\end{aligned}
\end{equation}
So we see that the whole integral just evaluate to the second term, giving:
\begin{equation}
I_{ij} = -2\pi i \sum_{cs}\frac{w_{i c \sigma}^s w_{j c \sigma}^s}{(\Omega_s + \epsilon_{c \sigma} - \epsilon_{i \sigma})(\Omega_s + \epsilon_{c \sigma} - \epsilon_{j \sigma})}
\end{equation}
So, the expression for $D_{ij}$ is:
\begin{equation}
D_{ij} = \bra{i\sigma } \gamma _0^\sigma \ket{j\sigma } + \frac{2\pi i^2}{2\pi} \sum_{cs}\frac{w_{ic} w_{jc}}{(\Omega_s + \epsilon_{c \sigma} - \epsilon_{i \sigma})(\Omega_s + \epsilon_{c \sigma} - \epsilon_{j \sigma})}
\end{equation}
The first term is the matrix element of the noninteracting part of the density matrix, so this just simplifies to $\delta _{ij}$ and then we relabel the virtual index $c\rightarrow a$:
\begin{equation}
D_{ij} = \delta _{ij} - \sum_{as}\frac{w_{ia} w_{ja}}{(\Omega_s + \epsilon_{a \sigma} - \epsilon_{i \sigma})(\Omega_s + \epsilon_{a \sigma} - \epsilon_{j \sigma})}
\end{equation}


\section{Fully Virtual Block}
For the fully virtual block, we need to consider third to last and last terms of the integral in equation \ref{eqn:long_integral}:
\begin{equation}
\begin{aligned}
I_{ab} =& \sum_{ks} w_{a k \sigma}^s w_{b k \sigma}^s \int \mathrm{d} \omega \mathrm{e}^{\mathrm{i \omega \eta}} \frac{1}{(\omega-\epsilon_{k \sigma}+\Omega_s-\mathrm{i} \eta)(\omega-\epsilon_{a \sigma}+\mathrm{i} \eta)(\omega-\epsilon_{b \sigma}+\mathrm{i} \eta)}\\
& + \sum_{cs} w_{a c \sigma}^s w_{b c \sigma}^s \int \mathrm{d} \omega \mathrm{e}^{\mathrm{i \omega \eta}} \frac{1}{(\omega-\epsilon_{c \sigma}-\Omega_s+\mathrm{i} \eta)(\omega-\epsilon_{a \sigma}+\mathrm{i} \eta)(\omega-\epsilon_{b \sigma}+\mathrm{i} \eta)}
\end{aligned}
\end{equation}
Due to the contour that is chosen for this case, we have poles for the first term at just $\omega_{11} = \epsilon _{a \sigma } - \mathrm{i} \eta$ and $\omega_{12} = \epsilon _{b \sigma } - \mathrm{i} \eta$. Using the Cauchy residue theorem from equation \ref{eqn:cauchy_residue}:
\begin{equation}
f_1(\omega) = \frac{e^{i\omega \eta }}{(\omega-\epsilon_{k \sigma}+\Omega_s-\mathrm{i} \eta)(\omega-\epsilon_{a \sigma}+\mathrm{i} \eta)(\omega-\epsilon_{b \sigma}+\mathrm{i} \eta)}
\end{equation}
Plugging in $\omega_{11} = \epsilon _{a \sigma } - \mathrm{i} \eta$, we get:
% \begin{equation}
% \phi_{\omega_{11}}(\omega) = (\omega - \epsilon_{a \sigma} + \mathrm{i} \eta) f_1(\omega) = \frac{e^{i\omega \eta }}{(\omega-\epsilon_{k \sigma}+\Omega_s-\mathrm{i} \eta)(\omega-\epsilon_{b \sigma}+\mathrm{i} \eta)}
% \end{equation}
% Evaluating this at the pole, we get:
% \begin{equation}
% \phi_{\omega_{11}}(\epsilon_{a \sigma} - \mathrm{i} \eta) = \frac{e^{i(\epsilon_{a \sigma} - \mathrm{i} \eta) \eta }}{(\epsilon_{a \sigma} - \mathrm{i} \eta-\epsilon_{k \sigma}+\Omega_s-\mathrm{i} \eta)(\epsilon_{a \sigma} - \mathrm{i} \eta-\epsilon_{b \sigma}+\mathrm{i} \eta)}
% \end{equation}
% In the limit $\eta \to 0$, we get:
\begin{equation}
\boxed{\phi_{\omega_{11}}(\epsilon_{a \sigma} - \mathrm{i} \eta) = \frac{1}{(\epsilon_{a \sigma} -\epsilon_{k \sigma}+\Omega_s)(\epsilon_{a \sigma} -\epsilon_{b \sigma})}}
\end{equation}
% Now we consider the second pole at $\omega_{12} = \epsilon _{b\sigma } - \mathrm{i} \eta$. We consider:
% \begin{equation}
% \phi_{\omega_{12}}(\omega) = (\omega - \epsilon_{b \sigma} - \mathrm{i} \eta) f_1(\omega) = \frac{e^{i\omega \eta }}{(\omega-\epsilon_{k \sigma}+\Omega_s-\mathrm{i} \eta)(\omega-\epsilon_{a \sigma}+\mathrm{i} \eta)}
% \end{equation}
% Plugging in $\omega_{12} = \epsilon _{b\sigma } - \mathrm{i} \eta$, we get:
% \begin{equation}
% \phi_{\omega_{12}}(\epsilon_{b \sigma} - \mathrm{i} \eta) = \frac{e^{i(\epsilon_{b \sigma} - \mathrm{i} \eta) \eta }}{(\epsilon_{b \sigma} - \mathrm{i} \eta-\epsilon_{k \sigma}+\Omega_s-\mathrm{i} \eta)(\epsilon_{b \sigma} - \mathrm{i} \eta-\epsilon_{a \sigma}+\mathrm{i} \eta)}
% \end{equation}
% In the limit $\eta \to 0$, we get:
\begin{equation}
\boxed{\phi_{\omega_{12}}(\epsilon_{b \sigma} - \mathrm{i} \eta) = \frac{1}{(\epsilon_{b \sigma} -\epsilon_{k \sigma}+\Omega_s)(\epsilon_{b \sigma} -\epsilon_{a \sigma})}}
\end{equation}
So the first term of the integral will be given by:
\begin{equation}
\begin{aligned}
2\pi i \sum_{ks} w_{a k \sigma}^s w_{b k \sigma}^s & \left( \frac{1}{(\epsilon_{a \sigma} -\epsilon_{k \sigma}+\Omega_s)(\epsilon_{a \sigma} -\epsilon_{b \sigma})} + \frac{1}{(\epsilon_{b \sigma} -\epsilon_{k \sigma}+\Omega_s)(\epsilon_{b \sigma} -\epsilon_{a \sigma})} \right)
\end{aligned}
\end{equation}
We move on to the second integral in the virtual block. It has now three poles in the fully virtual contour: $\omega_{21} = \epsilon _{c\sigma } + \Omega_s - \mathrm{i} \eta$, $\omega_{22} = \epsilon _{a\sigma } - \mathrm{i} \eta$, and $\omega_{23} = \epsilon _{b\sigma } - \mathrm{i} \eta$. We have $f_2(\omega)$ as:
\begin{equation}
f_2(\omega) = \frac{e^{i\omega \eta }}{(\omega-\epsilon_{c \sigma}-\Omega_s+\mathrm{i} \eta)(\omega-\epsilon_{a \sigma}+\mathrm{i} \eta)(\omega-\epsilon_{b \sigma}+\mathrm{i} \eta)}
\end{equation}
So $\phi_{\omega_{21}}(\omega _{21})$ is:
% \begin{equation}
% \phi_{\omega_{21}}(\omega) = (\omega - \epsilon_{c \sigma} - \Omega_s + \mathrm{i} \eta) f_2(\omega) = \frac{e^{i\omega \eta }}{(\omega-\epsilon_{a \sigma}+\mathrm{i} \eta)(\omega-\epsilon_{b \sigma}+\mathrm{i} \eta)}
% \end{equation}
% Plugging in $\omega_{21} = \epsilon _{c\sigma } + \Omega_s - \mathrm{i} \eta$, we get:
% \begin{equation}
% \phi_{\omega_{21}}(\epsilon_{c \sigma} + \Omega_s - \mathrm{i} \eta) = \frac{e^{i(\epsilon_{c \sigma} + \Omega_s - \mathrm{i} \eta) \eta }}{(\epsilon_{c \sigma} + \Omega_s - \mathrm{i} \eta-\epsilon_{a \sigma}+\mathrm{i} \eta)(\epsilon_{c \sigma} + \Omega_s - \mathrm{i} \eta-\epsilon_{b \sigma}+\mathrm{i} \eta)}
% \end{equation}
% In the limit $\eta \to 0$, we get:
\begin{equation}
\boxed{\phi_{\omega_{21}}(\epsilon_{c \sigma} + \Omega_s - \mathrm{i} \eta) = \frac{1}{(\epsilon_{c \sigma} + \Omega_s -\epsilon_{a \sigma})(\epsilon_{c \sigma} + \Omega_s -\epsilon_{b \sigma})}}
\end{equation}
Now we consider the second pole at $\omega_{22} = \epsilon _{a\sigma } - \mathrm{i} \eta$.
% \begin{equation}
% \phi_{\omega_{22}}(\omega) = (\omega - \epsilon_{a \sigma} - \mathrm{i} \eta) f_2(\omega) = \frac{e^{i\omega \eta }}{(\omega-\epsilon_{c \sigma}-\Omega_s+\mathrm{i} \eta)(\omega-\epsilon_{b \sigma}+\mathrm{i} \eta)}
% \end{equation}
% Plugging in $\omega_{22} = \epsilon _{a\sigma } - \mathrm{i} \eta$, we get:
% \begin{equation}
% \phi_{\omega_{22}}(\epsilon_{a \sigma} - \mathrm{i} \eta) = \frac{e^{i(\epsilon_{a \sigma} - \mathrm{i} \eta) \eta }}{(\epsilon_{a \sigma} - \mathrm{i} \eta-\epsilon_{c \sigma}-\Omega_s+\mathrm{i} \eta)(\epsilon_{a \sigma} - \mathrm{i} \eta-\epsilon_{b \sigma}+\mathrm{i} \eta)}
% \end{equation}
% In the limit $\eta \to 0$, we get:
\begin{equation}
\boxed{\phi_{\omega_{22}}(\epsilon_{a \sigma} - \mathrm{i} \eta) = \frac{1}{(\epsilon_{a \sigma} -\epsilon_{c \sigma}-\Omega_s)(\epsilon_{a \sigma} -\epsilon_{b \sigma})}}
\end{equation}
Now we consider the third pole at $\omega_{23} = \epsilon _{b\sigma } - \mathrm{i} \eta$.
% \begin{equation}
% \phi_{\omega_{23}}(\omega) = (\omega - \epsilon_{b \sigma} - \mathrm{i} \eta) f_2(\omega) = \frac{e^{i\omega \eta }}{(\omega-\epsilon_{c \sigma}-\Omega_s+\mathrm{i} \eta)(\omega-\epsilon_{a \sigma}+\mathrm{i} \eta)}
% \end{equation}
% Plugging in $\omega_{23} = \epsilon _{b\sigma } - \mathrm{i} \eta$, we get:
% \begin{equation}
% \phi_{\omega_{23}}(\epsilon_{b \sigma} - \mathrm{i} \eta) = \frac{e^{i(\epsilon_{b \sigma} - \mathrm{i} \eta) \eta }}{(\epsilon_{b \sigma} - \mathrm{i} \eta-\epsilon_{c \sigma}-\Omega_s+\mathrm{i} \eta)(\epsilon_{b \sigma} - \mathrm{i} \eta-\epsilon_{a \sigma}+\mathrm{i} \eta)}
% \end{equation}
% In the limit $\eta \to 0$, we get:
\begin{equation}
\boxed{\phi_{\omega_{23}}(\epsilon_{b \sigma} - \mathrm{i} \eta) = \frac{1}{(\epsilon_{b \sigma} -\epsilon_{c \sigma}-\Omega_s)(\epsilon_{b \sigma} -\epsilon_{a \sigma})}}
\end{equation}
So the second term of the integral will be given by:
\begin{equation}
\begin{aligned}
2\pi i \sum_{cs} w_{a c \sigma}^s w_{b c \sigma}^s & \left( \frac{1}{(\epsilon_{c \sigma} + \Omega_s -\epsilon_{a \sigma})(\epsilon_{c \sigma} + \Omega_s -\epsilon_{b \sigma})} \right. \\
& \left. + \frac{1}{(\epsilon_{a \sigma} -\epsilon_{c \sigma}-\Omega_s)(\epsilon_{a \sigma} -\epsilon_{b \sigma})} \right. \\
& \left. + \frac{1}{(\epsilon_{b \sigma} -\epsilon_{c \sigma}-\Omega_s)(\epsilon_{b \sigma} -\epsilon_{a \sigma})} \right)
\end{aligned}
\end{equation}
The results we got are summarized in the table:
\begin{table}[h]
\centering
\caption{Summary of Poles and their Residues}
\begin{tabular}{|c|c|c|}
\hline
Pole Notation & Position $\omega_0$ & Residue $\phi_{\omega_0}(\omega_0)$ \\
\hline
\multicolumn{3}{|c|}{Series $\omega_1$} \\
\hline
$\omega_{11}$ & $\epsilon_{a \sigma} - \mathrm{i} \eta$ & $\frac{1}{(\epsilon_{a \sigma} -\epsilon_{k \sigma}+\Omega_s)(\epsilon_{a \sigma} -\epsilon_{b \sigma})}$ \\
$\omega_{12}$ & $\epsilon_{b \sigma} - \mathrm{i} \eta$ & $\frac{1}{(\epsilon_{b \sigma} -\epsilon_{k \sigma}+\Omega_s)(\epsilon_{b \sigma} -\epsilon_{a \sigma})}$ \\
\hline
\multicolumn{3}{|c|}{Series $\omega_2$} \\
\hline
$\omega_{21}$ & $\epsilon_{c \sigma} + \Omega_s - \mathrm{i} \eta$ & $\frac{1}{(\epsilon_{c \sigma} + \Omega_s -\epsilon_{a \sigma})(\epsilon_{c \sigma} + \Omega_s -\epsilon_{b \sigma})}$ \\
$\omega_{22}$ & $\epsilon_{a \sigma} - \mathrm{i} \eta$ & $\frac{1}{(\epsilon_{a \sigma} -\epsilon_{c \sigma}-\Omega_s)(\epsilon_{a \sigma} -\epsilon_{b \sigma})}$ \\
$\omega_{23}$ & $\epsilon_{b \sigma} - \mathrm{i} \eta$ & $\frac{1}{(\epsilon_{b \sigma} -\epsilon_{c \sigma}-\Omega_s)(\epsilon_{b \sigma} -\epsilon_{a \sigma})}$ \\
\end{tabular}
\end{table}
Adding the two terms together, we get:
\begin{equation}
\begin{aligned}
I_{ab} = 2\pi i \Bigg( & \sum_{ks} w_{a k \sigma}^s w_{b k \sigma}^s \left( \frac{1}{(\epsilon_{a \sigma} -\epsilon_{k \sigma}+\Omega_s)(\epsilon_{a \sigma} -\epsilon_{b \sigma})} + \frac{1}{(\epsilon_{b \sigma} -\epsilon_{k \sigma}+\Omega_s)(\epsilon_{b \sigma} -\epsilon_{a \sigma})} \right) \\
& + \sum_{cs} w_{a c \sigma}^s w_{b c \sigma}^s \left( \frac{1}{(\epsilon_{c \sigma} + \Omega_s -\epsilon_{a \sigma})(\epsilon_{c \sigma} + \Omega_s -\epsilon_{b \sigma})} \right. \\
& \left. + \frac{1}{(\epsilon_{a \sigma} -\epsilon_{c \sigma}-\Omega_s)(\epsilon_{a \sigma} -\epsilon_{b \sigma})} \right. \\
& \left. + \frac{1}{(\epsilon_{b \sigma} -\epsilon_{c \sigma}-\Omega_s)(\epsilon_{b \sigma} -\epsilon_{a \sigma})} \right) \Bigg)
\end{aligned}
\end{equation}
A similar simplification as the one done before gives:
\begin{equation}
I_{ab} = -2\pi i \sum_{ks}\frac{w_{ak} w_{bk}}{(\Omega_s + \epsilon_{k \sigma} - \epsilon_{a \sigma})(\Omega_s + \epsilon_{k \sigma} - \epsilon_{b \sigma})}
\end{equation}
So, the expression for $D_{ab}$ is:
\begin{equation}
D_{ab} = \bra{a\sigma } \gamma _0^\sigma \ket{b\sigma } + \frac{2\pi i^2}{2\pi} \sum_{ks}\frac{w_{ak} w_{bk}}{(\Omega_s + \epsilon_{k \sigma} - \epsilon_{a \sigma})(\Omega_s + \epsilon_{k \sigma} - \epsilon_{b \sigma})}
\end{equation}
The matrix element of the noninteracting density matrix does not mix virtual states and we relabel the occupied index $k\rightarrow i$:
\begin{equation}
D_{ab} = - \sum_{is}\frac{w_{ai} w_{bi}}{(\Omega_s + \epsilon_{i \sigma} - \epsilon_{a \sigma})(\Omega_s + \epsilon_{i \sigma} - \epsilon_{b \sigma})}
\end{equation}
Now, we want to consider the mixed block i.e. the second and fourth terms of the integral in equation \ref{eqn:long_integral}:
\begin{equation}
\begin{aligned}
I_{ib} =& \sum_{ks} w_{i k \sigma}^s w_{b k \sigma}^s \int \mathrm{d} \omega \mathrm{e}^{\mathrm{i \omega \eta}} \frac{1}{(\omega-\epsilon_{k \sigma}+\Omega_s-\mathrm{i} \eta)(\omega-\epsilon_{i \sigma}-\mathrm{i} \eta)(\omega-\epsilon_{b \sigma}+\mathrm{i} \eta)}\\
& + \sum_{cs} w_{i c \sigma}^s w_{b c \sigma}^s \int \mathrm{d} \omega \mathrm{e}^{\mathrm{i \omega \eta}} \frac{1}{(\omega-\epsilon_{c \sigma}-\Omega_s+\mathrm{i} \eta)(\omega-\epsilon_{i \sigma}-\mathrm{i} \eta)(\omega-\epsilon_{b \sigma}+\mathrm{i} \eta)}
\end{aligned}
\end{equation}
Due to the contour that is chosen for this case, we have poles for the first term which lies in the upper half of the complex plane at $\omega_{11} = \epsilon _{k \sigma } - \Omega_s + \mathrm{i} \eta$ and $\omega_{12} = \epsilon _{i \sigma } + \mathrm{i} \eta$.
Using the Cauchy residue theorem from equation \ref{eqn:cauchy_residue}:
\begin{equation}
f_1(\omega) = \frac{e^{i\omega \eta }}{(\omega-\epsilon_{k \sigma}+\Omega_s-\mathrm{i} \eta)(\omega-\epsilon_{i \sigma}+\mathrm{i} \eta)(\omega-\epsilon_{b \sigma}+\mathrm{i} \eta)}
\end{equation}
Plugging in $\omega_{11} = \epsilon _{k \sigma } - \Omega_s + \mathrm{i} \eta$, we get:
\begin{equation}
\boxed{\phi_{\omega_{11}}(\epsilon_{k \sigma } - \Omega_s + \mathrm{i} \eta) = \frac{1}{(\epsilon_{k \sigma} -\epsilon_{i \sigma}-\Omega_s)(\epsilon_{k \sigma} -\epsilon_{b \sigma}-\Omega_s)}}
\end{equation}
Now we consider the second pole at $\omega_{12} = \epsilon _{i \sigma } + \mathrm{i} \eta$.
\begin{equation}
\boxed{\phi_{\omega_{12}}(\epsilon_{i \sigma} + \mathrm{i} \eta) = \frac{1}{(\epsilon_{i \sigma} -\epsilon_{k \sigma}+\Omega_s)(\epsilon_{i \sigma} -\epsilon_{b \sigma})}}
\end{equation}
So the first term of the integral will be given by:
\begin{equation}
\begin{aligned}
2\pi i \sum_{ks} w_{i k \sigma}^s w_{b k \sigma}^s & \left( \frac{1}{(\epsilon_{k \sigma} -\epsilon_{i \sigma}-\Omega_s)(\epsilon_{k \sigma} -\epsilon_{b \sigma}-\Omega_s)} \right. \\
& \left. + \frac{1}{(\epsilon_{i \sigma} -\epsilon_{k \sigma}+\Omega_s)(\epsilon_{i \sigma} -\epsilon_{b \sigma})} \right)
\end{aligned}
\end{equation}
We move on to the second integral in the mixed block. It has two poles in $D_-$: $\omega_{21} = \epsilon _{c\sigma } + \Omega_s - \mathrm{i} \eta$ and $\omega_{22} = \epsilon _{b \sigma } - \mathrm{i} \eta$. We have $f_2(\omega)$ as:
\begin{equation}
f_2(\omega) = \frac{e^{i\omega \eta }}{(\omega-\epsilon_{c \sigma}-\Omega_s+\mathrm{i} \eta)(\omega-\epsilon_{i \sigma}-\mathrm{i} \eta)(\omega-\epsilon_{b \sigma}+\mathrm{i} \eta)}
\end{equation}
So $\phi_{\omega_{21}}(\omega _{21})$ is:
\begin{equation}
\boxed{\phi_{\omega_{21}}(\epsilon_{c \sigma} + \Omega_s - \mathrm{i} \eta) = \frac{1}{(\epsilon_{c \sigma} + \Omega_s -\epsilon_{i \sigma})(\epsilon_{c \sigma} + \Omega_s -\epsilon_{b \sigma})}}
\end{equation}
Now we consider the second pole at $\omega_{22} = \epsilon _{b \sigma } - \mathrm{i} \eta$.
\begin{equation}
\boxed{\phi_{\omega_{22}}(\epsilon_{b \sigma} - \mathrm{i} \eta) = \frac{1}{(\epsilon_{b \sigma} -\epsilon_{c \sigma}-\Omega_s)(\epsilon_{b \sigma} -\epsilon_{i \sigma})}}
\end{equation}
So the second term of the integral will be given by:
\begin{equation}
\begin{aligned}
2\pi i \sum_{cs} w_{i c \sigma}^s w_{b c \sigma}^s & \left( \frac{1}{(\epsilon_{c \sigma} + \Omega_s -\epsilon_{i \sigma})(\epsilon_{c \sigma} + \Omega_s -\epsilon_{b \sigma})} \right. \\
& \left. + \frac{1}{(\epsilon_{b \sigma} -\epsilon_{c \sigma}-\Omega_s)(\epsilon_{b \sigma} -\epsilon_{i \sigma})} \right)
\end{aligned}
\end{equation}
The results we got are summarized in the table:
\begin{table}[h]
\centering
\caption{Summary of Poles and their Residues}
\begin{tabular}{|c|c|c|}
\hline
Pole Notation & Position $\omega_0$ & Residue $\phi_{\omega_0}(\omega_0)$ \\
\hline
\multicolumn{3}{|c|}{Series $\omega_1$} \\
\hline
$\omega_{11}$ & $\epsilon_{k \sigma} - \Omega_s + \mathrm{i} \eta$ & $\frac{1}{(\epsilon_{k \sigma} -\epsilon_{i \sigma}-\Omega_s)(\epsilon_{k \sigma} -\epsilon_{b \sigma}-\Omega_s)}$ \\
$\omega_{12}$ & $\epsilon_{i \sigma} + \mathrm{i} \eta$ & $\frac{1}{(\epsilon_{i \sigma} -\epsilon_{k \sigma}+\Omega_s)(\epsilon_{i \sigma} -\epsilon_{b \sigma})}$ \\
\hline
\multicolumn{3}{|c|}{Series $\omega_2$} \\
\hline
$\omega_{21}$ & $\epsilon_{c \sigma} + \Omega_s - \mathrm{i} \eta$ & $\frac{1}{(\epsilon_{c \sigma} + \Omega_s -\epsilon_{i \sigma})(\epsilon_{c \sigma} + \Omega_s -\epsilon_{b \sigma})}$ \\
$\omega_{22}$ & $\epsilon_{b \sigma} - \mathrm{i} \eta$ & $\frac{1}{(\epsilon_{b \sigma} -\epsilon_{c \sigma}-\Omega_s)(\epsilon_{b \sigma} -\epsilon_{i \sigma})}$ \\
\end{tabular}
\end{table}
Adding the two terms together, we get:
\begin{equation}
\begin{aligned}
I_{ib} = 2\pi i \Bigg( & \sum_{ks} w_{i k \sigma}^s w_{b k \sigma}^s \left( \frac{1}{(\epsilon_{k \sigma} -\epsilon_{i \sigma}-\Omega_s)(\epsilon_{k \sigma} -\epsilon_{b \sigma}-\Omega_s)} \right. \\
& \left. + \frac{1}{(\epsilon_{i \sigma} -\epsilon_{k \sigma}+\Omega_s)(\epsilon_{i \sigma} -\epsilon_{b \sigma})} \right) \\
& + \sum_{cs} w_{i c \sigma}^s w_{b c \sigma}^s \left( \frac{1}{(\epsilon_{c \sigma} + \Omega_s -\epsilon_{i \sigma})(\epsilon_{c \sigma} + \Omega_s -\epsilon_{b \sigma})} \right. \\
& \left. + \frac{1}{(\epsilon_{b \sigma} -\epsilon_{c \sigma}-\Omega_s)(\epsilon_{b \sigma} -\epsilon_{i \sigma})} \right) \Bigg)
\end{aligned}
\end{equation}
Let us make some simplifications on the first term:
\begin{equation}
\begin{aligned}
& \left( \frac{1}{(\epsilon_{k \sigma} -\epsilon_{i \sigma}-\Omega_s)(\epsilon_{k \sigma} -\epsilon_{b \sigma}-\Omega_s)} \right. \\
& \left. + \frac{1}{(\epsilon_{i \sigma} -\epsilon_{k \sigma}+\Omega_s)(\epsilon_{i \sigma} -\epsilon_{b \sigma})} \right) \\
& = \frac{\left( \epsilon_{i\sigma } - \epsilon_{b\sigma } \right)}{\left(\epsilon_{k\sigma } - \epsilon_{i\sigma } - \Omega _s\right)\left(\epsilon_{k\sigma } - \epsilon_{b\sigma } - \Omega _s\right)\left(\epsilon_{i\sigma } - \epsilon_{b\sigma }\right)} - \frac{\left(\epsilon _{k\sigma } - \epsilon _{b\sigma} + \Omega _s \right)}{\left(\epsilon _{k\sigma } - \epsilon _{i\sigma} + \Omega _s\right)\left(\epsilon _{k\sigma } - \epsilon _{b\sigma} - \Omega _s\right)\left(\epsilon _{i\sigma } - \epsilon _{b\sigma}\right)}\\
& = - \frac{\left( \epsilon_{k\sigma } - \epsilon_{i\sigma } - \Omega _s \right)}{\left(\epsilon_{k\sigma } - \epsilon_{i\sigma } - \Omega _s\right)\left(\epsilon_{k\sigma } - \epsilon_{b\sigma } - \Omega _s\right)\left(\epsilon_{i\sigma } - \epsilon_{b\sigma }\right)} = -\frac{1}{\left(\epsilon_{k\sigma } - \epsilon_{b\sigma } - \Omega _s\right)\left(\epsilon_{i \sigma } - \epsilon_{b \sigma }\right)}
\end{aligned}
\end{equation}
Doing the same for the second term will give:
\begin{equation}
\begin{aligned}
& \left( \frac{1}{(\epsilon_{c \sigma} + \Omega_s -\epsilon_{i \sigma})(\epsilon_{c \sigma} + \Omega_s -\epsilon_{b \sigma})} \right. \\
& \left. + \frac{1}{(\epsilon_{b \sigma} -\epsilon_{c \sigma}-\Omega_s)(\epsilon_{b \sigma} -\epsilon_{i \sigma})} \right) \\
& = \frac{\left( \epsilon_{b\sigma } - \epsilon_{i\sigma } \right)}{\left(\epsilon_{c\sigma } + \Omega _s - \epsilon_{i\sigma}\right)\left(\epsilon_{c\sigma } + \Omega _s - \epsilon_{b\sigma}\right)\left(\epsilon_{b\sigma } - \epsilon_{i\sigma}\right)} - \frac{\left(\epsilon _{c \sigma} + \Omega _s - \epsilon _{i\sigma } \right)}{\left(\epsilon _{c \sigma} + \Omega _s - \epsilon _{i\sigma}\right)\left(\epsilon _{c \sigma} + \Omega _s - \epsilon _{b\sigma}\right)\left(\epsilon _{b \sigma} - \epsilon _{i\sigma}\right)}\\
& = \frac{\epsilon _{b\sigma } - \epsilon _{c \sigma} - \Omega _s}{\left(\epsilon _{c \sigma} + \Omega _s - \epsilon _{i\sigma}\right)\left(\epsilon _{c \sigma} + \Omega _s - \epsilon _{b\sigma}\right)\left(\epsilon _{b \sigma} - \epsilon _{i\sigma}\right)} = - \frac{1}{\left(\epsilon _{c \sigma} + \Omega _s - \epsilon _{i\sigma}\right)\left(\epsilon _{b \sigma} - \epsilon _{i\sigma}\right)}
\end{aligned}
\end{equation}
So, the expression for $D_{ib}$ is:
\begin{equation}
D_{ib} = \bra{i\sigma } \gamma _0^\sigma \ket{b\sigma } + \frac{2\pi i^2}{2\pi\left( \epsilon _{i\sigma } - \epsilon _{b\sigma } \right)} \left[ \sum_{ks} \frac{w_{ik}^s w_{bk}^s}{\epsilon _{k\sigma } - \epsilon _{b\sigma } - \Omega _s} - \sum_{cs} \frac{w_{ic}^s w_{bc}^s}{\epsilon _{c \sigma} + \Omega _s - \epsilon _{i\sigma}} \right]
\end{equation}
The matrix element of the noninteracting density matrix does not mix copied wait virtual states and we relabel the occupied index $k\rightarrow j$ and the virtual index $c\rightarrow a$:
\begin{equation}
D_{ib} = \frac{1}{\epsilon _{i\sigma } - \epsilon _{b\sigma }} \left[ \sum_{as} \frac{w_{ia}^s w_{ba}^s}{\epsilon _{i\sigma } - \epsilon _{a\sigma } - \Omega _s} - \sum_{js} \frac{w_{ij}^s w_{bj}^s}{\epsilon _{j\sigma } - \epsilon _{b\sigma } - \Omega _s} \right]
\end{equation}
I am just curious what would happen if we chose the opposite contour for the previous integrations. We would just have one pole in $D_-$ at $\omega_{11} = \epsilon _{b \sigma } - \mathrm{i} \eta$ for the first term and $\omega_{21} = \epsilon _{i \sigma } + \mathrm{i} \eta$ for the second term. The residues would be:
\begin{equation}
\begin{aligned}
\phi_{\omega_{11}}(\epsilon_{b \sigma } - \mathrm{i} \eta) & = \frac{1}{(\epsilon_{b \sigma} -\epsilon_{k \sigma}+\Omega_s)(\epsilon_{b \sigma} -\epsilon_{i \sigma})} \\
\phi_{\omega_{21}}(\epsilon_{i \sigma} + \mathrm{i} \eta) & = \frac{1}{(\epsilon_{i \sigma} -\epsilon_{c \sigma}-\Omega_s)(\epsilon_{i \sigma} -\epsilon_{b \sigma})}
\end{aligned}
\end{equation}
% Finally, we consider the last term of the integral in the mixed block:
\end{document}
