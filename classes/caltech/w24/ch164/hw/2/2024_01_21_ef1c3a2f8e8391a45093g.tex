\documentclass[10pt]{article}
\usepackage[utf8]{inputenc}
\usepackage[T1]{fontenc}
\usepackage{amsmath}
\usepackage{amsfonts}
\usepackage{amssymb}
\usepackage[version=4]{mhchem}
\usepackage{stmaryrd}
\usepackage{physics}


\usepackage{listings} % Required for insertion of code
\usepackage{xcolor} % Required for custom colors

% Define custom colors
\definecolor{codegreen}{rgb}{0,0.6,0}
\definecolor{codegray}{rgb}{0.5,0.5,0.5}
\definecolor{codepurple}{rgb}{0.58,0,0.82}
\definecolor{backcolour}{rgb}{0.95,0.95,0.92}

% Setup the style for code listings
\lstdefinestyle{mystyle}{
    backgroundcolor=\color{backcolour},   
    commentstyle=\color{codegreen},
    keywordstyle=\color{magenta},
    numberstyle=\tiny\color{codegray},
    stringstyle=\color{codepurple},
    basicstyle=\ttfamily\footnotesize,
    breakatwhitespace=false,         
    breaklines=true,                 
    captionpos=b,                    
    keepspaces=true,                 
    numbers=left,                    
    numbersep=5pt,                  
    showspaces=false,                
    showstringspaces=false,
    showtabs=false,                  
    tabsize=2
}

% Activate the style
\lstset{style=mystyle}


\title{Ch/ChE 164 Winter 2024 
 Homework Problem Set \#2 }

\author{}
\date{}


\begin{document}
\maketitle
Due Date: Thursday January 25, 2024 @ 11:59pm
\section{}
\begin{enumerate}
  \item The approximate partition function for a dense gas is of the form
\end{enumerate}

$$
Q(N, V, T)=\frac{1}{N !}\left(\frac{2 \pi m k T}{h^{2}}\right)^{\frac{3}{2} N}(V-N b)^{N} \exp \left[\frac{a N^{2}}{V k T}\right]
$$

where $a$ and $b$ are constants that are given in terms of molecular parameters.

(a) (10 pts.) Calculate the equation of state from this partition function. What equation of state is this?
\subsection{}
We know that the pressure can be obtained from:
\begin{equation}
  P = kT \left( \frac{\partial \ln Q}{\partial V} \right)_{N,T}
\end{equation}
So, when we perform this computation using symbolic adobe, we get:
\begin{equation}
  P = - \frac{N^{2} a}{V^{2}} - \frac{N T k}{N b - V}
\end{equation}
Rearranging this equation, we get:
\begin{equation}
  P = \frac{N k T}{V - N b} - \frac{N^{2} a}{V^{2}}
\end{equation}
which is the Van der Waals equation of state for gas at high pressure and low temperature.
% Inline Python code in the document
\begin{lstlisting}[language=Python]
from sympy import symbols, diff, exp, ln, factorial, pi, simplify, latex

# Define the symbols
N, V, T, m, k, h, a, b = symbols('N V T m k h a b')

# Partition function Q
Q = (1 / factorial(N)) * ((2 * pi * m * k * T) / h**2)**(3/2 * N) * (V - N * b)**N * exp(a * N**2 / (V * k * T))

# Natural logarithm of Q
ln_Q = ln(Q)

# Differentiate ln(Q) with respect to V
dlnQ_dV = diff(ln_Q, V)

# Pressure P
P = k * T * dlnQ_dV

# Simplify the expression for P
P_simplified = simplify(P)

# Convert the simplified expression to LaTeX
P_latex = latex(P_simplified)

print(P_latex)

\end{lstlisting}

(b) (18 pts.) Calculate the Helmholtz free energy and the heat capacities, $C_{v}$ and $C_{p}$.
\subsection{}
The Helmholtz free energy is given by:
\begin{equation}
  A = - k T \ln Q
\end{equation}
First we want to take the natural logarithm of the partition function:
\begin{align}
  \ln Q &= \ln \left( \frac{1}{N !}\left(\frac{2 \pi m k T}{h^{2}}\right)^{\frac{3}{2} N}(V-N b)^{N} \exp \left[\frac{a N^{2}}{V k T}\right] \right) \\
&= \ln \left( \frac{1}{N !} \right) + \ln \left( \left(\frac{2 \pi m k T}{h^{2}}\right)^{\frac{3}{2} N} \right) + \ln \left( (V-N b)^{N} \right) + \ln \left( \exp \left[\frac{a N^{2}}{V k T}\right] \right) \\
&= \ln \left( \frac{1}{N !} \right) + \frac{3}{2} N \ln \left( \frac{2 \pi m k T}{h^{2}} \right) + N \ln \left( V-N b \right) + \frac{a N^{2}}{V k T}
\end{align}
Now we can substitute this expression into the equation for the Helmholtz free energy:
\begin{align}
  A &= - k T \ln Q \\
&= - k T \left( \ln \left( \frac{1}{N !} \right) + \frac{3}{2} N \ln \left( \frac{2 \pi m k T}{h^{2}} \right) + N \ln \left( V-N b \right) + \frac{a N^{2}}{V k T} \right) \\
\end{align}
Next, we want to calculate the heat capacities. We know that:
\begin{equation}
  C_{v} = \left( \frac{\partial U}{\partial T} \right)_{V,N}
\end{equation}


The internal energy is related to the Helmholtz free energy by:
\begin{equation}
  U = A + TS
\end{equation}

Now the entropy is given by:
\begin{equation}
  S = - \left( \frac{\partial A}{\partial T} \right)_{V,N}
\end{equation}

We can calculate the entropy by starting with the expression for \( A \):
\begin{equation}
  A = - k T \left( \ln \left( \frac{1}{N !} \right) + \frac{3}{2} N \ln \left( \frac{2 \pi m k T}{h^{2}} \right) + N \ln \left( V-N b \right) + \frac{a N^{2}}{V k T} \right)
\end{equation}
and taking the derivative with respect to \( T \) at constant \( V \) and \( N \):
\begin{align}
  \left( \frac{\partial A}{\partial T} \right)_{V,N} &= - k \left[ \ln \left( \frac{1}{N !} \right) + \frac{3}{2} N \ln \left( \frac{2 \pi m k T}{h^{2}} \right) + N \ln \left( V-N b \right) + \frac{a N^{2}}{V k T} \right] \\
  &\quad - k T \left[ \frac{3}{2} N \frac{h^2}{2\pi mk T} \frac{2\pi mk}{h^2} - \frac{a N^{2}}{V k T^{2}} \right] \\
  &= - k \left[ \ln \left( \frac{1}{N !} \right) + \frac{3}{2} N \ln \left( \frac{2 \pi m k T}{h^{2}} \right) + N \ln \left( V-N b \right) + \frac{a N^{2}}{V k T} \right] - \frac{3N}{2T} + \frac{a N^{2}}{V T} \\
  &= - k \left[ \ln \left( \frac{1}{N !} \right) + \frac{3}{2} N \ln \left( 2 \pi m k T \right) - \frac{3}{2} N \ln \left( h^{2} \right) + N \ln \left( V-N b \right) + \frac{a N^{2}}{V k T} \right] - \frac{3N}{2k} + \frac{a N^{2}}{V T} \\
  &= -k \ln \left( \frac{1}{N !} \right) - \frac{3}{2} N k \ln \left( 2 \pi m k T \right) + \frac{3}{2} N k \ln \left( h^{2} \right) - N k \ln \left( V-N b \right) - \frac{3N}{2k}
\end{align}
Combining the logarithms, we get:
\begin{equation}
  \left( \frac{\partial A}{\partial T} \right)_{V,N} = k \ln (N!) - \frac{3}{2} N k \ln \left( \frac{2 \pi m k T}{h^{2}} \right) + Nk \ln \left( V-N b \right) - \frac{3N}{2k}
\end{equation}
So, the entropy is given by:
\begin{equation}
  S = - \left( \frac{\partial A}{\partial T} \right)_{V,N} = - k \ln (N!) + \frac{3}{2} N k \ln \left( \frac{2 \pi m k T}{h^{2}} \right) - Nk \ln \left( V-N b \right) + \frac{3N}{2k}
\end{equation}
Once you get S, cans use
\begin{equation}
  dS = \frac{dU}{T} + \frac{PdV}{T}
\end{equation}
At fixed volume, this becomes:
\begin{equation}
  dS = \frac{dU}{T}
\end{equation}
We want to find the derivative of the entropy with respect to temperature at constant volume:
\begin{equation}
  \left( \frac{\partial S}{\partial T} \right)_{V} = \frac{1}{T} \left( \frac{\partial U}{\partial T} \right)_{V}
\end{equation}
All but one term in the expression for the entropy is constant with respect to temperature, so we can write:
\begin{equation}
  \left( \frac{\partial S}{\partial T} \right)_{V} = \frac{3Nk}{2T}
\end{equation}
Multiplying by $T$ and then integrating, we get:
\begin{equation}
  U = \frac{3}{2}NkT
\end{equation}
And then we know the definition of heat capacity at constant volume is:
\begin{equation}
  C_{v} = \left( \frac{\partial U}{\partial T} \right)_{V,N}
\end{equation}
So, this is just:
\begin{equation}
  C_{v} = \frac{3}{2}Nk
\end{equation}
Then, to find the heat capacity at constant pressure, we can use the relation:
\begin{equation}
C_{p} = C_{V} - T \left( \frac{\left( \frac{\partial P}{\partial T} \right)_{N,T}^2}{\left( \frac{\partial P}{\partial V} \right)_{N,V}} \right)
\end{equation}
First we need to get an expression for pressure, which is the derivative of the Helmholtz free energy with respect to volume at constant temperature and number of particles:
\begin{equation}
  P = - \left( \frac{\partial A}{\partial V} \right)_{T,N}
\end{equation}
We can start with the expression for the Helmholtz free energy:
\begin{equation}
  A = - k T \left( \ln \left( \frac{1}{N !} \right) + \frac{3}{2} N \ln \left( \frac{2 \pi m k T}{h^{2}} \right) + N \ln \left( V-N b \right) + \frac{a N^{2}}{V k T} \right)
\end{equation}
only the last two terms depend on volume, so we can write:
\begin{equation}
  \left( \frac{\partial A}{\partial V} \right)_{T,N} = - k T \left( \frac{N}{V-N b} - \frac{a N^{2}}{V^{2} k T} \right)
\end{equation}
So, the pressure is given by:
\begin{equation}
  P = k T \left( \frac{N}{V-N b} - \frac{a N^{2}}{V^{2} k T} \right)
\end{equation}
Taking the derivative of this expression with respect to temperature at constant volume and number of particles, we get:
\begin{equation}
  \left( \frac{\partial P}{\partial T} \right)_{V,N} = \frac{N k}{V-N b}
\end{equation}
Now we need to take the derivative of pressure with respect to volume at constant temperature and number of particles:
\begin{equation}
  \left( \frac{\partial P}{\partial V} \right)_{T,N} = - \frac{N k T}{\left( V-N b \right)^{2}} + \frac{2 a N^{2}}{V^{3}}
\end{equation}
Now we can substitute these expressions into the equation for the heat capacity at constant pressure:
\begin{equation}
  C_{p} = C_{V} - T \left( \frac{\left( \frac{\partial P}{\partial T} \right)_{N,T}^2}{\left( \frac{\partial P}{\partial V} \right)_{N,V}} \right)
\end{equation}
\begin{equation}
  C_{p} = \frac{3}{2}Nk - T \left( \frac{\left( \frac{N k}{V-N b} \right)^{2}}{- \frac{N k T}{\left( V-N b \right)^{2}} + \frac{2 a N^{2}}{V^{3}}} \right)
\end{equation}
We simplify with SymPy:
\begin{equation}
  C_{p} =\frac{N k \left(6 N a \left(N b - V\right)^{2} - 5 T V^{3} k\right)}{2 \cdot \left(2 N a \left(N b - V\right)^{2} - T V^{3} k\right)}
\end{equation}
% Inline Python code in the document
\begin{lstlisting}[language=Python]
from sympy import symbols, simplify, Rational, latex

# Redefine the symbols
N, k, T, V, b, a = symbols('N k T V b a')

# Define the numerator and denominator of the fraction
numerator = (N * k / (V - N * b))**2
denominator = -N * k * T / (V - N * b)**2 + 2 * a * N**2 / V**3

# Simplify the fraction
fraction_simplified = simplify(numerator / denominator)

# Define the expression for Cp
Cp = Rational(3,2) * N * k - T * fraction_simplified

# Simplify the expression for Cp
Cp_simplified = simplify(Cp)

# Print the LaTeX syntax of the simplified expression
print(latex(Cp_simplified))

\end{lstlisting}
\section{}
\begin{enumerate}
  \setcounter{enumi}{1}
  \item (30 pts.) ( The probability distribution for a thermodynamic system can be alternatively obtained directly from Gibb's definition of entropy. Entropy is defined as $S=-k \sum_{\nu} P_{\nu} \ln P_{\nu}$ where $P_{\nu}$ is the probability of finding the system in state $\nu$. This is true regardless of the specification of the system. Thus for a $(\mathrm{N}, \mathrm{P}, \mathrm{T})$ system, $\nu$ specifies the system's energy state and volume, while for a $(\mu, \mathrm{V}$, T) system it specifies the energy state and the number of particles. The probability distribution is obtained by the minimization of the characteristic potential (for example Gibbs free energy, Helmholtz free energy, etc.) for a particular specification of the system. Obtain the probability distributions for the $(\mathrm{N}, \mathrm{P}, \mathrm{T})$ and $(\mu, \mathrm{V}, \mathrm{T})$ systems using this approach.
\\
We want to use the method of Lagrange multipliers. We have the simple constraint for both systems that the sum of the probabilities must be equal to 1:
\begin{equation}
  \sum_{\nu} P_{\nu} = 1
\end{equation}
We also have that the entropy for both systems is divined as:
\begin{equation}
  S = - k \sum_{\nu} P_{\nu} \ln P_{\nu}
\end{equation}
\subsection{}
We want to minimize the Gibbs free energy, which is given by:
\begin{equation}
  G = U - TS + PV
\end{equation}
We can write the Lagrangian as:
\begin{equation}
  \mathcal{L} = G - \alpha \left( \sum_{\nu} P_{\nu} - 1 \right)
\end{equation}
We want to minimize the Lagrangian with respect to $P_{\nu}$ . So, we take the derivative of the Lagrangian with respect to $P_{\nu}$ and set it equal to zero:
\begin{equation}
  \pdv{\mathcal{L}}{P_{\nu}} = \pdv{G}{P_{\nu}} - \alpha = 0
\end{equation}
Plugging in the expression for $G$ and also for the $S$ withing $G$, we get:
\begin{equation}
  \pdv{P_{\nu}} \left( U + PV + kT \sum_{\nu} P_{\nu} \ln P_{\nu} \right) - \alpha = E_{\nu} + PV + kT \ln P_{\nu} + kT - \alpha = 0
\end{equation}
Solving for $P_{\nu}$, we get:
\begin{equation}
  P_{\nu} = \exp \left[ \frac{ \alpha - E_{\nu} - PV}{kT} - 1 \right]
\end{equation}
Now, we consider the constraint that the sum of the probabilities must be equal to 1:
\begin{equation}
  \sum_{\nu} P_{\nu} = 1
\end{equation}
We can substitute in the expression for $P_{\nu}$:
\begin{equation}
  \sum_{\nu} \exp \left[ \frac{ \alpha - E_{\nu} - PV}{kT} - 1 \right] = 1
\end{equation}
or:
\begin{equation}
  1 = \exp \left[\beta \alpha - 1 \right] \sum_{\nu} \exp \left[ - \beta E_{\nu} - \beta PV \right]
\end{equation}
So, we define our partition function as:
\begin{equation}
  Q = \sum_{\nu} \exp \left[ - \beta E_{\nu} - \beta PV \right] = \frac{1}{\exp \left[\beta \alpha - 1 \right]}
\end{equation}
So, the probability for this distribution is given by:
\begin{equation}
  P_{\nu} = \frac{\exp \left[ - \beta E_{\nu} - \beta PV \right]}{Q}
\end{equation}
Next, we want to find the probability distribution for the $(\mu, \mathrm{V}, \mathrm{T})$ system. We want to use the Grand potential, which is given by:
\begin{equation}
  \Omega = U - TS - \mu N
\end{equation}
We can write the Lagrangian as:
\begin{equation}
  \mathcal{L} = \Omega - \alpha \left( \sum_{\nu} P_{\nu} - 1 \right)
\end{equation}
We want to minimize the Lagrangian with respect to $P_{\nu}$ . So, we take the derivative of the Lagrangian with respect to $P_{\nu}$ and set it equal to zero:
\begin{equation}
  \pdv{\mathcal{L}}{P_{\nu}} = \pdv{\Omega}{P_{\nu}} - \alpha = 0
\end{equation}
Plugging in the expression for $\Omega$ and also for the $S$ within $\Omega$, we get:
\begin{equation}
  \pdv{P_{\nu}} \left( U - TS - \mu N + kT \sum_{\nu} P_{\nu} \ln P_{\nu} \right) - \alpha = E_{\nu} - \mu N + kT \ln P_{\nu} + kT - \alpha = 0
\end{equation}
Solving for $P_{\nu}$, we get:
\begin{equation}
  P_{\nu} = \exp \left[ \frac{ \alpha - E_{\nu} + \mu N}{kT} - 1 \right]
\end{equation}
Now, we consider the constraint that the sum of the probabilities must be equal to 1:
\begin{equation}
  \sum_{\nu} P_{\nu} = 1
\end{equation}
We can substitute in the expression for $P_{\nu}$:
\begin{equation}
  \sum_{\nu} \exp \left[ \frac{ \alpha - E_{\nu} + \mu N}{kT} - 1 \right] = 1
\end{equation}
or:
\begin{equation}
  1 = \exp \left[\beta \alpha - 1 \right] \sum_{\nu} \exp \left[ - \beta E_{\nu} + \beta \mu N \right]
\end{equation}
So, we define our partition function as:
\begin{equation}
  Q = \sum_{\nu} \exp \left[ - \beta E_{\nu} + \beta \mu N \right] = \frac{1}{\exp \left[\beta \alpha - 1 \right]}
\end{equation}
So, the probability for this distribution is given by:
\begin{equation}
  P_{\nu} = \frac{\exp \left[ - \beta E_{\nu} + \beta \mu N \right]}{Q}
\end{equation}



\section{}
  \item The canonical partition function can be written in terms of energy levels as

\end{enumerate}

$$
Q(N, V, T)=\sum_{E} t_{E}=\sum_{E} \Omega(E, V, N) e^{-\beta E}
$$

$E$ is the total energy of the system. Argue that in the thermodynamic limit, the dominant contribution to the partition function comes from the largest term, denoted by $t_{E}^{*}$, which corresponds to the most probable value of $E$, denoted $E^{*}$.

a) (9 pts.) Show that about $E^{*}, t_{E}$ can be approximated by

$$
t_{E}=t_{E^{*}} \exp \left[\frac{-\left(E-E^{*}\right)^{2}}{2 \sigma_{E}^{2}}\right]
$$

Find $\sigma_{E}^{2}$ by inspection. (Do not make use of the method illustrated in class.)
\subsection{}
We want to maximize $t_{E}$ with respect to $E$. We can do this by taking the derivative of $t_{E}$ with respect to $E$ and setting it equal to zero:
\begin{equation}
  \pdv{t_{E}}{E} = 0
\end{equation}
Now, maximizing $t_{E}$ is equivalent to maximizing $\ln t_{E}$, so we can take the derivative of $\ln t_{E}$ with respect to $E$ and set it equal to zero:
\begin{equation}
  \pdv{\ln t_{E}}{E} = 0 
\end{equation}
Now, we know that $\ln t_{E}$ is given by:
\begin{equation}
  \ln t_{E} = \ln \left( \Omega(E, V, N) e^{-\beta E} \right) = \ln \left( \Omega(E, V, N) \right) - \beta E
\end{equation}
Now, we can Taylor expand the expression for $\ln t_{E}$ about $E^{*}$:
\begin{equation}
  \ln t_{E} = \ln \left( \Omega(E^{*}) \right) - \beta E^{*} + \left( E - E^{*} \right) \left[ \pdv{\ln \Omega(E)}{E} - \beta \right] + \frac{1}{2} \left( E - E^{*} \right)^{2} \left[ \pdv[2]{\ln \Omega(E)}{E} \right] + \cdots
\end{equation}
We neglect the higher order terms from now on.
Now we know that the first derivative of $\ln t_{E}$ with respect to $E$ evaluated at $E^{*}$ and the first derivative of $\ln \Omega(E)$ with respect to $E$ evaluated at $E^{*}$ are both equal to zero, so we can write:
\begin{equation}
  \pdv{\ln t_{E}}{E} \eval_{E^{*}} = 0 = \pdv{\ln \Omega(E)}{E} \eval_{E^{*}} - \beta_{bath}
\end{equation}
Now, the expression for $\ln t_{E}$ with $E$ evaluated at $E^{*}$ is:
\begin{equation}
  \ln t(E^{*}) = \ln \left( \Omega(E^{*}) \right) - \beta_{bath} E^{*}
\end{equation}
So, with this substitution and the fact that the first derivative vanishes at $E^*$, we can write the expression for $\ln t_{E}$ as:
\begin{equation}
  \ln t_{E} = \ln t(E^{*}) + \frac{1}{2} \left( E - E^{*} \right)^{2} \left[ \pdv[2]{\ln \Omega(E)}{E} \right]
\end{equation}
Now, we can exponentiate both sides:
\begin{equation}
  t_{E} = t(E^{*}) \exp \left[ \frac{1}{2} \left( E - E^{*} \right)^{2} \left[ \pdv[2]{\ln \Omega(E)}{E} \right] \right]
\end{equation}
This looks like the expression we want and now we are left to evaluate the second derivative of $\ln \Omega(E)$ with respect to $E$. Now, we know that the entropy is given by:
\begin{equation}
  S = k \ln \Omega(E)
\end{equation}
So, the second derivative of $\ln \Omega(E)$ with respect to $E$ is given by:
\begin{equation}
  \pdv[2]{\ln \Omega(E)}{E} = \frac{1}{k} \pdv[2]{S}{E}
\end{equation}
But we know the first derivative of $S$ with respect to $E$ is just the inverse temperature:
\begin{equation}
  \pdv{S}{E} = \frac{1}{T}
\end{equation}
So, the second derivative of $S$ with respect to $E$ is just:
\begin{equation}
  \pdv[2]{S}{E} = - \frac{1}{T^{2}} \pdv{T}{E}
\end{equation}
Now, we know that the first derivative of $E$ with respect to $T$ is just the heat capacity at constant volume:
\begin{equation}
  \pdv{E}{T} = C_{V}
\end{equation}
So, the second derivative of $S$ with respect to $E$ is given by:
\begin{equation}
  \pdv[2]{S}{E} = - \frac{1}{T^{2}} \pdv{T}{E} = - \frac{1}{T^{2}} \frac{1}{C_{V}}
\end{equation}
Substituting this expression into the expression for $t_{E}$, we get:
\begin{equation}
  t_{E} = t(E^{*}) \exp \left[ \frac{1}{2} \left( E - E^{*} \right)^{2} \left[ - \frac{1}{kT^{2} C_{V}} \right] \right]
\end{equation}
So, by inspection, we can see that:
\begin{equation}
  \sigma_{E}^{2} = kT^{2} C_{V}
\end{equation}
\subsection{}
b) (9 pts.) Now using the relation, $\sigma_{E}^{2}=k T^{2} C_{V}$ (obtained in class), and the result of part a), show that the error committed in replacing $\ln Q$ by $\ln t_{E^{*}}$ is only of order $\ln N$. (Hint: Consider the energy spacing as nearly continuous...)\\
We know that the partition function is given by:
\begin{equation}
  Q = \sum_{E} t_{E} = t_{E^{*}} \sum_{E} \exp \left[ \frac{-\left(E-E^{*}\right)^{2}}{2 \sigma_{E}^{2}} \right]
\end{equation}
So, taking the logarithm of both sides and subtracting $\ln t_{E^{*}}$ from both sides, we get:
\begin{equation}
  \ln Q - \ln t_{E^{*}} = \ln \left( \sum_{E} \exp \left[ \frac{-\left(E-E^{*}\right)^{2}}{2 \sigma_{E}^{2}} \right] \right)
\end{equation}
Since the energy spacing is nearly continuous, we can replace the sum with an integral to a good approximation:
\begin{equation}
  \ln Q - \ln t_{E^{*}} = \ln \left( \int \exp \left[ \frac{-\left(E-E^{*}\right)^{2}}{2 \sigma_{E}^{2}} \right] dE \right)
\end{equation}
Now, the integral of a Gaussian is given by:
\begin{equation}
  \int_{-\infty}^{\infty} \exp \left[ - \frac{\left( x - \mu \right)^{2}}{2 \sigma^{2}} \right] dx = \sqrt{2 \pi \sigma^{2}}
\end{equation}
So, we can write:
\begin{equation}
  \ln Q - \ln t_{E^{*}} = \ln \left( \sqrt{2 \pi \sigma_{E}^{2}} \right) = \ln \left( \sqrt{2 \pi k T^{2} C_{V}} \right) = \frac{1}{2} \ln \left( 2 \pi k T^{2} C_{V} \right)
\end{equation}
$k,T$ are intensive, but $C_{V}$ is extensive, so our result scales like $\frac{1}{2} \ln C_{V}$. We know that $C_{V}$ scales like $N$, so our result scales like $\frac{1}{2} \ln N$, which is of order $\ln N$.
\section{}
\begin{enumerate}
  \setcounter{enumi}{3}
  \item (from Chandler 3.18) Consider a system of $N$ localized non-interacting spins in a magnetic field $H$. Each spin has a magnetic moment of size $\mu$, and each can point either parallel or antiparallel to the field. Thus the energy of a particular state is
\end{enumerate}

$$
E_{\nu}=\sum_{i=1}^{N}-n_{i} \mu H, \quad n_{i}= \pm 1
$$

where the vector $n=\left\{n_{i}\right\}$ specifies the state, $\nu$, and $n_{i} \mu$ is the magnetic moment in the direction of the field.

(a) (9 pts.) Determine the average internal energy of this system as a function of $\beta, H$, and $N$ by employing an ensemble characterized by these variables.
\subsection{}
The partition function will be given by:
\begin{equation}
  Q = \sum_{\nu} \exp \left[ - \beta E_{\nu} \right] = \sum_{\nu} \exp \left[ \beta \mu H \sum_{i=1}^{N} n_{i} \right]
\end{equation}
We can take the sum out of the exponent, which becomes a product:
\begin{equation}
  Q = \sum_{\nu} \prod_{i=1}^{N} \exp \left[ \beta \mu H n_{i} \right]
\end{equation}
The sum over $\nu$ is just a sum over all possible values of $n_{i}$, which are $\pm 1$. So, we can write a series of sums:
\begin{equation}
  Q = \sum_{n_{1} = \pm 1} \sum_{n_{2} = \pm 1} \cdots \sum_{n_{N} = \pm 1} \prod_{i=1}^{N} \exp \left[ \beta \mu H n_{i} \right]
\end{equation}
We can swap the product with a sum over $i$:
\begin{equation}
  Q = \prod_{i=1}^{N}\sum_{n_{i} = \pm 1} \exp \left[ \beta \mu H n_{i} \right] = \prod_{i = 1}^{N} \left( \exp \left[ \beta \mu H \right] + \exp \left[ - \beta \mu H \right] \right)
\end{equation}
We can get the internal energy by taking the derivative of the partition function with respect to $\beta$:
\begin{equation}
  U = - \pdv{\ln Q}{\beta} = - \pdv{\beta} \left( \sum_{i = 1}^{N} \ln \left( \exp \left[ \beta \mu H \right] + \exp \left[ - \beta \mu H \right] \right) \right) = - \sum_{i = 1}^{N}\left( \frac{\mu H \exp \left[ \beta \mu H \right] - \mu H \exp \left[ - \beta \mu H \right]}{\exp \left[ \beta \mu H \right] + \exp \left[ - \beta \mu H \right]} \right)
\end{equation}
We recognize the formula for the hyperbolic tangent:
\begin{equation}
  U = - N \mu H \tanh \left( \beta \mu H \right)
\end{equation}
\subsection{}
(b) (8 pts.) Determine the average entropy of this system as a function of $\beta, H$, and $N$.\\
The Helmholtz free energy is given by:
\begin{equation}
  A = U - TS = -kT \ln Q
\end{equation}
Isolating for $S$, on the RHS we get:
\begin{equation}
  S = \frac{U}{T} + k \ln Q
\end{equation}
We can substitute in the expression for $U$ and for $Q$:
\begin{equation}
  S = - k N \beta \mu H \tanh \left( \beta \mu H \right) + k \ln \left( \prod_{i = 1}^{N} \left( \exp \left[ \beta \mu H \right] + \exp \left[ - \beta \mu H \right] \right) \right)
\end{equation}
Bringing the product outside of the logarithm turns it into a sum and that just becomes $N$:
\begin{equation}
  S = - k N \beta \mu H \tanh \left( \beta \mu H \right) + k N \ln \left( \exp \left[ \beta \mu H \right] + \exp \left[ - \beta \mu H \right] \right)
\end{equation}
\subsection{}


(c) (7 pts.) Determine the behavior of the energy and entropy for this system as $T \rightarrow 0$.\\
As $T \rightarrow 0$, $\beta \rightarrow \infty$, so $\tanh \left( \beta \mu H \right) \rightarrow 1$. So, the energy will be given by:
\begin{equation}
  U = - N \mu H
\end{equation}
While for this limit, the second term of the second term in the entropy will go to zero, so the entropy in the limit will be given by:
\begin{equation}
  S \rightarrow -kN \beta \mu H + kN \beta \mu H = 0
\end{equation}

\end{document}