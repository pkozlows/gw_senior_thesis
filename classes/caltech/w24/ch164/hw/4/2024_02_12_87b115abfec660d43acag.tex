\documentclass[12pt]{article}
\usepackage[utf8]{inputenc}
\usepackage[T1]{fontenc}
\usepackage{amsmath}
\usepackage{amsfonts}
\usepackage{amssymb}
\usepackage[version=4]{mhchem}
\usepackage{stmaryrd}
\usepackage{physics}
On Friday
\usepackage{listings} % Required for insertion of code
\usepackage{xcolor} % Required for custom colors

% Define custom colors
\definecolor{codegreen}{rgb}{0,0.6,0}
\definecolor{codegray}{rgb}{0.5,0.5,0.5}
\definecolor{codepurple}{rgb}{0.58,0,0.82}
\definecolor{backcolour}{rgb}{0.95,0.95,0.92}

% Setup the style for code listings
\lstdefinestyle{mystyle}{
    backgroundcolor=\color{backcolour},   
    commentstyle=\color{codegreen},
    keywordstyle=\color{magenta},
    numberstyle=\tiny\color{codegray},
    stringstyle=\color{codepurple},
    basicstyle=\ttfamily\footnotesize,
    breakatwhitespace=false,         
    breaklines=true,                 
    captionpos=b,                    
    keepspaces=true,                 
    numbers=left,                    
    numbersep=5pt,                  
    showspaces=false,                
    showstringspaces=false,
    showtabs=false,                  
    tabsize=2
}

% Activate the style
\lstset{style=mystyle}


\title{Ch/ChE 164 Winter 2024 }


\author{Homework Problem Set \#4}
\date{}


\begin{document}
\maketitle
Due Date: Thursday, February 15, 2024 @ 11:59pm PT

For all problems, please consider reasonable simplifications of your final results.
\section{}
\begin{enumerate}
  \item (15 pts.) (Adapted from Callen). Consider a mixture of two non-identical monatomic ideal gases.
\end{enumerate}
\subsection{}
\begin{itemize}
  \item Starting from the expression for the grand canonical partition function and taking the limit of small fugacity, show that the canonical partition function $Z$ is factorizable and
\end{itemize}

$$
Z=Z_{1} Z_{2}=\frac{1}{N_{1} !} q_{1}^{N_{1}} \frac{1}{N_{2} !} q_{2}^{N_{2}}
$$

(You may wish to use the occupancy representation $\mid n_{1} m_{1}, n_{2} m_{2} \ldots$ ), where $n_{1}$ denotes occupancy of energy level 1 of gas 1 , and $m_{1}$ denotes occupancy of energy level 1 of gas 2 , etc.).\\
First, I will say that an individual partition function for each molecule is given by:
\begin{equation}
  Z_i = \sum_{n_i} e^{-\beta \epsilon_i n_i}
\end{equation}
The occupations of the first and second gases are given by:
\begin{equation}
  n_1, n_2,...
\end{equation}
\begin{equation}
  m_1, m_2,...
\end{equation}
The number of particles in the first and second gases are given by:
\begin{equation}
  N_1 = \sum_{i} n_i
\end{equation}
\begin{equation}
  N_2 = \sum_{i} m_i
\end{equation}
Total energy of the first and second gases are given by:
\begin{equation}
  U_1 = \sum_{i} n_i \epsilon_i^{(1)}
\end{equation}
\begin{equation}
  U_2 = \sum_{i} m_i \epsilon_i^{(2)}
\end{equation}
Their sum is
\begin{equation}
  U = U_1 + U_2 = \sum_{i} n_i \epsilon_i^{(1)} + \sum_{i} m_i \epsilon_i^{(2)}
\end{equation}
The grand canonical partition function is given by:
\begin{equation}
  \begin{aligned}
    \Xi &= \sum_{\nu} e^{-\beta(U_{\nu}) +  \beta \mu N_{1} + \beta \mu N_{2}} \\
\end{aligned}
\end{equation}
Adding the various sums to the exponent gives:
\begin{equation}
  \Xi = \sum_{{n}} e^{-\beta \sum_{i} n_i \epsilon_i^{(1)} +  \beta \mu N_{1}} \sum_{{m}}e^{-\beta \sum_{i} m_i \epsilon_i^{(2)} +  \beta \mu N_{2}}
\end{equation}
where $\mu^{(1)}$ and $\mu^{(2)}$ are the chemical potentials of the first and second gases, respectively. We can factor this in two separate sums each over the individual molecule:
\begin{equation}
  \Xi = \sum_{{n}} e^{-\beta \sum_{i} n_i \epsilon_i^{(1)} +  \beta \mu^{(1)} N_{1}} \sum_{{m}}e^{-\beta \sum_{i} m_i \epsilon_i^{(2)} +  \beta \mu^{(2)} N_{2}}
\end{equation}
Now, let us only consider only the first sum:
\begin{equation}
  \Xi = \sum_{{n}} e^{-\beta \sum_{i} n_i \epsilon_i^{(1)} +  \beta \mu^{(1)} N_{1}} = \prod_{i} \sum_{n_i} e^{-\beta n_i (\epsilon_i^{(1)} - \mu^{(1)})}
\end{equation}
We factor the terms inside of the exponent:
\begin{equation}
  \sum_{n_i} e^{-\beta n_i (\epsilon_i^{(1)} - \mu^{(1)})}
\end{equation}
The term that looks like the fugacity is independent of the product over the states, so we can factor it out:
\begin{equation}
  \sum_{n_i} e^{-\beta n_i (\epsilon_i^{(1)} - \mu^{(1)})} = \prod_{i} \sum_{n_i} e^{-\beta n_i (\epsilon_i^{(1)} - \mu^{(1)})}
\end{equation} 
In summary, we have the total grand partition function as a product of the individual ones:
\begin{equation}
  \Xi = \prod_{i} \sum_{n_i} e^{-\beta n_i (\epsilon_i^{(1)} - \mu^{(1)})} \prod_{i} \sum_{m_i} e^{-\beta m_i (\epsilon_i^{(2)} - \mu^{(2)})} = \Xi_1 \Xi_2
\end{equation}
Now we can express the grant canonical partition function for fermions or bosons succinctly as:
\begin{equation}
  \Xi = \prod_{i} \left( 1 \mp e^{-\beta(\epsilon_i - \mu)} \right)^{\mp}
\end{equation}
where the upper sign is for fermions and the lower sign is for bosons. Taking the log of this:
\begin{equation}
  \ln \Xi = \mp \sum_{i} \ln \left( 1 \mp e^{-\beta(\epsilon_i - \mu)} \right)
\end{equation}
In the timid of low frugality, the term of $e^{\beta \mu}$ is small, so we can expand the log term using:
\begin{equation}
  \ln(1+x) = x - \frac{x^2}{2} + \frac{x^3}{3} - \ldots
\end{equation}
Keepi only the first term:
\begin{equation}
  \ln \Xi = \mp \sum_{i} \left( \mp e^{-\beta(\epsilon_i - \mu)} \right)
\end{equation}
Since the term that depends on $\mu$ is independent of the sum, we can factor it out:
\begin{equation}
  \ln \Xi = \mp \sum_{i} \left( \mp e^{-\beta(\epsilon_i - \mu)} \right) = e^{\beta \mu} \sum_{i} e^{-\beta \epsilon_i} 
\end{equation}
We recognize the summation as the single partial partition function $q$:
\begin{equation}
  \ln \Xi = e^{\beta \mu} q
\end{equation}
Now, we can exponent both sites to get:
\begin{equation}
  \Xi = e^{e^{\beta \mu} q}
\end{equation}
The power series expansion of the exponential function is given by:
\begin{equation}
  e^x = \sum_{N = 0}^{\infty} \frac{x^N}{N!} 
\end{equation}
So, we can write the partition function as:
\begin{equation}
  \Xi = \sum_{N = 0}^{\infty} \frac{e^{\beta \mu N} q^N}{N!}
\end{equation}
This is the partition function for only one of the molecules, so the one for both is given by:
\begin{equation}
  \Xi = \sum_{N_1 = 0}^{\infty} \sum_{N_2 = 0}^{\infty} \frac{e^{\beta \mu^{(1)} N_1} q_1^{N_1}}{N_1!} \frac{e^{\beta \mu^{(2)} N_2} q_2^{N_2}}{N_2!}
\end{equation}
We recognize the canonical partition function for each molecule:
\begin{equation}
  Z_1 = \left(\frac {1}{N_1!} q_1^{N_1}\right),
Z_2 = \left(\frac {1}{N_2!} q_2^{N_2}\right)
\end{equation}
So the grant canonical partition function for the entire system can be written as:
\begin{equation}
  \Xi = \sum_{N}^{
\infty } Z_1 Z_2 e^{\beta \mu^{(1)} N_1} e^{\beta \mu^{(2)} N_2} 
\end{equation}
or summing the chemical potentials:
\begin{equation}
  \Xi = \sum_{N}^{\infty} Z_1 Z_2 e^{\beta \mu N}
\end{equation}
So, we found the desired result of:
\begin{equation}
  Z = Z_1 Z_2 = \frac{1}{N_1!} q^{N_1} \frac{1}{N_2!} q^{N_2}
\end{equation}
\subsection{}
\begin{itemize}
  \item Compute the entropy and show that (comparing to the entropy of the two separate gases) there is an entropy of mixing of the form
\end{itemize}

$$
S_{\text {mixing }}=\left(-x_{1} \log x_{1}-x_{2} \log x_{2}\right) N k
$$

where $N$ is the total number of particles.
\section{}
\begin{enumerate}
  \setcounter{enumi}{1}
  \item In class we derived the heat capacity of the Fermi gas at low temperature by an intuitive argument, which $C_{v} \sim N k O\left(T / T_{F}\right)$. Here we will derive the precise form and constants (adapted from Callen).
\end{enumerate}

Denote the Fermi-Dirac distribution at temperature $T$ as $f(\epsilon, T)$ and the (temperature dependent) chemical potential by $\mu$ (note this is not the Fermi energy $\epsilon_{F}$ except when $T=0$ ). We will first derive a general result for an integral of the form (Sommerfeld expansion)

$$
I \equiv \int_{0}^{\infty} \phi(\epsilon) f(\epsilon, T) d \epsilon=\int_{0}^{\mu} \phi(\epsilon) d \epsilon+\frac{\pi^{2}}{6}(k T)^{2} \phi^{\prime}(\mu)+\frac{7 \pi^{4}}{360}(k T)^{4} \phi^{\prime \prime \prime}(\mu)+\ldots
$$

a) (10 pts.) Integrate $I$ by parts, and let $\Phi \equiv \int_{0}^{\epsilon} \phi\left(\epsilon^{\prime}\right) d \epsilon^{\prime}$. Then expanding $\Phi(\epsilon)$ in a power series in $\epsilon-\mu$ to third order, deduce
$$
I=-\sum_{m=0}^{\infty} \frac{1}{m !} \frac{d^{m} \Phi(\mu)}{d \mu^{m}} I_{m}
$$

where $I_{m}=\int_{0}^{\infty}(\epsilon-\mu)^{m} \frac{d f}{d \epsilon} d \epsilon=-\beta^{-m} \int_{-\beta \mu}^{\infty} \frac{e^{x}}{\left(e^{x}+1\right)^{2}} x^{m} d x$

\subsection{}
The fd distribution is given by:
\begin{equation}
  f(\epsilon, T) = \frac{1}{e^{\beta(\epsilon - \mu)} + 1}
\end{equation}
where $\beta = 1/kT$.
For the integration by parts we choose $u = f(\epsilon, T) \rightarrow du = \frac{df}{d\epsilon} d\epsilon$ and $dv = \phi(\epsilon) d\epsilon \rightarrow v = \Phi(\epsilon)$. The integral becomes:
\begin{equation}
  I = \left. \Phi(\epsilon) f(\epsilon, T) \right|_{0}^{\infty} - \int_{0}^{\infty} \Phi(\epsilon) \frac{df}{d\epsilon} d\epsilon
\end{equation}
The first term vanishes at the limits of integration, so we are left with:
\begin{equation}
  I = -\int_{0}^{\infty} \Phi(\epsilon) \frac{df}{d\epsilon} d\epsilon
\end{equation}
Now, we consider the form of the power series expansion of $\Phi(\epsilon)$ around $\mu$:
\begin{equation}
  \Phi(\epsilon) = \sum_{m=0}^{\infty} \frac{1}{m!}\frac{d^m\Phi(\mu)}{d\mu^m}(\epsilon - \mu)^m
\end{equation}
We can substitute this into the integral:
\begin{equation}
  I = -\int_{0}^{\infty} \sum_{m=0}^{\infty} \frac{1}{m!}\frac{d^m\Phi(\mu)}{d\mu^m}(\epsilon - \mu)^m \frac{df}{d\epsilon} d\epsilon
\end{equation}
We can take some of the summation terms outside of the integral:
\begin{equation}
  I = -\sum_{m=0}^{\infty} \frac{1}{m!}\frac{d^m\Phi(\mu)}{d\mu^m} \int_{0}^{\infty} (\epsilon - \mu)^m \frac{df}{d\epsilon} d\epsilon
\end{equation}
We can recognize the integral as the $I_m$ term:
\begin{equation}
  I = -\sum_{m=0}^{\infty} \frac{1}{m!}\frac{d^m\Phi(\mu)}{d\mu^m} I_m
\end{equation}


b) (5 pts.) Show that only an exponentially small error is made by taking the lower limit of integration as $-\infty$, and that then all terms with $m$ odd vanish.
\subsection{}
We have the integral:
\begin{equation}
  I_m = \int_{0}^{\infty} (\epsilon - \mu)^m \frac{df}{d\epsilon} d\epsilon
\end{equation}
The derivative of the fd distribution with respect to energy is given by:
\begin{equation}
  \frac{df}{d\epsilon} = -\beta e^{\beta \left(\epsilon - \mu\right)} \left(e^{\beta \left(\epsilon - \mu\right)} + 1\right)^{-2}
\end{equation}
We plug this into the integral:
\begin{equation}
  I_m = \int_{0}^{\infty} (\epsilon - \mu)^m \frac{-\beta e^{\beta \left(\epsilon - \mu\right)}}{\left(e^{\beta \left(\epsilon - \mu\right)} + 1\right)^2} d\epsilon
\end{equation}
We can make the substitution $x = \beta(\epsilon - \mu)$, so that $d\epsilon = \frac{1}{\beta} dx$. Also, the limits of integration become $-\beta\mu$ and $\infty$. The integral becomes:
\begin{equation}
  I_m = \int_{-\beta\mu}^{\infty} -x^m \frac{e^{x}}{\left(e^{x} + 1\right)^2} \beta ^{-m} dx
\end{equation}
We want to show that this interval can be well approximated by the integral from $-\infty$ to $\infty$. That is, we want to show that the integral between $-\beta \mu$ and $-\infty$ is exponentially small. We know that:
\begin{equation}
  \int_{-\infty}^{-\beta \mu} x^m e^x dx \geq \int_{-\infty}^{-\beta \mu} x^m \frac{e^x}{(e^x + 1)^2} dx
\end{equation}
because the smallest value for the denominator is 1.
Considering the former integral, The exponential term supresses the integrand within its bounds, so this is very small and within this domain, this integral effectively finishes. So we can well approximate:
\begin{equation}
  I_m \approx - \beta ^{-m}\int_{-\infty}^{\infty} x^m \frac{e^{x}}{\left(e^{x} + 1\right)^2} dx
\end{equation}
Since the exponential fraction is an even function, whenever the polynomial term is odd, the integrand is odd and so the integral vanities.\\
c) (5 pts.) Evaluate the first two non-vanishing terms and show that this agrees with the expansion of $I$.
\subsection{}
The first non vanishing term will be given by:
\begin{equation}
  I_0 = -\int_{-\infty}^{\infty} \frac{e^{x}}{\left(e^{x} + 1\right)^2} dx
\end{equation}
Now, since the whole term is:
\begin{equation}
  I = -\sum_{m=0}^{\infty} \frac{1}{m!}\frac{d^m\Phi(\mu)}{d\mu^m} I_m
\end{equation}
We can see that the first term is:
\begin{equation}
  I = +\Phi(\mu) I_0 = \Phi(\mu)\left[(e^{\beta \mu} + 1)\right]|_{-\infty}^{\infty} = \Phi(\mu)
\end{equation}
The second term is given by:
\begin{equation}
   - \frac{1}{2} \frac{d^2\Phi(\mu)}{d\varepsilon ^2} I_2
\end{equation}  
and then we have:
\begin{equation}
  I_{2} = -\beta^{-2} \int_{-\infty}^{\infty} x^2 \frac{e^{x}}{\left(e^{x} + 1\right)^2} dx
\end{equation}
Using SymPy, the above integral is $\pi^2/3$. So we have for this term: 
\begin{equation}
  - \frac{1}{2} \frac{d^2\Phi(\mu)}{d\varepsilon ^2} I_2 = \beta ^{2} \frac{1}{2} \frac{d^2\Phi(\mu)}{d\varepsilon ^2} \frac{\pi^2}{3} = (kT)^2 \frac{\pi^2}{6} \phi ^{\prime}(\mu)
\end{equation}
as given in the equation.

d) (10 pts.) Using the result for $I$, express $N$ in the form of such an integral and obtain an expansion for $N(V, T, \mu)$ in terms of $k T / \mu$ (to second order). Verify that $T \rightarrow 0$ yields the relation between $N$ and $\epsilon_{F}$ derived in class.
\subsection{}
The number of particles is given by:
\begin{equation}
  N = \sum_{\alpha } \expval{n_{\alpha}}
\end{equation}
which we can express in terms of an integral of the distribution mortified by the density of states:
\begin{equation}
  N = \int_{0}^{\infty} g(\epsilon) f(\epsilon, T) d\epsilon
\end{equation}
Form Callen, we have for the density of states:
\begin{equation}
  g(\epsilon) = \frac{V}{4 \pi^2}\left(\frac{2 m}{\hbar^2}\right)^{3 / 2} \varepsilon^{1 / 2}
\end{equation}
So the derivative of this with respect to $\epsilon$ is:
\begin{equation}
  g^{\prime}(\epsilon) = \frac{V}{4 \pi^2}\left(\frac{2 m}{\hbar^2}\right)^{3 / 2} \frac{1}{2} \varepsilon^{-1 / 2}
\end{equation}
e) (10 pts.) Invert this relationship to obtain $\mu(T)$ as a function of $k T / \epsilon_{F}$ (to second order) for fixed $N$.

f) (5 pts.) Similarly obtain an expansion for the internal energy $E$ as a function of $k T / \mu$ (to second order).

g) (5 pts.) Substituting in $\mu(T)$ into the energy expansion, obtain an expansion of $E$ in $k T / \epsilon_{F}$ to second order, and thus $C_{v}$. Hence see why we skipped the detailed computation in class.
e) (10 pts.) Invert this relationship to obtain $\mu(T)$ as a function of $k T / \epsilon_{F}$ (to second order) for fixed $N$.

f) (5 pts.) Similarly obtain an expansion for the internal energy $E$ as a function of $k T / \mu$ (to second order).

g) (5 pts.) Substituting in $\mu(T)$ into the energy expansion, obtain an expansion of $E$ in $k T / \epsilon_{F}$ to second order, and thus $C_{v}$. Hence see why we skipped the detailed computation in class.
\section{}
\begin{enumerate}
  \setcounter{enumi}{2}
  \item (20 pts.) Show that for the Bose-Einstein and Fermi-Dirac gas at low density and/or high temperature the equation of state is given by
\end{enumerate}

$$
p=k T \rho\left(1 \mp \frac{\rho \Lambda^{3}}{2^{5 / 2}}+\ldots\right)
$$

where $\Lambda=h / \sqrt{2 \pi m k T}$ is the thermal de Broglie wavelength, and the upper (lower) sign is for the Bose-Einstein (Fermi-Dirac) gas.

We start with the thermodynamic identity for the pressure $p$ in terms of the grand canonical partition function $\Xi$:
\begin{equation}
    pV\beta = \ln \Xi
\end{equation}
where $\beta = \frac{1}{kT}$, with $k$ being the Boltzmann constant and $T$ the temperature.

The grand canonical partition function for Bose-Einstein (BE) and Fermi-Dirac (FD) gases are given by:
\begin{align}
    \ln \Xi_{\text{BE}} &= \sum_{\nu} \ln \left(1 - e^{-\beta(\epsilon_{\nu} - \mu)}\right) \\
    \ln \Xi_{\text{FD}} &= \sum_{\nu} \ln \left(1 + e^{-\beta(\epsilon_{\nu} - \mu)}\right)
\end{align}
where $\epsilon_{\nu}$ is the energy of state $\nu$ and $\mu$ is the chemical potential.
We will go through the derivation for the Fermi-Dirac gas and then show how similar steps repeat for the Bose-Einstein gas, noting that a comprehensive deviation was given in lecture 8.\\
After performing a Taylor expansion to second order in the limit of small fugacity inside of the integral in spherical coordinates for this, we get:
\begin{equation}
\ln \Xi=\frac{V}{\Lambda^3}(2 s+1) f_{5 / 2}(\zeta)
\end{equation}
So, we have arrived at:
\begin{equation}
    p(T, \mu)\beta = \frac{2s+1}{\Lambda^3}f_{5/2}(\zeta)
\end{equation}
where:
\begin{equation}
f_{5 / 2}(\zeta)=\zeta-\frac{\zeta^2}{2^{5 / 2}}
\end{equation}
to the 2nd order.
For simplicity, we assume that $s=0$.
But now we want to get rid of the fugacity, and turn it into a density. For this, we can use the virial expansion derived in the notes:
\begin{equation}
\zeta=\Lambda^3 \rho+\frac{\left(\Lambda^3 \rho\right)^2}{2^{3 / 2}}
\end{equation}

\begin{equation}
    p = kT\rho\left(1 + \frac{\rho\Lambda^{3}}{2^{5/2}} + \ldots\right)
\end{equation}
For the BE case, we will have:
\begin{equation}
  g_{5 / 2}(\zeta)=\zeta+\frac{\zeta^2}{2^{5 / 2}}
\end{equation}
so the sign is flipped and we get a final result of:
\begin{equation}
    p = kT\rho\left(1 - \frac{\rho\Lambda^{3}}{2^{5/2}} + \ldots\right)
\end{equation}

% We know that the particle number averages for the distributions are given by:
% \begin{equation}
%   \begin{aligned}
%     \bar{n}_{\text{BE}} &= \frac{1}{e^{\beta(\epsilon - \mu)} - 1} \\
%     \bar{n}_{\text{FD}} &= \frac{1}{e^{\beta(\epsilon - \mu)} + 1}
%   \end{aligned}
% \end{equation}

\end{document}