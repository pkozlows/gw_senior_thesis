%%%%%%%%%%%%
%% Please rename this main.tex file and the output PDF to
%% [lastname_firstname_graduationyear]
%% before submission.
%%
%% This .tex file is for use with BibLaTeX. Please use
%% main-bibtex.tex instead if you prefer BibTeX.
%%%%%%%%%%%%

\documentclass[12pt]{caltech_thesis}
\usepackage[hyphens]{url}
\usepackage{amsmath}
\usepackage{lipsum}
\usepackage{physics}
\usepackage{graphicx}

\usepackage{todonotes}

%% Tentative: newtx for better-looking Times
\usepackage[utf8]{inputenc}
\usepackage[T1]{fontenc}
\usepackage{newtxtext,newtxmath}

% Must use biblatex to produce the Published Contents and Contributions, per-chapter bibliography (if desired), etc.
\usepackage[
    backend=biber,natbib,
    % IMPORTANT: load a style suitable for your discipline
    style=authoryear
]{biblatex}
\usepackage[colorlinks=true, linkcolor=blue, citecolor=blue, urlcolor=blue]{hyperref}
\usepackage{nomencl}
\makenomenclature
\renewcommand{\nomname}{Notation}

%% This code creates the groups
% -----------------------------------------
\usepackage{etoolbox}
\renewcommand\nomgroup[1]{%
  \item[\bfseries
    \ifstrequal{#1}{I}{Integrals}{%
    \ifstrequal{#1}{O}{Orbital indices}{}}%
  ]
}
% -----------------------------------------




% Name of your .bib file(s)
\addbibresource{example.bib}
\addbibresource{ownpubs.bib}

\begin{document}

\nomenclature[O]{$p,q,r,s$}{General}
\nomenclature[O]{$i,j,k,l$}{Occupied}
\nomenclature[O]{$a,b,c,d$}{Virtual}

% Do remember to remove the square bracket!
\title{[Thesis Title]}
\author{[Your Full Name]}

\degreeaward{[Name of Degree]}                 % Degree to be awarded
\university{California Institute of Technology}    % Institution name
\address{Pasadena, California}                     % Institution address
\unilogo{caltech.png}                                 % Institution logo
\copyyear{[Year Degree Conferred]}  % Year (of graduation) on diploma
\defenddate{[Exact Date]}          % Date of defense

\orcid{[Author ORCID]}

%% IMPORTANT: Select ONE of the rights statement below.
\rightsstatement{All rights reserved\todo[size=\footnotesize]{Choose one from the choices in the source code!! And delete this \texttt{todo} when you're done that. :-)}}
% \rightsstatement{All rights reserved except where otherwise noted}
% \rightsstatement{Some rights reserved. This thesis is distributed under a [name license, e.g., ``Creative Commons Attribution-NonCommercial-ShareAlike License'']}

%%  If you'd like to remove the Caltech logo from your title page, simply remove the "[logo]" text from the maketitle command
\maketitle[logo]
%\maketitle

\begin{acknowledgements} 	 
   [Add acknowledgements here. If you do not wish to add any to your thesis, you may simply add a blank titled Acknowledgements page.]
\end{acknowledgements}

\begin{abstract}
   [This abstract must provide a succinct and informative condensation of your work. Candidates are welcome to prepare a lengthier abstract for inclusion in the dissertation, and provide a shorter one in the CaltechTHESIS record.]
\end{abstract}

\tableofcontents
\listoffigures
\listoftables
\printnomenclature

\mainmatter

\chapter{Introduction}
\section{Motivation for $GW$: moving past a mean-field description}
$GW$ is a methodology that derives from the queen's function formalism that attempts to treat the correlation between electrons. 

Here's an example of a citation \citep{GMP81}. Here's another \citep{PP98}. These will appear in the big bibliography at the end of the thesis.
\index{bibliography}

If you're new to \LaTeX{} and would like to begin by learning the basics, please see our free online course available at:\\ \url{https://www.overleaf.com/latex/learn/free-online-introduction-to-latex-part-1} \index{LaTeX@\LaTeX}

You can define nomenclatures \index{nomenclature} as you talk about key terms in your thesis. So what's a galaxy? \nomenclature{Galaxy}{A system of stars independent from all other systems}


\section{This is a Section}
\lipsum[1-2]

\begin{figure}[hbt!]
\centering
\includegraphics[width=.3\textwidth]{caltech.png}
\caption{This is a figure}\label{fig:logo}
\index{figures}
\end{figure}

\subsection{This is a subsection}

\begin{table}[hbt!]
\centering
\begin{tabular}{ll}
\hline
Area & Count\\
\hline
North & 100\\
South & 200\\
East & 80\\
West & 140\\
\hline
\end{tabular}
\caption{This is a table}\label{tab:sample}
\index{tables}
\end{table}

\lipsum[3] 

\lipsum[4-5] 

Here's an endnote.\endnote{Endnotes are notes that you can use to explain text in a document.}

\section{This is Another Section}
\lipsum[6-7] 

\chapter{This is the Second Chapter}
\begin{refsection}
If you'd like to have separate bibliographies at the end of each chapter, put a \verb|refsection| around the material of each chapter, then cite as usual -- e.g.~\citep{GMP81,Ful83}. Then do a \verb|\printbibliography| just before the \verb|refsection| ends. \index{bibliography!by chapter}

\printbibliography[heading=subbibliography]
\end{refsection}


\chapter{This is the Third Chapter}

\publishedas{Cahn:etal:2015}

[You can have chapters that were published as part of your thesis. The text style of the body should be single column, as it was submitted to the publisher, not formatted as the publisher did.]

\chapter{This is the Fourth Chapter}
\chapter{This is the Fifth Chapter}
\chapter{This is the Sixth Chapter}
\chapter{This is the Seventh Chapter}
\chapter{This is the Eighth Chapter}

\printbibliography[heading=bibintoc]







\end{document}
