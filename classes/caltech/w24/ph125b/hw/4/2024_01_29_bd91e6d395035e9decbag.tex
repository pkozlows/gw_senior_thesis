\documentclass[12pt]{article}
\usepackage[utf8]{inputenc}
\usepackage[T1]{fontenc}
\usepackage{amsmath}
\usepackage{physics}
\usepackage{amsfonts}
\usepackage{amssymb}
\usepackage[version=4]{mhchem}
\usepackage{stmaryrd}
\usepackage{graphicx}
\usepackage[export]{adjustbox}
\graphicspath{ {./images/} }

\usepackage{listings} % Required for insertion of code
\usepackage{xcolor} % Required for custom colors

% Define custom colors
\definecolor{codegreen}{rgb}{0,0.6,0}
\definecolor{codegray}{rgb}{0.5,0.5,0.5}
\definecolor{codepurple}{rgb}{0.58,0,0.82}
\definecolor{backcolour}{rgb}{0.95,0.95,0.92}

% Setup the style for code listings
\lstdefinestyle{mystyle}{
    backgroundcolor=\color{backcolour},   
    commentstyle=\color{codegreen},
    keywordstyle=\color{magenta},
    numberstyle=\tiny\color{codegray},
    stringstyle=\color{codepurple},
    basicstyle=\ttfamily\footnotesize,
    breakatwhitespace=false,         
    breaklines=true,                 
    captionpos=b,                    
    keepspaces=true,                 
    numbers=left,                    
    numbersep=5pt,                  
    showspaces=false,                
    showstringspaces=false,
    showtabs=false,                  
    tabsize=2
}

% Activate the style
\lstset{style=mystyle}

\title{PROBLEMS: }

\author{}
\date{}


\begin{document}
\maketitle
READING: Section 18.3 and 18.4 in Shankar on time-dependent perturbation theory and on electromagnetic interactions.
\section{}
\begin{enumerate}
  \setcounter{enumi}{12}
  \item Consider an electron in a weak one-dimensional periodic potential ("lattice") $V(x)=$ $V(x+d)$. Assume the lattice has a size $L=N d$, and that we the have a periodic boundary condition on our wave functions: $\psi(x)=\psi(x+L)$. With this boundary condition, the unperturbed wave functions are plane waves, $\psi_{p}(x)=\frac{1}{\sqrt{L}} e^{i p x}$, where $p=2 \pi n / L, n=$ integer, and the unperturbed eigenenergies are $\varepsilon_{n}=\frac{p^{2}}{2 m}=$ $\left(\frac{2 \pi n}{L}\right)^{2} \frac{1}{2 m}$. We expand the potential in a Fourier series:
\end{enumerate}

$$
V(x)=\sum_{n=-\infty}^{\infty} e^{i n 2 \pi x / d} V_{n}
$$

(a) If we label our eigenfunctions by $|p\rangle=\frac{1}{\sqrt{L}} e^{2 \pi i n_{p} x / L}$, determine all nonvanishing matrix elements of $V$ :

$$
\langle q|V| p\rangle
$$First try
Express your answer in terms of $V_{n}$, and a condition involving $p$ and $q$ or equivalently on $n_{p}=L p / 2 \pi$ and $n_{q}=L q / 2 \pi$.
\subsection{}
The matrix elements of $V$ are given by the integral:
\begin{equation}
  \begin{aligned}
    \langle q|V| p\rangle &=\int_{0}^{L} \mathrm{d} x \frac{1}{\sqrt{L}} e^{-2 \pi i n_{q} x / L} V(x) \frac{1}{\sqrt{L}} e^{2 \pi i n_{p} x / L} \\
\end{aligned}
\end{equation}
We can plug in the Fourier series for $V(x)$ into the above equation:
\begin{equation}
  \begin{aligned}
    \langle q|V| p\rangle &=\int_{0}^{L} \mathrm{d} x \frac{1}{\sqrt{L}} e^{-2 \pi i n_{q} x / L} \sum_{n=-\infty}^{\infty} e^{i n 2 \pi x / d} V_{n} \frac{1}{\sqrt{L}} e^{2 \pi i n_{p} x / L} \\
\end{aligned}
\end{equation}
Bringing the sum and coefficients outside the integral:
\begin{equation}
  \begin{aligned}
    \langle q|V| p\rangle &=\frac{1}{L}\sum_{n=-\infty}^{\infty} V_{n}\int_{0}^{L} \mathrm{d} x e^{-2 \pi i n_{q} x / L} e^{i n 2 \pi x / d} e^{2 \pi i n_{p} x / L} \\
\end{aligned}
\end{equation}
We can use the expression we are given for $d$ to combine the exponentials:
\begin{equation}
  \begin{aligned}
    &=\frac{1}{L}\sum_{n=-\infty}^{\infty} V_{n}\int_{0}^{L} \mathrm{d} x e^{-2 \pi i n_{q} x / L} e^{i n 2 \pi x / (L/N)} e^{2 \pi i n_{p} x / L} \\
    &=\frac{1}{L}\sum_{n=-\infty}^{\infty} V_{n}\int_{0}^{L} \mathrm{d} x e^{\frac{2\pi i x}{L}\left( nN+n_{p}-n_{q}\right)}\\
\end{aligned}
\end{equation}
This integral is only non zero and evaluates to $L$ when $nN + n_{p} - n_{q} = 0$, leading to the condition for non-zero contributions:
\begin{equation}
  nN + n_{p} - n_{q} = 0
\end{equation}
where $n$ is an integer.
We can thus express the matrix element as:
\begin{equation}
  \langle q|V| p\rangle = \sum_{n=-\infty}^{\infty} V_{n} \delta_{n_{q}, n_{p} + nN}
\end{equation}
Most of the terms in the sum will be zero, as the Kronecker delta will only be non-zero when $n_{p}-n{q}$ is a multiple of $N$:
\begin{equation}
  \langle q|V| p\rangle = V_{\frac{n_{q}-n_{p}}{N}}
\end{equation}
\subsection{}
(b) Suppose $\varepsilon_{n_{p}}$ and $\varepsilon_{n_{q}}$ are not close to each other $\forall n_{q}\left(\neq n_{p}\right)$, given some $n_{p}$. Calculate the perturbed wave function in ordinary first order perturbation theory corresponding to unperturbed wave function $\psi_{p}(x)$. Also, calculate the energy to $2^{\text {nd }}$ order. Express your answer in terms of $V_{n}$.\\
We may consider the projection of the unperturbed wave function $\bra{q}$ onto the perturbed wave function $\ket{N^{(1)}}$:
\begin{equation}
  \bra{q}\ket{N^{(1)}}=\frac{1}{\varepsilon _{n_{p}}-\varepsilon _{n_{q}}}\bra{q}V\ket{p}
\end{equation}
Using the expression for the matrix elements of $V$ we derived in part (a), we can write the above equation as:
\begin{equation}
  \bra{q}\ket{N^{(1)}}=\frac{1}{\varepsilon _{n_{p}}-\varepsilon _{n_{q}}}V_{\frac{n_{p}-n_{q}}{N}}
\end{equation}
To first order, the perturbed wave function is:
\begin{equation}
  \ket{N^{(1)}}=\ket{p}+\sum_{n_{q}\neq n_{p}}\frac{1}{\varepsilon _{n_{p}}-\varepsilon _{n_{q}}}V_{\frac{n_{p}-n_{q}}{N}}\ket{q}
\end{equation}
where we earlier stated the condition on $n_{q}$ and $n_{p}$.
The energy to second order consists of the unperturbed energy plus the first order correction plus the second order correction:
\begin{equation}
  \varepsilon _{n_{p}}\approx\varepsilon _{n_{p}}^{(0)}+\bra{n_p}V\ket{n_p}+\sum_{n_{q}\neq n_{p}}\frac{\abs{\bra{q}V\ket{p}}^{2}}{\varepsilon _{n_{p}}-\varepsilon _{n_{q}}}
\end{equation}
where $n$ is an unperturbed eigenstate. We can take advantage of the condition on $n_{q}$ and $n_{p}$ and define $V_0$ as the matrix element of $V$ between the unperturbed eigenstate and itself:  
\begin{equation}
  \varepsilon _{n_{p}}\approx\varepsilon _{n_{p}}^{(0)}+V_{0}+\sum_{m'= -\infty}^{\infty}\frac{\abs{V_{m'}}^{2}}{\varepsilon _{n_{p}}-\varepsilon _{Nm'+n_{p}}}
\end{equation}
where we again have the same condition on $n_{q}$ and $n_{p}$.
\section{}
\begin{enumerate}
  \setcounter{enumi}{13}
  \item It may happen that we encounter a situation where two eigenvalues of $H_{0}$, call them $\varepsilon_{n}$ and $\varepsilon_{m}$, are nearly, but not quite equal. In this case, we don't seem to be able to use degenerate perturbation theory, and ordinary perturbation theory is likely to converge slowly. Let us try to deal with such a situation: Suppose the two eigenstates $|n\rangle$ and $|m\rangle$ of $H_{0}$ have nearly the same energy (and all other eigenstates don't suffer this disease, for simplicity). Let $H=H_{0}+V$, and write
\end{enumerate}

$$
\begin{aligned}
V & =\sum_{i, j}|i\rangle\langle i|V| j\rangle\langle j| \\
H_{0}|i\rangle & =\varepsilon_{i}|i\rangle,
\end{aligned}
$$

where

$$
\langle i \mid j\rangle=\delta_{i j} .
$$

Let

$$
V=V_{1}+V_{2}
$$

with

$$
\begin{aligned}
V_{1} \equiv & |m\rangle\langle m|V| m\rangle\langle m|+| n\rangle\langle n|V| n\rangle\langle n|+ \\
& +|m\rangle\langle m|V| n\rangle\langle n|+| n\rangle\langle n|V| m\rangle\langle m|
\end{aligned}
$$

and $V_{2}$ is everything else.

If we can solve exactly the problem with $H_{1}=H_{0}+V_{1}$, then the troublesome $1 /\left(\varepsilon_{n}-\varepsilon_{m}\right)$ terms are avoided by the exact treatment, and we may treat $V_{2}$ as a perturbation in ordinary perturbation theory (since $\left\langle i\left|V_{2}\right| j\right\rangle=0$ for $i, j=n, m$ ). All states $|i\rangle, i \neq n, m$, are eigenstates of $H_{1}$, since $V_{1}|i\rangle=0$ in this case. However, $|n\rangle$ and $|m\rangle$ are not in general eigenstates of $H_{1}$.

(a) Solve exactly for the eigenstates and eigenvalues of $H_{1}$, in the subspace spanned by $|n\rangle,|m\rangle$. Express your answer in terms of

$$
\varepsilon_{n}, \varepsilon_{m},\langle m|V| n\rangle,\langle n|V| n\rangle,\langle m|V| m\rangle .
$$

(You may also use the shorthand

$$
E_{n, m}^{(1)}=\varepsilon_{n, m}+\langle n, m|V| n, m\rangle
$$

if you find it convenient.)
\subsection{}
We want to diagonalize the matrix representation of $H_{1}$ in the subspace spanned by $|n\rangle$ and $|m\rangle$. We have:
\begin{equation}
  H_{1}=\mqty(\varepsilon_{n}+\langle n|V_1| n\rangle & \langle n|V_1| m\rangle\\
  \langle m|V_1| n\rangle & \varepsilon_{m}+\langle m|V_1| m\rangle)
\end{equation}
since we have that $H_1 \equiv H_0 + V_1$.
Solving the characteristic equation for the matrix $H_1$ with Sympy:
\subsection*{Eigenvalues}

\[
\lambda_{\pm} = \frac{\varepsilon_{n} + \varepsilon_{m} + \langle n|V| n\rangle + \langle m|V| m\rangle}{2} \pm \frac{\sqrt{(\varepsilon_{n} - \varepsilon_{m} + \langle n|V| n\rangle - \langle m|V| m\rangle)^2 + 4|\langle n|V| m\rangle|^2}}{2}
\]
or using the relation $E_{n}^{(1)}=\varepsilon_{n}+\langle n|V| n\rangle$ and $E_{m}^{(1)}=\varepsilon_{m}+\langle m|V| m\rangle$ both outside and inside the square root:
\[
\lambda_{\pm} = \frac{E_{n}^{(1)} + E_{m}^{(1}}{2} \pm \frac{\sqrt{(E_{n}^{(1)} - E_{m}^{(1)})^2 + 4|\langle n|V| m\rangle|^2}}{2}
\]

\subsection*{Eigenvectors}

\[
\mathbf{v}_{+} = \begin{pmatrix} 1 \\ \frac{\lambda_{+} - \varepsilon_{n} - \langle n|V| n\rangle}{\langle n|V| m\rangle} \end{pmatrix}, \quad \mathbf{v}_{-} = \begin{pmatrix} 1 \\ \frac{\lambda_{-} - \varepsilon_{n} - \langle n|V| n\rangle}{\langle n|V| m\rangle} \end{pmatrix}
\]

% Inline Python code in the document
\begin{lstlisting}[language=Python]
from sympy import symbols, Matrix, solve, sqrt, latex

# Define symbols
epsilon_n, epsilon_m = symbols('epsilon_n epsilon_m')
V_nn, V_mm, V_nm, V_mn = symbols('V_nn V_mm V_nm V_mn')

# Define the Hamiltonian matrix H_1
H_1 = Matrix([
    [epsilon_n + V_nn, V_nm],
    [V_mn, epsilon_m + V_mm]
])

# Diagonalize the matrix to find eigenvalues and eigenvectors
eigenvals = H_1.eigenvals()
eigenvects = H_1.eigenvects()

# Simplifying the eigenvalues and eigenvectors might be necessary
for i in range(len(eigenvals)):
    eigenvals[i] = eigenvals[i].simplify()
for i in range(len(eigenvects)):
    eigenvects[i][2][0] = eigenvects[i][2][0].simplify()
# Print the eigenvalues and eigenvectors in LaTeX form
print("Eigenvalues:")
for eigenval in eigenvals:
    print(latex(eigenval))
print("\nEigenvectors:")
for eigenvect in eigenvects:
    print(latex(eigenvect[2][0]))


\end{lstlisting}


\subsection{}
(b) Now consider the periodic potential of problem 13. What is the condition on $n_{p}$ (and hence on $p$ ) so that $|p\rangle$ will be nearly degenerate in energy with another eigenstate of $H_{0}$ ? You might find it convenient to define the "reciprocal lattice constant" $K \equiv 2 \pi / d$.
\subsection{}
Knowing that the eigenenergies are $\varepsilon_{n}=\frac{p^{2}}{2 m}=\left(\frac{2 \pi n}{L}\right)^{2} \frac{1}{2 m}$, we see that $\varepsilon_{n}=\varepsilon _{-n}$. Also, the expression that we found for $\varepsilon _n$ earlier was:
\begin{equation}
  \varepsilon _{n}=\varepsilon _{n}^{(0)}+V_{0}+\sum_{m'= -\infty}^{\infty}\frac{\abs{V_{m'}}^{2}}{\varepsilon _{n}-\varepsilon _{Nm'+n}}
\end{equation}
We see that this is the same as:
\begin{equation}
  \varepsilon _{n}=\varepsilon _{n}^{(0)}+V_{0}+\sum_{m'= \infty}^{- \infty}\frac{\abs{V_{-m'}}^{2}}{\varepsilon _{n}-\varepsilon _{-Nm'+n}}
\end{equation}
So furthermore, we see that $n_{-p}=-n_{p}$ while $V_{-m'}=V_{m'}^*$. So, we can write the expression for $\varepsilon _n$ as:
\begin{equation}
  \varepsilon _{n}=\varepsilon _{-n}^{(0)}+V_{0}+\sum_{m'= \infty}^{- \infty}\frac{\abs{V_{m'}}^{2}}{\varepsilon _{-n}-\varepsilon _{Nm'-n}} = \varepsilon _{-n}
\end{equation}
So, we see that the condition for $|p\rangle$ to be nearly degenerate in energy with another eigenstate of $H_{0}$ is that $n_q=n_{-p}=-n_{p}$, or equivalently $p=-q$.

(c) Assume that the condition in part (b) is satisfied, and use part (a) to solve this "almost degenerate" case for the eigenenergies. Try to make a sketch of the energy as a function of momentum ("dispersion relation"). Fig. 1 gives a start for momenta less than $\pi / d$.
\subsection{}
We can use the expression for the eigenvalues we found in part (a) to solve for the eigenenergies. We have:
\begin{equation}
  \lambda_{\pm} = \frac{E_{n}^{(1)} + E_{m}^{(1}}{2} \pm \frac{\sqrt{(E_{n}^{(1)} - E_{m}^{(1)})^2 + 4|\langle n|V| m\rangle|^2}}{2}
\end{equation}
If we let $m=-n$, this simplifies to:
\begin{equation}
  \lambda_{\pm} = \frac{E_{n}^{(1)} + E_{-n}^{(1}}{2} \pm \frac{\sqrt{(E_{n}^{(1)} - E_{-n}^{(1)})^2 + 4|\langle n|V| -n\rangle|^2}}{2}
\end{equation}
Since we know that $E_{n}^{(1)}=E_{-n}^{(1)}$, we can simplify the above equation to:
\begin{equation}
  \lambda_{\pm} = E_{n}^{(1)} \pm \frac{\sqrt{4|\langle n|V| -n\rangle|^2}}{2}
\end{equation}
Furthermore, we define the matrix element $\langle n|V| -n\rangle$ as $V_{2n}$:
\begin{equation}
  \lambda_{\pm} = E_{n}^{(1)} \pm |V_{2n}|
\end{equation}
We can define $p(\Delta )= \frac{2\pi}{L}(1+\Delta )$. We can then plot the dispersion relation for $p(\Delta )$. I am not able to write by hand so I will use a sketch that the friend that I worked with made:
\begin{figure}
  \centering
  \includegraphics[width=0.5\textwidth]{dr.png}
  \caption{Dispersion relation}
\end{figure}
\section{}
\begin{enumerate}
  \setcounter{enumi}{14}
  \item When we calculate the density of states for a free particle, we use a "box" of length $L$ (here, we consider one dimension), and impose periodic boundary conditions to ensure no net flux of particles into or out of the box. We have in mind that we can eventually let $L \rightarrow \infty$, and are really interested in quantities per unit length (or volume). However, we should really demonstrate our conclusion. So, let us justify more carefully the use of periodic boundary conditions, i.e., we wish to carefully convince ourselves that the intuitive rationale given above is in fact correct. To do this, consider a free particle in a one-dimensional "box" from $-L / 2$ to $L / 2$. Remembering that the Hilbert space of allowed states is a linear space, show that the periodic boundary condition:
\end{enumerate}

$$
\begin{aligned}
\psi(-L / 2) & =\psi(L / 2) \\
\psi^{\prime}(-L / 2) & =\psi^{\prime}(L / 2)
\end{aligned}
$$

gives acceptable wave functions. "Acceptable" here includes that the probability to find a particle in the box must be constant. Are there other acceptable choices?
\subsection{}
We may restate the condition of no net floods into or out of the box as:
\begin{equation}
  0=\pdv{t}\int_{-L / 2}^{L / 2} \abs{\psi(x, t)}^{2} \dd{x}
\end{equation}
That is, the probability to find a particle in the box must not change with time.
Now, if we have some acceptable wave function $\psi(x)$.
And another acceptable function $\phi $:
\begin{equation}
  \phi (x)=e^{\frac{i2\pi^2t}{mL^2}}\sin\left(\frac{2\pi x}{L}\right)
\end{equation}
$\phi$ here is such that it's value and derivative at $-L/2$ and $L/2$ are the same. Therefore, we have proven the existence of and acceptable wave function.
Any linear combination of $\psi(x)$ and $\phi(x)$ is also an acceptable wave function.
So, we plug in the linear combination of $\psi(x)$ and $\phi(x)$ into the condition of no net floods into or out of the box:
\begin{equation}
  \begin{aligned}
    0&=\pdv{t}\int_{-L / 2}^{L / 2} \abs{\psi(x)+\phi(x)}^{2} \dd{x}\\
    &=\pdv{t}\int_{-L / 2}^{L / 2} \abs{\psi(x)}^{2} \dd{x}+\pdv{t}\int_{-L / 2}^{L / 2} \abs{\phi(x)}^{2} \dd{x}+\pdv{t}\int_{-L / 2}^{L / 2}\psi^*(x)\phi(x)+\psi(x)\phi^*(x) \dd{x}\\
\end{aligned}
\end{equation}
We know the first two terms are zero, as $\psi(x)$ and $\phi(x)$ are acceptable wave functions. So, we are left with:
\begin{equation}
  0=\pdv{t}\left[\int_{-L / 2}^{L / 2} \psi^*(x)\phi(x)\dd{x}+\int_{-L / 2}^{L / 2} \psi(x)\phi^*(x) \dd{x}\right]
\end{equation}
We can write the above equation as:
\begin{equation}
  0=\pdv{t}\left[\int_{-L / 2}^{L / 2} \psi^*(x)\phi(x)\dd{x}\right]+\pdv{t}\left[\int_{-L / 2}^{L / 2} \psi(x)\phi^*(x) \dd{x}\right]
\end{equation}
Now, we can use the product rule to differentiate the above equation:
\begin{equation}
  0=\int_{-L / 2}^{L / 2} \pdv{t}\left[\psi^*(x)\phi(x)\right]\dd{x}+\int_{-L / 2}^{L / 2} \pdv{t}\left[\psi(x)\phi^*(x)\right] \dd{x}
\end{equation}
We can write the above equation as:
\begin{equation}
  0=\int_{-L / 2}^{L / 2} \left[\pdv{t}\psi^*(x)\right]\phi(x)+\psi^*(x)\left[\pdv{t}\phi(x)\right]\dd{x}+\int_{-L / 2}^{L / 2} \left[\pdv{t}\psi(x)\right]\phi^*(x)+\psi(x)\left[\pdv{t}\phi^*(x)\right] \dd{x}
\end{equation}
Rewriting the above equation to emphasize that the wave functions are time dependent
\begin{equation}
  0=\int_{-L / 2}^{L / 2} \left[\pdv{t}\psi^*(x,t)\right]\phi(x,t)+\psi^*(x,t)\left[\pdv{t}\phi(x,t)\right]\dd{x}+\int_{-L / 2}^{L / 2} \left[\pdv{t}\psi(x,t)\right]\phi^*(x,t)+\psi(x,t)\left[\pdv{t}\phi^*(x,t)\right] \dd{x}
\end{equation}
Now, the time dependent Schrödinger equation is:
\begin{equation}
  i\hbar\pdv{t}\psi(x,t)=\hat{H}\psi(x,t)
\end{equation}
So, we can write the above equation as:
Isolate the time derivative of the wave function
\begin{equation}
  \pdv{t}\psi(x,t)=\frac{1}{i\hbar}\hat{H}\psi(x,t)
\end{equation}
Now, for the complex conjugate of the wave function:
\begin{equation}
  \pdv{t}\psi^*(x,t)=-\frac{1}{i\hbar}\hat{H}\psi^*(x,t)
\end{equation}
We can plug in the time derivative of the wave function into the above equation:
\begin{equation}
  0=\int_{-L / 2}^{L / 2} \left[\frac{1}{i\hbar}\hat{H}\psi^*(x,t)\right]\phi(x,t)+\psi^*(x,t)\left[\frac{1}{i\hbar}\hat{H}\phi(x,t)\right]\dd{x}+\int_{-L / 2}^{L / 2} \left[\frac{1}{i\hbar}\hat{H}\psi(x,t)\right]\phi^*(x,t)+\psi(x,t)\left[\frac{1}{i\hbar}\hat{H}\phi^*(x,t)\right] \dd{x}
\end{equation}
The free-particle Hamiltonian is:
\begin{equation}
  \hat{H}=-\frac{\hbar^2}{2m}\pdv[2]{x}
\end{equation}
So, we can plug in the Hamiltonian into the above equation:
\begin{equation}
  \begin{aligned}
    0&=\int_{-L / 2}^{L / 2} \left[\frac{1}{i\hbar}\left(-\frac{\hbar^2}{2m}\pdv[2]{x}\right)\psi^*(x,t)\right]\phi(x,t)+\psi^*(x,t)\left[\frac{1}{i\hbar}\left(-\frac{\hbar^2}{2m}\pdv[2]{x}\right)\phi(x,t)\right]\dd{x}\\
    &+\int_{-L / 2}^{L / 2} \left[\frac{1}{i\hbar}\left(-\frac{\hbar^2}{2m}\pdv[2]{x}\right)\psi(x,t)\right]\phi^*(x,t)+\psi(x,t)\left[\frac{1}{i\hbar}\left(-\frac{\hbar^2}{2m}\pdv[2]{x}\right)\phi^*(x,t)\right] \dd{x}\\
  \end{aligned}
\end{equation}
I was not able to get further.
\section{}
\begin{enumerate}
  \setcounter{enumi}{15}
  \item Note: I have posted a note reviewing complex variables in the module for week 4 , in case it is helpful (to evaluate an integral).
\end{enumerate}

\begin{center}
\includegraphics[max width=\textwidth]{2024_01_29_bd91e6d395035e9decbag-3}
\end{center}

Figure 1: Energy versus momentum for the one-dimensional lattice problem.

Consider a proton (charge $e$ ) in a one dimensional harmonic oscillator potential with unperturbed Hamiltonian

$$
H_{0}=\frac{p^{2}}{2 m}+\frac{1}{2} m \omega^{2} x^{2}
$$

We add a small time-dependent electric field so that $H=H_{0}+V_{t}$ with

$$
V_{t}=\frac{e E x}{1+(t / \tau)^{2}}, \quad-\infty<t<\infty
$$

If the system is initially in the ground state at $t=-\infty$, what is the probability to observe it in the first excited state after a long time $(t=\infty)$ ?

Thus,

$$
{ }_{\infty}\langle 1 \mid 0\rangle_{-\infty}=\frac{e E}{i} \pi \tau e^{-\omega \tau} \frac{1}{\sqrt{2 m \omega}}=-\frac{i \pi \tau e E}{\sqrt{2 m \omega}} e^{-\omega \tau}
$$

Finally, the desired transition probability is

$$
P(1)=\left.\left.\right|_{\infty}\langle 1 \mid 0\rangle_{-\infty}\right|^{2}=-\frac{(\pi \tau e E)^{2}}{2 m \omega} e^{-2 \omega \tau}
$$
\subsection{}
We want to evaluate the desired transition probability using the formalism of time-dependent perturbation theory in the interaction picture. We have:
\begin{equation}
  \begin{aligned}
    P_{1} &= -\frac{i}{\hbar} \int_{-\infty}^{\infty} \mathrm{d} t\left\langle 1\left|V_{t}\right| 0\right\rangle e^{i \omega_{1, 0} t} \\
\end{aligned}
\end{equation}
I was not able to get further on this one.
\end{document}