\documentclass[12pt]{article}
\usepackage[utf8]{inputenc}
\usepackage[T1]{fontenc}
\usepackage{amsmath}
\usepackage{amsfonts}
\usepackage{amssymb}
\usepackage[version=4]{mhchem}
\usepackage{stmaryrd}
\usepackage{graphicx}
\usepackage{physics}

\usepackage{listings} % Required for insertion of code
\usepackage{xcolor} % Required for custom colors

% Define custom colors
\definecolor{codegreen}{rgb}{0,0.6,0}
\definecolor{codegray}{rgb}{0.5,0.5,0.5}
\definecolor{codepurple}{rgb}{0.58,0,0.82}
\definecolor{backcolour}{rgb}{0.95,0.95,0.92}

% Setup the style for code listings
\lstdefinestyle{mystyle}{
    backgroundcolor=\color{backcolour},   
    commentstyle=\color{codegreen},
    keywordstyle=\color{magenta},
    numberstyle=\tiny\color{codegray},
    stringstyle=\color{codepurple},
    basicstyle=\ttfamily\footnotesize,
    breakatwhitespace=false,         
    breaklines=true,                 
    captionpos=b,                    
    keepspaces=true,                 
    numbers=left,                    
    numbersep=5pt,                  
    showspaces=false,                
    showstringspaces=false,
    showtabs=false,                  
    tabsize=2
}

% Activate the style
\lstset{style=mystyle}


\title{PROBLEMS: }

\author{}
\date{}


\begin{document}
\maketitle
Physics $125 \mathrm{~b}$

Problem set number 7

Due midnight Wednesday, February 21, 2024

READING: Pages 592-607 in Shankar on Berry's phase.\\
\emph{I have an extension until end of the day on Saturday from the professor.}
\section{}
\begin{enumerate}
  \setcounter{enumi}{24}
  \item Let's continue a bit further with the dipole interaction of problem 24. You obtained, I hope, an expression for the "electric dipole Hamiltonian". But it might look a bit mysterious. If this really describes an electric dipole interaction with a uniform external field, we might expect the Hamiltonian to look something like
\end{enumerate}


\begin{equation*}
H_{D}^{\prime}=-q E Z \sin \omega t . \tag{1}
\end{equation*}


That is, $q Z$ looks like a dipole moment. Consider the expectation value $\left\langle n\left|P_{z}\right| 0\right\rangle$ with respect to two eigenstates of $H_{0}$. Show that this can be written in terms of an expectation value $\langle n|Z| 0\rangle$ and quantities such as the electron mass, $m$, and the energy difference between the states. Rewrite $H_{D}$ using this relation. Does it look more intuitive now?
\subsection{}
We want to consider the commutator between $H_0$ and $Z$. For the original Hamiltonian, we had the 
\begin{equation}
  H_{0}=\frac{P^{2}}{2m}+V(R)
\end{equation}
Since the potential only has a radial dependence, which means it only depends on $R$, and we know that $Z$ commutes with each of the Cartesian coordinates, we can say that $[V(R),Z]=0$.
We can expand the total momentum operator squared into its individual Cartesian components
\begin{equation}
  P^{2}=P_{x}^{2}+P_{y}^{2}+P_{z}^{2}
\end{equation}
Because the $Z$ operator commutes with the $P_{x}$ and $P_{y}$ operators, we have found that the commutator $[H_0,Z]$ is the same as $\frac{1}{2m}[P_{z}^{2},Z]$. Then, we know that:
\begin{equation}
  [P_{z}^{2},Z]=P_{z}[P_{z},Z]+[P_{z},Z]P_{z}
\end{equation}
and we can use the commutator $[P_{z},Z]=i\hbar$ to get:
\begin{equation}
  [P_{z}^{2},Z]=i\hbar P_{z}+i\hbar P_{z}=2i\hbar P_{z} \rightarrow [H_0,Z]=\frac{i\hbar}{m} P_{z}
\end{equation}
So now that us consider the virginal expectation value:
\begin{equation}
  \left\langle n\left|P_{z}\right| 0\right\rangle
\end{equation}
We can use the commutator to get:
\begin{equation}
  \left\langle n\left|P_{z}\right| 0\right\rangle = \frac{m}{i\hbar}\left\langle n\left|[H_0,Z\right]| 0\right\rangle
\end{equation}
Expanding the commutator gives:
\begin{equation}
  \left\langle n\left|P_{z}\right| 0\right\rangle = \frac{m}{i\hbar}\left\langle n\left|H_0Z-ZH_0\right| 0\right\rangle = \frac{m}{i\hbar}\left\langle n\left|H_0Z\right| 0\right\rangle - \frac{m}{i\hbar}\left\langle n\left|ZH_0\right| 0\right\rangle
\end{equation}
Let us call the unpertirbed energies of $|n\rangle$ and $|0\rangle$ $E_n$ and $E_0$ respectively. All of these entities are real, so:
\begin{equation}
  = \frac{m}{i\hbar}\left(E_n\left\langle n\left|Z\right| 0\right\rangle - E_0\left\langle n\left|Z\right| 0\right\rangle\right) = \frac{m}{i\hbar}\left(E_n-E_0\right)\left\langle n\left|Z\right| 0\right\rangle
\end{equation}
Now, the expectation value of the dipole antonian between the final and initial states is given by:
\begin{equation}
  \left\langle n\left|H_{D}\right| 0\right\rangle = -qE\sin(\omega t)\left\langle n\left|Z\right| 0\right\rangle
\end{equation}
The matrix element of the $Z$ operator between the final and initial states in terms of the expectation value of the momentum operator is:
\begin{equation}
  \left\langle n\left|Z\right| 0\right\rangle = \frac{i\hbar}{m(E_n-E_0)}\left\langle n\left|P_{z}\right| 0\right\rangle
\end{equation}
Or then solving for $P_z$ in terms of $Z$:
\begin{equation}
  \left\langle n\left|P_{z}\right| 0\right\rangle = -i\hbar m(E_n-E_0)\left\langle n\left|Z\right| 0\right\rangle
\end{equation}
Setting $\hbar = 1$ and doing $\omega = E_n-E_0$ gives:
\begin{equation}
  \left\langle n\left|P_{z}\right| 0\right\rangle = -i m\omega \left\langle n\left|Z\right| 0\right\rangle
\end{equation}
Things will hold generally, because we have the same expectation value on both sides:
\begin{equation}
  P_{z}=-i m\omega Z
\end{equation}
From the previous problem set, we had:
\begin{equation}
  H_D = \frac{q }{m}P_z \frac{E_0}{\omega} \sin(\omega t)
\end{equation}
Using our expression for $P_z$ gives:
\begin{equation}
  H_D = \frac{q}{m}\left(-i m\omega Z\right) \frac{E_0}{\omega} \sin(\omega t) = -qE\sin(\omega t)Z
\end{equation}
% So we can rewrite the dipole Hamiltonian as:
% \begin{equation}
%   \left\langle n\left|H_{D}\right| 0\right\rangle = -qE\sin(\omega t)\frac{i\hbar}{m(E_n-E_0)}\left\langle n\left|P_{z}\right| 0\right\rangle
% \end{equation}

This looks a lot more intuitive now as we have two charges separated by a distance $Z$ and then interacting with an electric field $E$. What we had previously was not so intuitive, as it is hard to imagine a deal moment in terms of the moment anthem I pervade or $P_z$.
\section{}
\begin{enumerate}
  \setcounter{enumi}{25}
  \item When we continue to apply our dipole approximation to an atom, we'll find that the dipole operator produces transitions from the ground state $|0\rangle$ to other eigenstates, $|n\rangle$, of $H_{0}$. The relative strengths of these transitions is given by an "oscillator strength":
\end{enumerate}


\begin{equation*}
f_{n 0}=2 m \omega_{n 0}|\langle n|Z| 0\rangle|^{2}, \tag{2}
\end{equation*}


where $\omega_{n 0}=E_{n}-E_{0}$ is the energy difference between the states. Note that $f_{n 0}$ is dimensionless. Demonstrate the oscillator strength "sum rule":


\begin{equation*}
\sum_{n} f_{n 0}=1 \tag{3}
\end{equation*}


The work you have done in problem 25 will likely come in handy here.
\subsection{}
That us only substitute the relation that we god for the expectation value of $Z$ in terms of the momentum operator only for one of the terms in the square:
\begin{equation}
  f_{n 0}=2 m \omega_{n 0}\left(\frac{1}{2}\langle n|Z| 0\rangle \frac{i\hbar}{m(E_n-E_0)}\left\langle 0\left|P_{z}\right| n\right\rangle - \frac{1}{2}\langle n|P_{z}| 0\rangle \frac{i\hbar}{m(E_n-E_0)}\left\langle 0\left|Z\right| n\right\rangle\right)
\end{equation}
We note that in one of the terms the complex number was congregated, so we have two reverse the sign of one of these.\\
Making a lot of cancellations including for the mass, for $\omega _{n0}= E_n-E_0$ and factoring out the term of $i\hbar$ gives:
\begin{equation}
  f_{n 0}=i\hbar\left(\langle n|[Z,P_z]| 0\rangle\right)
\end{equation}
Then, we also know that $[Z,P_z]=-i\hbar$, so we get:
\begin{equation}
  f_{n 0}=- i\hbar i\hbar \langle n|| 0\rangle
\end{equation}
When we consider the whole sum, only one of these terms will survive from Arthur normality, so we get:
\begin{equation}
  \sum_{n} f_{n 0}=-i\hbar i\hbar \sum_{n}\langle n|| 0\rangle = -i\hbar i\hbar = \hbar^2
\end{equation}
But we have that $\hbar = 1$, so we get:
\begin{equation}
  \sum_{n} f_{n 0}=1
\end{equation}

\section{}
\begin{enumerate}
  \setcounter{enumi}{26}
  \item We discussed Berry's phase in class, including the example of the Aharonov-Bohm effect. Let us try another example.
\end{enumerate}

Consider a spin- $1 / 2$ particle of magnetic moment $\mu$ in a magnetic field, $\mathbf{B}$. Let the strength of the magnetic field be a constant, but suppose that its direction is slowly changing. Suppose the magnetic field vector is swept through a closed curve (i.e., think of the tip of its vector as being varied along a closed curve on the surface of a sphere). Thus, two parameters describing the direction of the field are being varied, which we might take to be the polar angles $(\theta, \phi)$.

(a) What does "slow" mean? That is, on what time scale should the variation be slow, in order for the adiabatic approximation to be reasonable?
\subsection{}
This is a two state system with two energies:
\begin{equation}
  E_{\pm}=\pm |\mu| |B|
\end{equation}
The magnitude of the energy difference is:
\begin{equation}
  \Delta = 2|\mu| |B|
\end{equation}
We have the following condition from the nodes:
\begin{figure}
  \centering
  \includegraphics[scale=0.5]{adiabatic.png}
\end{figure}
So, our condition is:
\begin{equation}
  \Delta  \gg \frac{1}{T}
\end{equation}
Alternatively, we have:
\begin{equation}
  T \gg \frac{1}{\Delta}
\end{equation}

(b) What is the expectation value of the spin vector in the adiabatic ground state?
\subsection{}
When we just were considering the expectation value of the spent vector at $\theta = 0$, we got:
\begin{equation}
  \bra{\Psi _{\theta =0}^{(0)}}\mathbf{S}\ket{\Psi _{\theta =0}^{(0)}} = \frac{1}{2}\hat{e}_{z}
\end{equation}
But now the versal will simply be the same as the Fabius one but with the unit vector in the direction of the magnetic field:
\begin{equation}
  \bra{\Psi _{\theta \phi}^{(0)}}\mathbf{S}\ket{\Psi _{\theta \phi}^{(0)}} = \frac{1}{2}\hat{e}_{\theta , \phi}
\end{equation}
where $\hat{e}_{B}$ is the unit vector in the direction of the magnetic field and depends on the polar angles.
We know of this:
$\begin{aligned} & \hat{n}_z=\cos \theta \\ & \hat{n}_x=\sin \theta \cos \phi \\ & \hat{n}_y=\sin \theta \sin \phi\end{aligned}$
So, we can write the expectation value as:
\begin{equation}
  \bra{\Psi _{\theta \phi}^{(0)}}\mathbf{S}\ket{\Psi _{\theta \phi}^{(0)}} = \frac{1}{2}\left( \cos \theta \hat{e}_{z} + \sin \theta \cos \phi \hat{e}_{x} + \sin \theta \sin \phi \hat{e}_{y}\right)
\end{equation}

(c) Express the adiabatic ground state wave function, $\psi_{\theta \phi}^{(0)}$, in terms of the adiabaticallyvaried parameters. You can in principle achieve this from your result to part (b), but it will probably be easier to accomplish by performing a rotation on a vector to the desired polar angles. The text pages 329-333 discuss finite rotations. The discussion of spin-1/2 in Chapter 14 of the text may also be helpful. I have also uploaded to module 7 the supplementary note on angular momentum from $\mathrm{Ph} 125$ a.
\subsection{}
The expression for this in the $\hat{e}_{z}$ basis is:
\begin{equation}
\begin{gathered}
\mid \hat{n} \text { up }\rangle \equiv|\hat{n}+\rangle=\left[\begin{array}{c}
\cos (\theta / 2) e^{-i \phi / 2} \\
\sin (\theta / 2) e^{i \phi / 2}
\end{array}\right] \\
\mid \hat{n} \text { down }\rangle \equiv|\hat{n}-\rangle=\left[\begin{array}{c}
-\sin (\theta / 2) e^{-i \phi / 2} \\
\cos (\theta / 2) e^{i \phi / 2}
\end{array}\right]
\end{gathered}
\end{equation}
The first is for when the spin is aligned with the magnetic field and the second is for when the spin is opposite the magnetic field. The ground state is the one with the lowest energy, so we have that the ground state is the one with the spin aligned with the magnetic field. So, we have that:
\begin{equation}
  \ket{\Psi _{\theta \phi}^{(0)}} = \ket{\hat{n}+\rangle}
\end{equation}

(d) Suppose we let the $\mathbf{B}$ field direction rotate slowly around the 3-axis at constant $\theta$. Calculate Berry's phase for this situation. [Recall that Berry's phase is the change in the phase of the adiabatic wave function (ground state here) over one complete circuit in parameter space. We found that the Berry phase is given by:


\begin{equation*}
\gamma_{B}=i \oint d \boldsymbol{\alpha} \cdot\left\langle\psi_{\boldsymbol{\alpha}}^{(0)} \mid \nabla_{\boldsymbol{\alpha}} \psi_{\boldsymbol{\alpha}}^{(0)}\right\rangle \tag{4}
\end{equation*}


where $\boldsymbol{\alpha}$ is a vector in parameter space.] See if you can give a geometric interpretation to your answer, in terms of an amount of solid angle swept out by the circuit in parameter space.
\subsection{}
In our case, $\boldsymbol{\alpha}$ is the vector of the polar angles, but $\theta $ is constant, so we only have to consider the variation of $\phi$. 
So we want to consider the integral:
\begin{equation}
  \gamma_B=i\oint d \phi \left\langle\Psi _{\theta \phi}^{(0)}\mid \nabla_{\phi} \Psi _{\theta \phi}^{(0)}\right\rangle
\end{equation}
The ground state wave function is:
\begin{equation}
  \ket{\Psi _{\theta \phi}^{(0)}} = \ket{\hat{n}+\rangle} = \left[\begin{array}{c}
\cos (\theta / 2) e^{-i \phi / 2} \\
\sin (\theta / 2) e^{i \phi / 2}
\end{array}\right]
\end{equation}
The gradient of the wave function is:
\begin{equation}
  \nabla_{\phi}\ket{\Psi _{\theta \phi}^{(0)}} = \left[\begin{array}{c}
-\frac{i}{2}\cos (\theta / 2) e^{-i \phi / 2} \\
\frac{i}{2}\sin (\theta / 2) e^{i \phi / 2}
\end{array}\right]
\end{equation}
We evaluate this integral to give:
\begin{equation}
\gamma_B=i(-i \pi \cos (\theta)) = \pi \cos (\theta)
\end{equation}
% Inline Python code in the document
\begin{lstlisting}[language=Python]
from sympy import exp

# Correcting the expressions for the wave function components with exponential function
psi_0 = cos(theta/2) * exp(-I * phi / 2)
psi_1 = sin(theta/2) * exp(I * phi / 2)

# Correcting the gradient expressions
grad_psi_0 = -I/2 * psi_0
grad_psi_1 = I/2 * psi_1

# Inner product calculation with the corrected gradient
inner_product = (psi_0.conjugate() * grad_psi_0 + psi_1.conjugate() * grad_psi_1).simplify()

# Berry phase calculation
berry_phase = integrate(inner_product, (phi, 0, 2*pi))

berry_phase.simplify()
\end{lstlisting}
So when the magnetic field does one circuit and perimeter space, it acquires a phase of $\pi \cos (\theta)$. This is the solid angle swept out by the circuit in parameter space.

\section{}
\begin{enumerate}
  \setcounter{enumi}{27}
  \item Let us investigate our quantum electromagnetic field operators and in particular think about a notion for the energy density in the vacuum. We defined the quantum mechanical electromagnetic field operators $\hat{A}_{\mathrm{k}} \boldsymbol{\epsilon}$ and $\hat{A}_{\mathrm{k} \boldsymbol{\epsilon}}^{\dagger}$ :
\end{enumerate}


\begin{align*}
& \hat{A}_{\mathbf{k} \epsilon}\left|N_{\mathbf{k}_{1} \epsilon_{1}}, \ldots, N_{\mathbf{k} \epsilon}, \ldots\right\rangle=\sqrt{\frac{2 \pi}{\omega}} \sqrt{N_{\mathbf{k} \epsilon}}\left|N_{\mathbf{k}_{1} \epsilon_{1}}, \ldots, N_{\mathbf{k} \epsilon}-1, \ldots\right\rangle  \tag{5}\\
& \hat{A}_{\mathbf{k} \epsilon}^{\dagger}\left|N_{\mathbf{k}_{1} \epsilon_{1}}, \ldots, N_{\mathbf{k} \epsilon}, \ldots\right\rangle=\sqrt{\frac{2 \pi}{\omega}} \sqrt{N_{\mathbf{k} \epsilon}+1}\left|N_{\mathbf{k}_{1} \epsilon_{1}}, \ldots, N_{\mathbf{k} \epsilon}+1, \ldots\right\rangle . \tag{6}
\end{align*}


(a) Determine the commutation relations among these operators.
\subsection{}
We want to compute 3 commutation relations:
\begin{equation}
  [\hat{A}_{\mathbf{k} \epsilon},\hat{A}_{\mathbf{k^{\prime}} \epsilon^{\prime}}] = \hat{A}_{\mathbf{k} \epsilon}\hat{A}_{\mathbf{k^{\prime}} \epsilon^{\prime}} - \hat{A}_{\mathbf{k^{\prime}} \epsilon^{\prime}}\hat{A}_{\mathbf{k} \epsilon}
\end{equation}
We want to consider the effect of this on the arbor every state like $\ket{\cdots ,N_{\mathbf{k} \epsilon}, \cdots, N_{\mathbf{k^{\prime}} \epsilon^{\prime}}, \cdots}$.
\begin{equation}
\hat{A}_{\mathbf{k} \epsilon}\sqrt{\frac{2\pi}{\omega }} \sqrt{N_{\mathbf{k^{^{\prime}}\epsilon^{\prime}}}}\ket{\cdots ,N_{\mathbf{k} \epsilon}, \cdots, N_{\mathbf{k^{\prime}} \epsilon^{\prime}}-1, \cdots} - \hat{A}_{\mathbf{k^{\prime}} \epsilon^{\prime}}\sqrt{\frac{2\pi}{\omega }} \sqrt{N_{\mathbf{k} \epsilon}}\ket{\cdots ,N_{\mathbf{k} \epsilon}-1, \cdots, N_{\mathbf{k^{\prime}} \epsilon^{\prime}}, \cdots}
\end{equation}
Applying the second operator gives:
\begin{equation}
  \sqrt{\frac{2\pi}{\omega }} \sqrt{N_{\mathbf{k^{^{\prime}}\epsilon^{\prime}}}}\sqrt{\frac{2\pi}{\omega }} \sqrt{N_{\mathbf{k} \epsilon}}\ket{\cdots ,N_{\mathbf{k} \epsilon}-1, \cdots, N_{\mathbf{k^{\prime}} \epsilon^{\prime}}-1, \cdots} - \sqrt{\frac{2\pi}{\omega }} \sqrt{N_{\mathbf{k} \epsilon}}\sqrt{\frac{2\pi}{\omega }} \sqrt{N_{\mathbf{k^{^{\prime}}\epsilon^{\prime}}}}\ket{\cdots ,N_{\mathbf{k^{^{\prime}}\epsilon^{\prime}}}-1, \cdots, N_{\mathbf{k} \epsilon}-1, \cdots}
\end{equation}
We can factor out the constants:
\begin{equation}
  = \frac{2\pi}{\omega } \sqrt{N_{\mathbf{k^{^{\prime}}\epsilon^{\prime}}}}\sqrt{N_{\mathbf{k} \epsilon}}\left( \ket{\cdots ,N_{\mathbf{k} \epsilon}-1, \cdots, N_{\mathbf{k^{\prime}} \epsilon^{\prime}}-1, \cdots} - \ket{\cdots ,N_{\mathbf{k^{^{\prime}}\epsilon^{\prime}}}-1, \cdots, N_{\mathbf{k} \epsilon}-1, \cdots}\right) = 0
\end{equation}
Let us also consider the case of $\mathbf{k},\epsilon = \mathbf{k^{\prime}},\epsilon^{\prime}$:
\begin{equation}
  [\hat{A}_{\mathbf{k} \epsilon},\hat{A}_{\mathbf{k} \epsilon}] = \hat{A}_{\mathbf{k} \epsilon}\hat{A}_{\mathbf{k} \epsilon} - \hat{A}_{\mathbf{k} \epsilon}\hat{A}_{\mathbf{k} \epsilon} = 0
\end{equation}
So, this case is trivially 0.\\
\begin{equation}
  [\hat{A}_{\mathbf{k} \epsilon}^{\dagger},\hat{A}_{\mathbf{k^{\prime}} \epsilon^{\prime}}^{\dagger}]
\end{equation}
By symmetry, this is also 0.\\
\begin{equation}
  [\hat{A}_{\mathbf{k} \epsilon},\hat{A}^\dagger_{\mathbf{k^{\prime}} \epsilon^{\prime}}] = \hat{A}_{\mathbf{k} \epsilon}\hat{A}^\dagger_{\mathbf{k^{\prime}} \epsilon^{\prime}} - \hat{A}^\dagger_{\mathbf{k^{\prime}} \epsilon^{\prime}}\hat{A}_{\mathbf{k} \epsilon}
\end{equation}
Apply this to the state $\ket{\cdots ,N_{\mathbf{k} \epsilon}, \cdots, N_{\mathbf{k^{\prime}} \epsilon^{\prime}}, \cdots}$:
\begin{equation}
  \hat{A}_{\mathbf{k} \epsilon}\sqrt{\frac{2\pi}{\omega }} \sqrt{N_{\mathbf{k^{^{\prime}}\epsilon^{\prime}}}+1}\ket{\cdots ,N_{\mathbf{k} \epsilon}, \cdots, N_{\mathbf{k^{\prime}} \epsilon^{\prime}}+1, \cdots} - \hat{A}^\dagger_{\mathbf{k^{\prime}} \epsilon^{\prime}}\sqrt{\frac{2\pi}{\omega }} \sqrt{N_{\mathbf{k} \epsilon}}\ket{\cdots ,N_{\mathbf{k} \epsilon}-1, \cdots, N_{\mathbf{k^{\prime}} \epsilon^{\prime}}, \cdots}
\end{equation}
Applying the second operator gives:
\begin{equation}
  =\sqrt{\frac{2\pi}{\omega }} \sqrt{N_{\mathbf{k^{^{\prime}}\epsilon^{\prime}}}+1}\sqrt{\frac{2\pi}{\omega }} \sqrt{N_{\mathbf{k} \epsilon}}\ket{\cdots ,N_{\mathbf{k} \epsilon}-1, \cdots, N_{\mathbf{k^{\prime}} \epsilon^{\prime}}+1, \cdots}
\end{equation}
\begin{equation}
  - \sqrt{\frac{2\pi}{\omega }} \sqrt{N_{\mathbf{k} \epsilon}}\sqrt{\frac{2\pi}{\omega }} \sqrt{N_{\mathbf{k^{^{\prime}}\epsilon^{\prime}}}+1}\ket{\cdots ,N_{\mathbf{k} \epsilon}-1, \cdots, N_{\mathbf{k^{\prime}} \epsilon^{\prime}}+1, \cdots}
\end{equation}
We can factor out the constants:
\begin{equation}
  = \frac{2\pi}{\omega } \sqrt{N_{\mathbf{k^{^{\prime}}\epsilon^{\prime}}}+1}\sqrt{N_{\mathbf{k} \epsilon}}\left( \ket{\cdots ,N_{\mathbf{k} \epsilon}-1, \cdots, N_{\mathbf{k^{\prime}} \epsilon^{\prime}}+1, \cdots} - \ket{\cdots ,N_{\mathbf{k} \epsilon}-1, \cdots, N_{\mathbf{k^{\prime}} \epsilon^{\prime}}+1, \cdots}\right) = 0
\end{equation}
Now, we consider the case of $\mathbf{k},\epsilon = \mathbf{k^{\prime}},\epsilon^{\prime}$:
\begin{equation}
  [\hat{A}_{\mathbf{k} \epsilon},\hat{A}^\dagger_{\mathbf{k} \epsilon}] = \hat{A}_{\mathbf{k} \epsilon}\hat{A}^\dagger_{\mathbf{k} \epsilon} - \hat{A}^\dagger_{\mathbf{k} \epsilon}\hat{A}_{\mathbf{k} \epsilon}
\end{equation}
Apply this to the state $\ket{\cdots ,N_{\mathbf{k} \epsilon}, \cdots}$:
\begin{equation}
  \hat{A}_{\mathbf{k} \epsilon}\sqrt{\frac{2\pi}{\omega }} \sqrt{N_{\mathbf{k} \epsilon}+1}\ket{\cdots ,N_{\mathbf{k} \epsilon}+1, \cdots} - \hat{A}^\dagger_{\mathbf{k} \epsilon}\sqrt{\frac{2\pi}{\omega }} \sqrt{N_{\mathbf{k} \epsilon}}\ket{\cdots ,N_{\mathbf{k} \epsilon}-1, \cdots}
\end{equation}
Applying the second operator gives:
\begin{equation}
  =\sqrt{\frac{2\pi}{\omega }} \sqrt{N_{\mathbf{k} \epsilon}+1}\sqrt{\frac{2\pi}{\omega }} \sqrt{N_{\mathbf{k} \epsilon}+1}\ket{\cdots ,N_{\mathbf{k} \epsilon}, \cdots}  - \sqrt{\frac{2\pi}{\omega }} \sqrt{N_{\mathbf{k} \epsilon}}\sqrt{\frac{2\pi}{\omega }} \sqrt{N_{\mathbf{k} \epsilon}}\ket{\cdots ,N_{\mathbf{k} \epsilon}, \cdots}
\end{equation}
We can factor out the constants:
\begin{equation}
  = \frac{2\pi}{\omega } \left( \left( N_{\mathbf{k} \epsilon}+1\right) - N_{\mathbf{k} \epsilon}\right)\ket{\cdots ,N_{\mathbf{k} \epsilon}, \cdots} = \frac{2\pi}{\omega }\ket{\cdots ,N_{\mathbf{k} \epsilon}, \cdots}
\end{equation}
So, we have that:
\begin{equation}
  [\hat{A}_{\mathbf{k} \epsilon},\hat{A}^\dagger_{\mathbf{k} \epsilon}] = \frac{2\pi}{\omega }
\end{equation}
and thus we have a relation involving the kronecker delta for this one:
\begin{equation}
  [\hat{A}_{\mathbf{k} \epsilon},\hat{A}^\dagger_{\mathbf{k^{\prime}} \epsilon^{\prime}}] = \frac{2\pi}{\omega }\delta_{\mathbf{k}\mathbf{k^{\prime}}}\delta_{\epsilon\epsilon^{\prime}}
\end{equation}

(b) We may define the quantum mechanical electric field operator according to:


\begin{equation*}
\hat{\mathbf{E}}(\mathbf{x})=\frac{1}{\sqrt{V}} \sum_{\mathbf{k} \boldsymbol{\epsilon}}\left(-i \omega \hat{A}_{\mathbf{k} \boldsymbol{\epsilon}} \boldsymbol{\epsilon} e^{i \mathbf{k} \cdot \mathbf{x}}+i \omega \hat{A}_{\mathbf{k} \boldsymbol{\epsilon}}^{\dagger} \boldsymbol{\epsilon}^{*} e^{-i \mathbf{k} \cdot \mathbf{x}}\right) . \tag{7}
\end{equation*}


Make sure this definition makes sense to you. We know that classically we can compute the energy density in an electromagnetic field as proportional to the squared electric (plus squared magnetic) fields. Compute the expectation value:


\begin{equation*}
\left\langle\Omega\left|\hat{\mathbf{E}}(\mathbf{x}) \cdot \hat{\mathbf{E}}\left(\mathbf{x}^{\prime}\right)\right| \Omega\right\rangle \tag{8}
\end{equation*}


Try to express your answer in terms of a dimensionless integral in one dimension (which will perhaps look divergent). Make sure the dimension of the coefficient is what you expect.
\subsection{}
The dot product of the electric field operator multiplied by itself, with the second term carrying a prime, is:
\begin{equation}
  \hat{\mathbf{E}}(\mathbf{x}) \cdot \hat{\mathbf{E}}\left(\mathbf{x}^{\prime}\right) = \frac{1}{V}\sum_{\mathbf{k} \boldsymbol{\epsilon}}\sum_{\mathbf{k}^{\prime} \boldsymbol{\epsilon}^{\prime}}\left(-i \omega \hat{A}_{\mathbf{k} \boldsymbol{\epsilon}} \boldsymbol{\epsilon} e^{i \mathbf{k} \cdot \mathbf{x}}+i \omega \hat{A}_{\mathbf{k} \boldsymbol{\epsilon}}^{\dagger} \boldsymbol{\epsilon}^{*} e^{-i \mathbf{k} \cdot \mathbf{x}}\right)\cdot \left(-i \omega \hat{A}_{\mathbf{k}^{\prime} \boldsymbol{\epsilon}^{\prime}} \boldsymbol{\epsilon}^{\prime} e^{i \mathbf{k}^{\prime} \cdot \mathbf{x}^{\prime}}+i \omega \hat{A}_{\mathbf{k}^{\prime} \boldsymbol{\epsilon}^{\prime}}^{\dagger} \boldsymbol{\epsilon}^{\prime *} e^{-i \mathbf{k}^{\prime} \cdot \mathbf{x}^{\prime}}\right)
\end{equation}
Since we are operating on the vacuum state, we know that terms with the creation operator followed by the annihilation operator vanish, we can factor out the $\omega^2$. And we also know from the previous part that even with the ideal ordering of the second quantization operators, the terms vanish unless we have that they are operating on the same modes and polarization vectors, so it simplifies to only one sum. Furthermore, this expression is shorter, so we now consider its expectation value inside the vacuum state:
\begin{equation}
  \left\langle\Omega\left|\hat{\mathbf{E}}(\mathbf{x}) \cdot \hat{\mathbf{E}}\left(\mathbf{x}^{\prime}\right)\right| \Omega\right\rangle = \frac{1}{V}\sum_{\mathbf{k} \boldsymbol{\epsilon}}\omega ^{2}\left\langle\Omega\left|\hat{A}_{\mathbf{k} \boldsymbol{\epsilon}} \boldsymbol{\epsilon} e^{i \mathbf{k} \cdot \mathbf{x}}\hat{A}_{\mathbf{k} \boldsymbol{\epsilon}}^{\dagger} \boldsymbol{\epsilon}^{*} e^{-i \mathbf{k} \cdot \mathbf{x}^{\prime}}\right| \Omega\right\rangle
\end{equation}
We can also take the resulting exponential combination outside of the matrix element and simplify:
\begin{equation}
  = \frac{1}{V}\sum_{\mathbf{k} \boldsymbol{\epsilon}} \omega ^{2}|\boldsymbol{\epsilon}|^2 e^{i \textbf{k}\left( \textbf{x}-\textbf{x}^{\prime}\right)}\left\langle\Omega\left|\hat{A}_{\mathbf{k} \boldsymbol{\epsilon}} \hat{A}_{\mathbf{k} \boldsymbol{\epsilon}}^{\dagger}\right| \Omega\right\rangle
\end{equation}
We know that the norm of the polarization vector will always be 1:
\begin{equation}
  = \frac{1}{V}\sum_{\mathbf{k} \boldsymbol{\epsilon}} \omega ^{2}e^{i \textbf{k}\left( \textbf{x}-\textbf{x}^{\prime}\right)}\left\langle\Omega\left|\hat{A}_{\mathbf{k} \boldsymbol{\epsilon}} \hat{A}_{\mathbf{k} \boldsymbol{\epsilon}}^{\dagger}\right| \Omega\right\rangle
\end{equation}
The matrix element will add a factor of $\frac{2\pi}{\omega }$:
\begin{equation}
  = \frac{2\pi }{V}\sum_{\mathbf{k} \boldsymbol{\epsilon}} \omega e^{i \textbf{k}\left( \textbf{x}-\textbf{x}^{\prime}\right)}
\end{equation}
Since the spectrum of modes is nearly continuous, we can approximate this as an integral:
\begin{equation}
  = \frac{2\pi \ }{V}\int d^3k \omega e^{i \textbf{k}\left( \textbf{x}-\textbf{x}^{\prime}\right)} = \frac{2\pi }{V}\iiint \omega \frac{k^{2} \dd{k}\dd{\Omega}}{(2\pi)^3} e^{i \textbf{k}\left( \textbf{x}-\textbf{x}^{\prime}\right)}
\end{equation}
There is no element in the interpret that depends on the solid ankle, so this just introduces a factor of $4\pi$ and we know that the frequency is related to the wave vector by a dispersion relation $\omega = ck$, so we can write:
\begin{equation}
  = \frac{8 \pi ^2  c}{8\pi^3V}\int_{0}^{\infty } k^3 \dd{k} e^{i \textbf{k}\left( \textbf{x}-\textbf{x}^{\prime}\right)} = \frac{c}{\pi V}\int_{0}^{\infty } k^3 \dd{k} e^{i \textbf{k}\left( \textbf{x}-\textbf{x}^{\prime}\right)}
\end{equation}



(c) Now let's consider the average, $\hat{\overline{\mathbf{E}}}(\mathbf{x})$, of $\hat{\mathbf{E}}(\mathbf{x})$ over a small volume $\mathcal{V}$. That is, we are interested in the average energy density in a small volume. What is


\begin{equation*}
\left\langle\Omega\left|[\hat{\overline{\mathbf{E}}}(\mathbf{x})]^{2}\right| \Omega\right\rangle, \tag{9}
\end{equation*}


and what happens as $\mathcal{V} \rightarrow 0$ ? You may wish to think about your result in terms of harmonic oscillators and zero point energies.
\subsection{}
We want to consider the average of the electric fueled over a small volume $V^{^{\prime}}$ different from the volume for this expression $V$. We have:
\begin{equation}
  \hat{\overline{\mathbf{E}}}(\mathbf{x})=\frac{1}{V^{^{\prime}}} \int_{V^{^{\prime}}}\hat{\mathbf{E}}(\mathbf{x})\dd[3]{x} 
\end{equation}
We will say that $V^{\prime}$ is the volume of a cube which we will later take to be small with side length $2L$ centered at a $\vec{x}$. We want to calculate the average field inside of this Cube. Expanding out the electric field operator with then the expression for the average field:
\begin{equation}
  \hat{\overline{\mathbf{E}}}(\mathbf{x})=\frac{1}{V^{^{\prime}}} \int_{V^{^{\prime}}}\frac{1}{\sqrt{V}} \sum_{\mathbf{k} \boldsymbol{\epsilon}}\left(-i \omega \hat{A}_{\mathbf{k} \boldsymbol{\epsilon}} \boldsymbol{\epsilon} e^{i \mathbf{k} \cdot \mathbf{x}}+i \omega \hat{A}_{\mathbf{k} \boldsymbol{\epsilon}}^{\dagger} \boldsymbol{\epsilon}^{*} e^{-i \mathbf{k} \cdot \mathbf{x}}\right)\dd[3]{x}
\end{equation}
We can bring outside the regular volume since it is just a constant, we can take out the summation from the integral as well, and we can also take out the factor of $i \omega$:
\begin{equation}
  \hat{\overline{\mathbf{E}}}(\mathbf{x})=\frac{1}{V^{^{\prime}}\sqrt{V}} \sum_{\mathbf{k} \boldsymbol{\epsilon}}\left(-i \omega \hat{A}_{\mathbf{k} \boldsymbol{\epsilon}} \boldsymbol{\epsilon} \int_{V^{^{\prime}}}e^{i \mathbf{k} \cdot \mathbf{x}}\dd[3]{x}+i \omega \hat{A}_{\mathbf{k} \boldsymbol{\epsilon}}^{\dagger} \boldsymbol{\epsilon}^{*} \left(\int_{V^{^{\prime}}}e^{i \mathbf{k} \cdot \mathbf{x}}\dd[3]{x}\right)^*\right)
\end{equation}
We start out by computing the into Gal, in a single dimension, but then we can easily gene rasta it to all 3 dimensions:
\begin{equation}
  \int_{x-L}^{x+L}e^{ikx}\dd{x} = \frac{1}{ik}e^{ikx}\Bigg|_{x-L}^{x+L} = \frac{1}{ik}\left(e^{ik(x+L)}-e^{ik(x-L)}\right)
\end{equation}
We can factor out a common factor of $e^{ikx}$ from the exponentials and then also substitute the sine function for the exponentials:
\begin{equation}
  = \frac{1}{ik}e^{ikx}\left(2i\sin(kL)\right) = \frac{2}{k}e^{ikx}\sin(kL)
\end{equation}
In three dimensions, this will simply become the product of the three one-dimensional integrals:
\begin{equation}
  \int_{V^{^{\prime}}}e^{i \mathbf{k} \cdot \mathbf{x}}\dd[3]{x} = 8 e^{i \mathbf{k} \cdot \mathbf{x}}\left( \frac{\sin(kL)}{k}\right)^3
\end{equation}
We can now substitute this back into the expression for the average electric field and then we also know that $V^{^{\prime}} = (2L)^3$:
\begin{equation}
  \hat{\overline{\mathbf{E}}}(\mathbf{x})=\frac{-8i}{8L^3\sqrt{V}} \sum_{\mathbf{k} \boldsymbol{\epsilon}}\omega \left( \hat{A}_{\mathbf{k} \boldsymbol{\epsilon}} \boldsymbol{\epsilon} e^{i \mathbf{k} \cdot \mathbf{x}}- \hat{A}_{\mathbf{k} \boldsymbol{\epsilon}}^{\dagger} \boldsymbol{\epsilon}^{*} e^{-i \mathbf{k} \cdot \mathbf{x}}\right) (\frac{\sin(kL)}{k})^3
\end{equation}
\begin{equation}
  = - \frac{i}{\sqrt{V}} \sum_{\mathbf{k} \boldsymbol{\epsilon}}\omega \left( \hat{A}_{\mathbf{k} \boldsymbol{\epsilon}} \boldsymbol{\epsilon} e^{i \mathbf{k} \cdot \mathbf{x}}- \hat{A}_{\mathbf{k} \boldsymbol{\epsilon}}^{\dagger} \boldsymbol{\epsilon}^{*} e^{-i \mathbf{k} \cdot \mathbf{x}}\right) (\frac{\sin(kL)}{kL})^3
\end{equation}
Now, we consider the expectation value of the square:
\begin{equation}
  \left\langle\Omega\left|[\hat{\overline{\mathbf{E}}}(\mathbf{x})]^{2}\right| \Omega\right\rangle
\end{equation}
By the sang logic as in the copious problem, we know that only one term survives from this multiplication:
\begin{equation}
  = +\frac{1}{V} \sum_{\mathbf{k} \boldsymbol{\epsilon}}\omega^2 \left\langle\Omega\left|\hat{A}_{\mathbf{k} \boldsymbol{\epsilon}} \hat{A}_{\mathbf{k} \boldsymbol{\epsilon}}^{\dagger}\right| \Omega\right\rangle (\frac{\sin(kL)}{kL})^6
\end{equation}
We know that the matrix element will add a factor of $\frac{2\pi}{\omega }$:
\begin{equation}
  = \frac{2\pi}{V} \sum_{\mathbf{k} \boldsymbol{\epsilon}}\omega \left(\frac{\sin(kL)}{kL}\right)^6
\end{equation}
Since we want to take the limit of a small cube, we have that $L \rightarrow 0$:
\begin{equation}
  \lim_{L \rightarrow 0}\frac{\sin(kL)}{kL} =\frac{kL}{kL} = 1
\end{equation}
So, the matrix element simplifies to:
\begin{equation}
  \left\langle\Omega\left|[\hat{\overline{\mathbf{E}}}(\mathbf{x})]^{2}\right| \Omega\right\rangle = \frac{2\pi}{V} \sum_{\mathbf{k} \boldsymbol{\epsilon}}\omega
\end{equation}
Again, we can approximate the sum as and integral:
\begin{equation}
  = \frac{2\pi}{V}\iiint \omega \frac{k^{2} \dd{k}\dd{\Omega}}{(2\pi)^3}
\end{equation}
Again, the integral over the solid angle $\dd{\Omega}$ will just introduce a factor of $4\pi$, and we know that the frequency is related to the wave vector by a dispersion relation $\omega = ck$, so we can write:
\begin{equation}
  = \frac{ c}{\pi V}\int_{0}^{\infty } k^3 \dd{k}
\end{equation}
To get a sensual enter form this integral, we have to substitute the upper limit $\Gamma \equiv \infty$:
\begin{equation}
  = \frac{ c}{\pi V}\int_{0}^{\Gamma } k^3 \dd{k} = \frac{ c}{\pi V}\frac{k^4}{4}\Bigg|_{0}^{\Gamma} = \frac{ c}{4\pi V}\Gamma^4
\end{equation}
This expectation value seems to be essentially infinite but other wise I am not sure how to interpret this result.

% \subsection{}
% The average of the electric field over a small volume $\mathcal{V}$ is given by:
% \begin{equation}
%   \hat{\overline{\mathbf{E}}}(\mathbf{x})=\mathcal{V}^{-3/2}\sum_{\mathbf{k} \boldsymbol{\epsilon}}\left(-i \omega \hat{A}_{\mathbf{k} \boldsymbol{\epsilon}} \boldsymbol{\epsilon} e^{i \mathbf{k} \cdot \mathbf{x}}+i \omega \hat{A}_{\mathbf{k} \boldsymbol{\epsilon}}^{\dagger} \boldsymbol{\epsilon}^{*} e^{-i \mathbf{k} \cdot \mathbf{x}}\right)
% \(end{equation})
% so the square of this relation is:
% \begin{equation}
%   (\hat{\overline{\mathbf{E}}}(\mathbf{x}))^2=\mathcal{V}^{-3}\sum_{\mathbf{k} \boldsymbol{\epsilon}}\left(-i \omega \hat{A}_{\mathbf{k} \boldsymbol{\epsilon}} \boldsymbol{\epsilon} e^{i \mathbf{k} \cdot \mathbf{x}}+i \omega \hat{A}_{\mathbf{k} \boldsymbol{\epsilon}}^{\dagger} \boldsymbol{\epsilon}^{*} e^{-i \mathbf{k} \cdot \mathbf{x}}\right)\sum_{\mathbf{k^{\prime}} \boldsymbol{\epsilon^{\prime}}}\left(-i \omega \hat{A}_{\mathbf{k^{\prime}} \boldsymbol{\epsilon^{\prime}}} \boldsymbol{\epsilon^{\prime}} e^{i \mathbf{k^{\prime}} \cdot \mathbf{x}}+i \omega \hat{A}_{\mathbf{k^{\prime}} \boldsymbol{\epsilon^{\prime}}}^{\dagger} \boldsymbol{\epsilon^{\prime}}^{*} e^{-i \mathbf{k^{\prime}} \cdot \mathbf{x}}\right)
% \end{equation}
% Moving the summations out font gives:
% \begin{equation}
%   (\hat{\overline{\mathbf{E}}}(\mathbf{x}))^2=\mathcal{V}^{-3}\sum_{\mathbf{k} \boldsymbol{\epsilon}}\sum_{\mathbf{k^{\prime}} \boldsymbol{\epsilon^{\prime}}}\left(-i \omega \hat{A}_{\mathbf{k} \boldsymbol{\epsilon}} \boldsymbol{\epsilon} e^{i \mathbf{k} \cdot \mathbf{x}}+i \omega \hat{A}_{\mathbf{k} \boldsymbol{\epsilon}}^{\dagger} \boldsymbol{\epsilon}^{*} e^{-i \mathbf{k} \cdot \mathbf{x}}\right)\left(-i \omega \hat{A}_{\mathbf{k^{\prime}} \boldsymbol{\epsilon^{\prime}}} \boldsymbol{\epsilon^{\prime}} e^{i \mathbf{k^{\prime}} \cdot \mathbf{x}}+i \omega \hat{A}_{\mathbf{k^{\prime}} \boldsymbol{\epsilon^{\prime}}}^{\dagger} \boldsymbol{\epsilon^{\prime}}^{*} e^{-i \mathbf{k^{\prime}} \cdot \mathbf{x}}\right)
% \end{equation}
% We multiply this out to get and factor out the $\omega^2$ to get:
% \begin{equation}
% \begin{aligned}
% =\frac{\omega ^2}{\mathcal{V}^{3}}\sum_{\mathbf{k}, \boldsymbol{\epsilon}}\sum_{\mathbf{k^{\prime}}, \boldsymbol{\epsilon^{\prime}}}\Bigg( 
% &-e^{i \mathbf{x}\cdot (\mathbf{k}+\mathbf{k^{\prime}})}\hat{A}_{\mathbf{k}, \boldsymbol{\epsilon}} \boldsymbol{\epsilon} \hat{A}_{\mathbf{k^{\prime}}, \boldsymbol{\epsilon^{\prime}}} \boldsymbol{\epsilon^{\prime}} \\
% &- e^{-i \mathbf{x}\cdot (\mathbf{k}+\mathbf{k^{\prime}})}\hat{A}_{\mathbf{k}, \boldsymbol{\epsilon}}^{\dagger} \boldsymbol{\epsilon}^{*} \hat{A}_{\mathbf{k^{\prime}}, \boldsymbol{\epsilon^{\prime}}}^{\dagger} \boldsymbol{\epsilon^{\prime}}^{*}\\
% &+ e^{i \mathbf{x}\cdot (\mathbf{k}-\mathbf{k^{\prime}})}\hat{A}_{\mathbf{k}, \boldsymbol{\epsilon}} \boldsymbol{\epsilon} \hat{A}_{\mathbf{k^{\prime}}, \boldsymbol{\epsilon^{\prime}}}^{\dagger} \boldsymbol{\epsilon^{\prime}}^{*} \\
% &+ e^{i \mathbf{x}\cdot (\mathbf{k}-\mathbf{k^{\prime}})}\hat{A}_{\mathbf{k}, \boldsymbol{\epsilon}}^{\dagger} \boldsymbol{\epsilon}^{*} \hat{A}_{\mathbf{k^{\prime}}, \boldsymbol{\epsilon^{\prime}}} \boldsymbol{\epsilon^{\prime}}\Bigg)
% \end{aligned}
% \end{equation}
% Since we are operating on the vacuum state, the terms which have an annihilation operator on the right side or a creation operator on the left side it well vanish, so we are just left with the third term:
% \begin{equation}
%   =\frac{\omega ^2}{\mathcal{V}^{3}}\sum_{\mathbf{k}, \boldsymbol{\epsilon}}\sum_{\mathbf{k^{\prime}}, \boldsymbol{\epsilon^{\prime}}}e^{i \mathbf{x}\cdot (\mathbf{k}-\mathbf{k^{\prime}})}\hat{A}_{\mathbf{k}, \boldsymbol{\epsilon}} \boldsymbol{\epsilon} \hat{A}_{\mathbf{k^{\prime}}, \boldsymbol{\epsilon^{\prime}}}^{\dagger} \boldsymbol{\epsilon^{\prime}}^{*}
% \end{equation}
% Now, we insert this relation between two vacuum states:
% \begin{equation}
%   \left\langle\Omega\left|[\hat{\overline{\mathbf{E}}}(\mathbf{x})]^{2}\right| \Omega\right\rangle = \frac{\omega ^2}{\mathcal{V}^{3}}\sum_{\mathbf{k}, \boldsymbol{\epsilon}}\sum_{\mathbf{k^{\prime}}, \boldsymbol{\epsilon^{\prime}}}e^{i \mathbf{x}\cdot (\mathbf{k}-\mathbf{k^{\prime}})}\boldsymbol{\epsilon} \boldsymbol{\epsilon^{\prime}}^{*}\left\langle\Omega\left|\hat{A}_{\mathbf{k}, \boldsymbol{\epsilon}} \hat{A}_{\mathbf{k^{\prime}}, \boldsymbol{\epsilon^{\prime}}}^{\dagger}\right| \Omega\right\rangle
% \end{equation}
% The summations will pick out only the term which has the same $\mathbf{k}$ and $\boldsymbol{\epsilon}$, so we get two kronecker deltas instead of the sums:
% \begin{equation}
%   \left\langle\Omega\left|[\hat{\overline{\mathbf{E}}}(\mathbf{x})]^{2}\right| \Omega\right\rangle = \frac{\omega ^2}{\mathcal{V}^{3}}|\boldsymbol{\varepsilon }^2|\left\langle\Omega\left|\hat{A}_{\mathbf{k}, \boldsymbol{\epsilon}} \hat{A}_{\mathbf{k}, \boldsymbol{\epsilon}}^{\dagger}\right| \Omega\right\rangle
% \end{equation}
% It was showed in the notes that this expectation value just becomes $\frac{2\pi}{\omega }$, so we get:
% \begin{equation}
%   \left\langle\Omega\left|[\hat{\overline{\mathbf{E}}}(\mathbf{x})]^{2}\right| \Omega\right\rangle = \frac{2\pi\omega }{\mathcal{V}^{3}}|\boldsymbol{\varepsilon }^2|
% \end{equation}
% As the volume goes to zero, the energy density goes to infinity, which is the zero-point energy of the electromagnetic field. The volume is proportional to the wavelength in a cavity, and this is inversely related to the frequency of a harmonic oscillator. As the frequency of a harmonic oscillator goes to infinity, the energy also goes to infinity.
 
\end{document}