\documentclass[12pt]{article}
\usepackage[utf8]{inputenc}
\usepackage[T1]{fontenc}
\usepackage{amsmath}
\usepackage{amsfonts}
\usepackage{amssymb}
\usepackage[version=4]{mhchem}
\usepackage{stmaryrd}

\usepackage{listings} % Required for insertion of code
\usepackage{xcolor} % Required for custom colors

% Define custom colors
\definecolor{codegreen}{rgb}{0,0.6,0}
\definecolor{codegray}{rgb}{0.5,0.5,0.5}
\definecolor{codepurple}{rgb}{0.58,0,0.82}
\definecolor{backcolour}{rgb}{0.95,0.95,0.92}

% Setup the style for code listings
\lstdefinestyle{mystyle}{
    backgroundcolor=\color{backcolour},   
    commentstyle=\color{codegreen},
    keywordstyle=\color{magenta},
    numberstyle=\tiny\color{codegray},
    stringstyle=\color{codepurple},
    basicstyle=\ttfamily\footnotesize,
    breakatwhitespace=false,         
    breaklines=true,                 
    captionpos=b,                    
    keepspaces=true,                 
    numbers=left,                    
    numbersep=5pt,                  
    showspaces=false,                
    showstringspaces=false,
    showtabs=false,                  
    tabsize=2
}

% Activate the style
\lstset{style=mystyle}


\title{Ch/ChE 164 Winter 2024 }


\author{Homework Problem Set \#4}
\date{}


\begin{document}
\maketitle
Due Date: Thursday, February 15, 2024 @ 11:59pm PT

For all problems, please consider reasonable simplifications of your final results.
\section{}
\begin{enumerate}
  \item (15 pts.) (Adapted from Callen). Consider a mixture of two non-identical monatomic ideal gases.
\end{enumerate}
\subsection{}
\begin{itemize}
  \item Starting from the expression for the grand canonical partition function and taking the limit of small fugacity, show that the canonical partition function $Z$ is factorizable and
\end{itemize}

$$
Z=Z_{1} Z_{2}=\frac{1}{N_{1} !} q_{1}^{N_{1}} \frac{1}{N_{2} !} q_{2}^{N_{2}}
$$

(You may wish to use the occupancy representation $\mid n_{1} m_{1}, n_{2} m_{2} \ldots$ ), where $n_{1}$ denotes occupancy of energy level 1 of gas 1 , and $m_{1}$ denotes occupancy of energy level 1 of gas 2 , etc.).\\
First, I will say that an individual partition function for each molecule is given by:
\begin{equation}
  Z_i = \sum_{n_i} e^{-\beta \epsilon_i n_i}
\end{equation}
The occupations of the first and second gases are given by:
\begin{equation}
  n_1, n_2,...
\end{equation}
\begin{equation}
  m_1, m_2,...
\end{equation}
The number of particles in the first and second gases are given by:
\begin{equation}
  N_1 = \sum_{i} n_i
\end{equation}
\begin{equation}
  N_2 = \sum_{i} m_i
\end{equation}
Total energy of the first and second gases are given by:
\begin{equation}
  U_1 = \sum_{i} n_i \epsilon_i
\end{equation}
\begin{equation}
  U_2 = \sum_{i} m_i \epsilon_i
\end{equation}
Their sum is
\begin{equation}
  U = U_1 + U_2 = \sum_{i} n_i \epsilon_i + \sum_{i} m_i \epsilon_i
\end{equation}
The grand canonical partition function is given by:
\begin{equation}
  \begin{aligned}
    \Xi &= \sum_{\nu} e^{-\beta(U_{\nu}) +  \beta \mu N_{1} + \beta \mu N_{2}} \\
\end{aligned}
\end{equation}
Adding the various sums to the exponent gives:
\begin{equation}
  \Xi = \sum_{\nu} e^{-\beta \sum_{i} n_i \epsilon_i - \beta \sum_{i} m_i \epsilon_i +  \beta \mu^{(1)} \sum_{i} n_i + \beta \mu^{(2)} \sum_{i} m_i}
\end{equation}
where $\mu^{(1)}$ and $\mu^{(2)}$ are the chemical potentials of the first and second gases, respectively. We can factor this in two separate sums each over the individual molecule:
\begin{equation}
  \Xi = \sum_{{n}} e^{-\beta \sum_{i} n_i \epsilon_i +  \beta \mu^{(1)} \sum_{i} n_i} \sum_{{m}}e^{-\beta \sum_{i} m_i \epsilon_i +  \beta \mu^{(2)} \sum_{i} m_i} 
\end{equation}
In 
\subsection{}
The sum over the states can be written as a product of sums over the states of the first and second gases:
\begin{itemize}
  \item Compute the entropy and show that (comparing to the entropy of the two separate gases) there is an entropy of mixing of the form
\end{itemize}

$$
S_{\text {mixing }}=\left(-x_{1} \log x_{1}-x_{2} \log x_{2}\right) N k
$$

where $N$ is the total number of particles.
Look at Callen pg. 393 ch 18
\section{}
\begin{enumerate}
  \setcounter{enumi}{1}
  \item In class we derived the heat capacity of the Fermi gas at low temperature by an intuitive argument, which $C_{v} \sim N k O\left(T / T_{F}\right)$. Here we will derive the precise form and constants (adapted from Callen).
\end{enumerate}

Denote the Fermi-Dirac distribution at temperature $T$ as $f(\epsilon, T)$ and the (temperature dependent) chemical potential by $\mu$ (note this is not the Fermi energy $\epsilon_{F}$ except when $T=0$ ). We will first derive a general result for an integral of the form (Sommerfeld expansion)

$$
I \equiv \int_{0}^{\infty} \phi(\epsilon) f(\epsilon, T) d \epsilon=\int_{0}^{\mu} \phi(\epsilon) d \epsilon+\frac{\pi^{2}}{6}(k T)^{2} \phi^{\prime}(\mu)+\frac{7 \pi^{4}}{360}(k T)^{4} \phi^{\prime \prime \prime}(\mu)+\ldots
$$

a) (10 pts.) Integrate $I$ by parts, and let $\Phi \equiv \int_{0}^{\epsilon} \phi\left(\epsilon^{\prime}\right) d \epsilon^{\prime}$. Then expanding $\Phi(\epsilon)$ in a power series in $\epsilon-\mu$ to third order, deduce
The fd distribution is given by:
\begin{equation}
  f(\epsilon, T) = \frac{1}{e^{\beta(\epsilon - \mu)} + 1}
\end{equation}
where $\beta = 1/kT$. The number of particles is a sum over all sxztates $\alpha$:
\begin{equation}
  N = \sum_{\alpha} f(\epsilon_{\alpha}, T)
\end{equation}

$$
I=-\sum_{m=0}^{\infty} \frac{1}{m !} \frac{d^{m} \Phi(\mu)}{d \mu^{m}} I_{m}
$$

where $I_{m}=\int_{0}^{\infty}(\epsilon-\mu)^{m} \frac{d f}{d \epsilon} d \epsilon=-\beta^{-m} \int_{-\beta \mu}^{\infty} \frac{e^{x}}{\left(e^{x}+1\right)^{2}} x^{m} d x$

b) (5 pts.) Show that only an exponentially small error is made by taking the lower limit of integration as $-\infty$, and that then all terms with $m$ odd vanish.

c) (5 pts.) Evaluate the first two non-vanishing terms and show that this agrees with the expansion of $I$.

d) (10 pts.) Using the result for $I$, express $N$ in the form of such an integral and obtain an expansion for $N(V, T, \mu)$ in terms of $k T / \mu$ (to second order). Verify that $T \rightarrow 0$ yields the relation between $N$ and $\epsilon_{F}$ derived in class.

e) (10 pts.) Invert this relationship to obtain $\mu(T)$ as a function of $k T / \epsilon_{F}$ (to second order) for fixed $N$.

f) (5 pts.) Similarly obtain an expansion for the internal energy $E$ as a function of $k T / \mu$ (to second order).

g) (5 pts.) Substituting in $\mu(T)$ into the energy expansion, obtain an expansion of $E$ in $k T / \epsilon_{F}$ to second order, and thus $C_{v}$. Hence see why we skipped the detailed computation in class.
\section{}
\begin{enumerate}
  \setcounter{enumi}{2}
  \item (20 pts.) Show that for the Bose-Einstein and Fermi-Dirac gas at low density and/or high temperature the equation of state is given by
\end{enumerate}

$$
p=k T \rho\left(1 \mp \frac{\rho \Lambda^{3}}{2^{5 / 2}}+\ldots\right)
$$

where $\Lambda=h / \sqrt{2 \pi m k T}$ is the thermal de Broglie wavelength, and the upper (lower) sign is for the Bose-Einstein (Fermi-Dirac) gas.

We start with the thermodynamic identity for the pressure $p$ in terms of the grand canonical partition function $\Xi$:
\begin{equation}
    pV\beta = \ln \Xi
\end{equation}
where $\beta = \frac{1}{kT}$, with $k$ being the Boltzmann constant and $T$ the temperature.

The grand canonical partition function for Bose-Einstein (BE) and Fermi-Dirac (FD) gases are given by:
\begin{align}
    \ln \Xi_{\text{BE}} &= \sum_{\nu} \ln \left(1 - e^{-\beta(\epsilon_{\nu} - \mu)}\right) \\
    \ln \Xi_{\text{FD}} &= \sum_{\nu} \ln \left(1 + e^{-\beta(\epsilon_{\nu} - \mu)}\right)
\end{align}
where $\epsilon_{\nu}$ is the energy of state $\nu$ and $\mu$ is the chemical potential.
We will go through the derivation for the Fermi-Dirac gas and then show how similar steps repeat for the Bose-Einstein gas, noting that a comprehensive deviation was given in lecture 8.\\
After performing a Taylor expansion to second order in the limit of small fugacity inside of the integral in spherical coordinates for this, we get:
\begin{equation}
\ln \Xi=\frac{V}{\Lambda^3}(2 s+1) f_{5 / 2}(\zeta)
\end{equation}
So, we have arrived at:
\begin{equation}
    p(T, \mu)\beta = \frac{2s+1}{\Lambda^3}f_{5/2}(\zeta)
\end{equation}
where:
\begin{equation}
f_{5 / 2}(\zeta)=\zeta-\frac{\zeta^2}{2^{5 / 2}}
\end{equation}
to the 2nd order.
For simplicity, we assume that $s=0$.
But now we want to get rid of the fugacity, and turn it into a density. For this, we can use the virial expansion derived in the notes:
\begin{equation}
\zeta=\Lambda^3 \rho+\frac{\left(\Lambda^3 \rho\right)^2}{2^{3 / 2}}
\end{equation}

\begin{equation}
    p = kT\rho\left(1 + \frac{\rho\Lambda^{3}}{2^{5/2}} + \ldots\right)
\end{equation}
For the BE case, we will have:
\begin{equation}
  g_{5 / 2}(\zeta)=\zeta+\frac{\zeta^2}{2^{5 / 2}}
\end{equation}
so the sign is flipped and we get a final result of:
\begin{equation}
    p = kT\rho\left(1 - \frac{\rho\Lambda^{3}}{2^{5/2}} + \ldots\right)
\end{equation}

% We know that the particle number averages for the distributions are given by:
% \begin{equation}
%   \begin{aligned}
%     \bar{n}_{\text{BE}} &= \frac{1}{e^{\beta(\epsilon - \mu)} - 1} \\
%     \bar{n}_{\text{FD}} &= \frac{1}{e^{\beta(\epsilon - \mu)} + 1}
%   \end{aligned}
% \end{equation}

\end{document}