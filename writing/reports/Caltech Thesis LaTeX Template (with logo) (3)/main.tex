%%%%%%%%%%%%
%% Please rename this main.tex file and the output PDF to
%% [lastname_firstname_graduationyear]
%% before submission.
%%
%% This .tex file is for use with BibLaTeX. Please use
%% main-bibtex.tex instead if you prefer BibTeX.
%%%%%%%%%%%%

\documentclass[12pt]{caltech_thesis}
\usepackage[hyphens]{url}
\usepackage{lipsum}
\usepackage{graphicx}

\usepackage{todonotes}

%% Tentative: newtx for better-looking Times
\usepackage[utf8]{inputenc}
\usepackage[T1]{fontenc}
\usepackage{newtxtext,newtxmath}

% Must use biblatex to produce the Published Contents and Contributions, per-chapter bibliography (if desired), etc.
\usepackage[
    backend=biber,natbib,
    % IMPORTANT: load a style suitable for your discipline
    style=authoryear
]{biblatex}
\usepackage{nomencl}
\makenomenclature
%% This code creates the groups
% -----------------------------------------
\usepackage{etoolbox}
\renewcommand\nomgroup[1]{%
  \item[\bfseries
  \ifstrequal{#1}{M}{Molecular orbital basis}{%
  \ifstrequal{#1}{N}{Number sets}{%
  \ifstrequal{#1}{O}{Other symbols}{}}}%
]}
% -----------------------------------------

% Name of your .bib file(s)
\addbibresource{example.bib}
\addbibresource{ownpubs.bib}

\begin{document}

% Do remember to remove the square bracket!
\title{[Thesis Title]}
\author{[Your Full Name]}

\degreeaward{[Name of Degree]}                 % Degree to be awarded
\university{California Institute of Technology}    % Institution name
\address{Pasadena, California}                     % Institution address
\unilogo{caltech.png}                                 % Institution logo
\copyyear{[Year Degree Conferred]}  % Year (of graduation) on diploma
\defenddate{[Exact Date]}          % Date of defense

\orcid{[Author ORCID]}

%% IMPORTANT: Select ONE of the rights statement below.
\rightsstatement{All rights reserved\todo[size=\footnotesize]{Choose one from the choices in the source code!! And delete this \texttt{todo} when you're done that. :-)}}
% \rightsstatement{All rights reserved except where otherwise noted}
% \rightsstatement{Some rights reserved. This thesis is distributed under a [name license, e.g., ``Creative Commons Attribution-NonCommercial-ShareAlike License'']}

%%  If you'd like to remove the Caltech logo from your title page, simply remove the "[logo]" text from the maketitle command
\maketitle[logo]
%\maketitle

\begin{acknowledgements} 	 
   [Add acknowledgements here. If you do not wish to add any to your thesis, you may simply add a blank titled Acknowledgements page.]
\end{acknowledgements}

\begin{abstract}
   [This abstract must provide a succinct and informative condensation of your work. Candidates are welcome to prepare a lengthier abstract for inclusion in the dissertation, and provide a shorter one in the CaltechTHESIS record.]
\end{abstract}

%% Uncomment the `iknowhattodo' option to dismiss the instruction in the PDF.
\begin{publishedcontent}%[iknowwhattodo]
% List your publications and contributions here.
\nocite{Cahn:etal:2015,Cahn:etal:2016}
\end{publishedcontent}

\tableofcontents
\listoffigures
\listoftables
\printnomenclature

\mainmatter

\chapter{nomenclature}
What follows is uses the restricted Hartree-Fock formalism with doubly occupied spatial orbitals.
\begin{tabular}{p{0.2\textwidth} p{0.8\textwidth}}
Symbol & Description \\
\hline
\(i,j,k,l\) & Occupied orbitals \\
\(a,b,c,d\) & Virtual orbitals \\
\(p,q,r,s\) & General MO basis \\
\(\mu,\nu,\lambda,\sigma\) & AO basis \\
\((\mu\nu|\lambda\sigma)\) & Two-electron integrals \\

\end{tabular}





\chapter{$G_0W_0$}
\section{Interpretive procedure}
The procedure that was used to compute the quasiparticle energies is given by the below equation:
\begin{equation}
    \delta_{pq}F_{pq}^{HF}[\gamma^{DFT}] + \Sigma_{p}^{corr}(\varepsilon_{p}^{QP}) = \varepsilon_{p}^{QP}
\end{equation}
We explain the notation starting from left to right. The first term corresponds to taking the diagonal $\delta_{pq}$ of the Hartree-Fock matrix $F_{pq}^{HF}$ evaluated at a given electron density $\gamma$. These electron densities are obtained from a previous mean-field calculation, either $\gamma_{DFT}$ or $\gamma_{HF}$. The second term evaluates the real part of the correlation self-energy for the $\varepsilon_{p}^{QP}$ determined in the previous  iteration. The right side of the equality gives the updated quasiparticle energy.
\subsection{The Fock Matrix}
In the basis of atomic orbitals, this is given by:
\begin{equation}
F_{\mu\nu}^{HF} = h_{\mu\nu} + \sum_{\lambda\sigma}P_{\lambda\sigma}(\mu\nu|\lambda\sigma) - \frac{1}{2}\sum_{\lambda\sigma}P_{\lambda\sigma}(\mu\lambda|\nu\sigma)
\end{equation}
where $h_{\mu\nu}$ is the one-electron part of the Hamiltonian, $P_{\lambda\sigma}$ is the density matrix, and $(\mu\nu|\lambda\sigma)$ is one of the two-electron integrals. Cite Szabo. This is the simple form of the Hartree-Fock matrix that we want to use here and not the DFT Fock matrix. We transform this Fock matrix into the MO basis with:
\begin{equation}
   F_{pq} = \sum_{\mu} \sum_{\nu} C_{\mu p}^{*}F_{\mu\nu}C_{\nu q}
\end{equation}
where $C$ is the matrix of MO coefficients. Another useful identity is for the density matrix in terms of the MO coefficients from the mean-field calculation: 
\begin{equation}
P_{\mu\nu} = 2\sum_{i=1}^{N/2}C_{\mu i}C_{\nu i}^{*}
\end{equation}
We note that the sum runs only over the $N/2$ occupied \emph{spatial} orbitals.
\subsection{Real Correlation-Solve Energy}
This is the second term in 2.1. It is dynamic, as opposed to the previous Fock term that was discussed, as it is updated with a new quasiparticle energy in each iteration. In the case of the $G_0W_0$ approximation, we are only interested in the diagonal element of $\Sigma^{corr}$ corresponding to the orbital with index $p$. This function is evaluated at the QP energy $\varepsilon_{p}^{QP}$ just obtained in the previous iteration. We will go into greater detail about the form of $\Sigma^{corr}$ in the next chapter.


\chapter{This is the Third Chapter}

\publishedas{Cahn:etal:2015}

[You can have chapters that w3ere published as part of your thesis. The text style of the body should be single column, as it was submitted to the publisher, not formatted as the publisher did.]

\chapter{This is the Fourth Chapter}
\chapter{This is the Fifth Chapter}
\chapter{This is the Sixth Chapter}
\chapter{This is the Seventh Chapter}
\chapter{This is the Eighth Chapter}

\printbibliography[heading=bibintoc]

\appendix

\chapter{Questionnaire}
\chapter{Consent Form}

\printindex

\theendnotes

%% Pocket materials at the VERY END of thesis
\pocketmaterial
\extrachapter{Pocket Material: Map of Case Study Solar Systems} 


\end{document}
