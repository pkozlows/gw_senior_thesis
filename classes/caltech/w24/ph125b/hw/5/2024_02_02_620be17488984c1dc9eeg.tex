\documentclass[12pt]{article}
\usepackage[utf8]{inputenc}
\usepackage[T1]{fontenc}
\usepackage{amsmath}
\usepackage{amsfonts}
\usepackage{physics}

\usepackage{amssymb}
\usepackage[version=4]{mhchem}
\usepackage{stmaryrd}

\usepackage{listings} % Required for insertion of code
\usepackage{xcolor} % Required for custom colors

% Define custom colors
\definecolor{codegreen}{rgb}{0,0.6,0}
\definecolor{codegray}{rgb}{0.5,0.5,0.5}
\definecolor{codepurple}{rgb}{0.58,0,0.82}
\definecolor{backcolour}{rgb}{0.95,0.95,0.92}

% Setup the style for code listings
\lstdefinestyle{mystyle}{
    backgroundcolor=\color{backcolour},   
    commentstyle=\color{codegreen},
    keywordstyle=\color{magenta},
    numberstyle=\tiny\color{codegray},
    stringstyle=\color{codepurple},
    basicstyle=\ttfamily\footnotesize,
    breakatwhitespace=false,         
    breaklines=true,                 
    captionpos=b,                    
    keepspaces=true,                 
    numbers=left,                    
    numbersep=5pt,                  
    showspaces=false,                
    showstringspaces=false,
    showtabs=false,                  
    tabsize=2
}

% Activate the style
\lstset{style=mystyle}

\title{PROBLEMS: }

\author{}
\date{}


\begin{document}
\maketitle
Physics $125 \mathrm{~b}$

Midterm (Problem set number 5)

Due midnight Wednesday, February 7, 2024

READING: Section 18.5 in Shankar on the interaction of atoms with electromagnetic radiation.

This is the Ph 125a "midterm", aka problem set 5 .

Collaboration prohibited: Please do this problem set without consulting with or working with other students. This is the only restriction compared with our usual problem sets. You may use Piazza and attend office hours.\\
\emph{I have an extension until 2-11 11:59 PM from the professor.}
\section{}
\begin{enumerate}
  \setcounter{enumi}{16}
  \item See if you can generalize the result for the first Born approximation:
\end{enumerate}

$$
\frac{d \sigma}{d \Omega^{\prime}}=\frac{m^{2}}{(2 \pi)^{2}}\left|\hat{V}\left(\mathbf{p}^{\prime}-\mathbf{p}\right)\right|^{2}
$$

to the case where the scattered particle (mass $m_{f}$ ) may have a different mass than the incident particle (mass $m_{i}$ ).
\subsection{}
Looking over the note in 7.3 on approximate methods, I found the derivation for the Born approximation. Basically, this scattering rate is given by an expression of the form:
\begin{equation}
  \dd{\sigma }= \frac{1}{\text{flux}}\dd{\Gamma }
\end{equation}
and the idea is that the masses that go into this flux and rate are different. For the flux, we care about the mass of the incident particle $m_{i}$,
and then for the differential rate, there is an expression of the form:
\begin{align}
  \dd{\Gamma }&= \int_{\mathbf{p'}\in \dd{\Omega '}}\frac{L^3}{(2\pi)^3}\dd^3{\mathbf{p'}} 2 \pi \frac{\abs{\hat{V}(\mathbf{p'}-\mathbf{p})}^2}{L^6}\delta (E_{f}-E_{i})\\
&= \frac{\dd{\Omega '}}{L^3}\frac{m_{f}p_f}{(2\pi)^2}\abs{\hat{V}(\mathbf{p'}-\mathbf{p})}^2
\end{align}
The flux is given by:
\begin{equation}
  \text{flux} = \frac{p_i}{m_i L^3}
\end{equation}
And this term divided by the flux gives us the differential cross section in the first Born approximation, which for equal masses (and momenta) is expressed as:
\begin{equation}
  \frac{\dd{\sigma }}{\dd{\Omega '}}= \frac{m^2}{(2\pi)^2}\abs{\hat{V}(\mathbf{p'}-\mathbf{p})}^2
\end{equation}
This time the two masses will be different from each other, so we can write the formula as:
\begin{equation}
  \frac{\dd{\sigma }}{\dd{\Omega '}}_{\text{diffms}}= \frac{m_{f}p_fm_{i}}{p_i(2\pi)^2}\abs{\hat{V}(\mathbf{p'}-\mathbf{p})}^2
\end{equation}
\section{}
\begin{enumerate}
  \setcounter{enumi}{17}
  \item We consider the potential (called the "Yukawa potential"):
\end{enumerate}

$$
V(\mathbf{x})=\frac{K e^{-\mu r}}{r}, \quad r=|\mathbf{x}|
$$

with real parameters $K$ and $\mu>0$. The parameter $K$ can be regarded as the "strength" of the potential ("interaction"), and $\frac{1}{\mu}$ is effectively the "range" of distance over which the potential is important. $\mu$ itself has units of mass - note that as $\mu \rightarrow 0$ we obtain the Coulomb potential: $\mu$ can be thought of as the mass of an "exchanged particle" which mediates the force. In electromagnetism, this is the photon, hence $\mu \rightarrow m_{\gamma}=0$

(a) Find a condition on $K$ and $\mu$ which guarantees that there are at least $n$ bound states in this potential. You will likely fashion and use some kind of "comparison" theorem in arriving at your result. You should give at least a "heuristically convincing" argument, if you don't actually prove it.
\subsection{}
We know that for any bound states at all, we need $K<0$ or else we would just be talking about the free particle. Let us consider the Yukawa potential as a perturbation for our original Coulomb potential where we have $H_0 = \frac{p^2}{2m} + \frac{K}{r}$ and the perturbation $V= K\left(\frac{e^{- \mu r}}{r} - \frac{1}{r}\right)$:
\begin{equation}
  H = H_0 + V
\end{equation}

It is always the case that $K<0$, and the quantity inside of the parentheses will always be negative, so let's consider the inside of the parentheses. We can expand it in a Taylor series:
\begin{equation}
  e^{-\mu r} = 1 - \mu r + \ldots
\end{equation}
So we can write the inside of the parentheses as:
\begin{equation}
  \frac{e^{-\mu r}}{r} - \frac{1}{r} = -\mu + \ldots
\end{equation}
So, we can say the maximum of the perturbation goes approximately like the positive quantity:
\begin{equation}
  V \approx -\mu K
\end{equation}
We know that the unperturbed energies in the Coulomb potential go like:
\begin{equation}
  E_n = \frac{a}{n^2}
\end{equation}
So, if we want at least $n$ bound states, we need to the sum of the perturbation and the Couluomb energy to be less than 0, which can give us a condition on $K$ and $\mu$:
\begin{equation}
  -\mu K - \frac{a}{n^2} < 0
\end{equation}
where $a$ is a positive constant.\\
(b) Using the Born approximation for the differential cross section that we developed in our discussion of time-dependent perturbation theory, calculate the differential cross section, $\frac{d \sigma}{d \Omega}$, for scattering on this potential. Consider the limit $\mu \rightarrow 0$ and compare with the classical Rutherford cross section (Is this calculation valid?).
\subsection{}
We follow from 19.3.8 in Shankar. When the Yukawa potential is given by:
\begin{equation}
  V(r) = \frac{K e^{-\mu r}}{r}
\end{equation}
The scattering amplitude is given by:
\begin{equation}
  f(\theta) = -\frac{2\mu _{reduced}K}{\hbar^2q}\int_{0}^{\infty}\frac{e^{iqr'}-e^{-iqr'}}{2i}e^{-\mu r'}\dd{r'}
\end{equation}
where $\mu _{reduced}$ is the reduced mass of the system.
This simplifies to:
\begin{equation}
  f(\theta) = -\frac{2\mu _{reduced}K}{\hbar^(\mu^2 + q^2)}
\end{equation}
The differential cross section in the Born approximation for the Yukawa potential is given by:
\begin{equation}
  \frac{\dd{\sigma }}{\dd{\Omega }} = \abs{f(\theta)}^2
\end{equation}
Now, we know that the square of the momentum $q$ transferred to the particle is given by:
\begin{equation}
  q^2 = 4k^2\sin^2\left(\frac{\theta}{2}\right)
\end{equation}
So we can write the differential cross section as:
\begin{equation}
  \frac{\dd{\sigma }}{\dd{\Omega }} = \frac{4\mu^2_{reduced}K^2}{\hbar^4(\mu^2 + q^2)^2} = \frac{4\mu^2_{reduced}K^2}{\hbar^4(\mu^2 + 4k^2\sin^2\left(\frac{\theta}{2}\right))^2}
\end{equation}
To compare with the classical Rutherford cross section, we need to consider the limit $\mu \rightarrow 0$ and let us set $K=Ze^2$. We can write the differential cross section as:
\begin{equation}
  \frac{\dd{\sigma }}{\dd{\Omega }} = \frac{4\mu^2_{reduced}(Ze^2)^2}{\hbar^4(4k^2\sin^2\left(\frac{\theta}{2}\right))^2} = \frac{(Ze^2)^2}{16E^2\sin^4\left(\frac{\theta}{2}\right)}
\end{equation}
where we have used the fact that the momentum $p$ is given by:
\begin{equation}
  p = \frac{\hbar k}{2}
\end{equation}
and the energy $E$ is given by:
\begin{equation}
  E = \frac{p^2}{2\mu_{reduced}}
\end{equation}
Although the result is the same as the Rutherford cross section, the calculation is not completely valid because the formulation developed is for a potential that vanishes faster than the inverse of the radius, which the Coulomb does not. For a comprehensive discussion, see Shankar pg. 532.\\
(c) Integrate your differential cross section over all solid angles to obtain the "total cross section". Again, consider the limit $\mu \rightarrow 0$. Assuming the Rutherford cross section holds, what is the total cross section for scattering on a Coulomb potential?
\subsection{}
Here we are interested in the total cross section, which is given by:
\begin{equation}
  \sigma = \int \frac{\dd{\sigma}}{\dd{\Omega}}\dd{\Omega}
\end{equation}
Plugging in our expression for the differential cross section:
\begin{equation}
  \sigma = \int \frac{4\mu^2_{reduced}K^2}{\hbar^4(\mu^2 + 4k^2\sin^2\left(\frac{\theta}{2}\right))^2}\dd{\Omega}
\end{equation}
We can write the differential solid angle as:
\begin{equation}
  \dd{\Omega} = 2\pi \sin\theta \dd{\theta}
\end{equation}
Pulling out all the terms that do not depend on $\theta$ and inserting this identity, we have:
\begin{equation}
  \sigma = \frac{8\pi\mu^2_{reduced}K^2}{\hbar^4}\int_0^{\pi} \frac{1}{(\mu^2 + 4k^2\sin^2\left(\frac{\theta}{2}\right))^2}\sin\theta \dd{\theta}
\end{equation}
At this point, I can remark, that as $\mu \rightarrow 0$, the integrand diverges at $\theta = 0$ and $\theta = \pi$. So, we can't compute thee total cross section for scattering on a Coulomb potential using this method. However, when the Yukawa potential does not merely reduce to the Coulomb potential in this way, the addition of the $\mu^2$ term in the denominator of the integrand will cause the integral to converge.\\
\section{}
\begin{enumerate}
  \setcounter{enumi}{18}
  \item Suppose we have a system consisting of two spin- $\frac{1}{2}$ 's $\left(\mathbf{S}_{1}\right.$ and $\left.\mathbf{S}_{2}\right)$, with an interaction Hamiltonian $a(t) \mathbf{S}_{1} \cdot \mathbf{S}_{2}$. Assume that $a( \pm \infty)=0$ and that $a(t)$ is significantly different from zero in an interval of order $\tau$ in width about $t=0$.
(a) Suppose at very early times $(t \rightarrow-\infty)$, the system is in state
\end{enumerate}

$$
|\psi(-\infty)\rangle=|+-\rangle
$$

where the state is labeled by the $z$ components of $\mathbf{S}_{1}$ and $\mathbf{S}_{2}$, respectively. Without making any approximations, what is the state of the system at $t=\infty$ ? What is the probability, $P(+-\rightarrow-+)$, that the state is observed to be $|-+\rangle$ at $t=\infty$ ? Show that this probability depends only on $\int_{-\infty}^{\infty} a(t) d t$.
\subsection{}
We know that:
\begin{equation}
  (\mathbf{S})^2 = (S_1 + S_2)^2 \rightarrow S^2 = S_1^2 + S_2^2 + 2\mathbf{S}_1 \cdot \mathbf{S}_2
\end{equation}
so we can write the Hamiltonian as:
\begin{equation}
  H = a(t) \mathbf{S}_1 \cdot \mathbf{S}_2 = \frac{1}{2}a(t) \left(S^2 - S_1^2 - S_2^2\right) 
\end{equation}
We want to transform current basis into the basis of the total spin and its $z$-component. Our current basis is:
\begin{equation}
  \ket{+-} = \ket{S_1^z = +\frac{1}{2}, S_2^z = -\frac{1}{2}}
\end{equation}
and:
\begin{equation}
  \ket{-+} = \ket{S_1^z = -\frac{1}{2}, S_2^z = +\frac{1}{2}}
\end{equation}
The total $S_z$ of 0 must be conserved, so we have the singlet:
\begin{equation}
  \ket{S^2 = 0, S^z = 0} = \frac{1}{\sqrt{2}}\left(\ket{+-} - \ket{-+}\right)
\end{equation}
and one of the triplets:
\begin{equation}
  \ket{S^2 = 1, S^z = 0} = \frac{1}{\sqrt{2}}\left(\ket{+-} + \ket{-+}\right)
\end{equation}
We can express our initial state in terms of these states:
\begin{equation}
  \ket{+-} = \frac{1}{\sqrt{2}}\left(\ket{S^2 = 1, S^z = 0} + \ket{S^2 = 0, S^z = 0}\right)
\end{equation}
and
\begin{equation}
  \ket{-+} = \frac{1}{\sqrt{2}}\left(\ket{S^2 = 1, S^z = 0} - \ket{S^2 = 0, S^z = 0}\right)
\end{equation}
We seek to figure out the eigenvalues of the Hamiltonian in the basis of the total spin and its $z$-component. We can write the Hamiltonian as:
\begin{equation}
  H = a(t) \mathbf{S}_1 \cdot \mathbf{S}_2 = \frac{1}{2}a(t) \left(S^2 - S_1^2 - S_2^2\right)
\end{equation}
So, it operates on the states as:
\begin{equation}
  H\ket{S^2 = 0, S^z = 0} = \frac{1}{2}a(t)\left(0 - \frac{3}{4} - \frac{3}{4}\right)\ket{S^2 = 0, S^z = 0} = -\frac{3}{4}a(t)\ket{S^2 = 0, S^z = 0}
\end{equation}
and:
\begin{equation}
  H\ket{S^2 = 1, S^z = 0} = \frac{1}{2}a(t)\left(2 - \frac{3}{4} - \frac{3}{4}\right)\ket{S^2 = 1, S^z = 0} = \frac{1}{4}a(t)\ket{S^2 = 1, S^z = 0}
\end{equation}
So, the action on the initial state is:
\begin{equation}
  H\ket{+-} = \frac{1}{\sqrt{2}}\left(\frac{1}{4}a(t)|S^2 = 1, S^z = 0\rangle - \frac{3}{4}a(t)|S^2 = 0, S^z = 0\rangle\right)
\end{equation}
For the state $\ket{-+}$, we have:
\begin{equation}
  H\ket{-+} = \frac{1}{\sqrt{2}}\left(\frac{1}{4}a(t)|S^2 = 1, S^z = 0\rangle + \frac{3}{4}a(t)|S^2 = 0, S^z = 0\rangle\right)
\end{equation}
With the time-dependent Schrödinger equation, we can write:
\begin{equation}
  i \pdv{t} \ket{\psi (t)} = H\ket{\psi (t)} = a(t) \mathbf{S}_1 \cdot \mathbf{S}_2 \ket{\psi (t)}
\end{equation}
We can express $\ket{\psi (t)}$ as a two-component vector in the basis of the total spin and its $z$-component:
\begin{equation}
  \ket{\psi (t)} = \begin{pmatrix} c_1(t) \\ c_2(t) \end{pmatrix}
\end{equation}
where $c_1(t)$ and $c_2(t)$ are the probability amplitudes for the states $\ket{S^2 = 0, S^z = 0}$ and $\ket{S^2 = 1, S^z = 0}$, respectively. Then, we have the time-dependent Schrödinger equation with only one component:
\begin{equation}
  i \partial_{t} c_1(t) = -\frac{3}{4}a(t)c_1(t)
\end{equation}
Dividing through by $c_1(t)$ and multiplying by $-i$ gives us:
\begin{equation}
  \frac{\partial_{t} c_1(t)}{c_1(t)} = +i\frac{3}{4} a(t)
\end{equation}
We recognize the left-hand side as the derivative of the natural logarithm of $c_1(t)$, so we can write:
\begin{equation}
  \ln(c_1(t)) = \frac{3i}{4}\int_{-\infty}^{t} a(t')\dd{t'}
\end{equation}
We can exponentiate both sides to solve for $c_1(t)$:
\begin{equation}
  c_1(t) = e^{\frac{3i}{4}\int_{-\infty}^{t} a(t')\dd{t'}}
\end{equation}
We can do the same for $c_2(t)$:
\begin{equation}
  c_2(t) = e^{-\frac{i}{4}\int_{-\infty}^{t} a(t')\dd{t'}}
\end{equation}
Now, we that we have the propagors for the eigenstates of the Hamiltonian, we can write the state at $t=\infty$ as:
\begin{equation}
  \ket{\psi (\infty)} = \frac{1}{\sqrt{2}}\begin{pmatrix} e^{\frac{3i}{4}\int_{-\infty}^{\infty} a(t')\dd{t'}} \\ e^{-\frac{i}{4}\int_{-\infty}^{\infty} a(t')\dd{t'}} \end{pmatrix}
\end{equation}
We can write the probability of the state $\ket{+-}$ transitioning to the state $\ket{-+}$ as the projection of the state $\ket{-+}$ onto the state $\ket{\psi (\infty)}$:
\begin{equation}
  P(+-\rightarrow-+) = \abs{\braket{-+}{\psi (\infty)}}^2
\end{equation}
In the basis of the total spin and its $z$-component, we can write the state $\ket{-+}$ as:
\begin{equation}
  \ket{-+} = \frac{1}{\sqrt{2}}\left( 
  \begin{pmatrix} 0 \\ 1 \end{pmatrix} - \begin{pmatrix} 1 \\ 0 \end{pmatrix}
  \right) = \frac{1}{\sqrt{2}}\begin{pmatrix} -1 \\ 1 \end{pmatrix}
\end{equation}
So we can write the probability as:
\begin{equation}
  P(+-\rightarrow-+) = \abs{\frac{1}{\sqrt{2}}\begin{pmatrix} -1 & 1 \end{pmatrix}\cdot \frac{1}{\sqrt{2}}\begin{pmatrix} e^{\frac{3i}{4}\int_{-\infty}^{\infty} a(t')\dd{t'}} \\ e^{-\frac{i}{4}\int_{-\infty}^{\infty} a(t')\dd{t'}} \end{pmatrix}}^2
\end{equation}
which simplifies to:
\begin{equation}
  P(+-\rightarrow-+) = \abs{\frac{1}{2}\left(-e^{\frac{3i}{4}\int_{-\infty}^{\infty} a(t')\dd{t'}} + e^{-\frac{i}{4}\int_{-\infty}^{\infty} a(t')\dd{t'}}\right)}^2
\end{equation}

(b) Now calculate $P(+-\rightarrow-+)$ in first-order time-dependent perturbation theory. What is a condition for the validity of first-order time-dependent perturbation theory?
\subsection{}
For the total Hamiltonian, we can choose:
\begin{equation}
  H = H_0 + V_t = 0 + a(t) \mathbf{S}_1 \cdot \mathbf{S}_2
\end{equation}
Ordinarily, in first-order time-dependent perturbation theory, we would write the transition probability as:
\begin{equation}
  P(+-\rightarrow-+) = \abs{-i\int_{-\infty}^{\infty} e^{i\omega_{+-}t}\bra{f}V_t\ket{i}\dd{t}}^2
\end{equation}
But here the unperturbed Hamiltonian is 0,so the exponential factor is just 1, and the transition probability is just:
\begin{equation}
  P(+-\rightarrow-+) = \abs{-i\int_{-\infty}^{\infty} \bra{f}V_t\ket{i}\dd{t}}^2
\end{equation}
The Hamiltonian acting on the initial state is $\ket{\Psi (\infty)}$ and the final state is $\ket{f} = \ket{-+}$.
In general, $\ket{\Psi (t)}$ is given by:
\begin{equation}
  \ket{\Psi (t)} = \ket{\Psi (0)} - i\int_{-\infty}^{\infty} V_t\ket{\Psi (t)}\dd{t}
\end{equation}
and the overlap of the final state with the state at $t=\infty$ is given by:
\begin{equation}
  \braket{f}{\Psi (\infty)} = \braket{f}{\Psi (0)} - i\int_{-\infty}^{\infty} \braket{f}{V_t}{\Psi (t)}\dd{t}
\end{equation}
the first term on the right-hand side is 0, so this is just:
\begin{equation}
  \braket{f}{\Psi (\infty)} = -i\int_{-\infty}^{\infty} \bra{f}V_t\ket{\Psi (t)}\dd{t}
\end{equation}
 We can write the transition probability as:
\begin{equation}
  P(+-\rightarrow-+) = \abs{-i\int_{-\infty}^{\infty} \bra{-+}a(t) \mathbf{S}_1 \cdot \mathbf{S}_2\ket{+-}\dd{t}}^2
\end{equation}
We found the action of the perturbation on the state $\ket{+-}$ in the previous part 
So we can write the transition probability as:
\begin{equation}
  P(+-\rightarrow-+) = \abs{\frac{1}{i\hbar}\int_{-\infty}^{\infty} \frac{1}{\sqrt{2}}\left(\frac{1}{4}a(t)\bra{-+}|S^2 = 1, S^z = 0\rangle - \frac{3}{4}a(t)\bra{-+}|S^2 = 0, S^z = 0\rangle\right)\dd{t}}^2
\end{equation}
Now, $\bra{-+}$ is given by:
\begin{equation}
  \bra{-+} = \frac{1}{\sqrt{2}}\left(\bra{S^2 = 1, S^z = 0} - \bra{S^2 = 0, S^z = 0}\right)
\end{equation}
So we can write the transition probability as:
\begin{equation}
  P(+-\rightarrow-+) = \abs{\frac{1}{2i\hbar}\int_{-\infty}^{\infty} \left(\frac{1}{4}a(t) + \frac{3}{4}a(t)\right)\dd{t}}^2 = \abs{\frac{1}{2i\hbar}\int_{-\infty}^{\infty} a(t)\dd{t}}^2
\end{equation}
The validity of my first order perturbation theory calculation depends on whether my perturbation was chosen to be small and, in this case, whether my transition probability $g(a)$ from my PT calculation was similar to the transition probability $f(a)$ from the exact solution, which wasn't completely the case for me.\\

\begin{enumerate}
  \setcounter{enumi}{19}
  \item With reference to the previous problem, consider the "Zeeman effect", with the addition of a uniform, static magnetic field of strength $B_{0}$ in the direction of the $z$-axis. Thus, we have another term in the Hamiltonian:
\end{enumerate}

$$
H_{0}=-B_{0}\left(g_{1} S_{1 z}+g_{2} S_{2 z}\right),
$$

where $g_{1}$ and $g_{2}$ are the gyromagnetic ratios of the two spins. We assume a gaussian form for $a(t)$ :

$$
a(t)=a(0) e^{-(t / \tau)^{2}}
$$

Repeat your first order perturbation theory calculation of the previous problem in the presence of the magnetic field and with this explicit form for $a(t)$, and determine $P(+-\rightarrow-+)$. Discuss the variation of this probability, for given $a(0)$ and $\tau$, as a function of $B_{0}$.
\subsection{}
Now, the unperturbed Hamiltonian gives non-zero energies for the states $\ket{+-}$ and $\ket{-+}$:
\begin{equation}
  H_0\ket{+-} = -B_0\left(g_1\frac{1}{2} + g_2\left(-\frac{1}{2}\right)\right)\ket{+-} = -\frac{1}{2}B_0(g_1 - g_2)\ket{+-}
\end{equation}
and:
\begin{equation}
  H_0\ket{-+} = -B_0\left(g_1\left(-\frac{1}{2}\right) + g_2\frac{1}{2}\right)\ket{-+} = -\frac{1}{2}B_0(g_2 - g_1)\ket{-+}
\end{equation}
So, now the expression for $e^{i\left(E_f-E_i\right)t}$ is not just 1, but:
\begin{equation}
  e^{i\left(E_f-E_i\right)t} = e^{\frac{-iB_0t}{2}\left((g_2 - g_1) - (g_1 - g_2)\right)} = e^{iB_0t(g_2 - g_1)}
\end{equation}
That is, the transition probability is now:
\begin{equation}
  P(+-\rightarrow-+) = \abs{-i\int_{-\infty}^{\infty} e^{i\omega_{+-}t}\bra{f}V_t\ket{i}\dd{t}}^2
\end{equation}
where the exponential factor is now:
\begin{equation}
  e^{i\omega_{+-}t} = e^{iB_0t(g_2 - g_1)}
\end{equation}
The perturbative piece will be the same as before, so we can write the transition probability as:
\begin{equation}
  P(+-\rightarrow-+) = \abs{-i\int_{-\infty}^{\infty} e^{iB_0t(g_2 - g_1)}\bra{-+}a(t) \mathbf{S}_1 \cdot \mathbf{S}_2\ket{+-}\dd{t}}^2 = \abs{\frac{1}{2i\hbar}\int_{-\infty}^{\infty} a(t)e^{iB_0t(g_2 - g_1)}\dd{t}}^2
\end{equation}
But we know that $a(t)$ is given by:
\begin{equation}
  a(t) = a(0)e^{-(t/\tau)^2}
\end{equation}
So we can write the transition probability as:
\begin{equation}
  P(+-\rightarrow-+) = \abs{\frac{a(0)}{2i\hbar}\int_{-\infty}^{\infty} e^{-(t/\tau)^2}e^{iB_0t(g_2 - g_1)}\dd{t}}^2
\end{equation}
Adding the exponents, we have:
\begin{equation}
  e^{-(t/\tau)^2}e^{iB_0t(g_2 - g_1)} = e^{-(t/\tau)^2 + iB_0t(g_2 - g_1)} = \exp\left(\left(-\frac{1}{\tau^2}\left(t- \frac{i\tau^2 B_0(g_2 - g_1)}{2}\right)\right)^2 - \frac{B_0^2 \tau^2 (g_2 - g_1)^2}{4}\right)
\end{equation}
Bringing the latter term out of the integral, we have:
\begin{equation}
  P(+-\rightarrow-+) = \abs{\frac{a(0)}{2i\hbar}e^{-\frac{B_0^2 \tau^2 (g_2 - g_1)^2}{4}}\int_{-\infty}^{\infty} \exp\left(\left(-\frac{1}{\tau^2}\left(t- \frac{i\tau^2 B_0(g_2 - g_1)}{2}\right)\right)^2\right)\dd{t}}^2
\end{equation}
We can recognize the integral as the Gaussian integral, with:
\begin{equation}
  \int_{-\infty}^{\infty} \exp\left(-a(t-b)^2\right)\dd{t} = \sqrt{\frac{\pi}{a}}
\end{equation}
where here we have $a = \frac{1}{\tau^2}$. So, the integral above evaluates to:
\begin{equation}
  \sqrt{\frac{\pi}{\frac{1}{\tau^2}}} = \sqrt{\pi}\tau
\end{equation}
So we can write the transition probability as:
\begin{equation}
  P(+-\rightarrow-+) = \abs{\frac{a(0)}{2i\hbar}e^{-\frac{B_0^2 \tau^2 (g_2 - g_1)^2}{4}}\sqrt{\pi}\tau}^2
\end{equation}
For a given $a(0)$ and $\tau$, the transition probability will decrease exponentially with $B_0$.\\
\end{document}