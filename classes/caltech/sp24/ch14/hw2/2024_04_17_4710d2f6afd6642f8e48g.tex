\documentclass[12pt]{article}
\usepackage[utf8]{inputenc}
\usepackage[T1]{fontenc}
\usepackage{amsmath}
\usepackage{amsfonts}
\usepackage{amssymb}
\usepackage[version=4]{mhchem}
\usepackage{stmaryrd}

\usepackage{listings} % Required for insertion of code
\usepackage{xcolor} % Required for custom colors

% Define custom colors
\definecolor{codegreen}{rgb}{0,0.6,0}
\definecolor{codegray}{rgb}{0.5,0.5,0.5}
\definecolor{codepurple}{rgb}{0.58,0,0.82}
\definecolor{backcolour}{rgb}{0.95,0.95,0.92}

% Setup the style for code listings
\lstdefinestyle{mystyle}{
    backgroundcolor=\color{backcolour},   
    commentstyle=\color{codegreen},
    keywordstyle=\color{magenta},
    numberstyle=\tiny\color{codegray},
    stringstyle=\color{codepurple},
    basicstyle=\ttfamily\footnotesize,
    breakatwhitespace=false,         
    breaklines=true,                 
    captionpos=b,                    
    keepspaces=true,                 
    numbers=left,                    
    numbersep=5pt,                  
    showspaces=false,                
    showstringspaces=false,
    showtabs=false,                  
    tabsize=2
}

% Activate the style
\lstset{style=mystyle}

\title{Ch14 Spring term 2024 }

\author{}
\date{}


\begin{document}
\maketitle
Problem set 2

due April 18, 2024

please indicate length of time needed for this problem set

unless otherwise stated, assume $\mathrm{T}=25^{\circ} \mathrm{C}$, that activities = concentrations and $\mathrm{Kw}=10^{-14}$. unless otherwise specified, give answers to 3 significant figures.
\section{}
Calculate the $\mathrm{pH}$ to the nearest $0.01 \mathrm{pH}$ unit for
Assume $\mathrm{HCl}$ is fully dissociated, and $\mathrm{Kw}=10^{-14}$.\\\\
If we assume that $\mathrm{HCl}$ is fully dissociated, then the concentration of $\mathrm{H}^{+}$ is equal to the concentration of $\mathrm{HCl}$, and the $\mathrm{pH}$ is given by $\mathrm{pH}=-\log _{10}\left(\left[\mathrm{H}^{+}\right]\right)$.
\subsection{}
a $10^{-2} \mathrm{M}$ solution of $\mathrm{HCl}$
\subsubsection{Answer}
This translates to a concentration of $\mathrm{H}^{+}$ of $10^{-2} \mathrm{M}$. Next, we can calculate the $\mathrm{pH}$ as follows:
\begin{equation}
  \mathrm{pH}=-\log _{10}\left(10^{-2}\right)=2
\end{equation}
\subsection{}
a $10^{-10} \mathrm{M}$ solution of $\mathrm{HCl}$
\subsubsection{Answer}
This translates to a concentration of $\mathrm{H}^{+}$ of $10^{-10} \mathrm{M}$. Next, we can calculate the $\mathrm{pH}$ as follows:
\begin{equation}
  \mathrm{pH}=-\log _{10}\left(10^{-10}\right)=10
\end{equation}
\section{}


for problems 2-4, assume that the concentrations are such that the "weak acid approximation" $\left(H^{+}\right) \sim \sqrt{K_{a} C_{0}}$ (or the corresponding "weak base approximation") is valid.
\subsection{}
Calculate the $\mathrm{pH}$ to the nearest $0.01 \mathrm{pH}$ unit for the following solutions. Use pKa $=4.75$ for acetic acid, $\mathrm{pK}_{\mathrm{b}}=4.75$ for ammonia, and $\mathrm{Kw}=10^{-14}$.
\subsubsection{}
$10^{-2} \mathrm{M}$ solution of acetic acid
\subsubsection{Answer}
Because the pKa of acetic acid is $4.75$, its $K_{a}$ is $10^{-4.75}$. The concentration of acetic acid is $10^{-2} \mathrm{M}$, so the concentration of $\mathrm{H}^{+}$ is given by the equation
\begin{equation}
  \left[\mathrm{H}^{+}\right]=\sqrt{K_{a} C_{0}}=\sqrt{10^{-4.75} \times 10^{-2}}=10^{-3.375}
\end{equation}
Next, we can calculate the $\mathrm{pH}$ as follows:
\begin{equation}
  \mathrm{pH}=-\log _{10}\left(10^{-3.375}\right)=3.375
\end{equation}
and with the correct number of significant figures, the $\mathrm{pH}$ is $3.38$.
\subsubsection{}
$10^{-2} \mathrm{M}$ solution of $\mathrm{NH}_{3}$
\subsubsection{Answer}
Because the $\mathrm{pK}_{\mathrm{b}}$ of ammonia is $4.75$, its $K_{b}$ is $10^{-4.75}$. The concentration of ammonia is $10^{-2} \mathrm{M}$, so the concentration of $\mathrm{OH}^{-}$ is given by the equation
\begin{equation}
  \left[\mathrm{OH}^{-}\right]=\sqrt{K_{b} C_{0}}=\sqrt{10^{-4.75} \times 10^{-2}}=10^{-3.375}
\end{equation}
Next, we can calculate the $\mathrm{pOH}$ as follows:
\begin{equation}
  \mathrm{pOH}=-\log _{10}\left(10^{-3.375}\right)=3.375
\end{equation}
and with the correct number of significant figures, the $\mathrm{pOH}$ is $3.38$. Finally, we can calculate the $\mathrm{pH}$ as follows:
\begin{equation}
  \mathrm{pH}=14-\mathrm{pOH}=14-3.38=10.62
\end{equation}
\section{}
In a $0.1 \mathrm{M}$ solution of $\mathrm{NH}_{4} \mathrm{Cl}$ in water, calculate the concentrations of $\mathrm{NH}_{4}{ }^{+}$and $\mathrm{NH}_{3}$. The $\mathrm{pK}_{\mathrm{a}}$ of $\mathrm{NH}_{4}{ }^{+}$is 9.25 .
\subsection{Answer}
The concentration of $\mathrm{NH}_{4}{ }^{+}$ is trivially $0.1 \mathrm{M}$. As for the concentration of $\mathrm{NH}_{3}$, we can use equations to calculate $K_b$:
\begin{equation}
  K_{b}=\frac{K_{w}}{K_{a}}
\end{equation}
Then, we want to calculate the concentration of $\mathrm{OH}^{-}$:
\begin{equation}
  \left[\mathrm{OH}^{-}\right]=\sqrt{K_{b} C_{0}}
\end{equation}
Finally, we can calculate the concentration of $\mathrm{NH}_{3}$ using the equilibrium equation:
\begin{equation}
  \left[\mathrm{NH}_{3}\right]=\frac{\left[\mathrm{NH}_{4}{ }^{+}\right]\left[\mathrm{OH}^{-}\right]}{K_{b}}
\end{equation}
We end up with
\begin{equation}
  \left[\mathrm{NH}_{3}\right]=7.50 \mathrm{M} \times 10^{-3}
\end{equation}
% Inline Python code in the document
\begin{lstlisting}[language=Python]
# Constants for calculations
Ka = 10**(-9.25)
Kw = 10**(-14)
C0_NH4Cl = 0.1  # Molarity of NH4Cl, thus [NH4+]

# Calculate Kb for NH3
Kb = Kw / Ka

# Calculate [OH-] using the approximation for weak bases
OH_minus = (Kb * C0_NH4Cl)**0.5
H_plus = Kw / OH_minus

# Calculate [NH3] using the relation with OH- and NH4+
NH3 = (OH_minus * C0_NH4Cl) / Kb

NH3
\end{lstlisting}
\section{}
The $\mathrm{pH}$ of a $0.10 \mathrm{M}$ solution of a certain amine, $\mathrm{R}-\mathrm{NH}_{2}$, is 11.80. What is the $\mathrm{pK}_\mathrm{b}$ of this amine?
\subsection{Answer}
From the pH, we calculate the concentration of $\mathrm{H}^{+}$ as follows:
\begin{equation}
  \left[\mathrm{H}^{+}\right] = 10^{-\mathrm{pH}}
\end{equation}
Using this, the concentration of $\mathrm{OH}^{-}$ is determined by the water ion-product, $K_{w}$:
\begin{equation}
  \left[\mathrm{OH}^{-}\right] = \frac{K_{w}}{\left[\mathrm{H}^{+}\right]}
\end{equation}
Assuming that all the amine is initially present as $\mathrm{R}-\mathrm{NH}_{2}$ and the solution is 0.10 M, the concentration of $\mathrm{R}-\mathrm{NH}_{3}^{+}$ formed by the reaction of $\mathrm{R}-\mathrm{NH}_{2}$ with water is equal to $\left[\mathrm{OH}^{-}\right]$. The equilibrium constant for the base, $K_{b}$, is then:
\begin{equation}
  K_{b} = \frac{\left[\mathrm{OH}^{-}\right]^{2}}{0.10}
\end{equation}
Finally, the $\mathrm{pK}_\mathrm{b}$ is calculated using:
\begin{equation}
  \mathrm{pK}_\mathrm{b} = -\log_{10}(K_{b})
\end{equation}
This gives us a $\mathrm{pK}_\mathrm{b}$ of 3.40.
% Inline Python code in the document
\begin{lstlisting}[language=Python]
import math
# Constants
pH = 11.80
C0_amine = 0.10  # Molarity of the amine
Kw = 10**-14

# Calculate [H+]
H_plus = 10**(-pH)

# Calculate [OH-]
OH_minus = Kw / H_plus

# Assuming [OH-] equals the amount of amine that reacted and [R-NH2] is approximately equal to the initial concentration
Kb = (OH_minus**2) / C0_amine

# Calculate pKb
pKb = -math.log10(Kb)

pKb

\end{lstlisting}

\section{}
Tris(hydroxymethyl)aminomethane (Tris) is a frequently used buffer component for biochemical studies. The structure of Tris in the basic form will be abbreviated in this problem as $\mathrm{R}-\mathrm{NH}_{2}$. The acid dissociation equilibrium may be represented as
$$
\mathrm{R}-\mathrm{NH}_{3}^{+} \leftrightarrow \mathrm{R}-\mathrm{NH}_{2}+\mathrm{H}^{+} \quad \mathrm{pKa}=8.10 ; \Delta \mathrm{H}^{\circ}=+50 \mathrm{~kJ} / \mathrm{mole}
$$

where $\mathrm{pKa}=-\log _{10} \mathrm{~K}_{\mathrm{a}}$, and $\mathrm{K}_{\mathrm{a}}$ is the equilibrium constant for this acid dissociation reaction. $\mathrm{K}_{\mathrm{a}}$ is a function of temperature since $\Delta \mathrm{H}^{\circ}$ is non-zero.

Although Tris is widely used, it has the serious drawback that the $\mathrm{pH}$ of a buffered solution varies significantly with temperature. Based on the van't Hoff equation that we've discussed in class, numerically evaluate $\partial p K_{a} / \partial T$, the change in pKa with a change in temperature, for Tris buffer at $\mathrm{T}=25^{\circ} \mathrm{C}$.

If this derivative is assumed to be constant, independent of temperature (this is not actually true, but is a reasonable approximation), calculate the expected pKa of Tris buffer at $0{ }^{\circ} \mathrm{C}$.
\subsection{Answer}
The van't Hoff equation in our case is given by:
\begin{equation}
  \left( \frac{\partial \ln K_{a}}{\partial T} \right) _{p} = \frac{\Delta H^{\circ}}{RT^{2}}
\end{equation}
The relationship between $K_{a}$ and $\mathrm{pKa}$ is given by:
\begin{equation}
  pKa = -\log_{10}(K_{a})
\end{equation}
Differentiating this equation with respect to $T$ gives:
\begin{equation}
  \frac{\partial pKa}{\partial T} = - \frac{1}{\ln(10)} \frac{\partial \ln K_{a}}{\partial T}
\end{equation}
Substituting the van't Hoff equation into this gives:
\begin{equation}
  \frac{\partial pKa}{\partial T} = - \frac{\Delta H^{\circ}}{RT^{2} \ln(10)}
\end{equation}
Evaluating this at $T=25^{\circ} \mathrm{C}$ gives:
\begin{equation}
  \frac{\partial pKa}{\partial T} = - \frac{50000}{(8.314 \times 298.15^{2}) \ln(10)} = -0.0294 \frac{1}{K}
\end{equation}
Finally, we can calculate the expected $\mathrm{pKa}$ at $0^{\circ} \mathrm{C}$ as follows:
\begin{equation}
  \mathrm{pKa}_{0^{\circ} \mathrm{C}} = \mathrm{pKa}_{25^{\circ} \mathrm{C}} + \left( \frac{\partial pKa}{\partial T} \right) \times (0 - 25)
\end{equation}
This one gives us a $\mathrm{pKa}$ of $8.83$ at $0^{\circ} \mathrm{C}$.
% Inline Python code in the document
% Inline Python code in the document
\begin{lstlisting}[language=Python]
import numpy as np

# Constants
R = 8.314  # J/(molxK)
Delta_H = 50000  # J/mol
T = 298.15  # K
pKa_25 = 8.10

# Calculate the partial derivative of pKa with respect to temperature
partial_pKa_partial_T = -1 / np.log(10) * (Delta_H / (R * T**2))

# Calculate the pKa at 0 degrees Celsius (273.15 K)
Delta_T = 273.15 - 298.15
pKa_0 = pKa_25 + partial_pKa_partial_T * Delta_T

partial_pKa_partial_T, pKa_0
\end{lstlisting}

\section{}
A solution with the following composition is prepared

0.15 liter of $1.0 \mathrm{M}$ ammonium hydroxide $\left(\mathrm{NH}_{4}{ }^{+} \mathrm{OH}^{-}\right)$

0.10 liter of $0.5 \mathrm{M}$ ammonium chloride $\left(\mathrm{NH}_{4}^{+} \mathrm{Cl}^{-}\right)$

0.10 liter of $1.0 \mathrm{M}$ acetic acid $\left(\mathrm{CH}_{3} \mathrm{CO}_{2} \mathrm{H}\right)$

distilled $\mathrm{H}_{2} \mathrm{O}$ is added to a final volume of 1.0 liter.

Using either graphical (such as Excel ${ }^{\circledR}$ ) or numerical (such as Mathematica ${ }^{\circledR}$ ) methods, calculate the $\mathrm{pH}$ of this solution without any approximations. You may wish to start from the charge balance equation, incorporating mass balance and equilibrium relationships to get an equation that may be solved for $\left(\mathrm{H}^{+}\right)$.

The pKa's of acetic acid and ammonium are $\mathrm{pK}_{1}=4.75$ and $\mathrm{pK}_{2}=9.25$, respectively. Hint: the final $\mathrm{pH}$ is $\sim 9$.
\subsection{Answer}
We want to start by setting up mass balance equations. First we are interested in the total concentration of $\mathrm{NH}_{4}^{+}$ in the solution. There are two sources for this; the ammonium hydroxide and the ammonium chloride. The total concentration of $\mathrm{NH}_{4}^{+}$ is given by:
\begin{equation}
  \left[\mathrm{NH}_{4}^{+}\right] = \left[\mathrm{NH}_{}^{+}\right]_{\text {hydroxide }} + \left[\mathrm{NH}_{4}^{+}\right]_{\text {chloride }}
\end{equation}
and then, we are interested in the concentration of $\mathrm{CH}_{3} \mathrm{CO}_{2} \mathrm{H}$ in the solution. The total concentration of $\mathrm{CH}_{3} \mathrm{CO}_{2} \mathrm{H}$ is given by:
\begin{equation}
  \left[\mathrm{CH}_{3} \mathrm{CO}_{2} \mathrm{H}\right] = \left[\mathrm{CH}_{3} \mathrm{CO}_{2} \mathrm{H}\right]
\end{equation}
% Inline Python code in the document
\begin{lstlisting}[language=Python]
# Initial concentrations
V_total = 1.0  # Total volume in liters
C_NH4OH = 0.15 * 1.0 / V_total  # Concentration of NH4+ from NH4OH
C_NH4Cl = 0.10 * 0.5 / V_total  # Concentration of NH4+ from NH4Cl
C_CH3COOH = 0.10 * 1.0 / V_total  # Initial concentration of CH3COOH

# Total NH4+ concentration
C_NH4_plus_total = C_NH4OH + C_NH4Cl
\end{lstlisting}
We also know the constance of
% Inline Python code in the document
\begin{lstlisting}[language=Python]
# Constants
Kw = 10**-14  # Water ionization constant
Ka_acetic = 10**-4.75  # Acetic acid dissociation constant
Ka_ammonium = 10**-9.25  # Ammonium hydrolysis constant
\end{lstlisting}
Next, we can set up the charge balance equation. The total charge in the solution is given by:
\begin{equation}
  \left[\mathrm{H}^{+}\right] + \left[\mathrm{NH}_{4}^{+}\right] = \left[\mathrm{OH}^{-}\right] + \left[\mathrm{CH}_{3} \mathrm{CO}_{2}^{-}\right] + \left[\mathrm{Cl}^{-}\right] 
\end{equation}
% Inline Python code in the document
\begin{lstlisting}[language=Python]
eq1 = H_plus + NH4_plus - OH_minus - CH3COO_minus - C_Cl_minus  # Charge balance
\end{lstlisting}
Then we want to use the mass balance for ammonium
% Inline Python code in the document
\begin{lstlisting}[language=Python]
eq2 = NH4_plus + NH3 - C_NH4_plus_total  # Total NH4 balance
\end{lstlisting}
and the mass balance for acetic acid
% Inline Python code in the document
\begin{lstlisting}[language=Python]
eq3 = CH3COOH + CH3COO_minus - C_CH3COOH  # Total acetic acid balance
\end{lstlisting}
Then, we want to consider the auto-ionization of water
% Inline Python code in the document
\begin{lstlisting}[language=Python]
eq4 = H_plus * OH_minus - Kw  # Water ionization
\end{lstlisting}
the equilibrium for the ammonia
% Inline Python code in the document
\begin{lstlisting}[language=Python]
eq5 = NH3 * H_plus - NH4_plus * Ka_ammonium  # Ammonium hydrolysis
\end{lstlisting}
and the equilibrium for acetic acid
% Inline Python code in the document
\begin{lstlisting}[language=Python]
eq6 = CH3COO_minus * H_plus - CH3COOH * Ka_acetic  # Acetic acid dissociation
\end{lstlisting}
Finally, we can solve the system of equations given our knowledge of initial values
% Inline Python code in the document
\begin{lstlisting}[language=Python]
# Initial guesses for [H+], [NH4+], [NH3], [OH-], [CH3COOH], [CH3COO-]
initial_guesses = [1e-7, C_NH4_plus_total, 0.01, 1e-7, C_CH3COOH, 0.01]
solution = fsolve(equations, initial_guesses)
\end{lstlisting}
Then we use the following to get the pH
% Inline Python code in the document
\begin{lstlisting}[language=Python]
pH = -np.log10(solution[0])
\end{lstlisting}
This gives a final value of $\mathrm{pH} = 8.77$.

\end{document}