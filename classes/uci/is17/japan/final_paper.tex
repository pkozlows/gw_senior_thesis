\documentclass[12pt,letterpaper]{article}
\usepackage[american]{babel}
\usepackage{csquotes}
\usepackage[style=mla,backend=biber]{biblatex}
\usepackage[letterpaper]{geometry}
\usepackage{times}
\geometry{top=1.0in, bottom=1.0in, left=1.0in, right=1.0in}
\usepackage{lipsum}  

%Doublespacing
\usepackage{setspace}
\doublespacing

%Rotating tables (e.g. sideways when too long)
\usepackage{rotating}

%Fancy-header package to modify header/page numbering (Put your surname in the surname field)
\usepackage{fancyhdr}
\pagestyle{fancy}
\lhead{} 
\chead{} 
\rhead{Kozlowski \thepage}
\lfoot{} 
\cfoot{} 
\rfoot{} 
\renewcommand{\headrulewidth}{0pt} 
\renewcommand{\footrulewidth}{0pt} 
%To make sure we actually have header 0.5in away from top edge
%12pt is one-sixth of an inch. Subtract this from 0.5in to get headsep value
\setlength\headsep{0.333in}

%You put your name title and professor name in here 
\renewcommand{\maketitle}{\makemlaheader}

%You put your name title and professor name in here 
\renewcommand{\maketitle}{\makemlaheader}
\newcommand{\C}{\autocite}

\newcommand{\makemlaheader}{
Patryk Kozlowski \\
Eve Darian-Smith\\
IS 17\\
\begin{center}\textnormal{The Fukushima disaster and why nuclear power is the future}\end{center}

% for some reason, this blank line is necessary
}

\addbibresource{citations.bib} % you would use your own bib file here
\begin{document}
\begin{flushleft}

%%%%First page name, class, etc
%%%%Changes paragraph indentation to 0.5in
\setlength{\parindent}{0.5in}

\maketitle
There is scientific consensus regarding the fact of anthropogenic climate change due to fossil fuel consumption. According to a report by the United Nations International Panel on Climate Change, there is growing evidence of climate action, but it is not enough to reduce global heating to the target of 1.5C. \autocite{noauthor_latest_nodate} To try and reach this decarbonization goal, there has been a renewed attention on nuclear fission as an energy source, which has a long history of accidents, dating back to the disaster that occurred in Chernobyl in the 1980s to the meltdown of the Fukushima nuclear plant in 2011. This paper will start by examining the Fukushima wastewater crisis and then will explore the status of nuclear fission on the global stage, particularly in Germany, which was undergoing a phase-out of nuclear fission before the Russian invasion of Ukraine. Now, natural gas formerly supplied by Russia, which was seen as a transitional fossil fuel that is cleaner, is no longer a viable source of energy, due to geopolitical reasons. It will be argued that if there is a desire to reach the United Nations decarbonization goals, humanity must accept nuclear fission as an essential part of the energy mix, while other renewable energy sources become cheaper. The current issue regarding the controlled removal of radioactive wastewater into the Japanese ocean is not just a regional one, but it is important globally too, as being extremely nitpicky about how this wastewater is removed, merely reinforces an attitude which is set against nuclear fission, which is dangerous as humanity grapples with fossil fuel-driven climate change.

First, I should address the criticism to Japan's plan to release the Fukushima wastewater into the ocean. On the colourful end of the spectrum, neighboring geopolitical rival China is accusing Japan of using their common ocean as its private sewer dump. However, on more reasonable parts of the spectrum, human rights and environmental activists argue that not enough research has been done to justify dumping the wastewater into the ocean yet. \autocite{noauthor_fukushima_2023} It is true that water is treated like a commodity that is taken for granted and when it is tainted, this has proven to have devastating consequences. A famous, domestic example of this was the recent water crisis in Flint, Michigan. In what was largely an economically dominated move, the white municipality leaders decided to switch the water source from the Detroit water system to the Flint river. However, the similarities between Flint and Japan end there. The Flint river was known to be contaminated with lead, and white leaders purposely overlooked this element at the expense of the largely black citizenry, in a common case of racial injustice tied to environmental injustice. \autocite{fennell_limn_2016} In the case of Japan, there is a third-party watchdog involved, namely the International Atomic Energy Agency, which has deemed the plan as safe. \autocite{noauthor_iaea_2023} More generally, however, there is a negative attitude of the global community towards nuclear fission that is dangerous and will be explored in the following.

In particular, Germany, which had formerly used nuclear fission as a vital part of its energy mix, closed down most of its reactors. There was some debate about backtracking on this action after the natural gas supplied by Russia was no longer available after its invasion of Ukraine, but nevertheless, the German government continued with its original plan. The Green Party in Germany was a chief supporter of shutting down nuclear fission capacity since the twentieth century. Back then, the climate crisis was not so visceral, and nuclear fission seemed like an environmental concern, as this technology was not well controlled. \autocite{noauthor_q_2021} However, things have changed since then. In the wake of shutting down its nuclear fission capacity, the German government must instead rely on coal for its electricity generation. Nuclear fission is commonly known to generate more energy than coal. The primary concern has always been related with the safety of nuclear fission. However, it is ironic, because science has reached a consensus regarding the safety of nuclear fission versus coal. For one, \textquote{The deaths from nuclear power (deaths per terawatt hour) are far eclipsed by its competitors with only 0.03 deaths per terawatt hour, compared to 32.72 for Brown Coal.} Furthermore, there is large public concern about nuclear fission causing cancer, but studies have repeatedly shown that this is false, with fossil fuels presenting the greater risk. \autocite{cohen_greta_nodate} Then, there is the argument that societies should focus their efforts on saving energy. However, climate activist Greta Thunberg dispels this notion. She provides the example of her own Swedish government which doesn't tell its citizens to save energy in order to decrease emissions. She explains that in general, people tend to be wary of any form of government intervention and the fact that even the socialist-leaning Scandinavian country doesn't do this is telling. \autocite{noauthor_why_2022} Additionally, there have been recent advances in actinide chemistry for the efficient processing of waste relating to nuclear fission. \autocite{costa_peluzo_uranium_2022} While the scientific community recognizes the suffering that the disasters in Chernobyl and Fukushima have caused, there is a continued commitment to safe, civilian nuclear power.

There is a review of Bill Gates' recent book on climate change, in which Bill Gates talks about nuclear fission. The reviewer takes issue with Bill Gates' oversimplification of reducing issues with the technology to the number of direct accidents it causes, and also with the fact that Gates comments on this issue, as an owner of a company developing advanced nuclear fission reactors. \autocite{reader_resolutereader_2021} While it is true that nuclear fission does have some negative environmental impacts, it should be considered what nuclear fission has already done: \textquote{Over the past 50 years, nuclear energy reduced CO2 emissions by 60 gigatons- nearly two years’ worth of global energy emissions.} \autocite{cohen_greta_nodate} This technology must continue to be leveraged in order to prevent ecological collapse.

Now, recent advances with the technology of nuclear fission will be showcased. Particularly, there has been an advent of the so-called small modular reactors. They are smaller than their traditional counterparts, and so do not need the typically large power plant site in order to operate. They come in standardized parts that are easier to manufacture. Furthermore, they have robust safety features. \autocite{noauthor_what_2021}

A review of nuclear fission aptly summarizes the issue at hand. It explains that,
while it is naive to think that technology will save us from the climate crisis, this century presents us with a paradox of ambitious decarbonization goals amidst heavy energy demands. \autocite{noauthor_future_nodate} Only technology, of which nuclear fission is currently the best bet, can solve this problem.
\newpage


\printbibliography

\end{flushleft}
\end{document}
