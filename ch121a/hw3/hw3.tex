
\documentclass[12pt]{article}
\usepackage{amsmath} % for math symbols and equations
\usepackage{amsfonts} % for math fonts
\usepackage{amssymb} % for additional math symbols
\usepackage{geometry} % for page margins
\geometry{margin=1in} % set 1-inch page margins
\title{Ch 121a HW}
\author{Patryk Kozlowski}
\date{\today}
\begin{document}
\maketitle
\section{Problem 1}
\subsection{Part a}
 ISMEAR determines how the partial occupancies are set for each orbital. It provides many options to choose from, of which we only used one.
\subsection{Part b}
SIGMA specifies the width of the smearing in eV. Its default is 0.2.
\subsection{Part c}
If you have no a priori knowledge of your system, for instance, if you do not know whether your system is an insulator, semiconductor or metal then always use Gaussian smearing ISMEAR=0 in combination with a small SIGMA=0.03-0.05.
\section{Problem 2}
\subsection{Part a}
The 4 \AA\ box gives an electronic free energy of -15.457164 eV. The 15 \AA\ box gives an electronic free energy of -14.767794 eV. The 4 \AA\ box gives a lower energy, since the atoms are closer, thus there will be stronger vdW interactions.
\subsection{Part b}
Pure, dry air has a density of 1.293 kg/m$^3$ at a temperature of 273 K and a pressure of 101.325 kPa, according to NASA. The 4 \AA\ box gives a density of 726.7622882763201 kg/m$^3$. This is 562.5 times denser than pure air, which is not realistic. The 15 \AA\ box gives a density of 13.781566355462072 kg/m$^3$, which is more realistic.
\section{Problem 3}
Using the calculation results from carbon monoxide from 15 \AA\ box, because this is more similar to the actual density of air.
\subsection{Part a}
2.44 eV
\subsection{Part b}
2.41 eV
\subsection{Part c}
0.92 eV
\subsection{Part d}
The top site binds CO the strongest
\end{document}
