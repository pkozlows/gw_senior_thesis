
\documentclass[12pt]{article}
\usepackage{physics}
\usepackage{amsmath}

\title{agenda}
\author{Patryk Kozlowski}
\date{\today}
\begin{document}
\maketitle
\section{how can I get into graduate school?}
I only put it this way, because I have recently learned a lot of skills that will enable me to be successful in graduate school. that being said, I recognize that for the past two years, I have been doing a lot of therapy and not that much academics. I am ready to do a lot of learning now and in graduate school, but I just want us to figure out what I can do to get into a good graduate school so I can continue my learning there.
\subsection{what has changed since the last time we spoke.}
\subsubsection{voice coding}
I have become proficient with voice coding, which has enabled me to program and to also typeset in LaTeX. 
\subsubsection{full configuration interaction starter project}
I have been working on this some for the past few months and it is challenging me mentally in a good way, but I am wondering if I need to do something more research-based over the summer for graduate school purposes?
%(I am very much willing to learn something new, but i imagine it will require some more mentorship that is synchronious over zoom from a graduate student. for the full configuration interaction starter project, I have been raising a lot of github issues asynchronously and that has worked for this, as I have not been working on this project fulltime, but if I want to work on research project full time over the summer, it might be helpful for me for it to be synchronous.)
\subsubsection{my field of interest}
as a function of having to think about full configuration interaction, I have realized that I really like thinking about electronic structure theory. more specifically, I like having to think about how I have to implement the formalism of second quantization from theory to code.
\subsection{potential summer research}
I was talking to rui about this, and this is what he said:\\
\textit{I am about to start implementing gradient of electronic repulsion integrals in periodic systems, from CPU to GPU, into pyscf. This will be super technical, but you can ask Garnet whether you can participate, as long as you feel interested. There are some others like DMET (Density-Matrix-Embedding-Theory) studies going on in Chan group as well that fit into the category of “studying condensed matters”. 
}\\
O ya, I forgot to mention this earlier, but I am kind of interested in sustainability and condensed matter is right up this alley, so \emph{the category of “studying condensed matters} might be of interest to me?
\subsection{future undergraduate coursework}
you don't need to know all of the specifics, but other than chemistry requirements that I need to take still, I have some room for some physics courses of interest, and was just curious what you think of what I want to take. because it seems that I am interested in electronic structure and maybe specifically in periodic systems, I am thinking of taking some coursework in statistic physics (ph127ab if you are familiar?)
\subsection{what kind of graduate programs do I want to look into?}
my current thought is that graduate programs in chemistry, materials science, and physics (condensed-matter) might fit into my field of interest, as I stated earlier. my thought is that I would like a physics graduate program, since most of the clases I want to take at Caltech are in the physics option, but I have heard th at they can be pretty stingy about accepting chemistry under graduates to physics programs. do you have any thoughts about this?
\end{document}
