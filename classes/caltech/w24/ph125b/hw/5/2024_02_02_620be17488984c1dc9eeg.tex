\documentclass[12pt]{article}
\usepackage[utf8]{inputenc}
\usepackage[T1]{fontenc}
\usepackage{amsmath}
\usepackage{amsfonts}
\usepackage{physics}

\usepackage{amssymb}
\usepackage[version=4]{mhchem}
\usepackage{stmaryrd}

\usepackage{listings} % Required for insertion of code
\usepackage{xcolor} % Required for custom colors

% Define custom colors
\definecolor{codegreen}{rgb}{0,0.6,0}
\definecolor{codegray}{rgb}{0.5,0.5,0.5}
\definecolor{codepurple}{rgb}{0.58,0,0.82}
\definecolor{backcolour}{rgb}{0.95,0.95,0.92}

% Setup the style for code listings
\lstdefinestyle{mystyle}{
    backgroundcolor=\color{backcolour},   
    commentstyle=\color{codegreen},
    keywordstyle=\color{magenta},
    numberstyle=\tiny\color{codegray},
    stringstyle=\color{codepurple},
    basicstyle=\ttfamily\footnotesize,
    breakatwhitespace=false,         
    breaklines=true,                 
    captionpos=b,                    
    keepspaces=true,                 
    numbers=left,                    
    numbersep=5pt,                  
    showspaces=false,                
    showstringspaces=false,
    showtabs=false,                  
    tabsize=2
}

% Activate the style
\lstset{style=mystyle}

\title{PROBLEMS: }

\author{}
\date{}


\begin{document}
\maketitle
Physics $125 \mathrm{~b}$

Midterm (Problem set number 5)

Due midnight Wednesday, February 7, 2024

READING: Section 18.5 in Shankar on the interaction of atoms with electromagnetic radiation.

This is the Ph 125a "midterm", aka problem set 5 .

Collaboration prohibited: Please do this problem set without consulting with or working with other students. This is the only restriction compared with our usual problem sets. You may use Piazza and attend office hours.
\section{}
\begin{enumerate}
  \setcounter{enumi}{16}
  \item See if you can generalize the result for the first Born approximation:
\end{enumerate}

$$
\frac{d \sigma}{d \Omega^{\prime}}=\frac{m^{2}}{(2 \pi)^{2}}\left|\hat{V}\left(\mathbf{p}^{\prime}-\mathbf{p}\right)\right|^{2}
$$

to the case where the scattered particle (mass $m_{f}$ ) may have a different mass than the incident particle (mass $m_{i}$ ).
\subsection{}
Looking over the note in 7.3 on approximate methods, I found the derivation for the Born approximation. Basically, this scattering rate is given by an expression of the form:
\begin{equation}
  \dd{\sigma }= \frac{1}{\text{flux}}\dd{\Gamma }
\end{equation}
and the idea is that the masses that go into this flux and rate are different. For the flux, we care about the mass of the incident particle $m_{i}$,
and then for the differential rate, there is an expression of the form:
\begin{align}
  \dd{\Gamma }&= \int_{\mathbf{p'}\in \dd{\Omega '}}\frac{L^3}{(2\pi)^3}\dd^3{\mathbf{p'}} 2 \pi \frac{\abs{\hat{V}(\mathbf{p'}-\mathbf{p})}^2}{L^6}\delta (E_{f}-E_{i})\\
&= \frac{\dd{\Omega '}}{L^3}\frac{m_{f}p}{(2\pi)^2}\abs{\hat{V}(\mathbf{p'}-\mathbf{p})}^2
\end{align}
And this term divided by the flux gives us the differential cross section in the first Born approximation, which for equal masses is expressed as:
\begin{equation}
  \frac{\dd{\sigma }}{\dd{\Omega '}}= \frac{m^2}{(2\pi)^2}\abs{\hat{V}(\mathbf{p'}-\mathbf{p})}^2
\end{equation}
This time the two masses will be different from each other, so we can write the formula as:
\begin{equation}
  \frac{\dd{\sigma }}{\dd{\Omega '}}_{\text{diffms}}= \frac{m_{f}m_{i}}{(2\pi)^2}\abs{\hat{V}(\mathbf{p'}-\mathbf{p})}^2
\end{equation}
\section{}
\begin{enumerate}
  \setcounter{enumi}{17}
  \item We consider the potential (called the "Yukawa potential"):
\end{enumerate}

$$
V(\mathbf{x})=\frac{K e^{-\mu r}}{r}, \quad r=|\mathbf{x}|
$$

with real parameters $K$ and $\mu>0$. The parameter $K$ can be regarded as the "strength" of the potential ("interaction"), and $\frac{1}{\mu}$ is effectively the "range" of distance over which the potential is important. $\mu$ itself has units of mass - note that as $\mu \rightarrow 0$ we obtain the Coulomb potential: $\mu$ can be thought of as the mass of an "exchanged particle" which mediates the force. In electromagnetism, this is the photon, hence $\mu \rightarrow m_{\gamma}=0$

(a) Find a condition on $K$ and $\mu$ which guarantees that there are at least $n$ bound states in this potential. You will likely fashion and use some kind of "comparison" theorem in arriving at your result. You should give at least a "heuristically convincing" argument, if you don't actually prove it.
\subsection{}
We know that for any bound states at all, we need $K<0$ or else we would just be talking about the free particle. Let us consider the Yukawa potential as a perturbation for our original Coulomb potential where we have $H_0 = \frac{p^2}{2m} + \frac{K}{r}$ and the perturbation $V= K\left(\frac{e^{- \mu r}}{r} - \frac{1}{r}\right)$:
\begin{equation}
  H = H_0 + V
\end{equation}
It is always the case that $K<0$, and the quantity inside of the parentheses will always be negative, so let's maximize the inside of the parentheses. We can expand it in a Taylor series:
\begin{equation}
  e^{-\mu r} = 1 - \mu r + \frac{(\mu r)^2}{2} + \ldots
\end{equation}
So we can write the inside of the parentheses as:
\begin{equation}
  \frac{e^{-\mu r}}{r} - \frac{1}{r} = -\mu + \frac{\mu^2 r}{2} + \ldots
\end{equation}
We want to minimize this with respect to $\mu$:
\begin{equation}
  \dv{\mu}\left(-\mu + \frac{\mu^2 r}{2}\right) = 0
\end{equation}
We find that the minimizing value of $\mu$ is $\mu = \frac{1}{r}$. 
% Inline Python code in the document
\begin{lstlisting}[language=Python]
from sympy import *
mu, r = symbols('mu r')

# I want to solve this minimization problem by differentiating with respect to mu: \dv{\mu}\left(-\mu + \frac{\mu^2 r}{2}\right) = 0

minimization_problem = Eq(diff(-mu + mu**2 * r / 2, mu), 0)
mu_solution = solve(minimization_problem, mu)

print(latex(mu_solution[0].simplify()))
\end{lstlisting}

(b) Using the Born approximation for the differential cross section that we developed in our discussion of time-dependent perturbation theory, calculate the differential cross section, $\frac{d \sigma}{d \Omega}$, for scattering on this potential. Consider the limit $\mu \rightarrow 0$ and compare with the classical Rutherford cross section (Is this calculation valid?).
\subsection{}
We want to consider the equation:
\begin{equation}
  \frac{\dd{ u}}{\dd{\theta }} +u = - \kappa 
\end{equation}
where in this case $u= \frac{Ke^{-\mu r }}{r}$. We will end up with a result of $\frac{\dd{\sigma }}{\dd{\Omega }}=?$\\
(c) Integrate your differential cross section over all solid angles to obtain the "total cross section". Again, consider the limit $\mu \rightarrow 0$. Assuming the Rutherford cross section holds, what is the total cross section for scattering on a Coulomb potential?
\subsection{}
Here we probably want to consider:
\begin{equation}
  \sigma = \int_{\Omega }? \dd{\Omega } 
\end{equation}
\section{}
\begin{enumerate}
  \setcounter{enumi}{18}
  \item Suppose we have a system consisting of two spin- $\frac{1}{2}$ 's $\left(\mathbf{S}_{1}\right.$ and $\left.\mathbf{S}_{2}\right)$, with an interaction Hamiltonian $a(t) \mathbf{S}_{1} \cdot \mathbf{S}_{2}$. Assume that $a( \pm \infty)=0$ and that $a(t)$ is significantly different from zero in an interval of order $\tau$ in width about $t=0$.
(a) Suppose at very early times $(t \rightarrow-\infty)$, the system is in state
\end{enumerate}

$$
|\psi(-\infty)\rangle=|+-\rangle
$$

where the state is labeled by the $z$ components of $\mathbf{S}_{1}$ and $\mathbf{S}_{2}$, respectively. Without making any approximations, what is the state of the system at $t=\infty$ ? What is the probability, $P(+-\rightarrow-+)$, that the state is observed to be $|-+\rangle$ at $t=\infty$ ? Show that this probability depends only on $\int_{-\infty}^{\infty} a(t) d t$.
\subsection{}
We know $\mathbf{S}_{1} \cdot \mathbf{S}_{2}$ can be decomposed into the 
Don't just take the standard dod product between the spin operators, but rather express them in terms of J. it will be a 2 by 2 matrix. We also want to consider a propagator\\
(b) Now calculate $P(+-\rightarrow-+)$ in first-order time-dependent perturbation theory. What is a condition for the validity of first-order time-dependent perturbation theory?
\subsection{}
We want to compute:
\begin{equation}
  P(+-\rightarrow-+)=|\langle-+|U(\infty,-\infty)|+-\rangle|^{2}
\end{equation}
\begin{enumerate}
  \setcounter{enumi}{19}
  \item With reference to the previous problem, consider the "Zeeman effect", with the addition of a uniform, static magnetic field of strength $B_{0}$ in the direction of the $z$-axis. Thus, we have another term in the Hamiltonian:
\end{enumerate}

$$
H_{0}=-B_{0}\left(g_{1} S_{1 z}+g_{2} S_{2 z}\right),
$$

where $g_{1}$ and $g_{2}$ are the gyromagnetic ratios of the two spins. We assume a gaussian form for $a(t)$ :

$$
a(t)=a(0) e^{-(t / \tau)^{2}}
$$

Repeat your first order perturbation theory calculation of the previous problem in the presence of the magnetic field and with this explicit form for $a(t)$, and determine $P(+-\rightarrow-+)$. Discuss the variation of this probability, for given $a(0)$ and $\tau$, as a function of $B_{0}$.


\end{document}