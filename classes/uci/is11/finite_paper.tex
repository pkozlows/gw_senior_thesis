\documentclass[12pt,letterpaper]{article}
\usepackage[american]{babel}
\usepackage{csquotes}
\usepackage[style=mla,backend=biber]{biblatex}
\usepackage[letterpaper]{geometry}
\usepackage{times}
\geometry{top=1.0in, bottom=1.0in, left=1.0in, right=1.0in}
\usepackage{lipsum}  
%
%Doublespacing
%
\usepackage{setspace}
\doublespacing

%
%Rotating tables (e.g. sideways when too long)
%
\usepackage{rotating}


%
%Fancy-header package to modify header/page numbering (Put your surname in the surname field)
%
\usepackage{fancyhdr}
\pagestyle{fancy}
\lhead{} 
\chead{} 
\rhead{Kozlowski \thepage}
\lfoot{} 
\cfoot{} 
\rfoot{} 
\renewcommand{\headrulewidth}{0pt} 
\renewcommand{\footrulewidth}{0pt} 
%To make sure we actually have header 0.5in away from top edge
%12pt is one-sixth of an inch. Subtract this from 0.5in to get headsep value
\setlength\headsep{0.333in}

%
%Works cited environment
%(to start, use \begin{workscited...}, each entry preceded by \bibent)
% - from Ryan Alcock's MLA style file
%
\newcommand{\bibent}{\noindent \hangindent 40pt}
\newenvironment{workscited}{\newpage \begin{center} Works Cited \end{center}}{\newpage }


%You put your name title and professor name in here 
\renewcommand{\maketitle}{\makemlaheader}
\newcommand{\C}{\autocite}

\newcommand{\makemlaheader}{
Patryk Kozlowski \\
Eve Darian-Smith \\
IS 17\\
\today\\
\begin{center}\textnormal{Final Paper}\end{center}

% for some reason, this blank line is necessary
}






\addbibresource{citations.bib} % you would use your own bib file here
\begin{document}
\begin{flushleft}

%%%%First page name, class, etc

\maketitle




%%%%Title
There has been a rising trend of anti-minority sentiment, and particularly Islamophobia, across the globe. Recently in Sweden, an immigrant from Iraq publicly burned a copy of the Quran. This action was met with widespread disapproval from Muslim-majority nations in the Middle East, who went as far as to sever diplomatic ties with Sweden, and noticeably lighter disapproval from Western nations, who cited free speech concerns when justifying their restraint in condemning the action. \autocite{noauthor_outcry_nodate}

Culture has been defined in many ways, and there are often as many as seven concepts associated with the term. \autocite{noauthor_what_nodate} In the context of this situation, we will focus on three of them, namely group membership, structure/pattern, and refinement. Trivially, the group membership element involves the Western nation of Sweden, with the majority of its population identifying as Christian, in conflict with the non-Western nations of the Middle East, with their majority Muslim population. Structure/pattern involves culture as a framework through which the idea of burning the Quran is interpreted differently; to the West, it is no big deal, but to Muslim nations, it is a major desecration. Finally, there is the concept of refinement, where ethnocentrism is an important factor. The people of Sweden see the public burning of the Quran merely as an expression of the liberal value of freedom of speech. Therefore, the fact that Muslim nations take issue with this action is simply seen as a lack of refinement; the Western value of freedom of speech is seen as a marker of a civilized people from the Eurocentric point of view.

The freedom of speech is foundational in Western civilization. The ability to criticize people in power is often seen as what keeps democracy going. However, an ethnocentric point of view fails to understand that democracy is not often something that non-Western nations value as much. Furthermore, there are limits to this freedom of speech in Western nations. Defaming the head of state in many European countries is a crime, and Holocaust denial is justifiably forbidden. \autocite{noauthor_burning_nodate} Theoretically, desecrating the American flag is protected under First Amendment rights. However, practically, this could not be further from the case, as will be shown in the following paragraph. When you think about it, the American flag and the Quran are just materials. If these were not cultural artifacts, burning them would have no significance at all. Freedom of speech is used to encourage Islamophobia, but it is looked down upon when convenient with the American flag.

Take the case of Colin Kaepernick. He received media attention for choosing to kneel when the national anthem was played before football games in protest of racial injustice. The people in power in the sport of American football are largely white and conservative males, and they resisted this form of protest. Even though, based on merit, he should have been a quarterback for one of the NFL teams, he was all but blacklisted. He showed that he maintained game readiness by organizing workouts with NFL team scouts, but ultimately was not able to get a job. \autocite{noauthor_colin_2020} This serves to show that the freedom of speech can have very real consequences in Western society, so one could argue that it is not really protected.

A good comparison can be made between the mandatory saying of the Pledge of Allegiance in American schools, and the practice of veiling by Muslim women. In Western society, the former is held in high esteem, but the latter is met with hate. The answer to this quandary is ethnocentrism. Western society is able to comprehend a display of patriotism in the Pledge of Allegiance, but not a display of religious devotion in the practice of veiling that is not well understood.

A common example given to American students learning about freedom of speech is the example that you cannot yell "fire" in a crowded theater. This is because it is a direct threat to the safety of the people in the theater. Some might argue that the public burning of the Quran is not the same phenomenon. After all, no one is being hurt in a public burning of the Quran. While it is true that the burning of the Quran does not put anybody in direct danger, it has been shown that this phenomenon directly contributes to Islamophobic hate crimes. The Swedish National Crime Council found a recent increase in anti-Muslim hate crimes. This has also been the trend in nearby Denmark. These hate crimes are often directed towards women, for whom practicing their religion is synonymous with veiling. \autocite{boxerman_whats_2023}

Taking ethnocentrism aside, it is also important to consider where Sweden resides on the spectrum of low-context to high-context society. \autocite{christie_peace_2001} A common misconception would be that Sweden is, in fact, a high-context society. After all, it is well-known to Americans as a socialized state with a nationalized healthcare system. However, where Sweden lies on the context spectrum depends more on free speech context, in which it is very individualistic. When doing business in Sweden, words are often taken at face value. Exchanges are direct, and there's often not much hidden meaning. \autocite{altinkaya_cultural_nodate} In Sweden, more emphasis is placed on what is said rather than how it is said. So, for Swedish people, the public nature of the burning of the Quran doesn't matter so much as the message that is trying to be conveyed, which was a protest of Islamic ideology. However, for the majority of Middle Eastern countries in disagreement with this action, the public nature of the burning is quite imperative. It is normal that a Western nation, such as Sweden, might be in conflict with Islamic ideology, but the major problem is the public nature as to which this conflict is being expressed. Therefore, a public burning of the Quran may not mean that much to the Swedish people as it does to the Muslim nations who were outraged by the action.

Finally, it will be examined why confronting Islamophobia in Europe is especially important now, with the majority of Muslim nations in the Middle East becoming uninhabitable due to climate change in the next few centuries. If the European acceptance of refugees from war-torn Muslim countries such as Syria is to be a precedent, then avoiding conflict related to the mass migration of Muslim climate refugees will be a difficult, but nevertheless imperative task for avoiding conflict in the twenty-first century. \autocite{noauthor_arab_nodate} With heatwaves in the Middle East often exceeding fifty degrees Celsius, compounded by the flooding and sea level rise that the region will experience over the next few decades, people will be forced to migrate. \autocite{yabi_climate_nodate} The ability of Western nations to accept the cultural differences of Muslim refugees will be necessary to retain some semblance of organized civilization.

In his book on how culture shapes the climate debate, Arnold Hoffman introduces the idea of the climate broker. \autocite{hoffmann_how_nodate} To give a quick explanation, people on the right will never trust what a liberal figure, such as Al Gore, has to say about climate change, no matter how strong his scientific evidence is. However, they might be willing to trust a messenger who is more aligned with their tribe's conservative political view, like Pope Francis. On the climate front, the Pope has shown to have considered social and environmental justice concerns. Furthermore, he recently denounced the burning of the Quran and has been a strong advocate for religious tolerance. \autocite{noauthor_outcry_nodate} As more refugees from Muslim countries in the Middle East migrate to Europe, it is important that there is someone from the Christian majority that is willing to vouch for them.




\newpage

\printbibliography


\end{flushleft}
\end{document}