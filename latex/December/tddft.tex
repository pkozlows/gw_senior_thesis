\documentclass[12pt]{article}
\usepackage{amsmath}
\usepackage{physics}
\usepackage{simplewick}
\usepackage{simpler-wick}
\author{Patryk Kozlowski}
\title{TDDFT Routines}
\date{\today}
\begin{document}
\maketitle
\section{RPA matrix equations}
We have $\textbf{A}_{ia,jb}$ and $\textbf{B}_{ia,jb}$ defined as:
\begin{equation}
    \textbf{A}_{ia,jb} = \delta _{ij}\delta _{ab}(\varepsilon _{a}- \varepsilon _{i}) + (\underline{ia}||\underline{jb})
\end{equation}
\begin{equation}
    \textbf{B}_{ia,jb} = (\underline{ia}||\underline{bj})
\end{equation}
\subsection{Direct approximation}
In the divert approximation, which we are working with because it is simpler to start with, this becomes:
\begin{equation}
    \textbf{A}_{ia,jb} = \delta _{ij}\delta _{ab}(\varepsilon _{a}- \varepsilon _{i}) + (\underline{ia}|\underline{jb})
\end{equation}
\begin{equation}
    \textbf{B}_{ia,jb} = (\underline{ia}|\underline{bj})
\end{equation}
After the spin integration, we get:
\begin{equation}
    \textbf{A}_{ia,jb} = \delta _{ij}\delta _{ab}(\varepsilon _{a}- \varepsilon _{i}) + 2(ia|jb)
\end{equation}
\begin{equation}
    \textbf{B}_{ia,jb} = 2(ia|bj)
\end{equation}
We build the eigenvalue equation:
\begin{equation}
\begin{bmatrix}
A & B \\
-B & -A
\end{bmatrix}
\begin{bmatrix}
X \\
Y
\end{bmatrix}
= \omega
\begin{bmatrix}
1 & 0 \\
0 & -1
\end{bmatrix}
\begin{bmatrix}
X \\
Y
\end{bmatrix}
\end{equation}
\subsection{Tamm-Dancoff approximation}
In the Tamm-Dancoff approximation, we neglect $\textbf{B}$, which means that we have:
\begin{equation}
    \textbf{A}\textbf{X} = \omega \textbf{X}
\end{equation}
The eigenvalues $\omega $ are the excitation energies, but we need to do some work to find the excitation vectors $V_{pq}^{\mu }$. However, we will use the eigenvectors from this matrix equation to build the excitation vectors.
\subsection{Building the excitation doctors}
First, we consider within the direct approximation:
\begin{equation}
    \textbf{W}_{p,q,ia} = \sum_{\underline{p,q,i,a}} (\underline{pq}|\underline{ia}) 
\end{equation}
\subsubsection{Spin integration}
$\ket{\Phi  _i^a}$ has the CSF
\begin{equation}
    \ket{\Phi  _{singlet}} = \frac{1}{\sqrt{2}}(\ket{\Phi _{i\alpha }^{a\alpha  }} + \ket{\Phi _{i\beta }^{a\beta  }})
\end{equation}
We want to consider something like:
\begin{equation}
\bra{\Phi _0}
    \frac{1}{4}\sum_{pqrs} V_{pq\underline{rs}}a^{\dag}_{p}a^{\dag}_{q}a_{s}a_{r}\frac{1}{\sqrt{2}}(a^{\dag \alpha }_{a}a^{\alpha }_{i} + a^{\dag \beta }_{a}a^{\beta }_{i})\ket{\Phi _0}
\end{equation}
Consolidating constants out front and distributing the CSF terms:
\begin{equation}
    \frac{1}{4\sqrt{2}}V_{pq\underline{rs}}\bra{\Phi _0}a^{\dag}_{p}a^{\dag}_{q}a_{s}a_{r}a^{\dag \alpha }_{a}a^{\alpha }_{i}\ket{\Phi _0} + \frac{1}{4\sqrt{2}}V_{pq\underline{rs}}\bra{\Phi _0}a^{\dag}_{p}a^{\dag}_{q}a_{s}a_{r}a^{\dag \beta }_{a}a^{\beta }_{i}\ket{\Phi _0}
\end{equation}
Let's consider the first term:
\begin{equation}
    \frac{1}{4\sqrt{2}}\sum_{pqrs} V_{pq\underline{rs}}\bra{\Phi _0}a^{\dag}_{p}a^{\dag}_{q}a_{s}a_{r}a^{\dag \alpha }_{a}a^{\alpha }_{i}\ket{\Phi _0}
\end{equation}


\end{document}