\documentclass[12pt]{article}
\usepackage{amsmath}
\usepackage{physics}
\usepackage{graphicx}
\usepackage{hyperref}
\usepackage{listings} % Required for insertion of code
\usepackage{xcolor} % Required for custom colors

% Define custom colors
\definecolor{codegreen}{rgb}{0,0.6,0}
\definecolor{codegray}{rgb}{0.5,0.5,0.5}
\definecolor{codepurple}{rgb}{0.58,0,0.82}
\definecolor{backcolour}{rgb}{0.95,0.95,0.92}

% Setup the style for code listings
\lstdefinestyle{mystyle}{
    backgroundcolor=\color{backcolour},   
    commentstyle=\color{codegreen},
    keywordstyle=\color{magenta},
    numberstyle=\tiny\color{codegray},
    stringstyle=\color{codepurple},
    basicstyle=\ttfamily\footnotesize,
    breakatwhitespace=false,         
    breaklines=true,                 
    captionpos=b,                    
    keepspaces=true,                 
    numbers=left,                    
    numbersep=5pt,                  
    showspaces=false,                
    showstringspaces=false,
    showtabs=false,                  
    tabsize=2
}

% Activate the style
\lstset{style=mystyle}

\author{Patryk Kozlowski}
\title{Fock Routine}
\date{\today}
\begin{document}
\maketitle
This is probably a simple question, but I am not understanding why my Fock construction doesn't seem to match that of PySCF. My test case for the $G_0W_0$ energy matches that of PySCF when I use the dTDA for my TDDFT and a Fock matrix @HF density, which I assume means the density matrix here. Are their instances where this means something different? One you say to evaluate something at the HF/DFT density, is that synonymous with saying to use the density matrix for that mean field object? What I am doing seems correct, but when I compare it to
% Inline Python code in the document
\begin{lstlisting}[language=Python]
mf_hf.get_fock(dm=mf_dft.make_rdm1())
\end{lstlisting}
 I get different results. I assume this is why I am getting different results for the $G_0W_0$ energy.
I am using the following code to construct my Fock matrix:
% Inline Python code in the document
\begin{lstlisting}[language=Python]
def fock_dft(mf):
    '''Calculates the HF fock matrix using the DFT electron density in AO basis for a given set of molecular orbitals and occupations.

    Parameters:
    mf (object): The object representing the mean-field calculation.

    Returns:
    mo_fock (ndarray): The Fock matrix in the molecular orbital basis.
    '''
    # make the common variables
    n_orbitals = mf.mol.nao_nr()

    # initialize the ao_fock matrix in the atomic orbital basis
    ao_fock = np.zeros((n_orbitals, n_orbitals))

    # get the core hamiltonian
    h_core = mf.get_hcore()

    # get the coulumb term
    coulumb_matrix = np.zeros((n_orbitals, n_orbitals))
    coulumb_matrix += np.einsum('rs,pqrs->pq', mf.make_rdm1(), mf.mol.intor('int2e').reshape((n_orbitals, n_orbitals, n_orbitals, n_orbitals)))

    # get the exchange term
    exchange_matrix = np.zeros((n_orbitals, n_orbitals))
    exchange_matrix += np.einsum('rs,prqs->pq', mf.make_rdm1(), mf.mol.intor('int2e').reshape((n_orbitals, n_orbitals, n_orbitals, n_orbitals)))

    # add the terms
    ao_fock += (h_core + coulumb_matrix - 0.5*exchange_matrix)

    # convert the fock matrix to the molecular orbital basis
    mo_fock = np.einsum('ia,ij,jb->ab', mf.mo_coeff, ao_fock, mf.mo_coeff)

    return mo_fock
\end{lstlisting}
Maybe this is a simple question, but when you say that the HF Fock matrix is diagonal in the energies in the canonical representation, what does this mean? I am familiar with what the term canonical means in second quantization, that is, the appropriate ordering of orbitals. I don't know how to think about this in first quantization?
\end{document}