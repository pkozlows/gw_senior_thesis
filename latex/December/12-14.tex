\documentclass[12pt]{article}
\usepackage{amsmath}
\usepackage{physics}
    \usepackage{graphicx}
\usepackage[linesnumbered,ruled,vlined]{algorithm2e}
\usepackage{listings} % Add this package to include code listings
\usepackage{xcolor} % Required for custom colors
\usepackage{simplewick}

% Define custom colors
\definecolor{codegreen}{rgb}{0,0.6,0}
\definecolor{codegray}{rgb}{0.5,0.5,0.5}
\definecolor{codepurple}{rgb}{0.58,0,0.82}
\definecolor{backcolour}{rgb}{0.95,0.95,0.92}

% Set up the style for Python code listings
\lstdefinestyle{pythoncode}{
    backgroundcolor=\color{backcolour},   
    commentstyle=\color{codegreen},
    keywordstyle=\color{magenta},
    numberstyle=\tiny\color{codegray},
    stringstyle=\color{codepurple},
    basicstyle=\ttfamily\footnotesize,
    breakatwhitespace=false,         
    breaklines=true,                 
    captionpos=b,                    
    keepspaces=true,                 
    numbers=left,                    
    numbersep=5pt,                  
    showspaces=false,                
    showstringspaces=false,
    showtabs=false,                  
    tabsize=2,
    language=Python % Specify the language for syntax highlighting
}

% Apply the style to Python listings
\lstset{style=pythoncode}
\author{Patryk Kozlowski}
\title{G0W0}
\date{\today}
\begin{document}
\maketitle
\section{Deriving complicated spin integration}
We are trying to get from
\begin{equation}
    W_{p,q,i,a} = \sum_{\underline{p,q,i,a}} (\underline{p} \underline{q} | \underline{i} \underline{a} )
\end{equation}
to
\begin{equation}
    W_{p,q,i,a} = \sqrt{2} \sum_{p,q,i,a} (pq|ia)
\end{equation}
This deprecation will require work in second quantization using weeks thm:
We start with the two electron operator in second quantization:
\begin{equation}
    \hat{V} = \frac{1}{4} \sum_{pqrs} V_{pqrs} \hat{a}_{p}^{\dagger} \hat{a}_{q}^{\dagger} \hat{a}_{s} \hat{a}_{r}
\end{equation}
We are interested in how this acts on the singlet CSF:
\begin{equation}
    \ket{\Psi _{S}} = \frac{1}{\sqrt{2}} (a_{a}^{\alpha \dagger } a_{i}^{\alpha } + a_{a}^{\beta \dagger } a_{i}^{\beta })
\end{equation}
We act the operator on the singled state and use Wick's theorem to simplify:
\begin{equation}
    \hat{V} \ket{\Psi _{S}} = \frac{1}{4} \sum_{pqrs} V_{pqrs} \hat{a}_{p}^{\dagger} \hat{a}_{q}^{\dagger} \hat{a}_{s} \hat{a}_{r} \frac{1}{\sqrt{2}} (a_{a}^{\alpha \dagger } a_{i}^{\alpha } + a_{a}^{\beta \dagger } a_{i}^{\beta })
\end{equation}
\begin{equation}
    \hat{V} \ket{\Psi _{S}} = \frac{1}{4} \sum_{pqrs} V_{pqrs} \frac{1}{\sqrt{2}} \left(\hat{a}_{p}^{\dagger} \hat{a}_{q}^{\dagger} \hat{a}_{s} \hat{a}_{r} a_{a}^{\alpha \dagger } a_{i}^{\alpha } + \hat{a}_{p}^{\dagger} \hat{a}_{q}^{\dagger} \hat{a}_{s} \hat{a}_{r} a_{a}^{\beta \dagger } a_{i}^{\beta }\right)
\end{equation}

\end{document}